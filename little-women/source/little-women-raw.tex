% Options for packages loaded elsewhere
\PassOptionsToPackage{unicode}{hyperref}
\PassOptionsToPackage{hyphens}{url}
\documentclass[
  english,
]{article}
\usepackage{xcolor}
\usepackage{amsmath,amssymb}
\setcounter{secnumdepth}{-\maxdimen} % remove section numbering
\usepackage{iftex}
\ifPDFTeX
  \usepackage[T1]{fontenc}
  \usepackage[utf8]{inputenc}
  \usepackage{textcomp} % provide euro and other symbols
\else % if luatex or xetex
  \usepackage{unicode-math} % this also loads fontspec
  \defaultfontfeatures{Scale=MatchLowercase}
  \defaultfontfeatures[\rmfamily]{Ligatures=TeX,Scale=1}
\fi
\usepackage{lmodern}
\ifPDFTeX\else
  % xetex/luatex font selection
\fi
% Use upquote if available, for straight quotes in verbatim environments
\IfFileExists{upquote.sty}{\usepackage{upquote}}{}
\IfFileExists{microtype.sty}{% use microtype if available
  \usepackage[]{microtype}
  \UseMicrotypeSet[protrusion]{basicmath} % disable protrusion for tt fonts
}{}
\makeatletter
\@ifundefined{KOMAClassName}{% if non-KOMA class
  \IfFileExists{parskip.sty}{%
    \usepackage{parskip}
  }{% else
    \setlength{\parindent}{0pt}
    \setlength{\parskip}{6pt plus 2pt minus 1pt}}
}{% if KOMA class
  \KOMAoptions{parskip=half}}
\makeatother
\usepackage{longtable,booktabs,array}
\usepackage{calc} % for calculating minipage widths
% Correct order of tables after \paragraph or \subparagraph
\usepackage{etoolbox}
\makeatletter
\patchcmd\longtable{\par}{\if@noskipsec\mbox{}\fi\par}{}{}
\makeatother
% Allow footnotes in longtable head/foot
\IfFileExists{footnotehyper.sty}{\usepackage{footnotehyper}}{\usepackage{footnote}}
\makesavenoteenv{longtable}
\usepackage{graphicx}
\makeatletter
\newsavebox\pandoc@box
\newcommand*\pandocbounded[1]{% scales image to fit in text height/width
  \sbox\pandoc@box{#1}%
  \Gscale@div\@tempa{\textheight}{\dimexpr\ht\pandoc@box+\dp\pandoc@box\relax}%
  \Gscale@div\@tempb{\linewidth}{\wd\pandoc@box}%
  \ifdim\@tempb\p@<\@tempa\p@\let\@tempa\@tempb\fi% select the smaller of both
  \ifdim\@tempa\p@<\p@\scalebox{\@tempa}{\usebox\pandoc@box}%
  \else\usebox{\pandoc@box}%
  \fi%
}
% Set default figure placement to htbp
\def\fps@figure{htbp}
\makeatother
\ifLuaTeX
  \usepackage{luacolor}
  \usepackage[soul]{lua-ul}
\else
  \usepackage{soul}
\fi
\ifLuaTeX
\usepackage[bidi=basic,shorthands=off,]{babel}
\else
\usepackage[bidi=default,shorthands=off,]{babel}
\fi
\ifLuaTeX
  \usepackage{selnolig} % disable illegal ligatures
\fi
\setlength{\emergencystretch}{3em} % prevent overfull lines
\providecommand{\tightlist}{%
  \setlength{\itemsep}{0pt}\setlength{\parskip}{0pt}}
\usepackage{bookmark}
\IfFileExists{xurl.sty}{\usepackage{xurl}}{} % add URL line breaks if available
\urlstyle{same}
\hypersetup{
  pdftitle={Little Women; Or, Meg, Jo, Beth, and Amy},
  pdfauthor={Louisa May Alcott},
  pdflang={en},
  pdfsubject={March family (Fictitious characters) -\/- FictionMothers and daughters -\/- FictionBildungsromansNew England -\/- FictionFamily life -\/- New England -\/- FictionDomestic fictionSisters -\/- FictionYoung women -\/- FictionAutobiographical fiction},
  hidelinks,
  pdfcreator={LaTeX via pandoc}}

\title{Little Women; Or, Meg, Jo, Beth, and Amy}
\author{Louisa May Alcott}
\date{2011-08-16}

\begin{document}
\maketitle

\pandocbounded{\includegraphics[keepaspectratio]{303483661336987339_cover.png}}

\protect\phantomsection\label{wrap0000.xhtml}{}

\protect\phantomsection\label{6672479776654687619_37106-h-0.htm.xhtml}{}

\protect\phantomsection\label{6672479776654687619_37106-h-0.htm.xhtml_pg-header}
\begin{otherlanguage}{english}

\subsection[The Project Gutenberg eBook of ]{\texorpdfstring{The Project
Gutenberg eBook of
\protect\hypertarget{6672479776654687619_37106-h-0.htm.xhtml_pg-title-no-subtitle}{}\foreignlanguage{english}{Little
Women; Or, Meg, Jo, Beth, and
Amy}}{The Project Gutenberg eBook of Little Women; Or, Meg, Jo, Beth, and Amy}}\label{6672479776654687619_37106-h-0.htm.xhtml_pg-header-heading}

This ebook is for the use of anyone anywhere in the United States and
most other parts of the world at no cost and with almost no restrictions
whatsoever. You may copy it, give it away or re-use it under the terms
of the Project Gutenberg License included with this ebook or online at
\href{https://www.gutenberg.org}{www.gutenberg.org}. If you are not
located in the United States, you will have to check the laws of the
country where you are located before using this eBook.

\protect\phantomsection\label{6672479776654687619_37106-h-0.htm.xhtml_pg-machine-header}
\textbf{Title}: Little Women; Or, Meg, Jo, Beth, and Amy

\protect\phantomsection\label{6672479776654687619_37106-h-0.htm.xhtml_pg-header-authlist}
\textbf{Author}: Louisa May Alcott

\textbf{Illustrator}: Frank T. Merrill

\textbf{Release date}: August 16, 2011 {[}eBook \#37106{]}\\
Most recently updated: May 22, 2023

\textbf{Language}: English

\textbf{Credits}: David Edwards, Ernest Schaal, Robert Homa, and the
Online Distributed Proofreading Team

\protect\phantomsection\label{6672479776654687619_37106-h-0.htm.xhtml_pg-start-separator}
{*** START OF THE PROJECT GUTENBERG EBOOK LITTLE WOMEN; OR, MEG, JO,
BETH, AND AMY ***}

\end{otherlanguage}

\section{LITTLE
WOMEN.}\label{6672479776654687619_37106-h-0.htm.xhtml_pgepubid00000}

\begin{center}\rule{0.5\linewidth}{0.5pt}\end{center}

\protect\phantomsection\label{6672479776654687619_37106-h-0.htm.xhtml_b001.png}{}
\pandocbounded{\includegraphics[keepaspectratio]{303483661336987339_b001.png}}\\
\protect\phantomsection\label{6672479776654687619_37106-h-0.htm.xhtml_ebm_caption0}{
"They all drew to the fire, mother in the big chair, with Beth at her
feet"}

(See page 9) {Frontispiece}

\begin{center}\rule{0.5\linewidth}{0.5pt}\end{center}

LITTLE WOMEN

OR

Meg, Jo, Beth, and Amy

\hfill\break

BY

LOUISA M. ALCOTT

AUTHOR OF "LITTLE MEN," "AN OLD-FASHIONED GIRL"\\
"SPINNING-WHEEL STORIES," ETC.

\hfill\break

\emph{With more than 200 illustrations by Frank T. Merrill and a picture
of the Home of the Little Women by Edmund H. Garrett}

\hfill\break
\hfill\break
\hfill\break

BOSTON\\
LITTLE, BROWN, AND COMPANY

\begin{center}\rule{0.5\linewidth}{0.5pt}\end{center}

Entered according to Act of Congress, in the years 1868 and 1869, by

LOUISA M. ALCOTT,

In the Clerk\textquotesingle s office of the District Court of the
District of Massachusetts.

\emph{Copyright, 1880},

{By LOUISA M. ALCOTT.}

\emph{Copyright, 1896},

{By JOHN S. P. ALCOTT.}

\hfill\break
\hfill\break

BOSTON

{Alfred Mudge \& Son Inc. Printers}

\begin{center}\rule{0.5\linewidth}{0.5pt}\end{center}

\protect\phantomsection\label{6672479776654687619_37106-h-0.htm.xhtml_b002.png}{}
\pandocbounded{\includegraphics[keepaspectratio]{303483661336987339_b002.png}}

"Go then, my little Book, and show to all

That entertain and bid thee welcome shall,

What thou dost keep close shut up in thy breast;

And wish what thou dost show them may be blest

To them for good, may make them choose to be

Pilgrims better, by far, than thee or me.

Tell them of Mercy; she is one

Who early hath her pilgrimage begun.

Yea, let young damsels learn of her to prize

The world which is to come, and so be wise;

For little tripping maids may follow God

Along the ways which saintly feet have trod."

Adapted from {John Bunyan}.

\begin{center}\rule{0.5\linewidth}{0.5pt}\end{center}

\subsection{Contents}\label{6672479776654687619_37106-h-0.htm.xhtml_pgepubid00001}

\protect\phantomsection\label{6672479776654687619_37106-h-0.htm.xhtml_b003.png}{}
\pandocbounded{\includegraphics[keepaspectratio]{303483661336987339_b003.png}}

\begin{longtable}[]{@{}lll@{}}
\caption{Part First.}\tabularnewline
\toprule\noalign{}
Chapter & & \\
\midrule\noalign{}
\endfirsthead
\toprule\noalign{}
Chapter & & \\
\midrule\noalign{}
\endhead
\bottomrule\noalign{}
\endlastfoot
\protect\phantomsection\label{6672479776654687619_37106-h-0.htm.xhtml_contents}{}I.
& \hyperref[6672479776654687619_37106-h-0.htm.xhtml_I]{Playing Pilgrims}
& \\
II. & \hyperref[6672479776654687619_37106-h-0.htm.xhtml_II]{A Merry
Christmas} & \\
III. & \hyperref[6672479776654687619_37106-h-0.htm.xhtml_III]{The
Laurence Boy} & \\
IV. & \hyperref[6672479776654687619_37106-h-0.htm.xhtml_IV]{Burdens}
& \\
V. & \hyperref[6672479776654687619_37106-h-1.htm.xhtml_V]{Being
Neighborly} & \\
VI. & \hyperref[6672479776654687619_37106-h-1.htm.xhtml_VI]{Beth finds
the Palace Beautiful} & \\
VII. &
\hyperref[6672479776654687619_37106-h-1.htm.xhtml_VII]{Amy\textquotesingle s
Valley of Humiliation} & \\
VIII. & \hyperref[6672479776654687619_37106-h-1.htm.xhtml_VIII]{Jo meets
Apollyon} & \\
IX. & \hyperref[6672479776654687619_37106-h-1.htm.xhtml_IX]{Meg goes to
Vanity Fair} & \\
X. & \hyperref[6672479776654687619_37106-h-1.htm.xhtml_X]{The P. C. and
P. O.} & \\
XI. & \hyperref[6672479776654687619_37106-h-2.htm.xhtml_XI]{Experiments}
& \\
\protect\phantomsection\label{6672479776654687619_37106-h-0.htm.xhtml_contents1b}{}XII.
& \hyperref[6672479776654687619_37106-h-2.htm.xhtml_XII]{Camp Laurence}
& \\
XIII. & \hyperref[6672479776654687619_37106-h-2.htm.xhtml_XIII]{Castles
in the Air} & \\
XIV. & \hyperref[6672479776654687619_37106-h-2.htm.xhtml_XIV]{Secrets}
& \\
XV. & \hyperref[6672479776654687619_37106-h-2.htm.xhtml_XV]{A Telegram}
& \\
XVI. & \hyperref[6672479776654687619_37106-h-3.htm.xhtml_XVI]{Letters}
& \\
XVII. & \hyperref[6672479776654687619_37106-h-3.htm.xhtml_XVII]{Little
Faithful} & \\
XVIII. & \hyperref[6672479776654687619_37106-h-3.htm.xhtml_XVIII]{Dark
Days} & \\
XIX. &
\hyperref[6672479776654687619_37106-h-3.htm.xhtml_XIX]{Amy\textquotesingle s
Will} & \\
XX. &
\hyperref[6672479776654687619_37106-h-3.htm.xhtml_XX]{Confidential} & \\
XXI. & \hyperref[6672479776654687619_37106-h-3.htm.xhtml_XXI]{Laurie
makes Mischief, and Jo makes Peace} & \\
XXII. & \hyperref[6672479776654687619_37106-h-4.htm.xhtml_XXII]{Pleasant
Meadows} & \\
XXIII. & \hyperref[6672479776654687619_37106-h-4.htm.xhtml_XXIII]{Aunt
March settles the Question} & \\
\multicolumn{3}{@{}l@{}}{%
\protect\phantomsection\label{6672479776654687619_37106-h-0.htm.xhtml_contents2}{}Part
Second.} \\
XXIV. & \hyperref[6672479776654687619_37106-h-4.htm.xhtml_XXIV]{Gossip}
& \\
XXV. & \hyperref[6672479776654687619_37106-h-4.htm.xhtml_XXV]{The First
Wedding} & \\
XXVI. & \hyperref[6672479776654687619_37106-h-4.htm.xhtml_XXVI]{Artistic
Attempts} & \\
XXVII. &
\hyperref[6672479776654687619_37106-h-4.htm.xhtml_XXVII]{Literary
Lessons} & \\
XXVIII. &
\hyperref[6672479776654687619_37106-h-4.htm.xhtml_XXVIII]{Domestic
Experiences} & \\
XXIX. & \hyperref[6672479776654687619_37106-h-5.htm.xhtml_XXIX]{Calls}
& \\
XXX. &
\hyperref[6672479776654687619_37106-h-5.htm.xhtml_XXX]{Consequences}
& \\
XXXI. & \hyperref[6672479776654687619_37106-h-5.htm.xhtml_XXXI]{Our
Foreign Correspondent} & \\
XXXII. & \hyperref[6672479776654687619_37106-h-5.htm.xhtml_XXXII]{Tender
Troubles} & \\
XXXIII. &
\hyperref[6672479776654687619_37106-h-5.htm.xhtml_XXXIII]{Jo\textquotesingle s
Journal} & \\
XXXIV. & \hyperref[6672479776654687619_37106-h-6.htm.xhtml_XXXIV]{A
Friend} & \\
\protect\phantomsection\label{6672479776654687619_37106-h-0.htm.xhtml_contents2b}{}XXXV.
& \hyperref[6672479776654687619_37106-h-6.htm.xhtml_XXXV]{Heartache}
& \\
XXXVI. &
\hyperref[6672479776654687619_37106-h-6.htm.xhtml_XXXVI]{Beth\textquotesingle s
Secret} & \\
XXXVII. & \hyperref[6672479776654687619_37106-h-6.htm.xhtml_XXXVII]{New
Impressions} & \\
XXXVIII. & \hyperref[6672479776654687619_37106-h-6.htm.xhtml_XXXVIII]{On
the Shelf} & \\
XXXIX. & \hyperref[6672479776654687619_37106-h-6.htm.xhtml_XXXIX]{Lazy
Laurence} & \\
XL. & \hyperref[6672479776654687619_37106-h-7.htm.xhtml_XL]{The Valley
of the Shadow} & \\
XLI. & \hyperref[6672479776654687619_37106-h-7.htm.xhtml_XLI]{Learning
to Forget} & \\
XLII. & \hyperref[6672479776654687619_37106-h-7.htm.xhtml_XLII]{All
Alone} & \\
XLIII. &
\hyperref[6672479776654687619_37106-h-7.htm.xhtml_XLIII]{Surprises} & \\
XLIV. & \hyperref[6672479776654687619_37106-h-7.htm.xhtml_XLIV]{My Lord
and Lady} & \\
XLV. & \hyperref[6672479776654687619_37106-h-7.htm.xhtml_XLV]{Daisy and
Demi} & \\
XLVI. & \hyperref[6672479776654687619_37106-h-8.htm.xhtml_XLVI]{Under
the Umbrella} & \\
XLVII. &
\hyperref[6672479776654687619_37106-h-8.htm.xhtml_XLVII]{Harvest Time}
& \\
\end{longtable}

\protect\phantomsection\label{6672479776654687619_37106-h-0.htm.xhtml_b004.png}{}
\pandocbounded{\includegraphics[keepaspectratio]{303483661336987339_b004.png}}

\begin{center}\rule{0.5\linewidth}{0.5pt}\end{center}

\subsection{Illustrations}\label{6672479776654687619_37106-h-0.htm.xhtml_pgepubid00002}

\protect\phantomsection\label{6672479776654687619_37106-h-0.htm.xhtml_b005.png}{}
\pandocbounded{\includegraphics[keepaspectratio]{303483661336987339_b005.png}}

\begin{longtable}[]{@{}ll@{}}
\caption{{[}The Illustrations, designed by {Frank T. Merrill}, drawn,
engraved, and printed under the supervision of {George T.
Andrew}.{]}}\tabularnewline
\toprule\noalign{}
\endfirsthead
\endhead
\bottomrule\noalign{}
\endlastfoot
\hyperref[6672479776654687619_37106-h-0.htm.xhtml_b001.png]{They all
drew to the fire, mother in the big chair, with Beth at her feet} & \\
\hyperref[6672479776654687619_37106-h-0.htm.xhtml_b002.png]{Preface}
& \\
\hyperref[6672479776654687619_37106-h-0.htm.xhtml_b003.png]{Contents}
& \\
\hyperref[6672479776654687619_37106-h-0.htm.xhtml_b004.png]{Tail-piece
to Contents} & \\
\hyperref[6672479776654687619_37106-h-0.htm.xhtml_b005.png]{List of
Illustrations} & \\
\hyperref[6672479776654687619_37106-h-0.htm.xhtml_b006.png]{Tail-piece
to Illustrations} & \\
\hyperref[6672479776654687619_37106-h-0.htm.xhtml_b007.png]{Christmas
won\textquotesingle t be Christmas without any presents} & \\
\hyperref[6672479776654687619_37106-h-0.htm.xhtml_b008.png]{Beth put a
pair of slippers down to warm} & \\
\hyperref[6672479776654687619_37106-h-0.htm.xhtml_b009.png]{I used to be
so frightened when it was my turn to sit in the big chair} & \\
\hyperref[6672479776654687619_37106-h-0.htm.xhtml_b010.png]{Do it this
way, clasp your hands so} & \\
\hyperref[6672479776654687619_37106-h-0.htm.xhtml_b011.png]{It was a
cheerful, hopeful letter} & \\
\hyperref[6672479776654687619_37106-h-0.htm.xhtml_b012.png]{How you used
to play Pilgrim\textquotesingle s Progress} & \\
\hyperref[6672479776654687619_37106-h-0.htm.xhtml_b013.png]{No one but
Beth could get much music out of the old piano} & \\
\hyperref[6672479776654687619_37106-h-0.htm.xhtml_b014.png]{At nine they
stopped work and sung as usual} & \\
\hyperref[6672479776654687619_37106-h-0.htm.xhtml_b015.png]{Merry
Christmas} & \\
\hyperref[6672479776654687619_37106-h-0.htm.xhtml_b016.png]{The
procession set out} & \\
\hyperref[6672479776654687619_37106-h-0.htm.xhtml_b017.png]{Out came Meg
with gray horse-hair hanging about her face} & \\
\hyperref[6672479776654687619_37106-h-0.htm.xhtml_b018.png]{A little
figure in cloudy white} & \\
\hyperref[6672479776654687619_37106-h-0.htm.xhtml_b019.png]{The lovers
kneeling to receive Don Pedro\textquotesingle s blessing} & \\
\hyperref[6672479776654687619_37106-h-0.htm.xhtml_b020.png]{We talked
over the fence} & \\
\hyperref[6672479776654687619_37106-h-0.htm.xhtml_b021.png]{Tail-piece}
& \\
\hyperref[6672479776654687619_37106-h-0.htm.xhtml_b022.png]{Eating
apples and crying over the "Heir of Redclyffe"} & \\
\hyperref[6672479776654687619_37106-h-0.htm.xhtml_b023.png]{Jo undertook
to pinch the papered locks} & \\
\hyperref[6672479776654687619_37106-h-0.htm.xhtml_b024.png]{Mrs.
Gardiner greeted them} & \\
\hyperref[6672479776654687619_37106-h-0.htm.xhtml_b025.png]{Face to face
with the Laurence boy} & \\
\hyperref[6672479776654687619_37106-h-0.htm.xhtml_b026.png]{They sat
down on the stairs} & \\
\hyperref[6672479776654687619_37106-h-0.htm.xhtml_b027.png]{Tell about
the party} & \\
\hyperref[6672479776654687619_37106-h-0.htm.xhtml_b028.png]{The kitten
stuck like a burr just out of reach} & \\
\hyperref[6672479776654687619_37106-h-0.htm.xhtml_b029.png]{Curling
herself up in the big chair} & \\
\hyperref[6672479776654687619_37106-h-1.htm.xhtml_b030.png]{Reading that
everlasting Belsham} & \\
\hyperref[6672479776654687619_37106-h-1.htm.xhtml_b031.png]{He took her
by the ear! by the ear!} & \\
\hyperref[6672479776654687619_37106-h-1.htm.xhtml_b032.png]{Mr. Laurence
hooked up a big fish} & \\
\hyperref[6672479776654687619_37106-h-1.htm.xhtml_b033.png]{Tail-piece}
& \\
\hyperref[6672479776654687619_37106-h-1.htm.xhtml_b034.png]{Being
neighborly} & \\
\hyperref[6672479776654687619_37106-h-1.htm.xhtml_b035.png]{Laurie
opened the window} & \\
\hyperref[6672479776654687619_37106-h-1.htm.xhtml_b036.png]{Poll tweaked
off his wig} & \\
\hyperref[6672479776654687619_37106-h-1.htm.xhtml_b037.png]{Putting his
finger under her chin} & \\
\hyperref[6672479776654687619_37106-h-1.htm.xhtml_b038.png]{Please give
these to your mother} & \\
\hyperref[6672479776654687619_37106-h-1.htm.xhtml_b039.png]{Tail-piece}
& \\
\hyperref[6672479776654687619_37106-h-1.htm.xhtml_b040.png]{O sir, they
do care very much} & \\
\hyperref[6672479776654687619_37106-h-1.htm.xhtml_b041.png]{Mr. Laurence
often opened his study door} & \\
\hyperref[6672479776654687619_37106-h-1.htm.xhtml_b042.png]{She put both
arms around his neck and kissed him} & \\
\hyperref[6672479776654687619_37106-h-1.htm.xhtml_b043.png]{The Cyclops}
& \\
\hyperref[6672479776654687619_37106-h-1.htm.xhtml_b044.png]{Amy bore
without flinching several tingling blows} & \\
\hyperref[6672479776654687619_37106-h-1.htm.xhtml_b045.png]{You do know
her} & \\
\hyperref[6672479776654687619_37106-h-1.htm.xhtml_b046.png]{Girls, where
are you going?} & \\
\hyperref[6672479776654687619_37106-h-1.htm.xhtml_b047.png]{I burnt it
up} & \\
\hyperref[6672479776654687619_37106-h-1.htm.xhtml_b048.png]{Held Amy up
by his arms and hockey} & \\
\hyperref[6672479776654687619_37106-h-1.htm.xhtml_b049.png]{Packing the
go abroady trunk} & \\
\hyperref[6672479776654687619_37106-h-1.htm.xhtml_b050.png]{Meg\textquotesingle s
partner appeared} & \\
\hyperref[6672479776654687619_37106-h-1.htm.xhtml_b051.png]{Asked to be
introduced} & \\
\hyperref[6672479776654687619_37106-h-1.htm.xhtml_b052.png]{I
wouldn\textquotesingle t, Meg} & \\
\hyperref[6672479776654687619_37106-h-1.htm.xhtml_b053.png]{Holding a
hand of each, Mrs. March said, \&c.} & \\
\hyperref[6672479776654687619_37106-h-1.htm.xhtml_b054.png]{Mr.
Pickwick} & \\
\hyperref[6672479776654687619_37106-h-2.htm.xhtml_b055.png]{Jo threw
open the door of the closet} & \\
\hyperref[6672479776654687619_37106-h-2.htm.xhtml_b056.png]{Jo spent the
morning on the river} & \\
\hyperref[6672479776654687619_37106-h-2.htm.xhtml_b057.png]{Amy sat down
to draw} & \\
\hyperref[6672479776654687619_37106-h-2.htm.xhtml_b058.png]{O Pip! O
Pip!} & \\
\hyperref[6672479776654687619_37106-h-2.htm.xhtml_b059.png]{Miss Crocker
made a wry face} & \\
\hyperref[6672479776654687619_37106-h-2.htm.xhtml_b060.png]{We\textquotesingle ll
work like bees} & \\
\hyperref[6672479776654687619_37106-h-2.htm.xhtml_b061.png]{Beth was
post-mistress} & \\
\hyperref[6672479776654687619_37106-h-2.htm.xhtml_b062.png]{Amy capped
the climax by putting a clothes-pin on her nose} & \\
\hyperref[6672479776654687619_37106-h-2.htm.xhtml_b063.png]{Mr. Laurence
waving his hat} & \\
\hyperref[6672479776654687619_37106-h-2.htm.xhtml_b064.png]{Now, Miss
Jo, I\textquotesingle ll settle you} & \\
\hyperref[6672479776654687619_37106-h-2.htm.xhtml_b065.png]{A very merry
lunch it was} & \\
\hyperref[6672479776654687619_37106-h-2.htm.xhtml_b066.png]{He went
prancing down a quiet street} & \\
\hyperref[6672479776654687619_37106-h-2.htm.xhtml_b067.png]{"Oh, rise,"
she said} & \\
\hyperref[6672479776654687619_37106-h-2.htm.xhtml_b067a.png]{A stunning
blow from the big Greek lexicon} & \\
\hyperref[6672479776654687619_37106-h-2.htm.xhtml_b068.png]{He sneezed}
& \\
\hyperref[6672479776654687619_37106-h-2.htm.xhtml_b069.png]{The
Portuguese walked the plank} & \\
\hyperref[6672479776654687619_37106-h-2.htm.xhtml_b070.png]{Will you
give me a rose?} & \\
\hyperref[6672479776654687619_37106-h-2.htm.xhtml_b071.png]{Miss Kate
put up her glass} & \\
\hyperref[6672479776654687619_37106-h-2.htm.xhtml_b072.png]{Ellen Tree}
& \\
\hyperref[6672479776654687619_37106-h-2.htm.xhtml_b073.png]{Tail-piece}
& \\
\hyperref[6672479776654687619_37106-h-2.htm.xhtml_b074.png]{Swinging to
and fro in his hammock} & \\
\hyperref[6672479776654687619_37106-h-2.htm.xhtml_b075.png]{It was
rather a pretty little picture} & \\
\hyperref[6672479776654687619_37106-h-2.htm.xhtml_b076.png]{Waved a
brake before her face} & \\
\hyperref[6672479776654687619_37106-h-2.htm.xhtml_b077.png]{I see him
bow and smile} & \\
\hyperref[6672479776654687619_37106-h-2.htm.xhtml_b078.png]{Tail-piece}
& \\
\hyperref[6672479776654687619_37106-h-2.htm.xhtml_b079.png]{Jo was very
busy} & \\
\hyperref[6672479776654687619_37106-h-2.htm.xhtml_b080.png]{Hurrah for
Miss March} & \\
\hyperref[6672479776654687619_37106-h-2.htm.xhtml_b081.png]{Jo darted
away} & \\
\hyperref[6672479776654687619_37106-h-2.htm.xhtml_b082.png]{Jo laid
herself on the sofa and affected to read} & \\
\hyperref[6672479776654687619_37106-h-3.htm.xhtml_b083.png]{November is
the most disagreeable month in the year} & \\
\hyperref[6672479776654687619_37106-h-3.htm.xhtml_b084.png]{One of them
horrid telegraph things} & \\
\hyperref[6672479776654687619_37106-h-3.htm.xhtml_b085.png]{She came
suddenly upon Mr. Brooke} & \\
\hyperref[6672479776654687619_37106-h-3.htm.xhtml_b086.png]{The man
clipped} & \\
\hyperref[6672479776654687619_37106-h-3.htm.xhtml_b087.png]{Tail-piece}
& \\
\hyperref[6672479776654687619_37106-h-3.htm.xhtml_b088.png]{Letters}
& \\
\hyperref[6672479776654687619_37106-h-3.htm.xhtml_b089.png]{She rolled
away} & \\
\hyperref[6672479776654687619_37106-h-3.htm.xhtml_b090.png]{I wind the
clock} & \\
\hyperref[6672479776654687619_37106-h-3.htm.xhtml_b091.png]{Yours
Respectful, Hannah Mullet} & \\
\hyperref[6672479776654687619_37106-h-3.htm.xhtml_b092.png]{Tail-piece}
& \\
\hyperref[6672479776654687619_37106-h-3.htm.xhtml_b093.png]{It
didn\textquotesingle t stir, and I knew it was dead} & \\
\hyperref[6672479776654687619_37106-h-3.htm.xhtml_b094.png]{He sat down
beside her} & \\
\hyperref[6672479776654687619_37106-h-3.htm.xhtml_b095.png]{What do you
want now?} & \\
\hyperref[6672479776654687619_37106-h-3.htm.xhtml_b096.png]{Beth did
have the fever} & \\
\hyperref[6672479776654687619_37106-h-3.htm.xhtml_b097.png]{Gently
stroking her head as her mother used to do} & \\
\hyperref[6672479776654687619_37106-h-3.htm.xhtml_b098.png]{Amy\textquotesingle s
Will} & \\
\hyperref[6672479776654687619_37106-h-3.htm.xhtml_b099.png]{Polish up
the spoons and the fat silver teapot} & \\
\hyperref[6672479776654687619_37106-h-3.htm.xhtml_b100.png]{On his back,
with all his legs in the air} & \\
\hyperref[6672479776654687619_37106-h-3.htm.xhtml_b101.png]{I should
choose this} & \\
\hyperref[6672479776654687619_37106-h-3.htm.xhtml_b102.png]{Gravely
promenaded to and fro} & \\
\hyperref[6672479776654687619_37106-h-3.htm.xhtml_b103.png]{Amy\textquotesingle s
Will} & \\
\hyperref[6672479776654687619_37106-h-3.htm.xhtml_b104.png]{Tail-piece}
& \\
\hyperref[6672479776654687619_37106-h-3.htm.xhtml_b105.png]{Mrs. March
would not leave Beth\textquotesingle s side} & \\
\hyperref[6672479776654687619_37106-h-3.htm.xhtml_b106.png]{Tail-piece}
& \\
\hyperref[6672479776654687619_37106-h-3.htm.xhtml_b107.png]{Letters}
& \\
\hyperref[6672479776654687619_37106-h-3.htm.xhtml_b108.png]{Jo and her
mother were reading the note} & \\
\hyperref[6672479776654687619_37106-h-3.htm.xhtml_b109.png]{Get up and
don\textquotesingle t be a goose} & \\
\hyperref[6672479776654687619_37106-h-3.htm.xhtml_b110.png]{"Hold your
tongue!" cried Jo, covering her ears} & \\
\hyperref[6672479776654687619_37106-h-3.htm.xhtml_b111.png]{He stood at
the foot, like a lion in the path} & \\
\hyperref[6672479776654687619_37106-h-4.htm.xhtml_b112.png]{Beth was
soon able to lie on the study sofa all day} & \\
\hyperref[6672479776654687619_37106-h-4.htm.xhtml_b113.png]{The
Jungfrau} & \\
\hyperref[6672479776654687619_37106-h-4.htm.xhtml_b115.png]{Popping in
her head now and then} & \\
\hyperref[6672479776654687619_37106-h-4.htm.xhtml_b114.png]{He sat in
the big chair by Beth\textquotesingle s sofa} & \\
\hyperref[6672479776654687619_37106-h-4.htm.xhtml_b116.png]{Shall I tell
you how?} & \\
\hyperref[6672479776654687619_37106-h-4.htm.xhtml_b117.png]{Bless me,
what\textquotesingle s all this?} & \\
\hyperref[6672479776654687619_37106-h-4.htm.xhtml_b118.png]{For Mrs.
John Brooke} & \\
\hyperref[6672479776654687619_37106-h-4.htm.xhtml_b118a.png]{Home of the
Little Women} & \\
\hyperref[6672479776654687619_37106-h-4.htm.xhtml_b119.png]{The Dove
Cote} & \\
\hyperref[6672479776654687619_37106-h-4.htm.xhtml_b120.png]{A small
watchman\textquotesingle s rattle} & \\
\hyperref[6672479776654687619_37106-h-4.htm.xhtml_b121.png]{Tail-piece}
& \\
\hyperref[6672479776654687619_37106-h-4.htm.xhtml_b122.png]{The First
Wedding} & \\
\hyperref[6672479776654687619_37106-h-4.htm.xhtml_b123.png]{Artistic
Attempts} & \\
\hyperref[6672479776654687619_37106-h-4.htm.xhtml_b124.png]{Her foot
held fast in a panful of plaster} & \\
\hyperref[6672479776654687619_37106-h-4.htm.xhtml_b125.png]{Please
don\textquotesingle t, it\textquotesingle s mine} & \\
\hyperref[6672479776654687619_37106-h-4.htm.xhtml_b126.png]{Tail-piece}
& \\
\hyperref[6672479776654687619_37106-h-4.htm.xhtml_b127.png]{Literary
Lessons} & \\
\hyperref[6672479776654687619_37106-h-4.htm.xhtml_b128.png]{A check for
one hundred dollars} & \\
\hyperref[6672479776654687619_37106-h-4.htm.xhtml_b129.png]{Tail-piece}
& \\
\hyperref[6672479776654687619_37106-h-4.htm.xhtml_b130.png]{Domestic
Experiences} & \\
\hyperref[6672479776654687619_37106-h-5.htm.xhtml_b131.png]{Both felt
desperately uncomfortable} & \\
\hyperref[6672479776654687619_37106-h-5.htm.xhtml_b132.png]{A bargain, I
assure you, ma\textquotesingle am} & \\
\hyperref[6672479776654687619_37106-h-5.htm.xhtml_b133.png]{Laurie
heroically shut his eyes while something was put into his arms} & \\
\hyperref[6672479776654687619_37106-h-5.htm.xhtml_b134.png]{Calls} & \\
\hyperref[6672479776654687619_37106-h-5.htm.xhtml_b135.png]{She took the
saddle to the horse} & \\
\hyperref[6672479776654687619_37106-h-5.htm.xhtml_b136.png]{It might
have been worse} & \\
\hyperref[6672479776654687619_37106-h-5.htm.xhtml_b137.png]{The call at
Aunt March\textquotesingle s} & \\
\hyperref[6672479776654687619_37106-h-5.htm.xhtml_b138.png]{Tail-piece}
& \\
\hyperref[6672479776654687619_37106-h-5.htm.xhtml_b139.png]{You shall
have another table} & \\
\hyperref[6672479776654687619_37106-h-5.htm.xhtml_b140.png]{Bought up
the bouquets} & \\
\hyperref[6672479776654687619_37106-h-5.htm.xhtml_b141.png]{Tail-piece}
& \\
\hyperref[6672479776654687619_37106-h-5.htm.xhtml_b142.png]{Flo and I
ordered a hansom-cab} & \\
\hyperref[6672479776654687619_37106-h-5.htm.xhtml_b143.png]{Every one
was very kind, especially the officers} & \\
\hyperref[6672479776654687619_37106-h-5.htm.xhtml_b144.png]{I\textquotesingle ve
seen the imperial family several times} & \\
\hyperref[6672479776654687619_37106-h-5.htm.xhtml_b145.png]{Trying to
sketch the gray-stone lion\textquotesingle s head on the wall} & \\
\hyperref[6672479776654687619_37106-h-5.htm.xhtml_b146.png]{She leaned
her head upon her hands} & \\
\hyperref[6672479776654687619_37106-h-5.htm.xhtml_b147.png]{Now, this is
filling at the price} & \\
\hyperref[6672479776654687619_37106-h-5.htm.xhtml_b148.png]{Up with the
Bonnets of Bonnie Dundee} & \\
\hyperref[6672479776654687619_37106-h-5.htm.xhtml_b149.png]{I amused
myself by dropping gingerbread nuts over the seat} & \\
\hyperref[6672479776654687619_37106-h-5.htm.xhtml_b150.png]{Thou shalt
haf thy Bhaer} & \\
\hyperref[6672479776654687619_37106-h-5.htm.xhtml_b151.png]{He waved his
hand, sock and all} & \\
\hyperref[6672479776654687619_37106-h-5.htm.xhtml_b152.png]{Dis is mine
effalunt} & \\
\hyperref[6672479776654687619_37106-h-6.htm.xhtml_b153.png]{I sat down
upon the floor and read and looked and ate} & \\
\hyperref[6672479776654687619_37106-h-6.htm.xhtml_b154.png]{Tail-piece}
& \\
\hyperref[6672479776654687619_37106-h-6.htm.xhtml_b155.png]{In the
presence of three gentlemen} & \\
\hyperref[6672479776654687619_37106-h-6.htm.xhtml_b156.png]{A select
symposium} & \\
\hyperref[6672479776654687619_37106-h-6.htm.xhtml_b157.png]{He
doesn\textquotesingle t prink at his glass before coming} & \\
\hyperref[6672479776654687619_37106-h-6.htm.xhtml_b158.png]{Jo stuffed
the whole bundle into the stove} & \\
\hyperref[6672479776654687619_37106-h-6.htm.xhtml_b159.png]{He put the
sisters into the carriage} & \\
\hyperref[6672479776654687619_37106-h-6.htm.xhtml_b160.png]{He laid his
head down on the mossy post} & \\
\hyperref[6672479776654687619_37106-h-6.htm.xhtml_b161.png]{O Jo,
can\textquotesingle t you?} & \\
\hyperref[6672479776654687619_37106-h-6.htm.xhtml_b162.png]{Tail-piece}
& \\
\hyperref[6672479776654687619_37106-h-6.htm.xhtml_b163.png]{With her
head in Jo\textquotesingle s lap, while the wind blew healthfully over
her} & \\
\hyperref[6672479776654687619_37106-h-6.htm.xhtml_b164.png]{Tail-piece}
& \\
\hyperref[6672479776654687619_37106-h-6.htm.xhtml_b165.png]{He hurried
forward to meet her} & \\
\hyperref[6672479776654687619_37106-h-6.htm.xhtml_b166.png]{Here are
your flowers} & \\
\hyperref[6672479776654687619_37106-h-6.htm.xhtml_b167.png]{Demi and
Daisy} & \\
\hyperref[6672479776654687619_37106-h-6.htm.xhtml_b168.png]{Mornin\textquotesingle{}
now} & \\
\hyperref[6672479776654687619_37106-h-6.htm.xhtml_b169.png]{My dear man,
it\textquotesingle s a bonnet} & \\
\hyperref[6672479776654687619_37106-h-6.htm.xhtml_b170.png]{Tail-piece}
& \\
\hyperref[6672479776654687619_37106-h-6.htm.xhtml_b171.png]{Sat piping
on a stone while his goats skipped} & \\
\hyperref[6672479776654687619_37106-h-7.htm.xhtml_b172.png]{Laurie threw
himself down on the turf} & \\
\hyperref[6672479776654687619_37106-h-7.htm.xhtml_b173.png]{A rough
sketch of Laurie taming a horse} & \\
\hyperref[6672479776654687619_37106-h-7.htm.xhtml_b174.png]{The Valley
of the Shadow} & \\
\hyperref[6672479776654687619_37106-h-7.htm.xhtml_b175.png]{Tail-piece}
& \\
\hyperref[6672479776654687619_37106-h-7.htm.xhtml_b176.png]{Sat staring
up at the busts} & \\
\hyperref[6672479776654687619_37106-h-7.htm.xhtml_b177.png]{Turning the
ring thoughtfully upon his finger} & \\
\hyperref[6672479776654687619_37106-h-7.htm.xhtml_b178.png]{O Laurie,
Laurie, I knew you\textquotesingle d come} & \\
\hyperref[6672479776654687619_37106-h-7.htm.xhtml_b179.png]{How well we
pull together} & \\
\hyperref[6672479776654687619_37106-h-7.htm.xhtml_b180.png]{Jo and her
father} & \\
\hyperref[6672479776654687619_37106-h-7.htm.xhtml_b181.png]{Jo laid her
head on a comfortable rag-bag and cried} & \\
\hyperref[6672479776654687619_37106-h-7.htm.xhtml_b182.png]{A
substantial lifelike ghost leaning over her} & \\
\hyperref[6672479776654687619_37106-h-7.htm.xhtml_b183.png]{The tall
uncle proceeded to toss and tousle the small nephew} & \\
\hyperref[6672479776654687619_37106-h-7.htm.xhtml_b184.png]{O Mr. Bhaer,
I am so glad to see you} & \\
\hyperref[6672479776654687619_37106-h-7.htm.xhtml_b185.png]{Mr. Bhaer
sang heartily} & \\
\hyperref[6672479776654687619_37106-h-7.htm.xhtml_b186.png]{Mrs.
Laurence sitting in her mother\textquotesingle s lap} & \\
\hyperref[6672479776654687619_37106-h-7.htm.xhtml_b187.png]{They began
to pace up and down} & \\
\hyperref[6672479776654687619_37106-h-7.htm.xhtml_b188.png]{Tail-piece}
& \\
\hyperref[6672479776654687619_37106-h-7.htm.xhtml_b189.png]{Me loves
evvybody} & \\
\hyperref[6672479776654687619_37106-h-7.htm.xhtml_b190.png]{What makes
my legs go, dranpa?} & \\
\hyperref[6672479776654687619_37106-h-8.htm.xhtml_b191.png]{Dranpa,
it\textquotesingle s a We} & \\
\hyperref[6672479776654687619_37106-h-8.htm.xhtml_b192.png]{Tail-piece}
& \\
\hyperref[6672479776654687619_37106-h-8.htm.xhtml_b193.png]{Mr. Bhaer
and Jo were enjoying promenades} & \\
\hyperref[6672479776654687619_37106-h-8.htm.xhtml_b194.png]{Looking up
she saw Mr. Bhaer} & \\
\hyperref[6672479776654687619_37106-h-8.htm.xhtml_b195.png]{Does this
suit you, Mr. Bhaer?} & \\
\hyperref[6672479776654687619_37106-h-8.htm.xhtml_b196.png]{Under the
umbrella} & \\
\hyperref[6672479776654687619_37106-h-8.htm.xhtml_b197.png]{Tail-piece}
& \\
\hyperref[6672479776654687619_37106-h-8.htm.xhtml_b198.png]{Harvest
time} & \\
\hyperref[6672479776654687619_37106-h-8.htm.xhtml_b199.png]{Teddy bore a
charmed life} & \\
\hyperref[6672479776654687619_37106-h-8.htm.xhtml_b200.png]{Leaving Mrs.
March and her daughters under the festival tree} & \\
\hyperref[6672479776654687619_37106-h-8.htm.xhtml_b201.png]{Tail-piece}
& \\
\end{longtable}

\protect\phantomsection\label{6672479776654687619_37106-h-0.htm.xhtml_b006.png}{}
\pandocbounded{\includegraphics[keepaspectratio]{303483661336987339_b006.png}}

\begin{center}\rule{0.5\linewidth}{0.5pt}\end{center}

\subsection{I. Playing
Pilgrims.}\label{6672479776654687619_37106-h-0.htm.xhtml_pgepubid00003}

\protect\phantomsection\label{6672479776654687619_37106-h-0.htm.xhtml_b007.png}{}
\pandocbounded{\includegraphics[keepaspectratio]{303483661336987339_b007.png}}

\protect\phantomsection\label{6672479776654687619_37106-h-0.htm.xhtml_I}{}\hyperref[6672479776654687619_37106-h-0.htm.xhtml_contents]{I.}

PLAYING PILGRIMS.

"{Christmas} won\textquotesingle t be Christmas without any presents,"
grumbled Jo, lying on the rug.

"It\textquotesingle s so dreadful to be poor!" sighed Meg, looking down
at her old dress.

"I don\textquotesingle t think it\textquotesingle s fair for some girls
to have plenty of pretty things, and other girls nothing at all," added
little Amy, with an injured sniff.

"We\textquotesingle ve got father and mother and each other," said Beth
contentedly, from her corner.

The four young faces on which the firelight shone brightened at the
cheerful words, but darkened again as Jo said sadly,---

"We haven\textquotesingle t got father, and shall not have him for a
long time." She didn\textquotesingle t say "perhaps never," but each
silently added it, thinking of father far away, where the fighting was.

Nobody spoke for a minute; then Meg said in an altered tone,---

"You know the reason mother proposed not having any presents this
Christmas was because it is going to be a hard winter for every one; and
she thinks we ought not to spend money for pleasure, when our men are
suffering so in the army. We can\textquotesingle t do much, but we can
make our little sacrifices, and ought to do it gladly. But I am afraid I
don\textquotesingle t;" and Meg shook her head, as she thought
regretfully of all the pretty things she wanted.

"But I don\textquotesingle t think the little we should spend would do
any good. We\textquotesingle ve each got a dollar, and the army
wouldn\textquotesingle t be much helped by our giving that. I agree not
to expect anything from mother or you, but I do want to buy Undine and
Sintram for myself; I\textquotesingle ve wanted it \emph{so} long," said
Jo, who was a bookworm.

"I planned to spend mine in new music," said Beth, with a little sigh,
which no one heard but the hearth-brush and kettle-holder.

"I shall get a nice box of Faber\textquotesingle s drawing-pencils; I
really need them," said Amy decidedly.

"Mother didn\textquotesingle t say anything about our money, and she
won\textquotesingle t wish us to give up everything.
Let\textquotesingle s each buy what we want, and have a little fun;
I\textquotesingle m sure we work hard enough to earn it," cried Jo,
examining the heels of her shoes in a gentlemanly manner.

"I know \emph{I} do,---teaching those tiresome children nearly all day,
when I\textquotesingle m longing to enjoy myself at home," began Meg, in
the complaining tone again.

"You don\textquotesingle t have half such a hard time as I do," said Jo.
"How would you like to be shut up for hours with a nervous, fussy old
lady, who keeps you trotting, is never satisfied, and worries you till
you\textquotesingle re ready to fly out of the window or cry?"

"It\textquotesingle s naughty to fret; but I do think washing dishes and
keeping things tidy is the worst work in the world. It makes me cross;
and my hands get so stiff, I can\textquotesingle t practise well at
all;" and Beth looked at her rough hands with a sigh that any one could
hear that time.

"I don\textquotesingle t believe any of you suffer as I do," cried Amy;
"for you don\textquotesingle t have to go to school with impertinent
girls, who plague you if you don\textquotesingle t know your lessons,
and laugh at your dresses, and label your father if he
isn\textquotesingle t rich, and insult you when your nose
isn\textquotesingle t nice."

"If you mean \emph{libel}, I\textquotesingle d say so, and not talk
about \emph{labels}, as if papa was a pickle-bottle," advised Jo,
laughing.

"I know what I mean, and you needn\textquotesingle t be
\emph{statirical} about it. It\textquotesingle s proper to use good
words, and improve your \emph{vocabilary}," returned Amy, with dignity.

"Don\textquotesingle t peck at one another, children.
Don\textquotesingle t you wish we had the money papa lost when we were
little, Jo? Dear me! how happy and good we\textquotesingle d be, if we
had no worries!" said Meg, who could remember better times.

"You said the other day, you thought we were a deal happier than the
King children, for they were fighting and fretting all the time, in
spite of their money."

"So I did, Beth. Well, I think we are; for, though we do have to work,
we make fun for ourselves, and are a pretty jolly set, as Jo would say."

"Jo does use such slang words!" observed Amy, with a reproving look at
the long figure stretched on the rug. Jo immediately sat up, put her
hands in her pockets, and began to whistle.

"Don\textquotesingle t, Jo; it\textquotesingle s so boyish!"

"That\textquotesingle s why I do it."

"I detest rude, unlady-like girls!"

"I hate affected, niminy-piminy chits!"

"\textquotesingle Birds in their little nests agree,\textquotesingle"
sang Beth, the peace-maker, with such a funny face that both sharp
voices softened to a laugh, and the "pecking" ended for that time.

"Really, girls, you are both to be blamed," said Meg, beginning to
lecture in her elder-sisterly fashion. "You are old enough to leave off
boyish tricks, and to behave better, Josephine. It
didn\textquotesingle t matter so much when you were a little girl; but
now you are so tall, and turn up your hair, you should remember that you
are a young lady."

"I\textquotesingle m not! and if turning up my hair makes me one,
I\textquotesingle ll wear it in two tails till I\textquotesingle m
twenty," cried Jo, pulling off her net, and shaking down a chestnut
mane. "I hate to think I\textquotesingle ve got to grow up, and be Miss
March, and wear long gowns, and look as prim as a China-aster!
It\textquotesingle s bad enough to be a girl, anyway, when I like
boys\textquotesingle{} games and work and manners! I
can\textquotesingle t get over my disappointment in not being a boy; and
it\textquotesingle s worse than ever now, for I\textquotesingle m dying
to go and fight with papa, and I can only stay at home and knit, like a
poky old woman!" And Jo shook the blue army-sock till the needles
rattled like castanets, and her ball bounded across the room.

"Poor Jo! It\textquotesingle s too bad, but it can\textquotesingle t be
helped; so you must try to be contented with making your name boyish,
and playing brother to us girls," said Beth, stroking the rough head at
her knee with a hand that all the dish-washing and dusting in the world
could not make ungentle in its touch.

"As for you, Amy," continued Meg, "you are altogether too particular and
prim. Your airs are funny now; but you\textquotesingle ll grow up an
affected little goose, if you don\textquotesingle t take care. I like
your nice manners and refined ways of speaking, when you
don\textquotesingle t try to be elegant; but your absurd words are as
bad as Jo\textquotesingle s slang."

"If Jo is a tom-boy and Amy a goose, what am I, please?" asked Beth,
ready to share the lecture.

"You\textquotesingle re a dear, and nothing else," answered Meg warmly;
and no one contradicted her, for the "Mouse" was the pet of the family.

As young readers like to know "how people look," we will take this
moment to give them a little sketch of the four sisters, who sat
knitting away in the twilight, while the December snow fell quietly
without, and the fire crackled cheerfully within. It was a comfortable
old room, though the carpet was faded and the furniture very plain; for
a good picture or two hung on the walls, books filled the recesses,
chrysanthemums and Christmas roses bloomed in the windows, and a
pleasant atmosphere of home-peace pervaded it.

Margaret, the eldest of the four, was sixteen, and very pretty, being
plump and fair, with large eyes, plenty of soft, brown hair, a sweet
mouth, and white hands, of which she was rather vain. Fifteen-year-old
Jo was very tall, thin, and brown, and reminded one of a colt; for she
never seemed to know what to do with her long limbs, which were very
much in her way. She had a decided mouth, a comical nose, and sharp,
gray eyes, which appeared to see everything, and were by turns fierce,
funny, or thoughtful. Her long, thick hair was her one beauty; but it
was usually bundled into a net, to be out of her way. Round shoulders
had Jo, big hands and feet, a fly-away look to her clothes, and the
uncomfortable appearance of a girl who was rapidly shooting up into a
woman, and didn\textquotesingle t like it. Elizabeth---or Beth, as every
one called her---was a rosy, smooth-haired, bright-eyed girl of
thirteen, with a shy manner, a timid voice, and a peaceful expression,
which was seldom disturbed. Her father called her "Little Tranquillity,"
and the name suited her excellently; for she seemed to live in a happy
world of her own, only venturing out to meet the few whom she trusted
and loved. Amy, though the youngest, was a most important person,---in
her own opinion at least. A regular snow-maiden, with blue eyes, and
yellow hair, curling on her shoulders, pale and slender, and always
carrying herself like a young lady mindful of her manners. What the
characters of the four sisters were we will leave to be found out.

The clock struck six; and, having swept up the hearth, Beth put a pair
of slippers down to warm. Somehow the sight of the old shoes had a good
effect upon the girls; for mother was coming, and every one brightened
to welcome her. Meg stopped lecturing, and lighted the lamp, Amy got out
of the easy-chair without being asked, and Jo forgot how tired she was
as she sat up to hold the slippers nearer to the blaze.

\protect\phantomsection\label{6672479776654687619_37106-h-0.htm.xhtml_b008.png}{}
\pandocbounded{\includegraphics[keepaspectratio]{303483661336987339_b008.png}}

"They are quite worn out; Marmee must have a new pair."

"I thought I\textquotesingle d get her some with my dollar," said Beth.

"No, I shall!" cried Amy.

"I\textquotesingle m the oldest," began Meg, but Jo cut in with a
decided---

"I\textquotesingle m the man of the family now papa is away, and
\emph{I} shall provide the slippers, for he told me to take special care
of mother while he was gone."

"I\textquotesingle ll tell you what we\textquotesingle ll do," said
Beth; "let\textquotesingle s each get her something for Christmas, and
not get anything for ourselves."

"That\textquotesingle s like you, dear! What will we get?" exclaimed Jo.

Every one thought soberly for a minute; then Meg announced, as if the
idea was suggested by the sight of her own pretty hands, "I shall give
her a nice pair of gloves."

"Army shoes, best to be had," cried Jo.

"Some handkerchiefs, all hemmed," said Beth.

"I\textquotesingle ll get a little bottle of cologne; she likes it, and
it won\textquotesingle t cost much, so I\textquotesingle ll have some
left to buy my pencils," added Amy.

"How will we give the things?" asked Meg.

"Put them on the table, and bring her in and see her open the bundles.
Don\textquotesingle t you remember how we used to do on our birthdays?"
answered Jo.

\protect\phantomsection\label{6672479776654687619_37106-h-0.htm.xhtml_b009.png}{}
\pandocbounded{\includegraphics[keepaspectratio]{303483661336987339_b009.png}}

"I used to be \emph{so} frightened when it was my turn to sit in the big
chair with the crown on, and see you all come marching round to give the
presents, with a kiss. I liked the things and the kisses, but it was
dreadful to have you sit looking at me while I opened the bundles," said
Beth, who was toasting her face and the bread for tea, at the same time.

"Let Marmee think we are getting things for ourselves, and then surprise
her. We must go shopping to-morrow afternoon, Meg; there is so much to
do about the play for Christmas night," said Jo, marching up and down,
with her hands behind her back and her nose in the air.

"I don\textquotesingle t mean to act any more after this time;
I\textquotesingle m getting too old for such things," observed Meg, who
was as much a child as ever about "dressing-up" frolics.

"You won\textquotesingle t stop, I know, as long as you can trail round
in a white gown with your hair down, and wear gold-paper jewelry. You
are the best actress we\textquotesingle ve got, and
there\textquotesingle ll be an end of everything if you quit the
boards," said Jo. "We ought to rehearse to-night. Come here, Amy, and do
the fainting scene, for you are as stiff as a poker in that."

"I can\textquotesingle t help it; I never saw any one faint, and I
don\textquotesingle t choose to make myself all black and blue, tumbling
flat as you do. If I can go down easily, I\textquotesingle ll drop; if I
can\textquotesingle t, I shall fall into a chair and be graceful; I
don\textquotesingle t care if Hugo does come at me with a pistol,"
returned Amy, who was not gifted with dramatic power, but was chosen
because she was small enough to be borne out shrieking by the villain of
the piece.

\protect\phantomsection\label{6672479776654687619_37106-h-0.htm.xhtml_b010.png}{}
\pandocbounded{\includegraphics[keepaspectratio]{303483661336987339_b010.png}}

"Do it this way; clasp your hands so, and stagger across the room,
crying frantically, \textquotesingle Roderigo! save me! save
me!\textquotesingle" and away went Jo, with a melodramatic scream which
was truly thrilling.

Amy followed, but she poked her hands out stiffly before her, and jerked
herself along as if she went by machinery; and her "Ow!" was more
suggestive of pins being run into her than of fear and anguish. Jo gave
a despairing groan, and Meg laughed outright, while Beth let her bread
burn as she watched the fun, with interest.

"It\textquotesingle s no use! Do the best you can when the time comes,
and if the audience laugh, don\textquotesingle t blame me. Come on,
Meg."

Then things went smoothly, for Don Pedro defied the world in a speech of
two pages without a single break; Hagar, the witch, chanted an awful
incantation over her kettleful of simmering toads, with weird effect;
Roderigo rent his chains asunder manfully, and Hugo died in agonies of
remorse and arsenic, with a wild "Ha! ha!"

"It\textquotesingle s the best we\textquotesingle ve had yet," said Meg,
as the dead villain sat up and rubbed his elbows.

"I don\textquotesingle t see how you can write and act such splendid
things, Jo. You\textquotesingle re a regular Shakespeare!" exclaimed
Beth, who firmly believed that her sisters were gifted with wonderful
genius in all things.

"Not quite," replied Jo modestly. "I do think \textquotesingle The
Witch\textquotesingle s Curse, an Operatic Tragedy,\textquotesingle{} is
rather a nice thing; but I\textquotesingle d like to try Macbeth, if we
only had a trap-door for Banquo. I always wanted to do the killing part.
\textquotesingle Is that a dagger that I see before me?\textquotesingle"
muttered Jo, rolling her eyes and clutching at the air, as she had seen
a famous tragedian do.

"No, it\textquotesingle s the toasting fork, with
mother\textquotesingle s shoe on it instead of the bread.
Beth\textquotesingle s stage-struck!" cried Meg, and the rehearsal ended
in a general burst of laughter.

"Glad to find you so merry, my girls," said a cheery voice at the door,
and actors and audience turned to welcome a tall, motherly lady, with a
"can-I-help-you" look about her which was truly delightful. She was not
elegantly dressed, but a noble-looking woman, and the girls thought the
gray cloak and unfashionable bonnet covered the most splendid mother in
the world.

"Well, dearies, how have you got on to-day? There was so much to do,
getting the boxes ready to go to-morrow, that I didn\textquotesingle t
come home to dinner. Has any one called, Beth? How is your cold, Meg?
Jo, you look tired to death. Come and kiss me, baby."

While making these maternal inquiries Mrs. March got her wet things off,
her warm slippers on, and sitting down in the easy-chair, drew Amy to
her lap, preparing to enjoy the happiest hour of her busy day. The girls
flew about, trying to make things comfortable, each in her own way. Meg
arranged the tea-table; Jo brought wood and set chairs, dropping,
overturning, and clattering everything she touched; Beth trotted to and
fro between parlor and kitchen, quiet and busy; while Amy gave
directions to every one, as she sat with her hands folded.

As they gathered about the table, Mrs. March said, with a particularly
happy face, "I\textquotesingle ve got a treat for you after supper."

A quick, bright smile went round like a streak of sunshine. Beth clapped
her hands, regardless of the biscuit she held, and Jo tossed up her
napkin, crying, "A letter! a letter! Three cheers for father!"

"Yes, a nice long letter. He is well, and thinks he shall get through
the cold season better than we feared. He sends all sorts of loving
wishes for Christmas, and an especial message to you girls," said Mrs.
March, patting her pocket as if she had got a treasure there.

"Hurry and get done! Don\textquotesingle t stop to quirk your little
finger, and simper over your plate, Amy," cried Jo, choking in her tea,
and dropping her bread, butter side down, on the carpet, in her haste to
get at the treat.

Beth ate no more, but crept away, to sit in her shadowy corner and brood
over the delight to come, till the others were ready.

"I think it was so splendid in father to go as a chaplain when he was
too old to be drafted, and not strong enough for a soldier," said Meg
warmly.

"Don\textquotesingle t I wish I could go as a drummer, a
\emph{vivan}---what\textquotesingle s its name? or a nurse, so I could
be near him and help him," exclaimed Jo, with a groan.

"It must be very disagreeable to sleep in a tent, and eat all sorts of
bad-tasting things, and drink out of a tin mug," sighed Amy.

"When will he come home, Marmee?" asked Beth, with a little quiver in
her voice.

"Not for many months, dear, unless he is sick. He will stay and do his
work faithfully as long as he can, and we won\textquotesingle t ask for
him back a minute sooner than he can be spared. Now come and hear the
letter."

They all drew to the fire, mother in the big chair with Beth at her
feet, Meg and Amy perched on either arm of the chair, and Jo leaning on
the back, where no one would see any sign of emotion if the letter
should happen to be touching.

Very few letters were written in those hard times that were not
touching, especially those which fathers sent home. In this one little
was said of the hardships endured, the dangers faced, or the
homesickness conquered; it was a cheerful, hopeful letter, full of
lively descriptions of camp life, marches, and military news; and only
at the end did the writer\textquotesingle s heart overflow with fatherly
love and longing for the little girls at home.

\protect\phantomsection\label{6672479776654687619_37106-h-0.htm.xhtml_b011.png}{}
\pandocbounded{\includegraphics[keepaspectratio]{303483661336987339_b011.png}}

"Give them all my dear love and a kiss. Tell them I think of them by
day, pray for them by night, and find my best comfort in their affection
at all times. A year seems very long to wait before I see them, but
remind them that while we wait we may all work, so that these hard days
need not be wasted. I know they will remember all I said to them, that
they will be loving children to you, will do their duty faithfully,
fight their bosom enemies bravely, and conquer themselves so
beautifully, that when I come back to them I may be fonder and prouder
than ever of my little women."

Everybody sniffed when they came to that part; Jo wasn\textquotesingle t
ashamed of the great tear that dropped off the end of her nose, and Amy
never minded the rumpling of her curls as she hid her face on her
mother\textquotesingle s shoulder and sobbed out, "I \emph{am} a selfish
girl! but I\textquotesingle ll truly try to be better, so he
mayn\textquotesingle t be disappointed in me by and by."

"We all will!" cried Meg. "I think too much of my looks, and hate to
work, but won\textquotesingle t any more, if I can help it."

"I\textquotesingle ll try and be what he loves to call me,
\textquotesingle a little woman,\textquotesingle{} and not be rough and
wild; but do my duty here instead of wanting to be somewhere else," said
Jo, thinking that keeping her temper at home was a much harder task than
facing a rebel or two down South.

Beth said nothing, but wiped away her tears with the blue army-sock, and
began to knit with all her might, losing no time in doing the duty that
lay nearest her, while she resolved in her quiet little soul to be all
that father hoped to find her when the year brought round the happy
coming home.

\protect\phantomsection\label{6672479776654687619_37106-h-0.htm.xhtml_b012.png}{}
\pandocbounded{\includegraphics[keepaspectratio]{303483661336987339_b012.png}}

Mrs. March broke the silence that followed Jo\textquotesingle s words,
by saying in her cheery voice, "Do you remember how you used to play
Pilgrim\textquotesingle s Progress when you were little things? Nothing
delighted you more than to have me tie my piece-bags on your backs for
burdens, give you hats and sticks and rolls of paper, and let you travel
through the house from the cellar, which was the City of Destruction,
up, up, to the house-top, where you had all the lovely things you could
collect to make a Celestial City."

"What fun it was, especially going by the lions, fighting Apollyon, and
passing through the Valley where the hobgoblins were!" said Jo.

"I liked the place where the bundles fell off and tumbled down stairs,"
said Meg.

"My favorite part was when we came out on the flat roof where our
flowers and arbors and pretty things were, and all stood and sung for
joy up there in the sunshine," said Beth, smiling, as if that pleasant
moment had come back to her.

"I don\textquotesingle t remember much about it, except that I was
afraid of the cellar and the dark entry, and always liked the cake and
milk we had up at the top. If I wasn\textquotesingle t too old for such
things, I\textquotesingle d rather like to play it over again," said
Amy, who began to talk of renouncing childish things at the mature age
of twelve.

"We never are too old for this, my dear, because it is a play we are
playing all the time in one way or another. Our burdens are here, our
road is before us, and the longing for goodness and happiness is the
guide that leads us through many troubles and mistakes to the peace
which is a true Celestial City. Now, my little pilgrims, suppose you
begin again, not in play, but in earnest, and see how far on you can get
before father comes home."

"Really, mother? Where are our bundles?" asked Amy, who was a very
literal young lady.

"Each of you told what your burden was just now, except Beth; I rather
think she hasn\textquotesingle t got any," said her mother.

"Yes, I have; mine is dishes and dusters, and envying girls with nice
pianos, and being afraid of people."

Beth\textquotesingle s bundle was such a funny one that everybody wanted
to laugh; but nobody did, for it would have hurt her feelings very much.

"Let us do it," said Meg thoughtfully. "It is only another name for
trying to be good, and the story may help us; for though we do want to
be good, it\textquotesingle s hard work, and we forget, and
don\textquotesingle t do our best."

"We were in the Slough of Despond to-night, and mother came and pulled
us out as Help did in the book. We ought to have our roll of directions,
like Christian. What shall we do about that?" asked Jo, delighted with
the fancy which lent a little romance to the very dull task of doing her
duty.

"Look under your pillows, Christmas morning, and you will find your
guide-book," replied Mrs. March.

They talked over the new plan while old Hannah cleared the table; then
out came the four little work-baskets, and the needles flew as the girls
made sheets for Aunt March. It was uninteresting sewing, but to-night no
one grumbled. They adopted Jo\textquotesingle s plan of dividing the
long seams into four parts, and calling the quarters Europe, Asia,
Africa, and America, and in that way got on capitally, especially when
they talked about the different countries as they stitched their way
through them.

\protect\phantomsection\label{6672479776654687619_37106-h-0.htm.xhtml_b013.png}{}
\pandocbounded{\includegraphics[keepaspectratio]{303483661336987339_b013.png}}

At nine they stopped work, and sung, as usual, before they went to bed.
No one but Beth could get much music out of the old piano; but she had a
way of softly touching the yellow keys, and making a pleasant
accompaniment to the simple songs they sung. Meg had a voice like a
flute, and she and her mother led the little choir. Amy chirped like a
cricket, and Jo wandered through the airs at her own sweet will, always
coming out at the wrong place with a croak or a quaver that spoilt the
most pensive tune. They had always done this from the time they could
lisp

"Crinkle, crinkle, \textquotesingle ittle \textquotesingle tar,"

and it had become a household custom, for the mother was a born singer.
The first sound in the morning was her voice, as she went about the
house singing like a lark; and the last sound at night was the same
cheery sound, for the girls never grew too old for that familiar
lullaby.

\protect\phantomsection\label{6672479776654687619_37106-h-0.htm.xhtml_b014.png}{}
\pandocbounded{\includegraphics[keepaspectratio]{303483661336987339_b014.png}}

\begin{center}\rule{0.5\linewidth}{0.5pt}\end{center}

\subsection{II. A Merry
Christmas.}\label{6672479776654687619_37106-h-0.htm.xhtml_pgepubid00004}

\protect\phantomsection\label{6672479776654687619_37106-h-0.htm.xhtml_b015.png}{}
\pandocbounded{\includegraphics[keepaspectratio]{303483661336987339_b015.png}}

\protect\phantomsection\label{6672479776654687619_37106-h-0.htm.xhtml_II}{}\hyperref[6672479776654687619_37106-h-0.htm.xhtml_contents]{II.}

A MERRY CHRISTMAS.

{Jo} was the first to wake in the gray dawn of Christmas morning. No
stockings hung at the fireplace, and for a moment she felt as much
disappointed as she did long ago, when her little sock fell down because
it was so crammed with goodies. Then she remembered her
mother\textquotesingle s promise, and, slipping her hand under her
pillow, drew out a little crimson-covered book. She knew it very well,
for it was that beautiful old story of the best life ever lived, and Jo
felt that it was a true guide-book for any pilgrim going the long
journey. She woke Meg with a "Merry Christmas," and bade her see what
was under her pillow. A green-covered book appeared, with the same
picture inside, and a few words written by their mother, which made
their one present very precious in their eyes. Presently Beth and Amy
woke, to rummage and find their little books also,---one dove-colored,
the other blue; and all sat looking at and talking about them, while the
east grew rosy with the coming day.

In spite of her small vanities, Margaret had a sweet and pious nature,
which unconsciously influenced her sisters, especially Jo, who loved her
very tenderly, and obeyed her because her advice was so gently given.

"Girls," said Meg seriously, looking from the tumbled head beside her to
the two little night-capped ones in the room beyond, "mother wants us to
read and love and mind these books, and we must begin at once. We used
to be faithful about it; but since father went away, and all this war
trouble unsettled us, we have neglected many things. You can do as you
please; but \emph{I} shall keep my book on the table here, and read a
little every morning as soon as I wake, for I know it will do me good,
and help me through the day."

Then she opened her new book and began to read. Jo put her arm round
her, and, leaning cheek to cheek, read also, with the quiet expression
so seldom seen on her restless face.

"How good Meg is! Come, Amy, let\textquotesingle s do as they do.
I\textquotesingle ll help you with the hard words, and
they\textquotesingle ll explain things if we don\textquotesingle t
understand," whispered Beth, very much impressed by the pretty books and
her sisters\textquotesingle{} example.

"I\textquotesingle m glad mine is blue," said Amy; and then the rooms
were very still while the pages were softly turned, and the winter
sunshine crept in to touch the bright heads and serious faces with a
Christmas greeting.

"Where is mother?" asked Meg, as she and Jo ran down to thank her for
their gifts, half an hour later.

"Goodness only knows. Some poor creeter come a-beggin\textquotesingle,
and your ma went straight off to see what was needed. There never
\emph{was} such a woman for givin\textquotesingle{} away vittles and
drink, clothes and firin\textquotesingle," replied Hannah, who had lived
with the family since Meg was born, and was considered by them all more
as a friend than a servant.

"She will be back soon, I think; so fry your cakes, and have everything
ready," said Meg, looking over the presents which were collected in a
basket and kept under the sofa, ready to be produced at the proper time.
"Why, where is Amy\textquotesingle s bottle of cologne?" she added, as
the little flask did not appear.

"She took it out a minute ago, and went off with it to put a ribbon on
it, or some such notion," replied Jo, dancing about the room to take the
first stiffness off the new army-slippers.

"How nice my handkerchiefs look, don\textquotesingle t they? Hannah
washed and ironed them for me, and I marked them all myself," said Beth,
looking proudly at the somewhat uneven letters which had cost her such
labor.

"Bless the child! she\textquotesingle s gone and put
\textquotesingle Mother\textquotesingle{} on them instead of
\textquotesingle M. March.\textquotesingle{} How funny!" cried Jo,
taking up one.

"Isn\textquotesingle t it right? I thought it was better to do it so,
because Meg\textquotesingle s initials are \textquotesingle M.
M.,\textquotesingle{} and I don\textquotesingle t want any one to use
these but Marmee," said Beth, looking troubled.

"It\textquotesingle s all right, dear, and a very pretty idea,---quite
sensible, too, for no one can ever mistake now. It will please her very
much, I know," said Meg, with a frown for Jo and a smile for Beth.

"There\textquotesingle s mother. Hide the basket, quick!" cried Jo, as a
door slammed, and steps sounded in the hall.

Amy came in hastily, and looked rather abashed when she saw her sisters
all waiting for her.

"Where have you been, and what are you hiding behind you?" asked Meg,
surprised to see, by her hood and cloak, that lazy Amy had been out so
early.

"Don\textquotesingle t laugh at me, Jo! I didn\textquotesingle t mean
any one should know till the time came. I only meant to change the
little bottle for a big one, and I gave \emph{all} my money to get it,
and I\textquotesingle m truly trying not to be selfish any more."

As she spoke, Amy showed the handsome flask which replaced the cheap
one; and looked so earnest and humble in her little effort to forget
herself that Meg hugged her on the spot, and Jo pronounced her "a
trump," while Beth ran to the window, and picked her finest rose to
ornament the stately bottle.

"You see I felt ashamed of my present, after reading and talking about
being good this morning, so I ran round the corner and changed it the
minute I was up: and I\textquotesingle m \emph{so} glad, for mine is the
handsomest now."

Another bang of the street-door sent the basket under the sofa, and the
girls to the table, eager for breakfast.

"Merry Christmas, Marmee! Many of them! Thank you for our books; we read
some, and mean to every day," they cried, in chorus.

"Merry Christmas, little daughters! I\textquotesingle m glad you began
at once, and hope you will keep on. But I want to say one word before we
sit down. Not far away from here lies a poor woman with a little
new-born baby. Six children are huddled into one bed to keep from
freezing, for they have no fire. There is nothing to eat over there; and
the oldest boy came to tell me they were suffering hunger and cold. My
girls, will you give them your breakfast as a Christmas present?"

They were all unusually hungry, having waited nearly an hour, and for a
minute no one spoke; only a minute, for Jo exclaimed impetuously,---

"I\textquotesingle m so glad you came before we began!"

"May I go and help carry the things to the poor little children?" asked
Beth, eagerly.

"\emph{I} shall take the cream and the muffins," added Amy, heroically
giving up the articles she most liked.

Meg was already covering the buckwheats, and piling the bread into one
big plate.

"I thought you\textquotesingle d do it," said Mrs. March, smiling as if
satisfied. "You shall all go and help me, and when we come back we will
have bread and milk for breakfast, and make it up at dinner-time."

They were soon ready, and the procession set out. Fortunately it was
early, and they went through back streets, so few people saw them, and
no one laughed at the queer party.

\protect\phantomsection\label{6672479776654687619_37106-h-0.htm.xhtml_b016.png}{}
\pandocbounded{\includegraphics[keepaspectratio]{303483661336987339_b016.png}}

A poor, bare, miserable room it was, with broken windows, no fire,
ragged bed-clothes, a sick mother, wailing baby, and a group of pale,
hungry children cuddled under one old quilt, trying to keep warm.

How the big eyes stared and the blue lips smiled as the girls went in!

"Ach, mein Gott! it is good angels come to us!" said the poor woman,
crying for joy.

"Funny angels in hoods and mittens," said Jo, and set them laughing.

In a few minutes it really did seem as if kind spirits had been at work
there. Hannah, who had carried wood, made a fire, and stopped up the
broken panes with old hats and her own cloak. Mrs. March gave the mother
tea and gruel, and comforted her with promises of help, while she
dressed the little baby as tenderly as if it had been her own. The
girls, meantime, spread the table, set the children round the fire, and
fed them like so many hungry birds,---laughing, talking, and trying to
understand the funny broken English.

"Das ist gut!" "Die Engel-kinder!" cried the poor things, as they ate,
and warmed their purple hands at the comfortable blaze.

The girls had never been called angel children before, and thought it
very agreeable, especially Jo, who had been considered a "Sancho" ever
since she was born. That was a very happy breakfast, though they
didn\textquotesingle t get any of it; and when they went away, leaving
comfort behind, I think there were not in all the city four merrier
people than the hungry little girls who gave away their breakfasts and
contented themselves with bread and milk on Christmas morning.

"That\textquotesingle s loving our neighbor better than ourselves, and I
like it," said Meg, as they set out their presents, while their mother
was upstairs collecting clothes for the poor Hummels.

Not a very splendid show, but there was a great deal of love done up in
the few little bundles; and the tall vase of red roses, white
chrysanthemums, and trailing vines, which stood in the middle, gave
quite an elegant air to the table.

"She\textquotesingle s coming! Strike up, Beth! Open the door, Amy!
Three cheers for Marmee!" cried Jo, prancing about, while Meg went to
conduct mother to the seat of honor.

Beth played her gayest march, Amy threw open the door, and Meg enacted
escort with great dignity. Mrs. March was both surprised and touched;
and smiled with her eyes full as she examined her presents, and read the
little notes which accompanied them. The slippers went on at once, a new
handkerchief was slipped into her pocket, well scented with
Amy\textquotesingle s cologne, the rose was fastened in her bosom, and
the nice gloves were pronounced a "perfect fit."

There was a good deal of laughing and kissing and explaining, in the
simple, loving fashion which makes these home-festivals so pleasant at
the time, so sweet to remember long afterward, and then all fell to
work.

The morning charities and ceremonies took so much time that the rest of
the day was devoted to preparations for the evening festivities. Being
still too young to go often to the theatre, and not rich enough to
afford any great outlay for private performances, the girls put their
wits to work, and---necessity being the mother of invention,---made
whatever they needed. Very clever were some of their
productions,---pasteboard guitars, antique lamps made of old-fashioned
butter-boats covered with silver paper, gorgeous robes of old cotton,
glittering with tin spangles from a pickle factory, and armor covered
with the same useful diamond-shaped bits, left in sheets when the lids
of tin preserve-pots were cut out. The furniture was used to being
turned topsy-turvy, and the big chamber was the scene of many innocent
revels.

No gentlemen were admitted; so Jo played male parts to her
heart\textquotesingle s content, and took immense satisfaction in a pair
of russet-leather boots given her by a friend, who knew a lady who knew
an actor. These boots, an old foil, and a slashed doublet once used by
an artist for some picture, were Jo\textquotesingle s chief treasures,
and appeared on all occasions. The smallness of the company made it
necessary for the two principal actors to take several parts apiece; and
they certainly deserved some credit for the hard work they did in
learning three or four different parts, whisking in and out of various
costumes, and managing the stage besides. It was excellent drill for
their memories, a harmless amusement, and employed many hours which
otherwise would have been idle, lonely, or spent in less profitable
society.

On Christmas night, a dozen girls piled on to the bed which was the
dress-circle, and sat before the blue and yellow chintz curtains in a
most flattering state of expectancy. There was a good deal of rustling
and whispering behind the curtain, a trifle of lamp-smoke, and an
occasional giggle from Amy, who was apt to get hysterical in the
excitement of the moment. Presently a bell sounded, the curtains flew
apart, and the Operatic Tragedy began.

"A gloomy wood," according to the one play-bill, was represented by a
few shrubs in pots, green baize on the floor, and a cave in the
distance. This cave was made with a clothes-horse for a roof, bureaus
for walls; and in it was a small furnace in full blast, with a black pot
on it, and an old witch bending over it. The stage was dark, and the
glow of the furnace had a fine effect, especially as real steam issued
from the kettle when the witch took off the cover. A moment was allowed
for the first thrill to subside; then Hugo, the villain, stalked in with
a clanking sword at his side, a slouched hat, black beard, mysterious
cloak, and the boots. After pacing to and fro in much agitation, he
struck his forehead, and burst out in a wild strain, singing of his
hatred to Roderigo, his love for Zara, and his pleasing resolution to
kill the one and win the other. The gruff tones of
Hugo\textquotesingle s voice, with an occasional shout when his feelings
overcame him, were very impressive, and the audience applauded the
moment he paused for breath. Bowing with the air of one accustomed to
public praise, he stole to the cavern, and ordered Hagar to come forth
with a commanding "What ho, minion! I need thee!"

\protect\phantomsection\label{6672479776654687619_37106-h-0.htm.xhtml_b017.png}{}
\pandocbounded{\includegraphics[keepaspectratio]{303483661336987339_b017.png}}

Out came Meg, with gray horse-hair hanging about her face, a red and
black robe, a staff, and cabalistic signs upon her cloak. Hugo demanded
a potion to make Zara adore him, and one to destroy Roderigo. Hagar, in
a fine dramatic melody, promised both, and proceeded to call up the
spirit who would bring the love philter:---

"Hither, hither, from thy home,

Airy sprite, I bid thee come!

Born of roses, fed on dew,

Charms and potions canst thou brew?

Bring me here, with elfin speed,

The fragrant philter which I need;

Make it sweet and swift and strong,

Spirit, answer now my song!"

A soft strain of music sounded, and then at the back of the cave
appeared a little figure in cloudy white, with glittering wings, golden
hair, and a garland of roses on its head. Waving a wand, it sang,---

"Hither I come,

From my airy home,

Afar in the silver moon.

Take the magic spell,

And use it well,

Or its power will vanish soon!"

\protect\phantomsection\label{6672479776654687619_37106-h-0.htm.xhtml_b018.png}{}
\pandocbounded{\includegraphics[keepaspectratio]{303483661336987339_b018.png}}

And, dropping a small, gilded bottle at the witch\textquotesingle s
feet, the spirit vanished. Another chant from Hagar produced another
apparition,---not a lovely one; for, with a bang, an ugly black imp
appeared, and, having croaked a reply, tossed a dark bottle at Hugo, and
disappeared with a mocking laugh. Having warbled his thanks and put the
potions in his boots, Hugo departed; and Hagar informed the audience
that, as he had killed a few of her friends in times past, she has
cursed him, and intends to thwart his plans, and be revenged on him.
Then the curtain fell, and the audience reposed and ate candy while
discussing the merits of the play.

A good deal of hammering went on before the curtain rose again; but when
it became evident what a masterpiece of stage-carpentering had been got
up, no one murmured at the delay. It was truly superb! A tower rose to
the ceiling; half-way up appeared a window, with a lamp burning at it,
and behind the white curtain appeared Zara in a lovely blue and silver
dress, waiting for Roderigo. He came in gorgeous array, with plumed cap,
red cloak, chestnut love-locks, a guitar, and the boots, of course.
Kneeling at the foot of the tower, he sang a serenade in melting tones.
Zara replied, and, after a musical dialogue, consented to fly. Then came
the grand effect of the play. Roderigo produced a rope-ladder, with five
steps to it, threw up one end, and invited Zara to descend. Timidly she
crept from her lattice, put her hand on Roderigo\textquotesingle s
shoulder, and was about to leap gracefully down, when, "Alas! alas for
Zara!" she forgot her train,---it caught in the window; the tower
tottered, leaned forward, fell with a crash, and buried the unhappy
lovers in the ruins!

A universal shriek arose as the russet boots waved wildly from the
wreck, and a golden head emerged, exclaiming, "I told you so! I told you
so!" With wonderful presence of mind, Don Pedro, the cruel sire, rushed
in, dragged out his daughter, with a hasty aside,---

"Don\textquotesingle t laugh! Act as if it was all right!"---and,
ordering Roderigo up, banished him from the kingdom with wrath and
scorn. Though decidedly shaken by the fall of the tower upon him,
Roderigo defied the old gentleman, and refused to stir. This dauntless
example fired Zara: she also defied her sire, and he ordered them both
to the deepest dungeons of the castle. A stout little retainer came in
with chains, and led them away, looking very much frightened, and
evidently forgetting the speech he ought to have made.

Act third was the castle hall; and here Hagar appeared, having come to
free the lovers and finish Hugo. She hears him coming, and hides; sees
him put the potions into two cups of wine, and bid the timid little
servant "Bear them to the captives in their cells, and tell them I shall
come anon." The servant takes Hugo aside to tell him something, and
Hagar changes the cups for two others which are harmless. Ferdinando,
the "minion," carries them away, and Hagar puts back the cup which holds
the poison meant for Roderigo. Hugo, getting thirsty after a long
warble, drinks it, loses his wits, and, after a good deal of clutching
and stamping, falls flat and dies; while Hagar informs him what she has
done in a song of exquisite power and melody.

This was a truly thrilling scene, though some persons might have thought
that the sudden tumbling down of a quantity of long hair rather marred
the effect of the villain\textquotesingle s death. He was called before
the curtain, and with great propriety appeared, leading Hagar, whose
singing was considered more wonderful than all the rest of the
performance put together.

Act fourth displayed the despairing Roderigo on the point of stabbing
himself, because he has been told that Zara has deserted him. Just as
the dagger is at his heart, a lovely song is sung under his window,
informing him that Zara is true, but in danger, and he can save her, if
he will. A key is thrown in, which unlocks the door, and in a spasm of
rapture he tears off his chains, and rushes away to find and rescue his
lady-love.

Act fifth opened with a stormy scene between Zara and Don Pedro. He
wishes her to go into a convent, but she won\textquotesingle t hear of
it; and, after a touching appeal, is about to faint, when Roderigo
dashes in and demands her hand. Don Pedro refuses, because he is not
rich. They shout and gesticulate tremendously, but cannot agree, and
Roderigo is about to bear away the exhausted Zara, when the timid
servant enters with a letter and a bag from Hagar, who has mysteriously
disappeared. The latter informs the party that she bequeaths untold
wealth to the young pair, and an awful doom to Don Pedro, if he
doesn\textquotesingle t make them happy. The bag is opened, and several
quarts of tin money shower down upon the stage, till it is quite
glorified with the glitter. This entirely softens the "stern sire": he
consents without a murmur, all join in a joyful chorus, and the curtain
falls upon the lovers kneeling to receive Don Pedro\textquotesingle s
blessing in attitudes of the most romantic grace.

\protect\phantomsection\label{6672479776654687619_37106-h-0.htm.xhtml_b019.png}{}
\pandocbounded{\includegraphics[keepaspectratio]{303483661336987339_b019.png}}

Tumultuous applause followed, but received an unexpected check; for the
cot-bed, on which the "dress-circle" was built, suddenly shut up, and
extinguished the enthusiastic audience. Roderigo and Don Pedro flew to
the rescue, and all were taken out unhurt, though many were speechless
with laughter. The excitement had hardly subsided, when Hannah appeared,
with "Mrs. March\textquotesingle s compliments, and would the ladies
walk down to supper."

This was a surprise, even to the actors; and, when they saw the table,
they looked at one another in rapturous amazement. It was like Marmee to
get up a little treat for them; but anything so fine as this was
unheard-of since the departed days of plenty. There was
ice-cream,---actually two dishes of it, pink and white,---and cake and
fruit and distracting French bonbons, and, in the middle of the table,
four great bouquets of hot-house flowers!

It quite took their breath away; and they stared first at the table and
then at their mother, who looked as if she enjoyed it immensely.

"Is it fairies?" asked Amy,

"It\textquotesingle s Santa Claus," said Beth.

"Mother did it"; and Meg smiled her sweetest, in spite of her gray beard
and white eyebrows.

"Aunt March had a good fit, and sent the supper," cried Jo, with a
sudden inspiration.

"All wrong. Old Mr. Laurence sent it," replied Mrs. March.

"The Laurence boy\textquotesingle s grandfather! What in the world put
such a thing into his head? We don\textquotesingle t know him!"
exclaimed Meg.

"Hannah told one of his servants about your breakfast party. He is an
odd old gentleman, but that pleased him. He knew my father, years ago;
and he sent me a polite note this afternoon, saying he hoped I would
allow him to express his friendly feeling toward my children by sending
them a few trifles in honor of the day. I could not refuse; and so you
have a little feast at night to make up for the bread-and-milk
breakfast."

"That boy put it into his head, I know he did! He\textquotesingle s a
capital fellow, and I wish we could get acquainted. He looks as if
he\textquotesingle d like to know us; but he\textquotesingle s bashful,
and Meg is so prim she won\textquotesingle t let me speak to him when we
pass," said Jo, as the plates went round, and the ice began to melt out
of sight, with "Ohs!" and "Ahs!" of satisfaction.

"You mean the people who live in the big house next door,
don\textquotesingle t you?" asked one of the girls. "My mother knows old
Mr. Laurence; but says he\textquotesingle s very proud, and
doesn\textquotesingle t like to mix with his neighbors. He keeps his
grandson shut up, when he isn\textquotesingle t riding or walking with
his tutor, and makes him study very hard. We invited him to our party,
but he didn\textquotesingle t come. Mother says he\textquotesingle s
very nice, though he never speaks to us girls."

"Our cat ran away once, and he brought her back, and we talked over the
fence, and were getting on capitally,---all about cricket, and so
on,---when he saw Meg coming, and walked off. I mean to know him some
day; for he needs fun, I\textquotesingle m sure he does," said Jo
decidedly.

\protect\phantomsection\label{6672479776654687619_37106-h-0.htm.xhtml_b020.png}{}
\pandocbounded{\includegraphics[keepaspectratio]{303483661336987339_b020.png}}

"I like his manners, and he looks like a little gentleman; so
I\textquotesingle ve no objection to your knowing him, if a proper
opportunity comes. He brought the flowers himself; and I should have
asked him in, if I had been sure what was going on upstairs. He looked
so wistful as he went away, hearing the frolic, and evidently having
none of his own."

"It\textquotesingle s a mercy you didn\textquotesingle t, mother!"
laughed Jo, looking at her boots. "But we\textquotesingle ll have
another play, some time, that he \emph{can} see. Perhaps
he\textquotesingle ll help act; wouldn\textquotesingle t that be jolly?"

"I never had such a fine bouquet before! How pretty it is!" And Meg
examined her flowers with great interest.

"They \emph{are} lovely! But Beth\textquotesingle s roses are sweeter to
me," said Mrs. March, smelling the half-dead posy in her belt.

Beth nestled up to her, and whispered softly, "I wish I could send my
bunch to father. I\textquotesingle m afraid he isn\textquotesingle t
having such a merry Christmas as we are."

\protect\phantomsection\label{6672479776654687619_37106-h-0.htm.xhtml_b021.png}{}
\pandocbounded{\includegraphics[keepaspectratio]{303483661336987339_b021.png}}

\begin{center}\rule{0.5\linewidth}{0.5pt}\end{center}

\subsection{III. The Laurence
Boy.}\label{6672479776654687619_37106-h-0.htm.xhtml_pgepubid00005}

\protect\phantomsection\label{6672479776654687619_37106-h-0.htm.xhtml_b022.png}{}
\pandocbounded{\includegraphics[keepaspectratio]{303483661336987339_b022.png}}

\protect\phantomsection\label{6672479776654687619_37106-h-0.htm.xhtml_III}{}\hyperref[6672479776654687619_37106-h-0.htm.xhtml_contents]{III.}

THE LAURENCE BOY.

"{Jo! Jo!} where are you?" cried Meg, at the foot of the garret stairs.

"Here!" answered a husky voice from above; and, running up, Meg found
her sister eating apples and crying over the "Heir of Redclyffe,"
wrapped up in a comforter on an old three-legged sofa by the sunny
window. This was Jo\textquotesingle s favorite refuge; and here she
loved to retire with half a dozen russets and a nice book, to enjoy the
quiet and the society of a pet rat who lived near by, and
didn\textquotesingle t mind her a particle. As Meg appeared, Scrabble
whisked into his hole. Jo shook the tears off her cheeks, and waited to
hear the news.

"Such fun! only see! a regular note of invitation from Mrs. Gardiner for
to-morrow night!" cried Meg, waving the precious paper, and then
proceeding to read it, with girlish delight.

"\textquotesingle Mrs. Gardiner would be happy to see Miss March and
Miss Josephine at a little dance on New-Year\textquotesingle s
Eve.\textquotesingle{} Marmee is willing we should go; now what
\emph{shall} we wear?"

"What\textquotesingle s the use of asking that, when you know we shall
wear our poplins, because we haven\textquotesingle t got anything else?"
answered Jo, with her mouth full.

"If I only had a silk!" sighed Meg. "Mother says I may when
I\textquotesingle m eighteen, perhaps; but two years is an everlasting
time to wait."

"I\textquotesingle m sure our pops look like silk, and they are nice
enough for us. Yours is as good as new, but I forgot the burn and the
tear in mine. Whatever shall I do? the burn shows badly, and I
can\textquotesingle t take any out."

"You must sit still all you can, and keep your back out of sight; the
front is all right. I shall have a new ribbon for my hair, and Marmee
will lend me her little pearl pin, and my new slippers are lovely, and
my gloves will do, though they aren\textquotesingle t as nice as
I\textquotesingle d like."

"Mine are spoilt with lemonade, and I can\textquotesingle t get any new
ones, so I shall have to go without," said Jo, who never troubled
herself much about dress.

"You \emph{must} have gloves, or I won\textquotesingle t go," cried Meg
decidedly. "Gloves are more important than anything else; you
can\textquotesingle t dance without them, and if you
don\textquotesingle t I should be \emph{so} mortified."

"Then I\textquotesingle ll stay still. I don\textquotesingle t care much
for company dancing; it\textquotesingle s no fun to go sailing round; I
like to fly about and cut capers."

"You can\textquotesingle t ask mother for new ones, they are so
expensive, and you are so careless. She said, when you spoilt the
others, that she shouldn\textquotesingle t get you any more this winter.
Can\textquotesingle t you make them do?" asked Meg anxiously.

"I can hold them crumpled up in my hand, so no one will know how stained
they are; that\textquotesingle s all I can do. No! I\textquotesingle ll
tell you how we can manage---each wear one good one and carry a bad one;
don\textquotesingle t you see?"

"Your hands are bigger than mine, and you will stretch my glove
dreadfully," began Meg, whose gloves were a tender point with her.

"Then I\textquotesingle ll go without. I don\textquotesingle t care what
people say!" cried Jo, taking up her book.

"You may have it, you may! only don\textquotesingle t stain it, and do
behave nicely. Don\textquotesingle t put your hands behind you, or
stare, or say \textquotesingle Christopher Columbus!\textquotesingle{}
will you?"

"Don\textquotesingle t worry about me; I\textquotesingle ll be as prim
as I can, and not get into any scrapes, if I can help it. Now go and
answer your note, and let me finish this splendid story."

So Meg went away to "accept with thanks," look over her dress, and sing
blithely as she did up her one real lace frill; while Jo finished her
story, her four apples, and had a game of romps with Scrabble.

On New-Year\textquotesingle s Eve the parlor was deserted, for the two
younger girls played dressing-maids, and the two elder were absorbed in
the all-important business of "getting ready for the party." Simple as
the toilets were, there was a great deal of running up and down,
laughing and talking, and at one time a strong smell of burnt hair
pervaded the house. Meg wanted a few curls about her face, and Jo
undertook to pinch the papered locks with a pair of hot tongs.

\protect\phantomsection\label{6672479776654687619_37106-h-0.htm.xhtml_b023.png}{}
\pandocbounded{\includegraphics[keepaspectratio]{303483661336987339_b023.png}}

"Ought they to smoke like that?" asked Beth, from her perch on the bed.

"It\textquotesingle s the dampness drying," replied Jo.

"What a queer smell! it\textquotesingle s like burnt feathers," observed
Amy, smoothing her own pretty curls with a superior air.

"There, now I\textquotesingle ll take off the papers and
you\textquotesingle ll see a cloud of little ringlets," said Jo, putting
down the tongs.

She did take off the papers, but no cloud of ringlets appeared, for the
hair came with the papers, and the horrified hair-dresser laid a row of
little scorched bundles on the bureau before her victim.

"Oh, oh, oh! what \emph{have} you done? I\textquotesingle m spoilt! I
can\textquotesingle t go! My hair, oh, my hair!" wailed Meg, looking
with despair at the uneven frizzle on her forehead.

"Just my luck! you shouldn\textquotesingle t have asked me to do it; I
always spoil everything. I\textquotesingle m so sorry, but the tongs
were too hot, and so I\textquotesingle ve made a mess," groaned poor Jo,
regarding the black pancakes with tears of regret.

"It isn\textquotesingle t spoilt; just frizzle it, and tie your ribbon
so the ends come on your forehead a bit, and it will look like the last
fashion. I\textquotesingle ve seen many girls do it so," said Amy
consolingly.

"Serves me right for trying to be fine. I wish I\textquotesingle d let
my hair alone," cried Meg petulantly.

"So do I, it was so smooth and pretty. But it will soon grow out again,"
said Beth, coming to kiss and comfort the shorn sheep.

After various lesser mishaps, Meg was finished at last, and by the
united exertions of the family Jo\textquotesingle s hair was got up and
her dress on. They looked very well in their simple suits,---Meg in
silvery drab, with a blue velvet snood, lace frills, and the pearl pin;
Jo in maroon, with a stiff, gentlemanly linen collar, and a white
chrysanthemum or two for her only ornament. Each put on one nice light
glove, and carried one soiled one, and all pronounced the effect "quite
easy and fine." Meg\textquotesingle s high-heeled slippers were very
tight, and hurt her, though she would not own it, and
Jo\textquotesingle s nineteen hair-pins all seemed stuck straight into
her head, which was not exactly comfortable; but, dear me, let us be
elegant or die!

"Have a good time, dearies!" said Mrs. March, as the sisters went
daintily down the walk. "Don\textquotesingle t eat much supper, and come
away at eleven, when I send Hannah for you." As the gate clashed behind
them, a voice cried from a window,---

"Girls, girls! \emph{have} you both got nice pocket-handkerchiefs?"

"Yes, yes, spandy nice, and Meg has cologne on hers," cried Jo, adding,
with a laugh, as they went on, "I do believe Marmee would ask that if we
were all running away from an earthquake."

"It is one of her aristocratic tastes, and quite proper, for a real lady
is always known by neat boots, gloves, and handkerchief," replied Meg,
who had a good many little "aristocratic tastes" of her own.

"Now don\textquotesingle t forget to keep the bad breadth out of sight,
Jo. Is my sash right? and does my hair look \emph{very} bad?" said Meg,
as she turned from the glass in Mrs. Gardiner\textquotesingle s
dressing-room, after a prolonged prink.

"I know I shall forget. If you see me doing anything wrong, just remind
me by a wink, will you?" returned Jo, giving her collar a twitch and her
head a hasty brush.

"No, winking isn\textquotesingle t lady-like; I\textquotesingle ll lift
my eyebrows if anything is wrong, and nod if you are all right. Now hold
your shoulders straight, and take short steps, and don\textquotesingle t
shake hands if you are introduced to any one: it isn\textquotesingle t
the thing."

"How \emph{do} you learn all the proper ways? I never can.
Isn\textquotesingle t that music gay?"

\protect\phantomsection\label{6672479776654687619_37106-h-0.htm.xhtml_b024.png}{}
\pandocbounded{\includegraphics[keepaspectratio]{303483661336987339_b024.png}}

Down they went, feeling a trifle timid, for they seldom went to parties,
and, informal as this little gathering was, it was an event to them.
Mrs. Gardiner, a stately old lady, greeted them kindly, and handed them
over to the eldest of her six daughters. Meg knew Sallie, and was at her
ease very soon; but Jo, who didn\textquotesingle t care much for girls
or girlish gossip, stood about, with her back carefully against the
wall, and felt as much out of place as a colt in a flower-garden. Half a
dozen jovial lads were talking about skates in another part of the room,
and she longed to go and join them, for skating was one of the joys of
her life. She telegraphed her wish to Meg, but the eyebrows went up so
alarmingly that she dared not stir. No one came to talk to her, and one
by one the group near her dwindled away, till she was left alone. She
could not roam about and amuse herself, for the burnt breadth would
show, so she stared at people rather forlornly till the dancing began.
Meg was asked at once, and the tight slippers tripped about so briskly
that none would have guessed the pain their wearer suffered smilingly.
Jo saw a big red-headed youth approaching her corner, and fearing he
meant to engage her, she slipped into a curtained recess, intending to
peep and enjoy herself in peace. Unfortunately, another bashful person
had chosen the same refuge; for, as the curtain fell behind her, she
found herself face to face with the "Laurence boy."

\protect\phantomsection\label{6672479776654687619_37106-h-0.htm.xhtml_b025.png}{}
\pandocbounded{\includegraphics[keepaspectratio]{303483661336987339_b025.png}}

"Dear me, I didn\textquotesingle t know any one was here!" stammered Jo,
preparing to back out as speedily as she had bounced in.

But the boy laughed, and said pleasantly, though he looked a little
startled,---

"Don\textquotesingle t mind me; stay, if you like."

"Sha\textquotesingle n\textquotesingle t I disturb you?"

"Not a bit; I only came here because I don\textquotesingle t know many
people, and felt rather strange at first, you know."

"So did I. Don\textquotesingle t go away, please, unless
you\textquotesingle d rather."

The boy sat down again and looked at his pumps, till Jo said, trying to
be polite and easy,---

"I think I\textquotesingle ve had the pleasure of seeing you before; you
live near us, don\textquotesingle t you?"

"Next door"; and he looked up and laughed outright, for
Jo\textquotesingle s prim manner was rather funny when he remembered how
they had chatted about cricket when he brought the cat home.

That put Jo at her ease; and she laughed too, as she said, in her
heartiest way,---

"We did have such a good time over your nice Christmas present."

"Grandpa sent it."

"But you put it into his head, didn\textquotesingle t you, now?"

"How is your cat, Miss March?" asked the boy, trying to look sober,
while his black eyes shone with fun.

"Nicely, thank you, Mr. Laurence; but I am not Miss March,
I\textquotesingle m only Jo," returned the young lady.

"I\textquotesingle m not Mr. Laurence, I\textquotesingle m only Laurie."

"Laurie Laurence,---what an odd name!"

"My first name is Theodore, but I don\textquotesingle t like it, for the
fellows called me Dora, so I made them say Laurie instead."

"I hate my name, too---so sentimental! I wish every one would say Jo,
instead of Josephine. How did you make the boys stop calling you Dora?"

"I thrashed \textquotesingle em."

"I can\textquotesingle t thrash Aunt March, so I suppose I shall have to
bear it"; and Jo resigned herself with a sigh.

"Don\textquotesingle t you like to dance, Miss Jo?" asked Laurie,
looking as if he thought the name suited her.

"I like it well enough if there is plenty of room, and every one is
lively. In a place like this I\textquotesingle m sure to upset
something, tread on people\textquotesingle s toes, or do something
dreadful, so I keep out of mischief, and let Meg sail about.
Don\textquotesingle t you dance?"

"Sometimes; you see I\textquotesingle ve been abroad a good many years,
and haven\textquotesingle t been into company enough yet to know how you
do things here."

"Abroad!" cried Jo. "Oh, tell me about it! I love dearly to hear people
describe their travels."

Laurie didn\textquotesingle t seem to know where to begin; but
Jo\textquotesingle s eager questions soon set him going, and he told her
how he had been at school in Vevay, where the boys never wore hats, and
had a fleet of boats on the lake, and for holiday fun went walking trips
about Switzerland with their teachers.

"Don\textquotesingle t I wish I\textquotesingle d been there!" cried Jo.
"Did you go to Paris?"

"We spent last winter there."

"Can you talk French?"

"We were not allowed to speak any thing else at Vevay."

"Do say some! I can read it, but can\textquotesingle t pronounce."

"Quel nom a cette jeune demoiselle en les pantoufles jolis?" said Laurie
good-naturedly.

"How nicely you do it! Let me see,---you said, \textquotesingle Who is
the young lady in the pretty slippers,\textquotesingle{}
didn\textquotesingle t you?"

"Oui, mademoiselle."

"It\textquotesingle s my sister Margaret, and you knew it was! Do you
think she is pretty?"

"Yes; she makes me think of the German girls, she looks so fresh and
quiet, and dances like a lady."

Jo quite glowed with pleasure at this boyish praise of her sister, and
stored it up to repeat to Meg. Both peeped and criticised and chatted,
till they felt like old acquaintances. Laurie\textquotesingle s
bashfulness soon wore off; for Jo\textquotesingle s gentlemanly demeanor
amused and set him at his ease, and Jo was her merry self again, because
her dress was forgotten, and nobody lifted their eyebrows at her. She
liked the "Laurence boy" better than ever, and took several good looks
at him, so that she might describe him to the girls; for they had no
brothers, very few male cousins, and boys were almost unknown creatures
to them.

"Curly black hair; brown skin; big, black eyes; handsome nose; fine
teeth; small hands and feet; taller than I am; very polite, for a boy,
and altogether jolly. Wonder how old he is?"

It was on the tip of Jo\textquotesingle s tongue to ask; but she checked
herself in time, and, with unusual tact, tried to find out in a
roundabout way.

"I suppose you are going to college soon? I see you pegging away at your
books,---no, I mean studying hard"; and Jo blushed at the dreadful
"pegging" which had escaped her.

Laurie smiled, but didn\textquotesingle t seem shocked, and answered,
with a shrug,---

"Not for a year or two; I won\textquotesingle t go before seventeen,
anyway."

"Aren\textquotesingle t you but fifteen?" asked Jo, looking at the tall
lad, whom she had imagined seventeen already.

"Sixteen, next month."

"How I wish I was going to college! You don\textquotesingle t look as if
you liked it."

"I hate it! Nothing but grinding or skylarking. And I
don\textquotesingle t like the way fellows do either, in this country."

"What do you like?"

"To live in Italy, and to enjoy myself in my own way."

Jo wanted very much to ask what his own way was; but his black brows
looked rather threatening as he knit them; so she changed the subject by
saying, as her foot kept time, "That\textquotesingle s a splendid polka!
Why don\textquotesingle t you go and try it?"

"If you will come too," he answered, with a gallant little bow.

"I can\textquotesingle t; for I told Meg I wouldn\textquotesingle t,
because---" There Jo stopped, and looked undecided whether to tell or to
laugh.

"Because what?" asked Laurie curiously.

"You won\textquotesingle t tell?"

"Never!"

"Well, I have a bad trick of standing before the fire, and so I burn my
frocks, and I scorched this one; and, though it\textquotesingle s nicely
mended, it shows, and Meg told me to keep still, so no one would see it.
You may laugh, if you want to; it is funny, I know."

But Laurie didn\textquotesingle t laugh; he only looked down a minute,
and the expression of his face puzzled Jo, when he said very gently,---

"Never mind that; I\textquotesingle ll tell you how we can manage:
there\textquotesingle s a long hall out there, and we can dance grandly,
and no one will see us. Please come?"

Jo thanked him, and gladly went, wishing she had two neat gloves, when
she saw the nice, pearl-colored ones her partner wore. The hall was
empty, and they had a grand polka; for Laurie danced well, and taught
her the German step, which delighted Jo, being full of swing and spring.
When the music stopped, they sat down on the stairs to get their breath;
and Laurie was in the midst of an account of a
students\textquotesingle{} festival at Heidelberg, when Meg appeared in
search of her sister. She beckoned, and Jo reluctantly followed her into
a side-room, where she found her on a sofa, holding her foot, and
looking pale.

\protect\phantomsection\label{6672479776654687619_37106-h-0.htm.xhtml_b026.png}{}
\pandocbounded{\includegraphics[keepaspectratio]{303483661336987339_b026.png}}

"I\textquotesingle ve sprained my ankle. That stupid high heel turned,
and gave me a sad wrench. It aches so, I can hardly stand, and I
don\textquotesingle t know how I\textquotesingle m ever going to get
home," she said, rocking to and fro in pain.

"I knew you\textquotesingle d hurt your feet with those silly shoes.
I\textquotesingle m sorry. But I don\textquotesingle t see what you can
do, except get a carriage, or stay here all night," answered Jo, softly
rubbing the poor ankle as she spoke.

"I can\textquotesingle t have a carriage, without its costing ever so
much. I dare say I can\textquotesingle t get one at all; for most people
come in their own, and it\textquotesingle s a long way to the stable,
and no one to send."

"I\textquotesingle ll go."

"No, indeed! It\textquotesingle s past nine, and dark as Egypt. I
can\textquotesingle t stop here, for the house is full. Sallie has some
girls staying with her. I\textquotesingle ll rest till Hannah comes, and
then do the best I can."

"I\textquotesingle ll ask Laurie; he will go," said Jo, looking relieved
as the idea occurred to her.

"Mercy, no! Don\textquotesingle t ask or tell any one. Get me my
rubbers, and put these slippers with our things. I can\textquotesingle t
dance any more; but as soon as supper is over, watch for Hannah, and
tell me the minute she comes."

"They are going out to supper now. I\textquotesingle ll stay with you;
I\textquotesingle d rather."

"No, dear, run along, and bring me some coffee. I\textquotesingle m so
tired, I can\textquotesingle t stir!"

So Meg reclined, with rubbers well hidden, and Jo went blundering away
to the dining-room, which she found after going into a china-closet, and
opening the door of a room where \ul{old Mr. Gardiner} was taking a
little private refreshment. Making a dart at the table, she secured the
coffee, which she immediately spilt, thereby making the front of her
dress as bad as the back.

"Oh, dear, what a blunderbuss I am!" exclaimed Jo, finishing
Meg\textquotesingle s glove by scrubbing her gown with it.

"Can I help you?" said a friendly voice; and there was Laurie, with a
full cup in one hand and a plate of ice in the other.

"I was trying to get something for Meg, who is very tired, and some one
shook me; and here I am, in a nice state," answered Jo, glancing
dismally from the stained skirt to the coffee-colored glove.

"Too bad! I was looking for some one to give this to. May I take it to
your sister?"

"Oh, thank you! I\textquotesingle ll show you where she is. I
don\textquotesingle t offer to take it myself, for I should only get
into another scrape if I did."

Jo led the way; and, as if used to waiting on ladies, Laurie drew up a
little table, brought a second instalment of coffee and ice for Jo, and
was so obliging that even particular Meg pronounced him a "nice boy."
They had a merry time over the bonbons and mottoes, and were in the
midst of a quiet game of "Buzz," with two or three other young people
who had strayed in, when Hannah appeared. Meg forgot her foot, and rose
so quickly that she was forced to catch hold of Jo, with an exclamation
of pain.

"Hush! Don\textquotesingle t say anything," she whispered, adding aloud,
"It\textquotesingle s nothing. I turned my foot a little,
that\textquotesingle s all"; and limped up-stairs to put her things on.

Hannah scolded, Meg cried, and Jo was at her wits\textquotesingle{} end,
till she decided to take things into her own hands. Slipping out, she
ran down, and, finding a servant, asked if he could get her a carriage.
It happened to be a hired waiter, who knew nothing about the
neighborhood; and Jo was looking round for help, when Laurie, who had
heard what she said, came up, and offered his
grandfather\textquotesingle s carriage, which had just come for him, he
said.

"It\textquotesingle s so early! You can\textquotesingle t mean to go
yet?" began Jo, looking relieved, but hesitating to accept the offer.

"I always go early,---I do, truly! Please let me take you home?
It\textquotesingle s all on my way, you know, and it rains, they say."

That settled it; and, telling him of Meg\textquotesingle s mishap, Jo
gratefully accepted, and rushed up to bring down the rest of the party.
Hannah hated rain as much as a cat does; so she made no trouble, and
they rolled away in the luxurious close carriage, feeling very festive
and elegant. \ul{Laurie went on the box,} so Meg could keep her foot up,
and the girls talked over their party in freedom.

"I had a capital time. Did you?" asked Jo, rumpling up her hair, and
making herself comfortable.

"Yes, till I hurt myself. Sallie\textquotesingle s friend, Annie Moffat,
took a fancy to me, and asked me to come and spend a week with her, when
Sallie does. She is going in the spring, when the opera comes; and it
will be perfectly splendid, if mother only lets me go," answered Meg,
cheering up at the thought.

"I saw you dancing with the red-headed man I ran away from. Was he
nice?"

"Oh, very! His hair is auburn, not red; and he was very polite, and I
had a delicious redowa with him."

"He looked like a grasshopper in a fit, when he did the new step. Laurie
and I couldn\textquotesingle t help laughing. Did you hear us?"

"No; but it was very rude. What \emph{were} you about all that time,
hidden away there?"

Jo told her adventures, and, by the time she had finished, they were at
home. With many thanks, they said "Good night," and crept in, hoping to
disturb no one; but the instant their door creaked, two little
night-caps bobbed up, and two sleepy but eager voices cried out,---

"Tell about the party! tell about the party!"

With what Meg called "a great want of manners," Jo had saved some
bonbons for the little girls; and they soon subsided, after hearing the
most thrilling events of the evening.

"I declare, it really seems like being a fine young lady, to come home
from the party in a carriage, and sit in my dressing-gown, with a maid
to wait on me," said Meg, as Jo bound up her foot with arnica, and
brushed her hair.

"I don\textquotesingle t believe fine young ladies enjoy themselves a
bit more than we do, in spite of our burnt hair, old gowns, one glove
apiece, and tight slippers that sprain our ankles when we are silly
enough to wear them." And I think Jo was quite right.

\protect\phantomsection\label{6672479776654687619_37106-h-0.htm.xhtml_b027.png}{}
\pandocbounded{\includegraphics[keepaspectratio]{303483661336987339_b027.png}}

\begin{center}\rule{0.5\linewidth}{0.5pt}\end{center}

\subsection{IV.
Burdens.}\label{6672479776654687619_37106-h-0.htm.xhtml_pgepubid00006}

\protect\phantomsection\label{6672479776654687619_37106-h-0.htm.xhtml_b028.png}{}
\pandocbounded{\includegraphics[keepaspectratio]{303483661336987339_b028.png}}

\protect\phantomsection\label{6672479776654687619_37106-h-0.htm.xhtml_IV}{}\hyperref[6672479776654687619_37106-h-0.htm.xhtml_contents]{IV.}

BURDENS.

"{Oh} dear, how hard it does seem to take up our packs and go on,"
sighed Meg, the morning after the party; for, now the holidays were
over, the week of merry-making did not fit her for going on easily with
the task she never liked.

"I wish it was Christmas or New-Year all the time;
wouldn\textquotesingle t it be fun?" answered Jo, yawning dismally.

"We shouldn\textquotesingle t enjoy ourselves half so much as we do now.
But it does seem so nice to have little suppers and bouquets, and go to
parties, and drive home, and read and rest, and not work.
It\textquotesingle s like other people, you know, and I always envy
girls who do such things; I\textquotesingle m so fond of luxury," said
Meg, trying to decide which of two shabby gowns was the least shabby.

"Well, we can\textquotesingle t have it, so don\textquotesingle t let us
grumble, but shoulder our bundles and trudge along as cheerfully as
Marmee does. I\textquotesingle m sure Aunt March is a regular Old Man of
the Sea to me, but I suppose when I\textquotesingle ve learned to carry
her without complaining, she will tumble off, or get so light that I
sha\textquotesingle n\textquotesingle t mind her."

This idea tickled Jo\textquotesingle s fancy, and put her in good
spirits; but Meg didn\textquotesingle t brighten, for her burden,
consisting of four spoilt children, seemed heavier than ever. She
hadn\textquotesingle t heart enough even to make herself pretty, as
usual, by putting on a blue neck-ribbon, and dressing her hair in the
most becoming way.

"Where\textquotesingle s the use of looking nice, when no one sees me
but those cross midgets, and no one cares whether I\textquotesingle m
pretty or not?" she muttered, shutting her drawer with a jerk. "I shall
have to toil and moil all my days, with only little bits of fun now and
then, and get old and ugly and sour, because I\textquotesingle m poor,
and can\textquotesingle t enjoy my life as other girls do.
It\textquotesingle s a shame!"

So Meg went down, wearing an injured look, and wasn\textquotesingle t at
all agreeable at breakfast-time. Every one seemed rather out of sorts,
and inclined to croak. Beth had a headache, and lay on the sofa, trying
to comfort herself with the cat and three kittens; Amy was fretting
because her lessons were not learned, and she couldn\textquotesingle t
find her rubbers; Jo \emph{would} whistle and make a great racket
getting ready; Mrs. March was very busy trying to finish a letter, which
must go at once; and Hannah had the grumps, for being up late
didn\textquotesingle t suit her.

"There never \emph{was} such a cross family!" cried Jo, losing her
temper when she had upset an inkstand, broken both boot-lacings, and sat
down upon her hat.

"You\textquotesingle re the crossest person in it!" returned Amy,
washing out the sum, that was all wrong, with the tears that had fallen
on her slate.

"Beth, if you don\textquotesingle t keep these horrid cats down cellar
I\textquotesingle ll have them drowned," exclaimed Meg angrily, as she
tried to get rid of the kitten, which had scrambled up her back, and
stuck like a burr just out of reach.

Jo laughed, Meg scolded, Beth implored, and Amy wailed, because she
couldn\textquotesingle t remember how much nine times twelve was.

"Girls, girls, do be quiet one minute! I \emph{must} get this off by the
early mail, and you drive me distracted with your worry," cried Mrs.
March, crossing out the third spoilt sentence in her letter.

There was a momentary lull, broken by Hannah, who stalked in, laid two
hot turn-overs on the table, and stalked out again. These turn-overs
were an institution; and the girls called them "muffs," for they had no
others, and found the hot pies very comforting to their hands on cold
mornings. Hannah never forgot to make them, no matter how busy or grumpy
she might be, for the walk was long and bleak; the poor things got no
other lunch, and were seldom home before two.

"Cuddle your cats, and get over your headache, Bethy. Good-by, Marmee;
we are a set of rascals this morning, but we\textquotesingle ll come
home regular angels. Now then, Meg!" and Jo tramped away, feeling that
the pilgrims were not setting out as they ought to do.

They always looked back before turning the corner, for their mother was
always at the window, to nod and smile, and wave her hand to them.
Somehow it seemed as if they couldn\textquotesingle t have got through
the day without that; for, whatever their mood might be, the last
glimpse of that motherly face was sure to affect them like sunshine.

"If Marmee shook her fist instead of kissing her hand to us, it would
serve us right, for more ungrateful wretches than we are were never
seen," cried Jo, taking a remorseful satisfaction in the snowy walk and
bitter wind.

"Don\textquotesingle t use such dreadful expressions," said Meg, from
the depths of the vail in which she had shrouded herself like a nun sick
of the world.

"I like good strong words, that mean something," replied Jo, catching
her hat as it took a leap off her head, preparatory to flying away
altogether.

"Call yourself any names you like; but \emph{I} am neither a rascal nor
a wretch, and I don\textquotesingle t choose to be called so."

"You\textquotesingle re a blighted being, and decidedly cross to-day
because you can\textquotesingle t sit in the lap of luxury all the time.
Poor dear, just wait till I make my fortune, and you shall revel in
carriages and ice-cream and high-heeled slippers and posies and
red-headed boys to dance with."

"How ridiculous you are, Jo!" but Meg laughed at the nonsense, and felt
better in spite of herself.

"Lucky for you I am; for if I put on crushed airs, and tried to be
dismal, as you do, we should be in a nice state. Thank goodness, I can
always find something funny to keep me up. Don\textquotesingle t croak
any more, but come home jolly, there\textquotesingle s a dear."

Jo gave her sister an encouraging pat on the shoulder as they parted for
the day, each going a different way, each hugging her little warm
turn-over, and each trying to be cheerful in spite of wintry weather,
hard work, and the unsatisfied desires of pleasure-loving youth.

When Mr. March lost his property in trying to help an unfortunate
friend, the two oldest girls begged to be allowed to do something toward
their own support, at least. Believing that they could not begin too
early to cultivate energy, industry, and independence, their parents
consented, and both fell to work with the hearty good-will which in
spite of all obstacles, is sure to succeed at last. Margaret found a
place as nursery governess, and felt rich with her small salary. As she
said, she \emph{was} "fond of luxury," and her chief trouble was
poverty. She found it harder to bear than the others, because she could
remember a time when home was beautiful, life full of ease and pleasure,
and want of any kind unknown. She tried not to be envious or
discontented, but it was very natural that the young girl should long
for pretty things, gay friends, accomplishments, and a happy life. At
the Kings\textquotesingle{} she daily saw all she wanted, for the
children\textquotesingle s older sisters were just out, and Meg caught
frequent glimpses of dainty ball-dresses and bouquets, heard lively
gossip about theatres, concerts, sleighing parties, and merry-makings of
all kinds, and saw money lavished on trifles which would have been so
precious to her. Poor Meg seldom complained, but a sense of injustice
made her feel bitter toward every one sometimes, for she had not yet
learned to know how rich she was in the blessings which alone can make
life happy.

Jo happened to suit Aunt March, who was lame, and needed an active
person to wait upon her. The childless old lady had offered to adopt one
of the girls when the troubles came, and was much offended because her
offer was declined. Other friends told the Marches that they had lost
all chance of being remembered in the rich old lady\textquotesingle s
will; but the unworldly Marches only said,---

"We can\textquotesingle t give up our girls for a dozen fortunes. Rich
or poor, we will keep together and be happy in one another."

The old lady wouldn\textquotesingle t speak to them for a time, but
happening to meet Jo at a friend\textquotesingle s, something in her
comical face and blunt manners struck the old lady\textquotesingle s
fancy, and she proposed to take her for a companion. This did not suit
Jo at all; but she accepted the place since nothing better appeared,
and, to every one\textquotesingle s surprise, got on remarkably well
with her irascible relative. There was an occasional tempest, and once
Jo had marched home, declaring she couldn\textquotesingle t bear it any
longer; but Aunt March always cleared up quickly, and sent for her back
again with such urgency that she could not refuse, for in her heart she
rather liked the peppery old lady.

I suspect that the real attraction was a large library of fine books,
which was left to dust and spiders since Uncle March died. Jo remembered
the kind old gentleman, who used to let her build railroads and bridges
with his big dictionaries, tell her stories about the queer pictures in
his Latin books, and buy her cards of gingerbread whenever he met her in
the street. The dim, dusty room, with the busts staring down from the
tall book-cases, the cosy chairs, the globes, and, best of all, the
wilderness of books, in which she could wander where she liked, made the
library a region of bliss to her. The moment Aunt March took her nap, or
was busy with company, Jo hurried to this quiet place, and, curling
herself up in the easy-chair, devoured poetry, romance, history,
travels, and pictures, like a regular book-worm. But, like all
happiness, it did not last long; for as sure as she had just reached the
heart of the story, the sweetest verse of the song, or the most perilous
adventure of her traveller, a shrill voice called, "Josy-phine!
Josy-phine!" and she had to leave her paradise to wind yarn, wash the
poodle, or read Belsham\textquotesingle s Essays by the hour together.

\protect\phantomsection\label{6672479776654687619_37106-h-0.htm.xhtml_b029.png}{}
\pandocbounded{\includegraphics[keepaspectratio]{303483661336987339_b029.png}}

Jo\textquotesingle s ambition was to do something very splendid; what it
was she had no idea, as yet, but left it for time to tell her; and,
meanwhile, found her greatest affliction in the fact that she
couldn\textquotesingle t read, run, and ride as much as she liked. A
quick temper, sharp tongue, and restless spirit were always getting her
into scrapes, and her life was a series of ups and downs, which were
both comic and pathetic. But the training she received at Aunt
March\textquotesingle s was just what she needed; and the thought that
she was doing something to support herself made her happy, in spite of
the perpetual "Josy-phine!"

Beth was too bashful to go to school; it had been tried, but she
suffered so much that it was given up, and she did her lessons at home,
with her father. Even when he went away, and her mother was called to
devote her skill and energy to Soldiers\textquotesingle{} Aid Societies,
Beth went faithfully on by herself, and did the best she could. She was
a housewifely little creature, and helped Hannah keep home neat and
comfortable for the workers, never thinking of any reward but to be
loved. Long, quiet days she spent, not lonely nor idle, for her little
world was peopled with imaginary friends, and she was by nature a busy
bee. There were six dolls to be taken up and dressed every morning, for
Beth was a child still, and loved her pets as well as ever. Not one
whole or handsome one among them; all were outcasts till Beth took them
in; for, when her sisters outgrew these idols, they passed to her,
because Amy would have nothing old or ugly. Beth cherished them all the
more tenderly for that very reason, and set up a hospital for infirm
dolls. No pins were ever stuck into their cotton vitals; no harsh words
or blows were ever given them; no neglect ever saddened the heart of the
most repulsive: but all were fed and clothed, nursed and caressed, with
an affection which never failed. One forlorn fragment of
\emph{dollanity} had belonged to Jo; and, having led a tempestuous life,
was left a wreck in the rag-bag, from which dreary poorhouse it was
rescued by Beth, and taken to her refuge. Having no top to its head, she
tied on a neat little cap, and, as both arms and legs were gone, she hid
these deficiencies by folding it in a blanket, and devoting her best bed
to this chronic invalid. If any one had known the care lavished on that
dolly, I think it would have touched their hearts, even while they
laughed. She brought it bits of bouquets; she read to it, took it out to
breathe the air, hidden under her coat; she sung it lullabys, and never
went to bed without kissing its dirty face, and whispering tenderly, "I
hope you\textquotesingle ll have a good night, my poor dear."

Beth had her troubles as well as the others; and not being an angel, but
a very human little girl, she often "wept a little weep," as Jo said,
because she couldn\textquotesingle t take music lessons and have a fine
piano. She loved music so dearly, tried so hard to learn, and practised
away so patiently at the jingling old instrument, that it did seem as if
some one (not to hint Aunt March) ought to help her. Nobody did,
however, and nobody saw Beth wipe the tears off the yellow keys, that
wouldn\textquotesingle t keep in tune, when she was all alone. She sang
like a little lark about her work, never was too tired to play for
Marmee and the girls, and day after day said hopefully to herself, "I
know I\textquotesingle ll get my music some time, if I\textquotesingle m
good."

There are many Beths in the world, shy and quiet, sitting in corners
till needed, and living for others so cheerfully that no one sees the
sacrifices till the little cricket on the hearth stops chirping, and the
sweet, sunshiny presence vanishes, leaving silence and shadow behind.

If anybody had asked Amy what the greatest trial of her life was, she
would have answered at once, "My nose." When she was a baby, Jo had
accidentally dropped her into the coal-hod, and Amy insisted that the
fall had ruined her nose forever. It was not big, nor red, like poor
"Petrea\textquotesingle s"; it was only rather flat, and all the
pinching in the world could not give it an aristocratic point. No one
minded it but herself, and it was doing its best to grow, but Amy felt
deeply the want of a Grecian nose, and drew whole sheets of handsome
ones to console herself.

"Little Raphael," as her sisters called her, had a decided talent for
drawing, and was never so happy as when copying flowers, designing
fairies, or illustrating stories with queer specimens of art. Her
teachers complained that, instead of doing her sums, she covered her
slate with animals; the blank pages of her atlas were used to copy maps
on; and caricatures of the most ludicrous description came fluttering
out of all her books at unlucky moments. She got through her lessons as
well as she could, and managed to escape reprimands by being a model of
deportment. She was a great favorite with her mates, being
good-tempered, and possessing the happy art of pleasing without effort.
Her little airs and graces were much admired, so were her
accomplishments; for beside her drawing, she could play twelve tunes,
crochet, and read French without mispronouncing more than two thirds of
the words. She had a plaintive way of saying, "When papa was rich we did
so-and-so," which was very touching; and her long words were considered
"perfectly elegant" by the girls.

Amy was in a fair way to be spoilt; for every one petted her, and her
small vanities and selfishnesses were growing nicely. One thing,
however, rather quenched the vanities; she had to wear her
cousin\textquotesingle s clothes. Now Florence\textquotesingle s mamma
hadn\textquotesingle t a particle of taste, and Amy suffered deeply at
having to wear a red instead of a blue bonnet, unbecoming gowns, and
fussy aprons that did not fit. Everything was good, well made, and
little worn; but Amy\textquotesingle s artistic eyes were much
afflicted, especially this winter, when her school dress was a dull
purple, with yellow dots, and no trimming.

"My only comfort," she said to Meg, with tears in her eyes, "is, that
mother don\textquotesingle t take tucks in my dresses whenever
I\textquotesingle m naughty, as Maria Parks\textquotesingle{} mother
does. My dear, it\textquotesingle s really dreadful; for sometimes she
is so bad, her frock is up to her knees, and she can\textquotesingle t
come to school. When I think of this \emph{deggerredation}, I feel that
I can bear even my flat nose and purple gown, with yellow sky-rockets on
it."

Meg was Amy\textquotesingle s confidant and monitor, and, by some
strange attraction of opposites, Jo was gentle Beth\textquotesingle s.
To Jo alone did the shy child tell her thoughts; and over her big,
harum-scarum sister, Beth unconsciously exercised more influence than
any one in the family. The two older girls were a great deal to one
another, but each took one of the younger into her keeping, and watched
over her in her own way; "playing mother" they called it, and put their
sisters in the places of discarded dolls, with the maternal instinct of
little women.

"Has anybody got anything to tell? It\textquotesingle s been such a
dismal day I\textquotesingle m really dying for some amusement," said
Meg, as they sat sewing together that evening.

"I had a queer time with aunt to-day, and, as I got the best of it,
I\textquotesingle ll tell you about it," began Jo, who dearly loved to
tell stories. "I was reading that everlasting Belsham, and droning away
as I always do, for aunt soon drops off, and then I take out some nice
book, and read like fury till she wakes up. I actually made myself
sleepy; and, before she began to nod, I gave such a gape that she asked
me what I meant by opening my mouth wide enough to take the whole book
in at once.

\protect\phantomsection\label{6672479776654687619_37106-h-1.htm.xhtml}{}

\protect\phantomsection\label{6672479776654687619_37106-h-1.htm.xhtml_b030.png}{}
\pandocbounded{\includegraphics[keepaspectratio]{303483661336987339_b030.png}}

"\textquotesingle I wish I could, and be done with it,\textquotesingle{}
said I, trying not to be saucy.

"Then she gave me a long lecture on my sins, and told me to sit and
think them over while she just \textquotesingle lost\textquotesingle{}
herself for a moment. She never finds herself very soon; so the minute
her cap began to bob, like a top-heavy dahlia, I whipped the
\textquotesingle Vicar of Wakefield\textquotesingle{} out of my pocket,
and read away, with one eye on him, and one on aunt. I\textquotesingle d
just got to where they all tumbled into the water, when I forgot, and
laughed out loud. Aunt woke up; and, being more good-natured after her
nap, told me to read a bit, and show what frivolous work I preferred to
the worthy and instructive Belsham. I did my very best, and she liked
it, though she only said,---

"\textquotesingle I don\textquotesingle t understand what
it\textquotesingle s all about. Go back and begin it,
child.\textquotesingle{}

"Back I went, and made the Primroses as interesting as ever I could.
Once I was wicked enough to stop in a thrilling place, and say meekly,
\textquotesingle I\textquotesingle m afraid it tires you,
ma\textquotesingle am; sha\textquotesingle n\textquotesingle t I stop
now?\textquotesingle{}

"She caught up her knitting, which had dropped out of her hands, gave me
a sharp look through her specs, and said, in her short way,---

"\textquotesingle Finish the chapter, and don\textquotesingle t be
impertinent, miss.\textquotesingle"

"Did she own she liked it?" asked Meg.

"Oh, bless you, no! but she let old Belsham rest; and, when I ran back
after my gloves this afternoon, there she was, so hard at the Vicar that
she didn\textquotesingle t hear me laugh as I danced a jig in the hall,
because of the good time coming. What a pleasant life she might have, if
she only chose. I don\textquotesingle t envy her much, in spite of her
money, for after all rich people have about as many worries as poor
ones, I think," added Jo.

"That reminds me," said Meg, "that I\textquotesingle ve got something to
tell. It isn\textquotesingle t funny, like Jo\textquotesingle s story,
but I thought about it a good deal as I came home. At the Kings to-day I
found everybody in a flurry, and one of the children said that her
oldest brother had done something dreadful, and papa had sent him away.
I heard Mrs. King crying and Mr. King talking very loud, and Grace and
Ellen turned away their faces when they passed me, so I
shouldn\textquotesingle t see how red their eyes were. I
didn\textquotesingle t ask any questions, of course; but I felt so sorry
for them, and was rather glad I hadn\textquotesingle t any wild brothers
to do wicked things and disgrace the family."

"I think being disgraced in school is a great deal try\emph{inger} than
anything bad boys can do," said Amy, shaking her head, as if her
experience of life had been a deep one. "Susie Perkins came to school
to-day with a lovely red carnelian ring; I wanted it dreadfully, and
wished I was her with all my might. Well, she drew a picture of Mr.
Davis, with a monstrous nose and a hump, and the words,
\textquotesingle Young ladies, my eye is upon you!\textquotesingle{}
coming out of his mouth in a balloon thing. We were laughing over it,
when all of a sudden his eye \emph{was} on us, and he ordered Susie to
bring up her slate. She was \emph{parry}lized with fright, but she went,
and oh, what \emph{do} you think he did? He took her by the ear, the
ear! just fancy how horrid!---and led her to the recitation platform,
and made her stand there half an hour, holding that slate so every one
could see."

\protect\phantomsection\label{6672479776654687619_37106-h-1.htm.xhtml_b031.png}{}
\pandocbounded{\includegraphics[keepaspectratio]{303483661336987339_b031.png}}

"Didn\textquotesingle t the girls laugh at the picture?" asked Jo, who
relished the scrape.

"Laugh? Not one! They sat as still as mice; and Susie cried quarts, I
know she did. I didn\textquotesingle t envy her then; for I felt that
millions of carnelian rings wouldn\textquotesingle t have made me happy,
after that. I never, never should have got over such a agonizing
mortification." And Amy went on with her work, in the proud
consciousness of virtue, and the successful utterance of two long words
in a breath.

"I saw something that I liked this morning, and I meant to tell it at
dinner, but I forgot," said Beth, putting Jo\textquotesingle s
topsy-turvy basket in order as she talked. "When I went to get some
oysters for Hannah, Mr. Laurence was in the fish-shop; but he
didn\textquotesingle t see me, for I kept behind a barrel, and he was
busy with Mr. Cutter, the fish-man. A poor woman came in, with a pail
and a mop, and asked Mr. Cutter if he would let her do some scrubbing
for a bit of fish, because she hadn\textquotesingle t any dinner for her
children, and had been disappointed of a day\textquotesingle s work. Mr.
Cutter was in a hurry, and said \textquotesingle No,\textquotesingle{}
rather crossly; so she was going away, looking hungry and sorry, when
Mr. Laurence hooked up a big fish with the crooked end of his cane, and
held it out to her. She was so glad and surprised, she took it right in
her arms, and thanked him over and over. He told her to
\textquotesingle go along and cook it,\textquotesingle{} and she hurried
off, so happy! Wasn\textquotesingle t it good of him? Oh, she did look
so funny, hugging the big, slippery fish, and hoping Mr.
Laurence\textquotesingle s bed in heaven would be
\textquotesingle aisy.\textquotesingle"

\protect\phantomsection\label{6672479776654687619_37106-h-1.htm.xhtml_b032.png}{}
\pandocbounded{\includegraphics[keepaspectratio]{303483661336987339_b032.png}}

When they had laughed at Beth\textquotesingle s story, they asked their
mother for one; and, after a moment\textquotesingle s thought, she said
soberly,---

"As I sat cutting out blue flannel jackets to-day, at the rooms, I felt
very anxious about father, and thought how lonely and helpless we should
be, if anything happened to him. It was not a wise thing to do; but I
kept on worrying, till an old man came in, with an order for some
clothes. He sat down near me, and I began to talk to him; for he looked
poor and tired and anxious.

"\textquotesingle Have you sons in the army?\textquotesingle{} I asked;
for the note he brought was not to me.

"\textquotesingle Yes, ma\textquotesingle am. I had four, but two were
killed, one is a prisoner, and I\textquotesingle m going to the other,
who is very sick in a Washington hospital,\textquotesingle{} he answered
quietly.

"\textquotesingle You have done a great deal for your country,
sir,\textquotesingle{} I said, feeling respect now, instead of pity.

"\textquotesingle Not a mite more than I ought, ma\textquotesingle am.
I\textquotesingle d go myself, if I was any use; as I
ain\textquotesingle t, I give my boys, and give \textquotesingle em
free.\textquotesingle{}

"He spoke so cheerfully, looked so sincere, and seemed so glad to give
his all, that I was ashamed of myself. I\textquotesingle d given one
man, and thought it too much, while he gave four, without grudging them.
I had all my girls to comfort me at home; and his last son was waiting,
miles away, to say \textquotesingle good by\textquotesingle{} to him,
perhaps! I felt so rich, so happy, thinking of my blessings, that I made
him a nice bundle, gave him some money, and thanked him heartily for the
lesson he had taught me."

"Tell another story, mother,---one with a moral to it, like this. I like
to think about them afterwards, if they are real, and not too preachy,"
said Jo, after a minute\textquotesingle s silence.

Mrs. March smiled, and began at once; for she had told stories to this
little audience for many years, and knew how to please them.

"Once upon a time, there were four girls, who had enough to eat and
drink and wear, a good many comforts and pleasures, kind friends and
parents, who loved them dearly, and yet they were not contented." (Here
the listeners stole sly looks at one another, and began to sew
diligently.) "These girls were anxious to be good, and made many
excellent resolutions; but they did not keep them very well, and were
constantly saying, \textquotesingle If we only had
this,\textquotesingle{} or \textquotesingle If we could only do
that,\textquotesingle{} quite forgetting how much they already had, and
how many pleasant things they actually could do. So they asked an old
woman what spell they could use to make them happy, and she said,
\textquotesingle When you feel discontented, think over your blessings,
and be grateful.\textquotesingle" (Here Jo looked up quickly, as if
about to speak, but changed her mind, seeing that the story was not done
yet.)

"Being sensible girls, they decided to try her advice, and soon were
surprised to see how well off they were. One discovered that money
couldn\textquotesingle t keep shame and sorrow out of rich
people\textquotesingle s houses; another that, though she was poor, she
was a great deal happier, with her youth, health, and good spirits, than
a certain fretful, feeble old lady, who couldn\textquotesingle t enjoy
her comforts; a third that, disagreeable as it was to help get dinner,
it was harder still to have to go begging for it; and the fourth, that
even carnelian rings were not so valuable as good behavior. So they
agreed to stop complaining, to enjoy the blessings already possessed,
and try to deserve them, lest they should be taken away entirely,
instead of increased; and I believe they were never disappointed, or
sorry that they took the old woman\textquotesingle s advice."

"Now, Marmee, that is very cunning of you to turn our own stories
against us, and give us a sermon instead of a romance!" cried Meg.

"I like that kind of sermon. It\textquotesingle s the sort father used
to tell us," said Beth thoughtfully, putting the needles straight on
Jo\textquotesingle s cushion.

"I don\textquotesingle t complain near as much as the others do, and I
shall be more careful than ever now; for I\textquotesingle ve had
warning from Susie\textquotesingle s downfall," said Amy morally.

"We needed that lesson, and we won\textquotesingle t forget it. If we
do, you just say to us, as old Chloe did in \textquotesingle Uncle
Tom,\textquotesingle{} \textquotesingle Tink ob yer marcies, chillen!
tink ob yer marcies!\textquotesingle" added Jo, who could not, for the
life of her, help getting a morsel of fun out of the little sermon,
though she took it to heart as much as any of them.

\protect\phantomsection\label{6672479776654687619_37106-h-1.htm.xhtml_b033.png}{}
\pandocbounded{\includegraphics[keepaspectratio]{303483661336987339_b033.png}}

\begin{center}\rule{0.5\linewidth}{0.5pt}\end{center}

\subsection{V. Being
Neighborly.}\label{6672479776654687619_37106-h-1.htm.xhtml_pgepubid00007}

\protect\phantomsection\label{6672479776654687619_37106-h-1.htm.xhtml_V}{}\hyperref[6672479776654687619_37106-h-0.htm.xhtml_contents]{V.}

BEING NEIGHBORLY.

\protect\phantomsection\label{6672479776654687619_37106-h-1.htm.xhtml_b034.png}{}
\pandocbounded{\includegraphics[keepaspectratio]{303483661336987339_b034.png}}

{"What} in the world are you going to do now, Jo?" asked Meg, one snowy
afternoon, as her sister came tramping through the hall, in rubber
boots, old sack and hood, with a broom in one hand and a shovel in the
other.

"Going out for exercise," answered Jo, with a mischievous twinkle in her
eyes.

"I should think two long walks this morning would have been enough!
It\textquotesingle s cold and dull out; and I advise you to stay, warm
and dry, by the fire, as I do," said Meg, with a shiver.

"Never take advice! Can\textquotesingle t keep still all day, and, not
being a pussycat, I don\textquotesingle t like to doze by the fire. I
like adventures, and I\textquotesingle m going to find some."

Meg went back to toast her feet and read "Ivanhoe"; and Jo began to dig
paths with great energy. The snow was light, and with her broom she soon
swept a path all round the garden, for Beth to walk in when the sun came
out; and the invalid dolls needed air. Now, the garden separated the
Marches\textquotesingle{} house from that of Mr. Laurence. Both stood in
a suburb of the city, which was still country-like, with groves and
lawns, large gardens, and quiet streets. A low hedge parted the two
estates. On one side was an old, brown house, looking rather bare and
shabby, robbed of the vines that in summer covered its walls, and the
flowers which then surrounded it. On the other side was a stately stone
mansion, plainly betokening every sort of comfort and luxury, from the
big coach-house and well-kept grounds to the conservatory and the
glimpses of lovely things one caught between the rich curtains. Yet it
seemed a lonely, lifeless sort of house; for no children frolicked on
the lawn, no motherly face ever smiled at the windows, and few people
went in and out, except the old gentleman and his grandson.

To Jo\textquotesingle s lively fancy, this fine house seemed a kind of
enchanted palace, full of splendors and delights, which no one enjoyed.
She had long wanted to behold these hidden glories, and to know the
"Laurence boy," who looked as if he would like to be known, if he only
knew how to begin. Since the party, she had been more eager than ever,
and had planned many ways of making friends with him; but he had not
been seen lately, and Jo began to think he had gone away, when she one
day spied a brown face at an upper window, looking wistfully down into
their garden, where Beth and Amy were snow-balling one another.

"That boy is suffering for society and fun," she said to herself. "His
grandpa does not know what\textquotesingle s good for him, and keeps him
shut up all alone. He needs a party of jolly boys to play with, or
somebody young and lively. I\textquotesingle ve a great mind to go over
and tell the old gentleman so!"

The idea amused Jo, who liked to do daring things, and was always
scandalizing Meg by her queer performances. The plan of "going over" was
not forgotten; and when the snowy afternoon came, Jo resolved to try
what could be done. She saw Mr. Laurence drive off, and then sallied out
to dig her way down to the hedge, where she paused, and took a survey.
All quiet,---curtains down at the lower windows; servants out of sight,
and nothing human visible but a curly black head leaning on a thin hand
at the upper window.

"There he is," thought Jo, "poor boy! all alone and sick this dismal
day. It\textquotesingle s a shame! I\textquotesingle ll toss up a
snow-ball, and make him look out, and then say a kind word to him."

Up went a handful of soft snow, and the head turned at once, showing a
face which lost its listless look in a minute, as the big eyes
brightened and the mouth began to smile. Jo nodded and laughed, and
flourished her broom as she called out,---

"How do you do? Are you sick?"

\protect\phantomsection\label{6672479776654687619_37106-h-1.htm.xhtml_b035.png}{}
\pandocbounded{\includegraphics[keepaspectratio]{303483661336987339_b035.png}}

Laurie opened the window, and croaked out as hoarsely as a raven,---

"Better, thank you. I\textquotesingle ve had a bad cold, and been shut
up a week."

"I\textquotesingle m sorry. What do you amuse yourself with?"

"Nothing; it\textquotesingle s as dull as tombs up here."

"Don\textquotesingle t you read?"

"Not much; they won\textquotesingle t let me."

"Can\textquotesingle t somebody read to you?"

"Grandpa does, sometimes; but my books don\textquotesingle t interest
him, and I hate to ask Brooke all the time."

"Have some one come and see you, then."

"There isn\textquotesingle t any one I\textquotesingle d like to see.
Boys make such a row, and my head is weak."

"Isn\textquotesingle t there some nice girl who\textquotesingle d read
and amuse you? Girls are quiet, and like to play nurse."

"Don\textquotesingle t know any."

"You know us," began Jo, then laughed, and stopped.

"So I do! Will you come, please?" cried Laurie.

"I\textquotesingle m not quiet and nice; but I\textquotesingle ll come,
if mother will let me. I\textquotesingle ll go ask her. Shut that
window, like a good boy, and wait till I come."

With that, Jo shouldered her broom and marched into the house, wondering
what they would all say to her. Laurie was in a flutter of excitement at
the idea of having company, and flew about to get ready; for, as Mrs.
March said, he was "a little gentleman," and did honor to the coming
guest by brushing his curly pate, putting on a fresh collar, and trying
to tidy up the room, which, in spite of half a dozen servants, was
anything but neat. Presently there came a loud ring, then a decided
voice, asking for "Mr. Laurie," and a surprised-looking servant came
running up to announce a young lady.

"All right, show her up, it\textquotesingle s Miss Jo," said Laurie,
going to the door of his little parlor to meet Jo, who appeared, looking
rosy and kind and quite at her ease, with a covered dish in one hand and
Beth\textquotesingle s three kittens in the other.

"Here I am, bag and baggage," she said briskly. "Mother sent her love,
and was glad if I could do anything for you. Meg wanted me to bring some
of her blanc-mange; she makes it very nicely, and Beth thought her cats
would be comforting. I knew you\textquotesingle d laugh at them, but I
couldn\textquotesingle t refuse, she was so anxious to do something."

It so happened that Beth\textquotesingle s funny loan was just the
thing; for, in laughing over the kits, Laurie forgot his bashfulness,
and grew sociable at once.

"That looks too pretty to eat," he said, smiling with pleasure, as Jo
uncovered the dish, and showed the blanc-mange, surrounded by a garland
of green leaves, and the scarlet flowers of Amy\textquotesingle s pet
geranium.

"It isn\textquotesingle t anything, only they all felt kindly, and
wanted to show it. Tell the girl to put it away for your tea:
it\textquotesingle s so simple, you can eat it; and, being soft, it will
slip down without hurting your sore throat. What a cosy room this is!"

"It might be if it was kept nice; but the maids are lazy, and I
don\textquotesingle t know how to make them mind. It worries me,
though."

"I\textquotesingle ll right it up in two minutes; for it only needs to
have the hearth brushed, so,---and the things made straight on the
mantel-piece so,---and the books put here, and the bottles there, and
your sofa turned from the light, and the pillows plumped up a bit. Now,
then, you\textquotesingle re fixed."

And so he was; for, as she laughed and talked, Jo had whisked things
into place, and given quite a different air to the room. Laurie watched
her in respectful silence; and when she beckoned him to his sofa, he sat
down with a sigh of satisfaction, saying gratefully,---

"How kind you are! Yes, that\textquotesingle s what it wanted. Now
please take the big chair, and let me do something to amuse my company."

"No; I came to amuse you. Shall I read aloud?" and Jo looked
affectionately toward some inviting books near by.

"Thank you; I\textquotesingle ve read all those, and if you
don\textquotesingle t mind, I\textquotesingle d rather talk," answered
Laurie.

"Not a bit; I\textquotesingle ll talk all day if you\textquotesingle ll
only set me going. Beth says I never know when to stop."

"Is Beth the rosy one, who stays at home a good deal, and sometimes goes
out with a little basket?" asked Laurie, with interest.

"Yes, that\textquotesingle s Beth; she\textquotesingle s my girl, and a
regular good one she is, too."

"The pretty one is Meg, and the curly-haired one is Amy, I believe?"

"How did you find that out?"

Laurie colored up, but answered frankly, "Why, you see, I often hear you
calling to one another, and when I\textquotesingle m alone up here, I
can\textquotesingle t help looking over at your house, you always seem
to be having such good times. I beg your pardon for being so rude, but
sometimes you forget to put down the curtain at the window where the
flowers are; and when the lamps are lighted, it\textquotesingle s like
looking at a picture to see the fire, and you all round the table with
your mother; her face is right opposite, and it looks so sweet behind
the flowers, I can\textquotesingle t help watching it. I
haven\textquotesingle t got any mother, you know;" and Laurie poked the
fire to hide a little twitching of the lips that he could not control.

The solitary, hungry look in his eyes went straight to
Jo\textquotesingle s warm heart. She had been so simply taught that
there was no nonsense in her head, and at fifteen she was as innocent
and frank as any child. Laurie was sick and lonely; and, feeling how
rich she was in home-love and happiness, she gladly tried to share it
with him. Her face was very friendly and her sharp voice unusually
gentle as she said,---

"We\textquotesingle ll never draw that curtain any more, and I give you
leave to look as much as you like. I just wish, though, instead of
peeping, you\textquotesingle d come over and see us. Mother is so
splendid, she\textquotesingle d do you heaps of good, and Beth would
sing to you if \emph{I} begged her to, and Amy would dance; Meg and I
would make you laugh over our funny stage properties, and
we\textquotesingle d have jolly times. Wouldn\textquotesingle t your
grandpa let you?"

"I think he would, if your mother asked him. He\textquotesingle s very
kind, though he does not look so; and he lets me do what I like, pretty
much, only he\textquotesingle s afraid I might be a bother to
strangers," began Laurie, brightening more and more.

"We are not strangers, we are neighbors, and you needn\textquotesingle t
think you\textquotesingle d be a bother. We \emph{want} to know you, and
I\textquotesingle ve been trying to do it this ever so long. We
haven\textquotesingle t been here a great while, you know, but we have
got acquainted with all our neighbors but you."

"You see grandpa lives among his books, and doesn\textquotesingle t mind
much what happens outside. Mr. Brooke, my tutor, doesn\textquotesingle t
stay here, you know, and I have no one to go about with me, so I just
stop at home and get on as I can."

"That\textquotesingle s bad. You ought to make an effort, and go
visiting everywhere you are asked; then you\textquotesingle ll have
plenty of friends, and pleasant places to go to. Never mind being
bashful; it won\textquotesingle t last long if you keep going."

Laurie turned red again, but wasn\textquotesingle t offended at being
accused of bashfulness; for there was so much good-will in Jo, it was
impossible not to take her blunt speeches as kindly as they were meant.

"Do you like your school?" asked the boy, changing the subject, after a
little pause, during which he stared at the fire, and Jo looked about
her, well pleased.

"Don\textquotesingle t go to school; I\textquotesingle m a business
man---girl, I mean. I go to wait on my great-aunt, and a dear, cross old
soul she is, too," answered Jo.

Laurie opened his mouth to ask another question; but remembering just in
time that it wasn\textquotesingle t manners to make too many inquiries
into people\textquotesingle s affairs, he shut it again, and looked
uncomfortable. Jo liked his good breeding, and didn\textquotesingle t
mind having a laugh at Aunt March, so she gave him a lively description
of the fidgety old lady, her fat poodle, the parrot that talked Spanish,
and the library where she revelled. Laurie enjoyed that immensely; and
when she told about the prim old gentleman who came once to woo Aunt
March, and, in the middle of a fine speech, how Poll had tweaked his wig
off to his great dismay, the boy lay back and laughed till the tears ran
down his cheeks, and a maid popped her head in to see what was the
matter.

\protect\phantomsection\label{6672479776654687619_37106-h-1.htm.xhtml_b036.png}{}
\pandocbounded{\includegraphics[keepaspectratio]{303483661336987339_b036.png}}

"Oh! that does me no end of good. Tell on, please," he said, taking his
face out of the sofa-cushion, red and \ul{shining with merriment.}

Much elated with her success, Jo did "tell on," all about their plays
and plans, their hopes and fears for father, and the most interesting
events of the little world in which the sisters lived. Then they got to
talking about books; and to Jo\textquotesingle s delight, she found that
Laurie loved them as well as she did, and had read even more than
herself.

"If you like them so much, come down and see ours. Grandpa is out, so
you needn\textquotesingle t be afraid," said Laurie, getting up.

"I\textquotesingle m not afraid of anything," returned Jo, with a toss
of the head.

"I don\textquotesingle t believe you are!" exclaimed the boy, looking at
her with much admiration, though he privately thought she would have
good reason to be a trifle afraid of the old gentleman, if she met him
in some of his moods.

The atmosphere of the whole house being summer-like, Laurie led the way
from room to room, letting Jo stop to examine whatever struck her fancy;
and so at last they came to the library, where she clapped her hands,
and pranced, as she always did when especially delighted. It was lined
with books, and there were pictures and statues, and distracting little
cabinets full of coins and curiosities, and sleepy-hollow chairs, and
queer tables, and bronzes; and, best of all, a great open fireplace,
with quaint tiles all round it.

"What richness!" sighed Jo, sinking into the depth of a velvet chair,
and gazing about her with an air of intense satisfaction. "Theodore
Laurence, you ought to be the happiest boy in the world," she added
impressively.

"A fellow can\textquotesingle t live on books," said Laurie, shaking his
head, as he perched on a table opposite.

Before he could say more, a bell rung, and Jo flew up, exclaiming with
alarm, "Mercy me! it\textquotesingle s your grandpa!"

"Well, what if it is? You are not afraid of anything, you know,"
returned the boy, looking wicked.

"I think I am a little bit afraid of him, but I don\textquotesingle t
know why I should be. Marmee said I might come, and I
don\textquotesingle t think you\textquotesingle re any the worse for
it," said Jo, composing herself, though she kept her eyes on the door.

"I\textquotesingle m a great deal better for it, and ever so much
obliged. I\textquotesingle m only afraid you are very tired talking to
me; it was \emph{so} pleasant, I couldn\textquotesingle t bear to stop,"
said Laurie gratefully.

"The doctor to see you, sir," and the maid beckoned as she spoke.

"Would you mind if I left you for a minute? I suppose I must see him,"
said Laurie.

"Don\textquotesingle t mind me. I\textquotesingle m as happy as a
cricket here," answered Jo.

Laurie went away, and his guest amused herself in her own way. She was
standing before a fine portrait of the old gentleman, when the door
opened again, and, without turning, she said decidedly,
"I\textquotesingle m sure now that I shouldn\textquotesingle t be afraid
of him, for he\textquotesingle s got kind eyes, though his mouth is
grim, and he looks as if he had a tremendous will of his own. He
isn\textquotesingle t as handsome as \emph{my} grandfather, but I like
him."

"Thank you, ma\textquotesingle am," said a gruff voice behind her; and
there, to her great dismay, stood old Mr. Laurence.

Poor Jo blushed till she couldn\textquotesingle t blush any redder, and
her heart began to beat uncomfortably fast as she thought what she had
said. For a minute a wild desire to run away possessed her; but that was
cowardly, and the girls would laugh at her: so she resolved to stay, and
get out of the scrape as she could. A second look showed her that the
living eyes, under the bushy gray eyebrows, were kinder even than the
painted ones; and there was a sly twinkle in them, which lessened her
fear a good deal. The gruff voice was gruffer than ever, as the old
gentleman said abruptly, after that dreadful pause, "So
you\textquotesingle re not afraid of me, hey?"

"Not much, sir."

"And you don\textquotesingle t think me as handsome as your
grandfather?"

"Not quite, sir."

"And I\textquotesingle ve got a tremendous will, have I?"

"I only said I thought so."

"But you like me, in spite of it?"

"Yes, I do, sir."

That answer pleased the old gentleman; he gave a short laugh, shook
hands with her, and, putting his finger under her chin, turned up her
face, examined it gravely, and let it go, saying, with a nod,
"You\textquotesingle ve got your grandfather\textquotesingle s spirit,
if you haven\textquotesingle t his face. He \emph{was} a fine man, my
dear; but, what is better, he was a brave and an honest one, and I was
proud to be his friend."

\protect\phantomsection\label{6672479776654687619_37106-h-1.htm.xhtml_b037.png}{}
\pandocbounded{\includegraphics[keepaspectratio]{303483661336987339_b037.png}}

"Thank you, sir;" and Jo was quite comfortable after that, for it suited
her exactly.

"What have you been doing to this boy of mine, hey?" was the next
question, sharply put.

"Only trying to be neighborly, sir;" and Jo told how her visit came
about.

"You think he needs cheering up a bit, do you?"

"Yes, sir; he seems a little lonely, and young folks would do him good
perhaps. We are only girls, but we should be glad to help if we could,
for we don\textquotesingle t forget the splendid Christmas present you
sent us," said Jo eagerly.

"Tut, tut, tut! that was the boy\textquotesingle s affair. How is the
poor woman?"

"Doing nicely, sir;" and off went Jo, talking very fast, as she told all
about the Hummels, in whom her mother had interested richer friends than
they were.

"Just her father\textquotesingle s way of doing good. I shall come and
see your mother some fine day. Tell her so. There\textquotesingle s the
tea-bell; we have it early, on the boy\textquotesingle s account. Come
down, and go on being neighborly."

"If you\textquotesingle d like to have me, sir."

"Shouldn\textquotesingle t ask you, if I didn\textquotesingle t;" and
Mr. Laurence offered her his arm with old-fashioned courtesy.

"What \emph{would} Meg say to this?" thought Jo, as she was marched
away, while her eyes danced with fun as she imagined herself telling the
story at home.

"Hey! Why, what the dickens has come to the fellow?" said the old
gentleman, as Laurie came running down stairs, and brought up with a
start of surprise at the astonishing sight of Jo arm-in-arm with his
redoubtable grandfather.

"I didn\textquotesingle t know you\textquotesingle d come, sir," he
began, as Jo gave him a triumphant little glance.

"That\textquotesingle s evident, by the way you racket down stairs. Come
to your tea, sir, and behave like a gentleman;" and having pulled the
boy\textquotesingle s hair by way of a caress, Mr. Laurence walked on,
while Laurie went through a series of comic evolutions behind their
backs, which nearly produced an explosion of laughter from Jo.

The old gentleman did not say much as he drank his four cups of tea, but
he watched the young people, who soon chatted away like old friends, and
the change in his grandson did not escape him. There was color, light,
and life in the boy\textquotesingle s face now, vivacity in his manner,
and genuine merriment in his laugh.

"She\textquotesingle s right; the lad \emph{is} lonely.
I\textquotesingle ll see what these little girls can do for him,"
thought Mr. Laurence, as he looked and listened. He liked Jo, for her
odd, blunt ways suited him; and she seemed to understand the boy almost
as well as if she had been one herself.

If the Laurences had been what Jo called "prim and poky," she would not
have got on at all, for such people always made her shy and awkward; but
finding them free and easy, she was so herself, and made a good
impression. When they rose she proposed to go, but Laurie said he had
something more to show her, and took her away to the conservatory, which
had been lighted for her benefit. It seemed quite fairylike to Jo, as
she went up and down the walks, enjoying the blooming walls on either
side, the soft light, the damp \ul{sweet air, and the wonderful vines}
and trees that hung above her,---while her new friend cut the finest
flowers till his hands were full; then he tied them up, saying, with the
happy look Jo liked to see, "Please give these to your mother, and tell
her I like the medicine she sent me very much."

\protect\phantomsection\label{6672479776654687619_37106-h-1.htm.xhtml_b038.png}{}
\pandocbounded{\includegraphics[keepaspectratio]{303483661336987339_b038.png}}

They found Mr. Laurence standing before the fire in the great
drawing-room, but Jo\textquotesingle s attention was entirely absorbed
by a grand piano, which stood open.

"Do you play?" she asked, turning to Laurie with a respectful
expression.

"Sometimes," he answered modestly.

"Please do now. I want to hear it, so I can tell Beth."

"Won\textquotesingle t you first?"

"Don\textquotesingle t know how; too stupid to learn, but I love music
dearly."

So Laurie played, and Jo listened, with her nose luxuriously buried in
heliotrope and tea-roses. Her respect and regard for the "Laurence boy"
increased very much, for he played remarkably well, and
didn\textquotesingle t put on any airs. She wished Beth could hear him,
but she did not say so; only praised him till he was quite abashed, and
his grandfather came to the rescue. "That will do, that will do, young
lady. Too many sugar-plums are not good for him. His music
isn\textquotesingle t bad, but I hope he will do as well in more
important things. Going? Well, I\textquotesingle m much obliged to you,
and I hope you\textquotesingle ll come again. My respects to your
mother. Good-night, Doctor Jo."

He shook hands kindly, but looked as if something did not please him.
When they got into the hall, Jo asked Laurie if she had said anything
amiss. He shook his head.

"No, it was me; he doesn\textquotesingle t like to hear me play."

"Why not?"

"I\textquotesingle ll tell you some day. John is going home with you, as
I can\textquotesingle t."

"No need of that; I am not a young lady, and it\textquotesingle s only a
step. Take care of yourself, won\textquotesingle t you?"

"Yes; but you will come again, I hope?"

"If you promise to come and see us after you are well."

"I will."

"Good-night, Laurie!"

"Good-night, Jo, good-night!"

When all the afternoon\textquotesingle s adventures had been told, the
family felt inclined to go visiting in a body, for each found something
very attractive in the big house on the other side of the hedge. Mrs.
March wanted to talk of her father with the old man who had not
forgotten him; Meg longed to walk in the conservatory; Beth sighed for
the grand piano; and Amy was eager to see the fine pictures and statues.

"Mother, why didn\textquotesingle t Mr. Laurence like to have Laurie
play?" asked Jo, who was of an inquiring disposition.

"I am not sure, but I think it was because his son,
Laurie\textquotesingle s father, married an Italian lady, a musician,
which displeased the old man, who is very proud. The lady was good and
lovely and accomplished, but he did not like her, and never saw his son
after he married. They both died when Laurie was a little child, and
then his grandfather took him home. I fancy the boy, who was born in
Italy, is not very strong, and the old man is afraid of losing him,
which makes him so careful. Laurie comes naturally by his love of music,
for he is like his mother, and I dare say his grandfather fears that he
may want to be a musician; at any rate, his skill reminds him of the
woman he did not like, and so he
\textquotesingle glowered,\textquotesingle{} as Jo said."

"Dear me, how romantic!" exclaimed Meg.

"How silly!" said Jo. "Let him be a musician, if he wants to, and not
plague his life out sending him to college, when he hates to go."

"That\textquotesingle s why he has such handsome black eyes and pretty
manners, I suppose. Italians are always nice," said Meg, who was a
little sentimental.

"What do you know about his eyes and his manners? You never spoke to
him, hardly," cried Jo, who was \emph{not} sentimental.

"I saw him at the party, and what you tell shows that he knows how to
behave. That was a nice little speech about the medicine mother sent
him."

"He meant the blanc-mange, I suppose."

"How stupid you are, child! He meant you, of course."

"Did he?" and Jo opened her eyes as if it had never occurred to her
before.

"I never saw such a girl! You don\textquotesingle t know a compliment
when you get it," said Meg, with the air of a young lady who knew all
about the matter.

"I think they are great nonsense, and I\textquotesingle ll thank you not
to be silly, and spoil my fun. Laurie\textquotesingle s a nice boy, and
I like him, and I won\textquotesingle t have any sentimental stuff about
compliments and such rubbish. We\textquotesingle ll all be good to him,
because he hasn\textquotesingle t got any mother, and he \emph{may} come
over and see us, mayn\textquotesingle t he, Marmee?"

"Yes, Jo, your little friend is very welcome, and I hope Meg will
remember that children should be children as long as they can."

"I don\textquotesingle t call myself a child, and I\textquotesingle m
not in my teens yet," observed Amy. "What do you say, Beth?"

"I was thinking about our \textquotesingle Pilgrim\textquotesingle s
Progress,\textquotesingle" answered Beth, who had not heard a word. "How
we got out of the Slough and through the Wicket Gate by resolving to be
good, and up the steep hill by trying; and that maybe the house over
there, full of splendid things, is going to be our Palace Beautiful."

"We have got to get by the lions, first," said Jo, as if she rather
liked the prospect.

\protect\phantomsection\label{6672479776654687619_37106-h-1.htm.xhtml_b039.png}{}
\pandocbounded{\includegraphics[keepaspectratio]{303483661336987339_b039.png}}

\begin{center}\rule{0.5\linewidth}{0.5pt}\end{center}

\subsection{VI. Beth finds the Palace
Beautiful.}\label{6672479776654687619_37106-h-1.htm.xhtml_pgepubid00008}

\protect\phantomsection\label{6672479776654687619_37106-h-1.htm.xhtml_VI}{}\hyperref[6672479776654687619_37106-h-0.htm.xhtml_contents]{VI.}

BETH FINDS THE PALACE BEAUTIFUL.

{The} big house did prove a Palace Beautiful, though it took some time
for all to get in, and Beth found it very hard to pass the lions. Old
Mr. Laurence was the biggest one; but after he had called, said
something funny or kind to each one of the girls, and talked over old
times with their mother, nobody felt much afraid of him, except timid
Beth. The other lion was the fact that they were poor and Laurie rich;
for this made them shy of accepting favors which they could not return.
But, after a while, they found that he considered them the benefactors,
and could not do enough to show how grateful he was for Mrs.
March\textquotesingle s motherly welcome, their cheerful society, and
the comfort he took in that humble home of theirs. So they soon forgot
their pride, and interchanged kindnesses without stopping to think which
was the greater.

All sorts of pleasant things happened about that time; for the new
friendship flourished like grass in spring. Every one liked Laurie, and
he privately informed his tutor that "the Marches were regularly
splendid girls." With the delightful enthusiasm of youth, they took the
solitary boy into their midst, and made much of him, and he found
something very charming in the innocent companionship of these
simple-hearted girls. Never having known mother or sisters, he was quick
to feel the influences they brought about him; and their busy, lively
ways made him ashamed of the indolent life he led. He was tired of
books, and found people so interesting now that Mr. Brooke was obliged
to make very unsatisfactory reports; for Laurie was always playing
truant, and running over to the Marches.

"Never mind; let him take a holiday, and make it up afterwards," said
the old gentleman. "The good lady next door says he is studying too
hard, and needs young society, amusement, and exercise. I suspect she is
right, and that I\textquotesingle ve been coddling the fellow as if
I\textquotesingle d been his grandmother. Let him do what he likes, as
long as he is happy. He can\textquotesingle t get into mischief in that
little nunnery over there; and Mrs. March is doing more for him than we
can."

What good times they had, to be sure! Such plays and tableaux, such
sleigh-rides and skating frolics, such pleasant evenings in the old
parlor, and now and then such gay little parties at the great house. Meg
could walk in the conservatory whenever she liked, and revel in
bouquets; Jo browsed over the new library voraciously, and convulsed the
old gentleman with her criticisms; Amy copied pictures, and enjoyed
beauty to her heart\textquotesingle s content; and Laurie played "lord
of the manor" in the most delightful style.

But Beth, though yearning for the grand piano, could not pluck up
courage to go to the "Mansion of Bliss," as Meg called it. She went once
with Jo; but the old gentleman, not being aware of her infirmity, stared
at her so hard from under his heavy eyebrows, and said "Hey!" so loud,
that he frightened her so much her "feet chattered on the floor," she
told her mother; and she ran away, declaring she would never go there
any more, not even for the dear piano. No persuasions or enticements
could overcome her fear, till, the fact coming to Mr.
Laurence\textquotesingle s ear in some mysterious way, he set about
mending matters. During one of the brief calls he made, he artfully led
the conversation to music, and talked away about great singers whom he
had seen, fine organs he had heard, and told such charming anecdotes
that Beth found it impossible to stay in her distant corner, but crept
nearer and nearer, as if fascinated. At the back of his chair she
stopped, and stood listening, with her great eyes wide open, and her
cheeks red with the excitement of this unusual performance. Taking no
more notice of her than if she had been a fly, Mr. Laurence talked on
about Laurie\textquotesingle s lessons and teachers; and presently, as
if the idea had just occurred to him, he said to Mrs. March,---

"The boy neglects his music now, and I\textquotesingle m glad of it, for
he was getting too fond of it. But the piano suffers for want of use.
Wouldn\textquotesingle t some of your girls like to run over, and
practise on it now and then, just to keep it in tune, you know,
ma\textquotesingle am?"

Beth took a step forward, and pressed her hands tightly together to keep
from clapping them, for this was an irresistible temptation; and the
thought of practising on that splendid instrument quite took her breath
away. Before Mrs. March could reply, Mr. Laurence went on with an odd
little nod and smile,---

"They needn\textquotesingle t see or speak to any one, but run in at any
time; for I\textquotesingle m shut up in my study at the other end of
the house, Laurie is out a great deal, and the servants are never near
the drawing-room after nine o\textquotesingle clock."

Here he rose, as if going, and Beth made up her mind to speak, for that
last arrangement left nothing to be desired. "Please tell the young
ladies what I say; and if they don\textquotesingle t care to come, why,
never mind." Here a little hand slipped into his, and Beth looked up at
him with a face full of gratitude, as she said, in her earnest yet timid
way,---

"O sir, they do care, very, very much!"

\protect\phantomsection\label{6672479776654687619_37106-h-1.htm.xhtml_b040.png}{}
\pandocbounded{\includegraphics[keepaspectratio]{303483661336987339_b040.png}}

"Are you the musical girl?" he asked, without any startling "Hey!" as he
looked down at her very kindly.

"I\textquotesingle m Beth. I love it dearly, and I\textquotesingle ll
come, if you are quite sure nobody will hear me---and be disturbed," she
added, fearing to be rude, and trembling at her own boldness as she
spoke.

"Not a soul, my dear. The house is empty half the day; so come, and drum
away as much as you like, and I shall be obliged to you."

"How kind you are, sir!"

Beth blushed like a rose under the friendly look he wore; but she was
not frightened now, and gave the big hand a grateful squeeze, because
she had no words to thank him for the precious gift he had given her.
The old gentleman softly stroked the hair off her forehead, and,
stooping down, he kissed her, saying, in a tone few people ever
heard,---

"I had a little girl once, with eyes like these. God bless you, my dear!
Good day, madam;" and away he went, in a great hurry.

Beth had a rapture with her mother, and then rushed up to impart the
glorious news to her family of invalids, as the girls were not at home.
How blithely she sung that evening, and how they all laughed at her,
because she woke Amy in the night by playing the piano on her face in
her sleep. Next day, having seen both the old and young gentleman out of
the house, Beth, after two or three retreats, fairly got in at the
side-door, and made her way, as noiselessly as any mouse, to the
drawing-room, where her idol stood. Quite by accident, of course, some
pretty, easy music lay on the piano; and, with trembling fingers, and
frequent stops to listen and look about, Beth at last touched the great
instrument, and straightway forgot her fear, herself, and everything
else but the unspeakable delight which the music gave her, for it was
like the voice of a beloved friend.

She stayed till Hannah came to take her home to dinner; but she had no
appetite, and could only sit and smile upon every one in a general state
of beatitude.

\protect\phantomsection\label{6672479776654687619_37106-h-1.htm.xhtml_b041.png}{}
\pandocbounded{\includegraphics[keepaspectratio]{303483661336987339_b041.png}}

After that, the little brown hood slipped through the hedge nearly every
day, and the great drawing-room was haunted by a tuneful spirit that
came and went unseen. She never knew that Mr. Laurence often opened his
study-door to hear the old-fashioned airs he liked; she never saw Laurie
mount guard in the hall to warn the servants away; she never suspected
that the exercise-books and new songs which she found in the rack were
put there for her especial benefit; and when he talked to her about
music at home, she only thought how kind he was to tell things that
helped her so much. So she enjoyed herself heartily, and found, what
isn\textquotesingle t always the case, that her granted wish was all she
had hoped. Perhaps it was because she was so grateful for this blessing
that a greater was given her; at any rate, she deserved both.

"Mother, I\textquotesingle m going to work Mr. Laurence a pair of
slippers. He is so kind to me, I must thank him, and I
don\textquotesingle t know any other way. Can I do it?" asked Beth, a
few weeks after that eventful call of his.

"Yes, dear. It will please him very much, and be a nice way of thanking
him. The girls will help you about them, and I will pay for the making
up," replied Mrs. March, who took peculiar pleasure in granting
Beth\textquotesingle s requests, because she so seldom asked anything
for herself.

After many serious discussions with Meg and Jo, the pattern was chosen,
the materials bought, and the slippers begun. A cluster of grave yet
cheerful pansies, on a deeper purple ground, was pronounced very
appropriate and pretty; and Beth worked away early and late, with
occasional lifts over hard parts. She was a nimble little needle-woman,
and they were finished before any one got tired of them. Then she wrote
a very short, simple note, and, with Laurie\textquotesingle s help, got
them smuggled on to the study-table one morning before the old gentleman
was up.

When this excitement was over, Beth waited to see what would happen. All
that day passed, and a part of the next, before any acknowledgment
arrived, and she was beginning to fear she had offended her crotchety
friend. On the afternoon of the second day, she went out to do an
errand, and give poor Joanna, the invalid doll, her daily exercise. As
she came up the street, on her return, she saw three, yes, four, heads
popping in and out of the parlor windows, and the moment they saw her,
several hands were waved, and several joyful voices screamed,---

"Here\textquotesingle s a letter from the old gentleman! Come quick, and
read it!"

"O Beth, he\textquotesingle s sent you---" began Amy, gesticulating with
unseemly energy; but she got no further, for Jo quenched her by slamming
down the window.

Beth hurried on in a flutter of suspense. At the door, her sisters
seized and bore her to the parlor in a triumphal procession, all
pointing, and all saying at once, "Look there! look there!" Beth did
look, and turned pale with delight and surprise; for there stood a
little cabinet-piano, with a letter lying on the glossy lid, directed,
like a sign-board, to "Miss Elizabeth March."

"For me?" gasped Beth, holding on to Jo, and feeling as if she should
tumble down, it was such an overwhelming thing altogether.

"Yes; all for you, my precious! Isn\textquotesingle t it splendid of
him? Don\textquotesingle t you think he\textquotesingle s the dearest
old man in the world? Here\textquotesingle s the key in the letter. We
didn\textquotesingle t open it, but we are dying to know what he says,"
cried Jo, hugging her sister, and offering the note.

"You read it! I can\textquotesingle t, I feel so queer! Oh, it is too
lovely!" and Beth hid her face in Jo\textquotesingle s apron, quite
upset by her present.

Jo opened the paper, and began to laugh, for the first words she saw
were,---

\begin{quote}
"{Miss March}:\\
{"\emph{Dear Madam},---"}
\end{quote}

"How nice it sounds! I wish some one would write to me so!" said Amy,
who thought the old-fashioned address very elegant.

\begin{quote}
"\textquotesingle I have had many pairs of slippers in my life, but I
never had any that suited me so well as yours,\textquotesingle"
continued Jo. "\textquotesingle Heart\textquotesingle s-ease is my
favorite flower, and these will always remind me of the gentle giver. I
like to pay my debts; so I know you will allow "the old gentleman" to
send you something which once belonged to the little granddaughter he
lost. With hearty thanks and best wishes, \ul{I remain,}

"\textquotesingle Your grateful friend and humble servant,

"\textquotesingle James Laurence.\textquotesingle"
\end{quote}

"There, Beth, that\textquotesingle s an honor to be proud of,
I\textquotesingle m sure! Laurie told me how fond Mr. Laurence used to
be of the child who died, and how he kept all her little things
carefully. Just think, he\textquotesingle s given you her piano. That
comes of having big blue eyes and loving music," said Jo, trying to
soothe Beth, who trembled, and looked more excited than she had ever
been before.

"See the cunning brackets to hold candles, and the nice green silk,
puckered up, with a gold rose in the middle, and the pretty rack and
stool, all complete," added Meg, opening the instrument and displaying
its beauties.

"\textquotesingle Your humble servant, James Laurence\textquotesingle;
only think of his writing that to you. I\textquotesingle ll tell the
girls. They\textquotesingle ll think it\textquotesingle s splendid,"
said Amy, much impressed by the note.

"Try it, honey. Let\textquotesingle s hear the sound of the
baby-pianny," said Hannah, who always took a share in the family joys
and sorrows.

So Beth tried it; and every one pronounced it the most remarkable piano
ever heard. It had evidently been newly tuned and put in apple-pie
order; but, perfect as it was, I think the real charm of it lay in the
happiest of all happy faces which leaned over it, as Beth lovingly
touched the beautiful black and white keys and pressed the bright
pedals.

"You\textquotesingle ll have to go and thank him," said Jo, by way of a
joke; for the idea of the child\textquotesingle s really going never
entered her head.

"Yes, I mean to. I guess I\textquotesingle ll go now, before I get
frightened thinking about it." And, to the utter amazement of the
assembled family, Beth walked deliberately down the garden, through the
hedge, and in at the Laurences\textquotesingle{} door.

"Well, I wish I may die if it ain\textquotesingle t the queerest thing I
ever see! The pianny has turned her head! She\textquotesingle d never
have gone in her right mind," cried Hannah, staring after her, while the
girls were rendered quite speechless by the miracle.

They would have been still more amazed if they had seen what Beth did
afterward. If you will believe me, she went and knocked at the
study-door before she gave herself time to think; and when a gruff voice
called out, "Come in!" she did go in, right up to Mr. Laurence, who
looked quite taken aback, and held out her hand, saying, with only a
small quaver in her voice, "I came to thank you, sir, for---" But she
didn\textquotesingle t finish; for he looked so friendly that she forgot
her speech, and, only remembering that he had lost the little girl he
loved, she put both arms round his neck, and kissed him.

\protect\phantomsection\label{6672479776654687619_37106-h-1.htm.xhtml_b042.png}{}
\pandocbounded{\includegraphics[keepaspectratio]{303483661336987339_b042.png}}

If the roof of the house had suddenly flown off, the old gentleman
wouldn\textquotesingle t have been more astonished; but he liked
it,---oh, dear, yes, he liked it amazingly!---and was so touched and
pleased by that confiding little kiss that all his crustiness vanished;
and he just set her on his knee, and laid his wrinkled cheek against her
rosy one, feeling as if he had got his own little granddaughter back
again. Beth ceased to fear him from that moment, and sat there talking
to him as cosily as if she had known him all her life; for love casts
out fear, and gratitude can conquer pride. When she went home, he walked
with her to her own gate, shook hands cordially, and touched his hat as
he marched back again, looking very stately and erect, like a handsome,
soldierly old gentleman, as he was.

When the girls saw that performance, Jo began to dance a jig, by way of
expressing her satisfaction; Amy nearly fell out of the window in her
surprise; and Meg exclaimed, with uplifted hands, "Well, I do believe
the world is coming to an end!"

\begin{center}\rule{0.5\linewidth}{0.5pt}\end{center}

\subsection{VII. Amy\textquotesingle s Valley of
Humiliation.}\label{6672479776654687619_37106-h-1.htm.xhtml_pgepubid00009}

\protect\phantomsection\label{6672479776654687619_37106-h-1.htm.xhtml_VII}{}\hyperref[6672479776654687619_37106-h-0.htm.xhtml_contents]{VII.}

AMY\textquotesingle S VALLEY OF HUMILIATION.

\protect\phantomsection\label{6672479776654687619_37106-h-1.htm.xhtml_b043.png}{}
\pandocbounded{\includegraphics[keepaspectratio]{303483661336987339_b043.png}}

{"That} boy is a perfect Cyclops, isn\textquotesingle t he?" said Amy,
one day, as Laurie clattered by on horseback, with a flourish of his
whip as he passed.

"How dare you say so, when he\textquotesingle s got both his eyes? and
very handsome ones they are, too," cried Jo, who resented any slighting
remarks about her friend.

"I didn\textquotesingle t say anything about his eyes, and I
don\textquotesingle t see why you need fire up when I admire his
riding."

"Oh, my goodness! that little goose means a centaur, and she called him
a Cyclops," exclaimed Jo, with a burst of laughter.

"You needn\textquotesingle t be so rude; it\textquotesingle s only a
\textquotesingle lapse of lingy,\textquotesingle{} as Mr. Davis says,"
retorted Amy, finishing Jo with her Latin. "I just wish I had a little
of the money Laurie spends on that horse," she added, as if to herself,
yet hoping her sisters would hear.

"Why?" asked Meg kindly, for Jo had gone off in another laugh at
Amy\textquotesingle s second blunder.

"I need it so much; I\textquotesingle m dreadfully in debt, and it
won\textquotesingle t be my turn to have the rag-money for a month."

"In debt, Amy? What do you mean?" and Meg looked sober.

"Why, I owe at least a dozen pickled limes, and I can\textquotesingle t
pay them, you know, till I have money, for Marmee forbade my having
anything charged at the shop."

"Tell me all about it. Are limes the fashion now? It used to be pricking
bits of rubber to make balls;" and Meg tried to keep her countenance,
Amy looked so grave and important.

"Why, you see, the girls are always buying them, and unless you want to
be thought mean, you must do it, too. It\textquotesingle s nothing but
limes now, for every one is sucking them in their desks in school-time,
and trading them off for pencils, bead-rings, paper dolls, or something
else, at recess. If one girl likes another, she gives her a lime; if
she\textquotesingle s mad with her, she eats one before her face, and
don\textquotesingle t offer even a suck. They treat by turns; and
I\textquotesingle ve had ever so many, but haven\textquotesingle t
returned them; and I ought, for they are debts of honor, you know."

"How much will pay them off, and restore your credit?" asked Meg, taking
out her purse.

"A quarter would more than do it, and leave a few cents over for a treat
for you. Don\textquotesingle t you like limes?"

"Not much; you may have my share. Here\textquotesingle s the money. Make
it last as long as you can, for it isn\textquotesingle t very plenty,
you know."

"Oh, thank you! It must be so nice to have pocket-money!
I\textquotesingle ll have a grand feast, for I haven\textquotesingle t
tasted a lime this week. I felt delicate about taking any, as I
couldn\textquotesingle t return them, and I\textquotesingle m actually
suffering for one."

Next day Amy was rather late at school; but could not resist the
temptation of displaying, with pardonable pride, a moist brown-paper
parcel, before she consigned it to the inmost recesses of her desk.
During the next few minutes the rumor that Amy March had got twenty-four
delicious limes (she ate one on the way), and was going to treat,
circulated through her "set," and the attentions of her friends became
quite overwhelming. Katy Brown invited her to her next party on the
spot; Mary Kingsley insisted on lending her her watch till recess; and
Jenny Snow, a satirical young lady, who had basely twitted Amy upon her
limeless state, promptly buried the hatchet, and offered to furnish
answers to certain appalling sums. But Amy had not forgotten Miss
Snow\textquotesingle s cutting remarks about "some persons whose noses
were not too flat to smell other people\textquotesingle s limes, and
stuck-up people, who were not too proud to ask for them;" and she
instantly crushed "that Snow girl\textquotesingle s" hopes by the
withering telegram, "You needn\textquotesingle t be so polite all of a
sudden, for you won\textquotesingle t get any."

A distinguished personage happened to visit the school that morning, and
Amy\textquotesingle s beautifully drawn maps received praise, which
honor to her foe rankled in the soul of Miss Snow, and caused Miss March
to assume the airs of a studious young peacock. But, alas, alas! pride
goes before a fall, and the revengeful Snow turned the tables with
disastrous success. No sooner had the guest paid the usual stale
compliments, and bowed himself out, than Jenny, under pretence of asking
an important question, informed Mr. Davis, the teacher, that Amy March
had pickled limes in her desk.

Now Mr. Davis had declared limes a contraband article, and solemnly
vowed to publicly ferrule the first person who was found breaking the
law. This much-enduring man had succeeded in banishing chewing-gum after
a long and stormy war, had made a bonfire of the confiscated novels and
newspapers, had suppressed a private post-office, had forbidden
distortions of the face, nicknames, and caricatures, and done all that
one man could do to keep half a hundred rebellious girls in order. Boys
are trying enough to human patience, goodness knows! but girls are
infinitely more so, especially to nervous gentlemen, with tyrannical
tempers, and no more talent for teaching than Dr. Blimber. Mr. Davis
knew any quantity of Greek, Latin, Algebra, and ologies of all sorts, so
he was called a fine teacher; and manners, morals, feelings, and
examples were not considered of any particular importance. It was a most
unfortunate moment for denouncing Amy, and Jenny knew it. Mr. Davis had
evidently taken his coffee too strong that morning; there was an east
wind, which always affected his neuralgia; and his pupils had not done
him the credit which he felt he deserved: therefore, to use the
expressive, if not elegant, language of a school-girl, "he was as
nervous as a witch and as cross as a bear." The word "limes" was like
fire to powder; his yellow face flushed, and he rapped on his desk with
an energy which made Jenny skip to her seat with unusual rapidity.

"Young ladies, attention, if you please!"

At the stern order the buzz ceased, and fifty pairs of blue, black,
gray, and brown eyes were obediently fixed upon his awful countenance.

"Miss March, come to the desk."

Amy rose to comply with outward composure, but a secret fear oppressed
her, for the limes weighed upon her conscience.

"Bring with you the limes you have in your desk," was the unexpected
command which arrested her before she got out of her seat.

"Don\textquotesingle t take all," whispered her neighbor, a young lady
of great presence of mind.

Amy hastily shook out half a dozen, and laid the rest down before Mr.
Davis, feeling that any man possessing a human heart would relent when
that delicious perfume met his nose. Unfortunately, Mr. Davis
particularly detested the odor of the fashionable pickle, and disgust
added to his wrath.

"Is that all?"

"Not quite," stammered Amy.

"Bring the rest immediately."

With a despairing glance at her set, she obeyed.

"You are sure there are no more?"

"I never lie, sir."

"So I see. Now take these disgusting things two by two, and throw them
out of the window."

There was a simultaneous sigh, which created quite a little gust, as the
last hope fled, and the treat was ravished from their longing lips.
Scarlet with shame and anger, Amy went to and fro six dreadful times;
and as each doomed couple---looking oh! so plump and juicy---fell from
her reluctant hands, a shout from the street completed the anguish of
the girls, for it told them that their feast was being exulted over by
the little Irish children, who were their sworn foes. This---this was
too much; all flashed indignant or appealing glances at the inexorable
Davis, and one passionate lime-lover burst into tears.

As Amy returned from her last trip, Mr. Davis gave a portentous "Hem!"
and said, in his most impressive manner,---

"Young ladies, you remember what I said to you a week ago. I am sorry
this has happened, but I never allow my rules to be infringed, and I
\emph{never} break my word. Miss March, hold out your hand."

Amy started, and put both hands behind her, turning on him an imploring
look which pleaded for her better than the words she could not utter.
She was rather a favorite with "old Davis," as, of course, he was
called, and it\textquotesingle s my private belief that he \emph{would}
have broken his word if the indignation of one irrepressible young lady
had not found vent in a hiss. That hiss, faint as it was, irritated the
irascible gentleman, and sealed the culprit\textquotesingle s fate.

"Your hand, Miss March!" was the only answer her mute appeal received;
and, too proud to cry or beseech, Amy set her teeth, threw back her head
defiantly, and bore without flinching several tingling blows on her
little palm. They were neither many nor heavy, but that made no
difference to her. For the first time in her life she had been struck;
and the disgrace, in her eyes, was as deep as if he had knocked her
down.

\protect\phantomsection\label{6672479776654687619_37106-h-1.htm.xhtml_b044.png}{}
\pandocbounded{\includegraphics[keepaspectratio]{303483661336987339_b044.png}}

"You will now stand on the platform till recess," said Mr. Davis,
resolved to do the thing thoroughly, since he had begun.

That was dreadful. It would have been bad enough to go to her seat, and
see the pitying faces of her friends, or the satisfied ones of her few
enemies; but to face the whole school, with that shame fresh upon her,
seemed impossible, and for a second she felt as if she could only drop
down where she stood, and break her heart with crying. A bitter sense of
wrong, and the thought of Jenny Snow, helped her to bear it; and, taking
the ignominious place, she fixed her eyes on the stove-funnel above what
now seemed a sea of faces, and stood there, so motionless and white that
the girls found it very hard to study, with that pathetic figure before
them.

During the fifteen minutes that followed, the proud and sensitive little
girl suffered a shame and pain which she never forgot. To others it
might seem a ludicrous or trivial affair, but to her it was a hard
experience; for during the twelve years of her life she had been
governed by love alone, and a blow of that sort had never touched her
before. The smart of her hand and the ache of her heart were forgotten
in the sting of the thought,---

"I shall have to tell at home, and they will be so disappointed in me!"

The fifteen minutes seemed an hour; but they came to an end at last, and
the word "Recess!" had never seemed so welcome to her before.

"You can go, Miss March," said Mr. Davis, looking, as he felt,
uncomfortable.

He did not soon forget the reproachful glance Amy gave him, as she went,
without a word to any one, straight into the ante-room, snatched her
things, and left the place "forever," as she passionately declared to
herself. She was in a sad state when she got home; and when the older
girls arrived, some time later, an indignation meeting was held at once.
Mrs. March did not say much, but looked disturbed, and comforted her
afflicted little daughter in her tenderest manner. Meg bathed the
insulted hand with glycerine and tears; Beth felt that even her beloved
kittens would fail as a balm for griefs like this; Jo wrathfully
proposed that Mr. Davis be arrested without delay; and Hannah shook her
fist at the "villain," and pounded potatoes for dinner as if she had him
under her pestle.

No notice was taken of Amy\textquotesingle s flight, except by her
mates; but the sharp-eyed demoiselles discovered that Mr. Davis was
quite benignant in the afternoon, also unusually nervous. Just before
school closed, Jo appeared, wearing a grim expression, as she stalked up
to the desk, and delivered a letter from her mother; then collected
Amy\textquotesingle s property, and departed, carefully scraping the mud
from her boots on the door-mat, as if she shook the dust of the place
off her feet.

"Yes, you can have a vacation from school, but I want you to study a
little every day, with Beth," said Mrs. March, that evening. "I
don\textquotesingle t approve of corporal punishment, especially for
girls. I dislike Mr. Davis\textquotesingle s manner of teaching, and
don\textquotesingle t think the girls you associate with are doing you
any good, so I shall ask your father\textquotesingle s advice before I
send you anywhere else."

"That\textquotesingle s good! I wish all the girls would leave, and
spoil his old school. It\textquotesingle s perfectly maddening to think
of those lovely limes," sighed Amy, with the air of a martyr.

"I am not sorry you lost them, for you broke the rules, and deserved
some punishment for disobedience," was the severe reply, which rather
disappointed the young lady, who expected nothing but sympathy.

"Do you mean you are glad I was disgraced before the whole school?"
cried Amy.

"I should not have chosen that way of mending a fault," replied her
mother; "but I\textquotesingle m not sure that it won\textquotesingle t
do you more good than a milder method. You are getting to be rather
conceited, my dear, and it is quite time you set about correcting it.
You have a good many little gifts and virtues, but there is no need of
parading them, for conceit spoils the finest genius. There is not much
danger that real talent or goodness will be overlooked long; even if it
is, the consciousness of possessing and using it well should satisfy
one, and the great charm of all power is modesty."

"So it is!" cried Laurie, who was playing chess in a corner with Jo. "I
knew a girl, once, who had a really remarkable talent for music, and she
didn\textquotesingle t know it; never guessed what sweet little things
she composed when she was alone, and wouldn\textquotesingle t have
believed it if any one had told her."

"I wish I\textquotesingle d known that nice girl; maybe she would have
helped me, I\textquotesingle m so stupid," said Beth, who stood beside
him, listening eagerly.

"You do know her, and she helps you better than any one else could,"
answered Laurie, looking at her with such mischievous meaning in his
merry black eyes, that Beth suddenly turned very red, and hid her face
in the sofa-cushion, quite overcome by such an unexpected discovery.

\protect\phantomsection\label{6672479776654687619_37106-h-1.htm.xhtml_b045.png}{}
\pandocbounded{\includegraphics[keepaspectratio]{303483661336987339_b045.png}}

Jo let Laurie win the game, to pay for that praise of her Beth, who
could not be prevailed upon to play for them after her compliment. So
Laurie did his best, and sung delightfully, being in a particularly
lively humor, for to the Marches he seldom showed the moody side of his
character. When he was gone, Amy, who had been pensive all the evening,
said suddenly, as if busy over some new idea,---

"Is Laurie an accomplished boy?"

"Yes; he has had an excellent education, and has much talent; he will
make a fine man, if not spoilt by petting," replied her mother.

"And he isn\textquotesingle t conceited, is he?" asked Amy.

"Not in the least; that is why he is so charming, and we all like him so
much."

"I see; it\textquotesingle s nice to have accomplishments, and be
elegant; but not to show off, or get perked up," said Amy thoughtfully.

"These things are always seen and felt in a person\textquotesingle s
manner and conversation, if modestly used; but it is not necessary to
display them," said Mrs. March.

"Any more than it\textquotesingle s proper to wear all your bonnets and
gowns and ribbons at once, that folks may know you\textquotesingle ve
got them," added Jo; and the lecture ended in a laugh.

\begin{center}\rule{0.5\linewidth}{0.5pt}\end{center}

\subsection{VIII. Jo Meets
Apollyon.}\label{6672479776654687619_37106-h-1.htm.xhtml_pgepubid00010}

\protect\phantomsection\label{6672479776654687619_37106-h-1.htm.xhtml_b046.png}{}
\pandocbounded{\includegraphics[keepaspectratio]{303483661336987339_b046.png}}

\protect\phantomsection\label{6672479776654687619_37106-h-1.htm.xhtml_VIII}{}\hyperref[6672479776654687619_37106-h-0.htm.xhtml_contents]{VIII.}

JO MEETS APOLLYON.

"{Girls}, where are you going?" asked Amy, coming into their room one
Saturday afternoon, and finding them getting ready to go out, with an
air of secrecy which excited her curiosity.

"Never mind; little girls shouldn\textquotesingle t ask questions,"
returned Jo sharply.

Now if there \emph{is} anything mortifying to our feelings, when we are
young, it is to be told that; and to be bidden to "run away, dear," is
still more trying to us. Amy bridled up at this insult, and determined
to find out the secret, if she teased for an hour. Turning to Meg, who
never refused her anything very long, she said coaxingly, "Do tell me! I
should think you might let me go, too; for Beth is fussing over her
piano, and I haven\textquotesingle t got anything to do, and am
\emph{so} lonely."

"I can\textquotesingle t, dear, because you aren\textquotesingle t
invited," began Meg; but Jo broke in impatiently, "Now, Meg, be quiet,
or you will spoil it all. You can\textquotesingle t go, Amy; so
don\textquotesingle t be a baby, and whine about it."

"You are going somewhere with Laurie, I know you are; you were
whispering and laughing together, on the sofa, last night, and you
stopped when I came in. Aren\textquotesingle t you going with him?"

"Yes, we are; now do be still, and stop bothering."

Amy held her tongue, but used her eyes, and saw Meg slip a fan into her
pocket.

"I know! I know! you\textquotesingle re going to the theatre to see the
\textquotesingle Seven Castles!\textquotesingle" she cried; adding
resolutely, "and I \emph{shall} go, for mother said I might see it; and
I\textquotesingle ve got my rag-money, and it was mean not to tell me in
time."

"Just listen to me a minute, and be a good child," said Meg soothingly.
"Mother doesn\textquotesingle t wish you to go this week, because your
eyes are not well enough yet to bear the light of this fairy piece. Next
week you can go with Beth and Hannah, and have a nice time."

"I don\textquotesingle t like that half as well as going with you and
Laurie. Please let me; I\textquotesingle ve been sick with this cold so
long, and shut up, I\textquotesingle m dying for some fun. Do, Meg!
I\textquotesingle ll be ever so good," pleaded Amy, looking as pathetic
as she could.

"Suppose we take her. I don\textquotesingle t believe mother would mind,
if we bundle her up well," began Meg.

"If \emph{she} goes \emph{I} sha\textquotesingle n\textquotesingle t;
and if I don\textquotesingle t, Laurie won\textquotesingle t like it;
and it will be very rude, after he invited only us, to go and drag in
Amy. I should think she\textquotesingle d hate to poke herself where she
isn\textquotesingle t wanted," said Jo crossly, for she disliked the
trouble of overseeing a fidgety child, when she wanted to enjoy herself.

Her tone and manner angered Amy, who began to put her boots on, saying,
in her most aggravating way, "I \emph{shall} go; Meg says I may; and if
I pay for myself, Laurie hasn\textquotesingle t anything to do with it."

"You can\textquotesingle t sit with us, for our seats are reserved, and
you mustn\textquotesingle t sit alone; so Laurie will give you his
place, and that will spoil our pleasure; or he\textquotesingle ll get
another seat for you, and that isn\textquotesingle t proper, when you
weren\textquotesingle t asked. You
sha\textquotesingle n\textquotesingle t stir a step; so you may just
stay where you are," scolded Jo, crosser than ever, having just pricked
her finger in her hurry.

Sitting on the floor, with one boot on, Amy began to cry, and Meg to
reason with her, when Laurie called from below, and the two girls
hurried down, leaving their sister wailing; for now and then she forgot
her grown-up ways, and acted like a spoilt child. Just as the party was
setting out, Amy called over the banisters, in a threatening tone,
"You\textquotesingle ll be sorry for this, Jo March; see if you
ain\textquotesingle t."

"Fiddlesticks!" returned Jo, slamming the door.

They had a charming time, for "The Seven Castles of the Diamond Lake"
were as brilliant and wonderful as heart could wish. But, in spite of
the comical red imps, sparkling elves, and gorgeous princes and
princesses, Jo\textquotesingle s pleasure had a drop of bitterness in
it; the fairy queen\textquotesingle s yellow curls reminded her of Amy;
and between the acts she amused herself with wondering what her sister
would do to make her "sorry for it." She and Amy had had many lively
skirmishes in the course of their lives, for both had quick tempers, and
were apt to be violent when fairly roused. Amy teased Jo, and Jo
irritated Amy, and semi-occasional explosions occurred, of which both
were much ashamed afterward. Although the oldest, Jo had the least
self-control, and had hard times trying to curb the fiery spirit which
was continually getting her into trouble; her anger never lasted long,
and, having humbly confessed her fault, she sincerely repented, and
tried to do better. Her sisters used to say that they rather liked to
get Jo into a fury, because she was such an angel afterward. Poor Jo
tried desperately to be good, but her bosom enemy was always ready to
flame up and defeat her; and it took years of patient effort to subdue
it.

When they got home, they found Amy reading in the parlor. She assumed an
injured air as they came in; never lifted her eyes from her book, or
asked a single question. Perhaps curiosity might have conquered
resentment, if Beth had not been there to inquire, and receive a glowing
description of the play. On going up to put away her best hat,
Jo\textquotesingle s first look was toward the bureau; for, in their
last quarrel, Amy had soothed her feelings by turning
Jo\textquotesingle s top drawer upside down on the floor. Everything was
in its place, however; and after a hasty glance into her various
closets, bags, and boxes, Jo decided that Amy had forgiven and forgotten
her wrongs.

There Jo was mistaken; for next day she made a discovery which produced
a tempest. Meg, Beth, and Amy were sitting together, late in the
afternoon, when Jo burst into the room, looking excited, and demanding
breathlessly, "Has any one taken my book?"

Meg and Beth said "No," at once, and looked surprised; Amy poked the
fire, and said nothing. Jo saw her color rise, and was down upon her in
a minute.

"Amy, you\textquotesingle ve got it?"

"No, I haven\textquotesingle t."

"You know where it is, then?"

"No, I don\textquotesingle t."

"That\textquotesingle s a fib!" cried Jo, taking her by the shoulders,
and looking fierce enough to frighten a much braver child than Amy.

"It isn\textquotesingle t. I haven\textquotesingle t got it,
don\textquotesingle t know where it is now, and don\textquotesingle t
care."

"You know something about it, and you\textquotesingle d better tell at
once, or I\textquotesingle ll make you," and Jo gave her a slight shake.

"Scold as much as you like, you\textquotesingle ll never see your silly
old book again," cried Amy, getting excited in her turn.

"Why not?"

"I burnt it up."

\protect\phantomsection\label{6672479776654687619_37106-h-1.htm.xhtml_b047.png}{}
\pandocbounded{\includegraphics[keepaspectratio]{303483661336987339_b047.png}}

"What! my little book I was so fond of, and worked over, and meant to
finish before father got home? Have you really burnt it?" said Jo,
turning very pale, while her eyes kindled and her hands clutched Amy
nervously.

"Yes, I did! I told you I\textquotesingle d make you pay for being so
cross yesterday, and I have, so---"

Amy got no farther, for Jo\textquotesingle s hot temper mastered her,
and she shook Amy till her teeth chattered in her head; crying, in a
passion of grief and anger,---

"You wicked, wicked girl! I never can write it again, and
I\textquotesingle ll never forgive you as long as I live."

Meg flew to rescue Amy, and Beth to pacify Jo, but Jo was quite beside
herself; and, with a parting box on her sister\textquotesingle s ear,
she rushed out of the room up to the old sofa in the garret, and
finished her fight alone.

The storm cleared up below, for Mrs. March came home, and, having heard
the story, soon brought Amy to a sense of the wrong she had done her
sister. Jo\textquotesingle s book was the pride of her heart, and was
regarded by her family as a literary sprout of great promise. It was
only half a dozen little fairy tales, but Jo had worked over them
patiently, putting her whole heart into her work, hoping to make
something good enough to print. She had just copied them with great
care, and had destroyed the old manuscript, so that
Amy\textquotesingle s bonfire had consumed the loving work of several
years. It seemed a small loss to others, but to Jo it was a dreadful
calamity, and she felt that it never could be made up to her. Beth
mourned as for a departed kitten, and Meg refused to defend her pet;
Mrs. March looked grave and grieved, and Amy felt that no one would love
her till she had asked pardon for the act which she now regretted more
than any of them.

When the tea-bell rung, Jo appeared, looking so grim and unapproachable
that it took all Amy\textquotesingle s courage to say meekly,---

"Please forgive me, Jo; I\textquotesingle m very, very sorry."

"I never shall forgive you," was Jo\textquotesingle s stern answer; and,
from that moment, she ignored Amy entirely.

No one spoke of the great trouble,---not even Mrs. March,---for all had
learned by experience that when Jo was in that mood words were wasted;
and the wisest course was to wait till some little accident, or her own
generous nature, softened Jo\textquotesingle s resentment, and healed
the breach. It was not a happy evening; for, though they sewed as usual,
while their mother read aloud from Bremer, Scott, or Edgeworth,
something was wanting, and the sweet home-peace was disturbed. They felt
this most when singing-time came; for Beth could only play, Jo stood
dumb as a stone, and Amy broke down, so Meg and mother sung alone. But,
in spite of their efforts to be as cheery as larks, the flute-like
voices did not seem to chord as well as usual, and all felt out of tune.

As Jo received her good-night kiss, Mrs. March whispered gently,---

"My dear, don\textquotesingle t let the sun go down upon your anger;
forgive each other, help each other, and begin again to-morrow."

Jo wanted to lay her head down on that motherly bosom, and cry her grief
and anger all away; but tears were an unmanly weakness, and she felt so
deeply injured that she really \emph{couldn\textquotesingle t} quite
forgive yet. So she winked hard, shook her head, and said, gruffly
because Amy was listening,---

"It was an abominable thing, and she don\textquotesingle t deserve to be
forgiven."

With that she marched off to bed, and there was no merry or confidential
gossip that night.

Amy was much offended that her overtures of peace had been repulsed, and
began to wish she had not humbled herself, to feel more injured than
ever, and to plume herself on her superior virtue in a way which was
particularly exasperating. Jo still looked like a thunder-cloud, and
nothing went well all day. It was bitter cold in the morning; she
dropped her precious turn-over in the gutter, Aunt March had an attack
of fidgets, Meg was pensive, Beth \emph{would} look grieved and wistful
when she got home, and Amy kept making remarks about people who were
always talking about being good, and yet wouldn\textquotesingle t try,
when other people set them a virtuous example.

"Everybody is so hateful, I\textquotesingle ll ask Laurie to go skating.
He is always kind and jolly, and will put me to rights, I know," said Jo
to herself, and off she went.

Amy heard the clash of skates, and looked out with an impatient
exclamation,---

"There! she promised I should go next time, for this is the last ice we
shall have. But it\textquotesingle s no use to ask such a cross-patch to
take me."

"Don\textquotesingle t say that; you \emph{were} very naughty, and it
\emph{is} hard to forgive the loss of her precious little book; but I
think she might do it now, and I guess she will, if you try her at the
right minute," said Meg. "Go after them; don\textquotesingle t say
anything till Jo has got good-natured with Laurie, then take a quiet
minute, and just kiss her, or do some kind thing, and
I\textquotesingle m sure she\textquotesingle ll be friends again, with
all her heart."

"I\textquotesingle ll try," said Amy, for the advice suited her; and,
after a flurry to get ready, she ran after the friends, who were just
disappearing over the hill.

It was not far to the river, but both were ready before Amy reached
them. Jo saw her coming, and turned her back; Laurie did not see, for he
was carefully skating along the shore, sounding the ice, for a warm
spell had preceded the cold snap.

"I\textquotesingle ll go on to the first bend, and see if
it\textquotesingle s all right, before we begin to race," Amy heard him
say, as he shot away, looking like a young Russian, in his fur-trimmed
coat and cap.

Jo heard Amy panting after her run, stamping her feet and blowing her
fingers, as she tried to put her skates on; but Jo never turned, and
went slowly zigzagging down the river, taking a bitter, unhappy sort of
satisfaction in her sister\textquotesingle s troubles. She had cherished
her anger till it grew strong, and took possession of her, as evil
thoughts and feelings always do, unless cast out at once. As Laurie
turned the bend, he shouted back,---

"Keep near the shore; it isn\textquotesingle t safe in the middle."

Jo heard, but Amy was just struggling to her feet, and did not catch a
word. Jo glanced over her shoulder, and the little demon she was
harboring said in her ear,---

"No matter whether she heard or not, let her take care of herself."

Laurie had vanished round the bend; Jo was just at the turn, and Amy,
far behind, striking out toward the smoother ice in the middle of the
river. For a minute Jo stood still, with a strange feeling at her heart;
then she resolved to go on, but something held and turned her round,
just in time to see Amy throw up her hands and go down, with the sudden
crash of rotten ice, the splash of water, and a cry that made
Jo\textquotesingle s heart stand still with fear. She tried to call
Laurie, but her voice was gone; she tried to rush forward, but her feet
seemed to have no strength in them; and, for a second, she could only
stand motionless, staring, with a terror-stricken face, at the little
blue hood above the black water. Something rushed swiftly by her, and
Laurie\textquotesingle s voice cried out,---

"Bring a rail; quick, quick!"

How she did it, she never knew; but for the next few minutes she worked
as if possessed, blindly obeying Laurie, who was quite self-possessed,
and, lying flat, held Amy up by his arm and hockey till Jo dragged a
rail from the fence, and together they got the child out, more
frightened than hurt.

\protect\phantomsection\label{6672479776654687619_37106-h-1.htm.xhtml_b048.png}{}
\pandocbounded{\includegraphics[keepaspectratio]{303483661336987339_b048.png}}

"Now then, we must walk her home as fast as we can; pile our things on
her, while I get off these confounded skates," cried Laurie, wrapping
his coat round Amy, and tugging away at the straps, which never seemed
so intricate before.

Shivering, dripping, and crying, they got Amy home; and, after an
exciting time of it, she fell asleep, rolled in blankets, before a hot
fire. During the bustle Jo had scarcely spoken; but flown about, looking
pale and wild, with her things half off, her dress torn, and her hands
cut and bruised by ice and rails, and refractory buckles. When Amy was
comfortably asleep, the house quiet, and Mrs. March sitting by the bed,
she called Jo to her, and began to bind up the hurt hands.

"Are you sure she is safe?" whispered Jo, looking remorsefully at the
golden head, which might have been swept away from her sight forever
under the treacherous ice.

"Quite safe, dear; she is not hurt, and won\textquotesingle t even take
cold, I think, you were so sensible in covering and getting her home
quickly," replied her mother cheerfully.

"Laurie did it all; I only let her go. Mother, if she \emph{should} die,
it would be my fault"; and Jo dropped down beside the bed, in a passion
of penitent tears, telling all that had happened, bitterly condemning
her hardness of heart, and sobbing out her gratitude for being spared
the heavy punishment which might have come upon her.

"It\textquotesingle s my dreadful temper! I try to cure it; I think I
have, and then it breaks out worse than ever. O mother, what shall I do?
what shall I do?" cried poor Jo, in despair.

"Watch and pray, dear; never get tired of trying; and never think it is
impossible to conquer your fault," said Mrs. March, drawing the blowzy
head to her shoulder, and kissing the wet cheek so tenderly that Jo
cried harder than ever.

"You don\textquotesingle t know, you can\textquotesingle t guess how bad
it is! It seems as if I could do anything when I\textquotesingle m in a
passion; I get so savage, I could hurt any one, and enjoy it.
I\textquotesingle m afraid I \emph{shall} do something dreadful some
day, and spoil my life, and make everybody hate me. O mother, help me,
do help me!"

"I will, my child, I will. Don\textquotesingle t cry so bitterly, but
remember this day, and resolve, with all your soul, that you will never
know another like it. Jo, dear, we all have our temptations, some far
greater than yours, and it often takes us all our lives to conquer them.
You think your temper is the worst in the world; but mine used to be
just like it."

"Yours, mother? Why, you are never angry!" and, for the moment, Jo
forgot remorse in surprise.

"I\textquotesingle ve been trying to cure it for forty years, and have
only succeeded in controlling it. I am angry nearly every day of my
life, Jo; but I have learned not to show it; and I still hope to learn
not to feel it, though it may take me another forty years to do so."

The patience and the humility of the face she loved so well was a better
lesson to Jo than the wisest lecture, the sharpest reproof. She felt
comforted at once by the sympathy and confidence given her; the
knowledge that her mother had a fault like hers, and tried to mend it,
made her own easier to bear and strengthened her resolution to cure it;
though forty years seemed rather a long time to watch and pray, to a
girl of fifteen.

"Mother, are you angry when you fold your lips tight together, and go
out of the room sometimes, when Aunt March scolds, or people worry you?"
asked Jo, feeling nearer and dearer to her mother than ever before.

"Yes, I\textquotesingle ve learned to check the hasty words that rise to
my lips; and when I feel that they mean to break out against my will, I
just go away a minute, and give myself a little shake, for being so weak
and wicked," answered Mrs. March, with a sigh and a smile, as she
smoothed and fastened up Jo\textquotesingle s dishevelled hair.

"How did you learn to keep still? That is what troubles me---for the
sharp words fly out before I know what I\textquotesingle m about; and
the more I say the worse I get, till it\textquotesingle s a pleasure to
hurt people\textquotesingle s feelings, and say dreadful things. Tell me
how you do it, Marmee dear."

"My good mother used to help me---"

"As you do us---" interrupted Jo, with a grateful kiss.

"But I lost her when I was a little older than you are, and for years
had to struggle on alone, for I was too proud to confess my weakness to
any one else. I had a hard time, Jo, and shed a good many bitter tears
over my failures; for, in spite of my efforts, I never seemed to get on.
Then your father came, and I was so happy that I found it easy to be
good. But by and by, when I had four little daughters round me, and we
were poor, then the old trouble began again; for I am not patient by
nature, and it tried me very much to see my children wanting anything."

"Poor mother! what helped you then?"

"Your father, Jo. He never loses patience,---never doubts or
complains,---but always hopes, and works and waits so cheerfully, that
one is ashamed to do otherwise before him. He helped and comforted me,
and showed me that I must try to practise all the virtues I would have
my little girls possess, for I was their example. It was easier to try
for your sakes than for my own; a startled or surprised look from one of
you, when I spoke sharply, rebuked me more than any words could have
done; and the love, respect, and confidence of my children was the
sweetest reward I could receive for my efforts to be the woman I would
have them copy."

"O mother, if I\textquotesingle m ever half as good as you, I shall be
satisfied," cried Jo, much touched.

"I hope you will be a great deal better, dear; but you must keep watch
over your \textquotesingle bosom enemy,\textquotesingle{} as father
calls it, or it may sadden, if not spoil your life. You have had a
warning; remember it, and try with heart and soul to master this quick
temper, before it brings you greater sorrow and regret than you have
known to-day."

"I will try, mother; I truly will. But you must help me, remind me, and
keep me from flying out. I used to see father sometimes put his finger
on his lips, and look at you with a very kind, but sober face, and you
always folded your lips tight or went away: was he reminding you then?"
asked Jo softly.

"Yes; I asked him to help me so, and he never forgot it, but saved me
from many a sharp word by that little gesture and kind look."

Jo saw that her mother\textquotesingle s eyes filled and her lips
trembled, as she spoke; and, fearing that she had said too much, she
whispered anxiously, "Was it wrong to watch you, and to speak of it? I
didn\textquotesingle t mean to be rude, but it\textquotesingle s so
comfortable to say all I think to you, and feel so safe and happy here."

"My Jo, you may say anything to your mother, for it is my greatest
happiness and pride to feel that my girls confide in me, and know how
much I love them."

"I thought I\textquotesingle d grieved you."

"No, dear; but speaking of father reminded me how much I miss him, how
much I owe him, and how faithfully I should watch and work to keep his
little daughters safe and good for him."

"Yet you told him to go, mother, and didn\textquotesingle t cry when he
went, and never complain now, or seem as if you needed any help," said
Jo, wondering.

"I gave my best to the country I love, and kept my tears till he was
gone. Why should I complain, when we both have merely done our duty and
will surely be the happier for it in the end? If I don\textquotesingle t
seem to need help, it is because I have a better friend, even than
father, to comfort and sustain me. My child, the troubles and
temptations of your life are beginning, and may be many; but you can
overcome and outlive them all if you learn to feel the strength and
tenderness of your Heavenly Father as you do that of your earthly one.
The more you love and trust Him, the nearer you will feel to Him, and
the less you will depend on human power and wisdom. His love and care
never tire or change, can never be taken from you, but may become the
source of life-long peace, happiness, and strength. Believe this
heartily, and go to God with all your little cares, and hopes, and sins,
and sorrows, as freely and confidingly as you come to your mother."

Jo\textquotesingle s only answer was to hold her mother close, and, in
the silence which followed, the sincerest prayer she had ever prayed
left her heart without words; for in that sad, yet happy hour, she had
learned not only the bitterness of remorse and despair, but the
sweetness of self-denial and self-control; and, led by her
mother\textquotesingle s hand, she had drawn nearer to the Friend who
welcomes every child with a love stronger than that of any father,
tenderer than that of any mother.

Amy stirred, and sighed in her sleep; and, as if eager to begin at once
to mend her fault, Jo looked up with an expression on her face which it
had never worn before.

"I let the sun go down on my anger; I wouldn\textquotesingle t forgive
her, and to-day, if it hadn\textquotesingle t been for Laurie, it might
have been too late! How could I be so wicked?" said Jo, half aloud, as
she leaned over her sister, softly stroking the wet hair scattered on
the pillow.

As if she heard, Amy opened her eyes, and held out her arms, with a
smile that went straight to Jo\textquotesingle s heart. Neither said a
word, but they hugged one another close, in spite of the blankets, and
everything was forgiven and forgotten in one hearty kiss.

\begin{center}\rule{0.5\linewidth}{0.5pt}\end{center}

\subsection{IX. Meg goes to Vanity
Fair.}\label{6672479776654687619_37106-h-1.htm.xhtml_pgepubid00011}

\protect\phantomsection\label{6672479776654687619_37106-h-1.htm.xhtml_b049.png}{}
\pandocbounded{\includegraphics[keepaspectratio]{303483661336987339_b049.png}}

\protect\phantomsection\label{6672479776654687619_37106-h-1.htm.xhtml_IX}{}\hyperref[6672479776654687619_37106-h-0.htm.xhtml_contents]{IX.}

MEG GOES TO VANITY FAIR.

"I {do} think it was the most fortunate thing in the world that those
children should have the measles just now," said Meg, one April day, as
she stood packing the "go abroady" trunk in her room, surrounded by her
sisters.

"And so nice of Annie Moffat not to forget her promise. A whole
fortnight of fun will be regularly splendid," replied Jo, looking like a
windmill, as she folded skirts with her long arms.

"And such lovely weather; I\textquotesingle m so glad of that," added
Beth, tidily sorting neck and hair ribbons in her best box, lent for the
great occasion.

"I wish I was going to have a fine time, and wear all these nice
things," said Amy, with her mouth full of pins, as she artistically
replenished her sister\textquotesingle s cushion.

"I wish you were all going; but, as you can\textquotesingle t, I shall
keep my adventures to tell you when I come back. I\textquotesingle m
sure it\textquotesingle s the least I can do, when you have been so
kind, lending me things, and helping me get ready," said Meg, glancing
round the room at the very simple outfit, which seemed nearly perfect in
their eyes.

"What did mother give you out of the treasure-box?" asked Amy, who had
not been present at the opening of a certain cedar chest, in which Mrs.
March kept a few relics of past splendor, as gifts for her girls when
the proper time came.

"A pair of silk stockings, that pretty carved fan, and a lovely blue
sash. I wanted the violet silk; but there isn\textquotesingle t time to
make it over, so I must be contented with my old tarlatan."

"It will look nicely over my new muslin skirt, and the sash will set it
off beautifully. I wish I hadn\textquotesingle t smashed my coral
bracelet, for you might have had it," said Jo, who loved to give and
lend, but whose possessions were usually too dilapidated to be of much
use.

"There is a lovely old-fashioned pearl set in the treasure-box; but
mother said real flowers were the prettiest ornament for a young girl,
and Laurie promised to send me all I want," replied Meg. "Now, let me
see; there\textquotesingle s my new gray walking-suit---just curl up the
feather in my hat, Beth,---then my poplin, for Sunday, and the small
party,---it looks heavy for spring, doesn\textquotesingle t it? The
violet silk would be so nice; oh, dear!"

"Never mind; you\textquotesingle ve got the tarlatan for the big party,
and you always look like an angel in white," said Amy, brooding over the
little store of finery in which her soul delighted.

"It isn\textquotesingle t low-necked, and it doesn\textquotesingle t
sweep enough, but it will have to do. My blue house-dress looks so well,
turned and freshly trimmed, that I feel as if I\textquotesingle d got a
new one. My silk sacque isn\textquotesingle t a bit the fashion, and my
bonnet doesn\textquotesingle t look like Sallie\textquotesingle s; I
didn\textquotesingle t like to say anything, but I was sadly
disappointed in my umbrella. I told mother black, with a white handle,
but she forgot, and bought a green one, with a yellowish handle.
It\textquotesingle s strong and neat, so I ought not to complain, but I
know I shall feel ashamed of it beside Annie\textquotesingle s silk one
with a gold top," sighed Meg, surveying the little umbrella with great
disfavor.

"Change it," advised Jo.

"I won\textquotesingle t be so silly, or hurt Marmee\textquotesingle s
feelings, when she took so much pains to get my things.
It\textquotesingle s a nonsensical notion of mine, and
I\textquotesingle m not going to give up to it. My silk stockings and
two pairs of new gloves are my comfort. You are a dear, to lend me
yours, Jo. I feel so rich, and sort of elegant, with two new pairs, and
the old ones cleaned up for common;" and Meg took a refreshing peep at
her glove-box.

"Annie Moffat has blue and pink bows on her night-caps; would you put
some on mine?" she asked, as Beth brought up a pile of snowy muslins,
fresh from Hannah\textquotesingle s hands.

"No, I wouldn\textquotesingle t; for the smart caps
won\textquotesingle t match the plain gowns, without any trimming on
them. Poor folks shouldn\textquotesingle t rig," said Jo decidedly.

"I wonder if I shall \emph{ever} be happy enough to have real lace on my
clothes, and bows on my caps?" said Meg impatiently.

"You said the other day that you\textquotesingle d be perfectly happy if
you could only go to Annie Moffat\textquotesingle s," observed Beth, in
her quiet way.

"So I did! Well, I \emph{am} happy, and I \emph{won\textquotesingle t}
fret; but it does seem as if the more one gets the more one wants,
doesn\textquotesingle t it? There, now, the trays are ready, and
everything in but my ball-dress, which I shall leave for mother to
pack," said Meg, cheering up, as she glanced from the half-filled trunk
to the many-times pressed and mended white tarlatan, which she called
her "ball-dress," with an important air.

The next day was fine, and Meg departed, in style, for a fortnight of
novelty and pleasure. Mrs. March had consented to the visit rather
reluctantly, fearing that Margaret would come back more discontented
than she went. But she had begged so hard, and Sallie had promised to
take good care of her, and a little pleasure seemed so delightful after
a winter of irksome work, that the mother yielded, and the daughter went
to take her first taste of fashionable life.

The Moffats \emph{were} very fashionable, and simple Meg was rather
daunted, at first, by the splendor of the house and the elegance of its
occupants. But they were kindly people, in spite of the frivolous life
they led, and soon put their guest at her ease. Perhaps Meg felt,
without understanding why, that they were not particularly cultivated or
intelligent people, and that all their gilding could not quite conceal
the ordinary material of which they were made. It certainly was
agreeable to fare sumptuously, drive in a fine carriage, wear her best
frock every day, and do nothing but enjoy herself. It suited her
exactly; and soon she began to imitate the manners and conversation of
those about her; to put on little airs and graces, use French phrases,
crimp her hair, take in her dresses, and talk about the fashions as well
as she could. The more she saw of Annie Moffat\textquotesingle s pretty
things, the more she envied her, and sighed to be rich. Home now looked
bare and dismal as she thought of it, work grew harder than ever, and
she felt that she was a very destitute and much-injured girl, in spite
of the new gloves and silk stockings.

She had not much time for repining, however, for the three young girls
were busily employed in "having a good time." They shopped, walked,
rode, and called all day; went to theatres and operas, or frolicked at
home in the evening; for Annie had many friends, and knew how to
entertain them. Her older sisters were very fine young ladies, and one
was engaged, which was extremely interesting and romantic, Meg thought.
Mr. Moffat was a fat, jolly old gentleman, who knew her father; and Mrs.
Moffat, a fat, jolly old lady, who took as great a fancy to Meg as her
daughter had done. Every one petted her; and "Daisy," as they called
her, was in a fair way to have her head turned.

When the evening for the "small party" came, she found that the poplin
wouldn\textquotesingle t do at all, for the other girls were putting on
thin dresses, and making themselves very fine indeed; so out came the
tarlatan, looking older, limper, and shabbier than ever beside
Sallie\textquotesingle s crisp new one. Meg saw the girls glance at it
and then at one another, and her cheeks began to burn, for, with all her
gentleness, she was very proud. No one said a word about it, but Sallie
offered to dress her hair, and Annie to tie her sash, and Belle, the
engaged sister, praised her white arms; but in their kindness Meg saw
only pity for her poverty, and her heart felt very heavy as she stood by
herself, while the others laughed, chattered, and flew about like gauzy
butterflies. The hard, bitter feeling was getting pretty bad, when the
maid brought in a box of flowers. Before she could speak, Annie had the
cover off, and all were exclaiming at the lovely roses, heath, and fern
within.

"It\textquotesingle s for Belle, of course; George always sends her
some, but these are altogether ravishing," cried Annie, with a great
sniff.

"They are for Miss March, the man said. And here\textquotesingle s a
note," put in the maid, holding it to Meg.

"What fun! Who are they from? Didn\textquotesingle t know you had a
lover," cried the girls, fluttering about Meg in a high state of
curiosity and surprise.

"The note is from mother, and the flowers from Laurie," said Meg simply,
yet much gratified that he had not forgotten her.

"Oh, indeed!" said Annie, with a funny look, as Meg slipped the note
into her pocket, as a sort of talisman against envy, vanity, and false
pride; for the few loving words had done her good, and the flowers
cheered her up by their beauty.

Feeling almost happy again, she laid by a few ferns and roses for
herself, and quickly made up the rest in dainty bouquets for the
breasts, hair, or skirts of her friends, offering them so prettily that
Clara, the elder sister, told her she was "the sweetest little thing she
ever saw;" and they looked quite charmed with her small attention.
Somehow the kind act finished her despondency; and when all the rest
went to show themselves to Mrs. Moffat, she saw a happy, bright-eyed
face in the mirror, as she laid her ferns against her rippling hair, and
fastened the roses in the dress that didn\textquotesingle t strike her
as so \emph{very} shabby now.

She enjoyed herself very much that evening, for she danced to her
heart\textquotesingle s content; every one was very kind, and she had
three compliments. Annie made her sing, and some one said she had a
remarkably fine voice; Major Lincoln asked who "the fresh little girl,
with the beautiful eyes," was; and Mr. Moffat insisted on dancing with
her, because she "didn\textquotesingle t dawdle, but had some spring in
her," as he gracefully expressed it. So, altogether, she had a very nice
time, till she overheard a bit of a conversation, which disturbed her
extremely. She was sitting just inside the conservatory, waiting for her
partner to bring her an ice, when she heard a voice ask, on the other
side of the flowery wall,---

"How old is he?"

"Sixteen or seventeen, I should say," replied another voice.

"It would be a grand thing for one of those girls,
wouldn\textquotesingle t it? Sallie says they are very intimate now, and
the old man quite dotes \ul{on them."}

"Mrs M. has made her plans, I dare say, and will play her cards well,
early as it is. The girl evidently doesn\textquotesingle t think of it
yet," said Mrs. Moffat.

"She told that fib about her mamma, as if she did know, and colored up
when the flowers came, quite prettily. Poor thing! she\textquotesingle d
be so nice if she was only got up in style. Do you think
she\textquotesingle d be offended if we offered to lend her a dress for
Thursday?" \ul{asked another voice.}

"She\textquotesingle s proud, but I don\textquotesingle t believe
she\textquotesingle d mind, for that dowdy tarlatan is all she has got.
She may tear it to-night, and that will be a good excuse for offering a
decent one."

"We\textquotesingle ll see. I shall ask young Laurence, as a compliment
to her, and we\textquotesingle ll have fun about it afterward."

\protect\phantomsection\label{6672479776654687619_37106-h-1.htm.xhtml_b050.png}{}
\pandocbounded{\includegraphics[keepaspectratio]{303483661336987339_b050.png}}

Here Meg\textquotesingle s partner appeared, to find her looking much
flushed and rather agitated. She \emph{was} proud, and her pride was
useful just then, for it helped her hide her mortification, anger, and
disgust at what she had just heard; for, innocent and unsuspicious as
she was, she could not help understanding the gossip of her friends. She
tried to forget it, but could not, and kept repeating to herself, "Mrs.
M. has made her plans," "that fib about her mamma," and "dowdy
tarlatan," till she was ready to cry, and rush home to tell her troubles
and ask for advice. As that was impossible, she did her best to seem
gay; and, being rather excited, she succeeded so well that no one
dreamed what an effort she was making. She was very glad when it was all
over, and she was quiet in her bed, where she could think and wonder and
fume till her head ached and her hot cheeks were cooled by a few natural
tears. Those foolish, yet well-meant words, had opened a new world to
Meg, and much disturbed the peace of the old one, in which, till now,
she had lived as happily as a child. Her innocent friendship with Laurie
was spoilt by the silly speeches she had overheard; her faith in her
mother was a little shaken by the worldly plans attributed to her by
Mrs. Moffat, who judged others by herself; and the sensible resolution
to be contented with the simple wardrobe which suited a poor
man\textquotesingle s daughter, was weakened by the unnecessary pity of
girls who thought a shabby dress one of the greatest calamities under
heaven.

Poor Meg had a restless night, and got up heavy-eyed, unhappy, half
resentful toward her friends, and half ashamed of herself for not
speaking out frankly, and setting everything right. Everybody dawdled
that morning, and it was noon before the girls found energy enough even
to take up their worsted work. Something in the manner of her friends
struck Meg at once; they treated her with more respect, she thought;
took quite a tender interest in what she said, and looked at her with
eyes that plainly betrayed curiosity. All this surprised and flattered
her, though she did not understand it till Miss Belle looked up from her
writing, and said, with a sentimental air,---

"Daisy, dear, I\textquotesingle ve sent an invitation to your friend,
Mr. Laurence, for Thursday. We should like to know him, and
it\textquotesingle s only a proper compliment to you."

Meg colored, but a mischievous fancy to tease the girls made her reply
demurely,---

"You are very kind, but I\textquotesingle m afraid he
won\textquotesingle t come."

"Why not, \emph{chérie}?" asked Miss Belle.

"He\textquotesingle s too old."

"My child, what do you mean? What is his age, I beg to know!" cried Miss
Clara.

"Nearly seventy, I believe," answered Meg, counting stitches, to hide
the merriment in her eyes.

"You sly creature! Of course we meant the young man," exclaimed Miss
Belle, laughing.

"There isn\textquotesingle t any; Laurie is only a little boy," and Meg
laughed also at the queer look which the sisters exchanged as she thus
described her supposed lover.

"About your age," Nan said.

"Nearer my sister Jo\textquotesingle s; \emph{I} am seventeen in
August," returned Meg, tossing her head.

"It\textquotesingle s very nice of him to send you flowers,
isn\textquotesingle t it?" said Annie, looking wise about nothing.

"Yes, he often does, to all of us; for their house is full, and we are
so fond of them. My mother and old Mr. Laurence are friends, you know,
so it is quite natural that we children should play together;" and Meg
hoped they would say no more.

"It\textquotesingle s evident Daisy isn\textquotesingle t out yet," said
Miss Clara to Belle, with a nod.

"Quite a pastoral state of innocence all round," returned Miss Belle,
with a shrug.

"I\textquotesingle m going out to get some little matters for my girls;
can I do anything for you, young ladies?" asked Mrs. Moffat, lumbering
in, like an elephant, in silk and lace.

"No, thank you, ma\textquotesingle am," replied Sallie.
"I\textquotesingle ve got my new pink silk for Thursday, and
don\textquotesingle t want a thing."

"Nor I,---" began Meg, but stopped, because it occurred to her that she
\emph{did} want several things, and could not have them.

"What shall you wear?" asked Sallie.

"My old white one again, if I can mend it fit to be seen; it got sadly
torn last night," said Meg, trying to speak quite easily, but feeling
very uncomfortable.

"Why don\textquotesingle t you send home for another?" said Sallie, who
was not an observing young lady.

"I haven\textquotesingle t got any other." It cost Meg an effort to say
that, but Sallie did not see it, and exclaimed, in amiable surprise,---

"Only that? How funny---" She did not finish her speech, for Belle shook
her head at her, and broke in, saying kindly,---

"Not at all; where is the use of having a lot of dresses when she
isn\textquotesingle t out? There\textquotesingle s no need of sending
home, Daisy, even if you had a dozen, for I\textquotesingle ve got a
sweet blue silk laid away, which I\textquotesingle ve outgrown, and you
shall wear it, to please me, won\textquotesingle t you, dear?"

"You are very kind, but I don\textquotesingle t mind my old dress,
\ul{if you don\textquotesingle t;} it does well enough for a little girl
like me," said Meg.

"Now do let me please myself by dressing you up in style. I admire to do
it, and you\textquotesingle d be a regular little beauty, with a touch
here and there. I sha\textquotesingle n\textquotesingle t let any one
see you till you are done, and then we\textquotesingle ll burst upon
them like Cinderella and her godmother, going to the ball," said Belle,
in her persuasive tone.

Meg couldn\textquotesingle t refuse the offer so kindly made, for a
desire to see if she would be "a little beauty" after touching up,
caused her to accept, and forget all her former uncomfortable feelings
towards the Moffats.

On the Thursday evening, Belle shut herself up with her maid; and,
between them, they turned Meg into a fine lady. They crimped and curled
her hair, they polished her neck and arms with some fragrant powder,
touched her lips with coralline salve, to make them redder, and Hortense
would have added "a \emph{soupçon} of rouge," if Meg had not rebelled.
They laced her into a sky-blue dress, which was so tight she could
hardly breathe, and so low in the neck that modest Meg blushed at
herself in the mirror. A set of silver filagree was added, bracelets,
necklace, brooch, and even ear-rings, for Hortense tied them on, with a
bit of pink silk, which did not show. A cluster of tea-rosebuds at the
bosom, and a \emph{ruche}, reconciled Meg to the display of her pretty
white shoulders, and a pair of high-heeled blue silk boots satisfied the
last wish of her heart. A laced handkerchief, a plumy fan, and a bouquet
in a silver holder finished her off; and Miss Belle surveyed her with
the satisfaction of a little girl with a newly dressed doll.

"Mademoiselle is charmante, très jolie, is she not?" cried Hortense,
clasping her hands in an affected rapture.

"Come and show yourself," said Miss Belle, leading the way to the room
where the others were waiting.

As Meg went rustling after, with her long skirts trailing, her ear-rings
tinkling, her curls waving, and her heart beating, she felt as if her
"fun" had really begun at last, for the mirror had plainly told her that
she \emph{was} "a little beauty." Her friends repeated the pleasing
phrase enthusiastically; and, for several minutes, she stood, like the
jackdaw in the fable, enjoying her borrowed plumes, while the rest
chattered like a party of magpies.

"While I dress, do you drill her, Nan, in the management of her skirt,
and those French heels, or she will trip herself up. Take your silver
butterfly, and catch up that long curl on the left side of her head,
Clara, and don\textquotesingle t any of you disturb the charming work of
my hands," said Belle, as she hurried away, looking well pleased with
her success.

"I\textquotesingle m afraid to go down, I feel so queer and stiff and
half-dressed," said Meg to Sallie, as the bell rang, and Mrs. Moffat
sent to ask the young ladies to appear at once.

"You don\textquotesingle t look a bit like yourself, but you are very
nice. I\textquotesingle m nowhere beside you, for Belle has heaps of
taste, and you\textquotesingle re quite French, I assure you. Let your
flowers hang; don\textquotesingle t be so careful of them, and be sure
you don\textquotesingle t trip," returned Sallie, trying not to care
that Meg was prettier than herself.

\protect\phantomsection\label{6672479776654687619_37106-h-1.htm.xhtml_b051.png}{}
\pandocbounded{\includegraphics[keepaspectratio]{303483661336987339_b051.png}}

Keeping that warning carefully in mind, Margaret got safely down stairs,
and sailed into the drawing-rooms, where the Moffats and a few early
guests were assembled. She very soon discovered that there is a charm
about fine clothes which attracts a certain class of people, and secures
their respect. Several young ladies, who had taken no notice of her
before, were very affectionate all of a sudden; several young gentlemen,
who had only stared at her at the other party, now not only stared, but
asked to be introduced, and said all manner of foolish but agreeable
things to her; and several old ladies, who sat on sofas, and criticised
the rest of the party, inquired who she was, with an air of interest.
She heard Mrs. Moffat reply to one of them,---

"Daisy March---father a colonel in the army---one of our first families,
but reverses of fortune, you know; intimate friends of the Laurences;
sweet creature, I assure you; my Ned is quite wild about her."

"Dear me!" said the old lady, putting up her glass for another
observation of Meg, who tried to look as if she had not heard, and been
rather shocked at Mrs. Moffat\textquotesingle s fibs.

The "queer feeling" did not pass away, but she imagined herself acting
the new part of fine lady, and so got on pretty well, though the tight
dress gave her a side-ache, the train kept getting under her feet, and
she was in constant fear lest her ear-rings should fly off, and get lost
or broken. She was flirting her fan and laughing at the feeble jokes of
a young gentleman who tried to be witty, when she suddenly stopped
laughing and looked confused; for, just opposite, she saw Laurie. He was
staring at her with undisguised surprise, and disapproval also, she
thought; for, though he bowed and smiled, yet something in his honest
eyes made her blush, and wish she had her old dress on. To complete her
confusion, she saw Belle nudge Annie, and both glance from her to
Laurie, who, she was happy to see, looked unusually boyish and shy.

"Silly creatures, to put such thoughts into my head! I
won\textquotesingle t care for it, or let it change me a bit," thought
Meg, and rustled across the room to shake hands with her friend.

"I\textquotesingle m glad you came, I was afraid you
wouldn\textquotesingle t," she said, with her most grown-up air.

"Jo wanted me to come, and tell her how you looked, so I did;" answered
Laurie, without turning his eyes upon her, though he half smiled at her
maternal tone.

"What shall you tell her?" asked Meg, full of curiosity to know his
opinion of her, yet feeling ill at ease with him, for the first time.

"I shall say I didn\textquotesingle t know you; for you look so
grown-up, and unlike yourself, I\textquotesingle m quite afraid of you,"
he said, fumbling at his glove-button.

"How absurd of you! The girls dressed me up for fun, and I rather like
it. Wouldn\textquotesingle t Jo stare if she saw me?" said Meg, bent on
making him say whether he thought her improved or not.

"Yes, I think she would," returned Laurie gravely.

"Don\textquotesingle t you like me so?" asked Meg.

"No, I don\textquotesingle t," was the blunt reply.

"Why not?" in an anxious tone.

He glanced at her frizzled head, bare shoulders, and fantastically
trimmed dress, with an expression that abashed her more than his answer,
which had not a particle of his usual politeness about it.

"I don\textquotesingle t like fuss and feathers."

That was altogether too much from a lad younger than herself; and Meg
walked away, saying petulantly,---

"You are the rudest boy I ever saw."

Feeling very much ruffled, she went and stood at a quiet window, to cool
her cheeks, for the tight dress gave her an uncomfortably brilliant
color. As she stood there, Major Lincoln passed by; and, a minute after,
she heard him saying to his mother,---

"They are making a fool of that little girl; I wanted you to see her,
but they have spoilt her entirely; she\textquotesingle s nothing but a
doll, to-night."

"Oh, dear!" sighed Meg; "I wish I\textquotesingle d been sensible, and
worn my own things; then I should not have disgusted other people, or
felt so uncomfortable and ashamed myself."

She leaned her forehead on the cool pane, and stood half hidden by the
curtains, never minding that her favorite waltz had begun, till some one
touched her; and, turning, she saw Laurie, looking penitent, as he said,
with his very best bow, and his hand out,---

"Please forgive my rudeness, and come and dance with me."

"I\textquotesingle m afraid it will be too disagreeable to you," said
Meg, trying to look offended, and failing entirely.

"Not a bit of it; I\textquotesingle m dying to do it. Come,
I\textquotesingle ll be good; I don\textquotesingle t like your gown,
but I do think you are---just splendid;" and he waved his hands, as if
words failed to express his admiration.

Meg smiled and relented, and whispered, as they stood waiting to catch
the time,---

"Take care my skirt don\textquotesingle t trip you up;
it\textquotesingle s the plague of my life, and I was a goose to wear
it."

"Pin it round your neck, and then it will be useful," said Laurie,
looking down at the little blue boots, which he evidently approved of.

Away they went, fleetly and gracefully; for, having practised at home,
they were well matched, and the blithe young couple were a pleasant
sight to see, as they twirled merrily round and round, feeling more
friendly than ever after their small tiff.

"Laurie, I want you to do me a favor; will you?" said Meg, as he stood
fanning her, when her breath gave out, which it did very soon, though
she would not own why.

"Won\textquotesingle t I!" said Laurie, with alacrity.

"Please don\textquotesingle t tell them at home about my dress to-night.
They won\textquotesingle t understand the joke, and it will worry
mother."

"Then why did you do it?" said Laurie\textquotesingle s eyes, so plainly
that Meg hastily added,---

"I shall tell them, myself, all about it, and
\textquotesingle\textquotesingle fess\textquotesingle{} to mother how
silly I\textquotesingle ve been. But I\textquotesingle d rather do it
myself; so you\textquotesingle ll not tell, will you?"

"I give you my word I won\textquotesingle t; only what shall I say when
they ask me?"

"Just say I looked pretty well, and was having a good time."

"I\textquotesingle ll say the first, with all my heart; but how about
the other? You don\textquotesingle t look as if you were having a good
time; are you?" and Laurie looked at her with an expression which made
her answer, in a whisper,---

"No; not just now. Don\textquotesingle t think I\textquotesingle m
horrid; I only wanted a little fun, but this sort
doesn\textquotesingle t pay, I find, and I\textquotesingle m getting
tired of it."

"Here comes Ned Moffat; what does he want?" said Laurie, knitting his
black brows, as if he did not regard his young host in the light of a
pleasant addition to the party.

"He put his name down for three dances, and I suppose
he\textquotesingle s coming for them. What a bore!" said Meg, assuming a
languid air, which amused Laurie immensely.

He did not speak to her again till supper-time, when he saw her drinking
champagne with Ned and his friend Fisher, who were behaving "like a pair
of fools," as Laurie said to himself, for he felt a brotherly sort of
right to watch over the Marches, and fight their battles whenever a
defender was needed.

\protect\phantomsection\label{6672479776654687619_37106-h-1.htm.xhtml_b052.png}{}
\pandocbounded{\includegraphics[keepaspectratio]{303483661336987339_b052.png}}

"You\textquotesingle ll have a splitting headache to-morrow, if you
drink much of that. I wouldn\textquotesingle t Meg; your mother
doesn\textquotesingle t like it, you know," he whispered, leaning over
her chair, as Ned turned to refill her glass, and Fisher stooped to pick
up her fan.

"I\textquotesingle m not Meg, to-night; I\textquotesingle m
\textquotesingle a doll,\textquotesingle{} who does all sorts of crazy
things. To-morrow I shall put away my \textquotesingle fuss and
feathers,\textquotesingle{} and be desperately good again," she
answered, with an affected little laugh.

"Wish to-morrow was here, then," muttered Laurie, walking off,
ill-pleased at the change he saw in her.

Meg danced and flirted, chattered and giggled, as the other girls did;
after supper she undertook the German, and blundered through it, nearly
upsetting her partner with her long skirt, and romping in a way that
scandalized Laurie, who looked on and meditated a lecture. But he got no
chance to deliver it, for Meg kept away from him till he came to say
good-night.

"Remember!" she said, trying to smile, for the splitting headache had
already begun.

"Silence à la mort," replied Laurie, with a melodramatic flourish, as he
went away.

This little bit of by-play excited Annie\textquotesingle s curiosity;
but Meg was too tired for gossip, and went to bed, feeling as if she had
been to a masquerade, and hadn\textquotesingle t enjoyed herself as much
as she expected. She was sick all the next day, and on Saturday went
home, quite used up with her fortnight\textquotesingle s fun, and
feeling that she had "sat in the lap of luxury" long enough.

"It does seem pleasant to be quiet, and not have company manners on all
the time. Home \emph{is} a nice place, though it isn\textquotesingle t
splendid," said Meg, looking about her with a restful expression, as she
sat with her mother and Jo on the Sunday evening.

"I\textquotesingle m glad to hear you say so, dear, for I was afraid
home would seem dull and poor to you, after your fine quarters," replied
her mother, who had given her many anxious looks that day; for motherly
eyes are quick to see any change in children\textquotesingle s faces.

Meg had told her adventures gayly, and said over and over what a
charming time she had had; but something still seemed to weigh upon her
spirits, and, when the younger girls were gone to bed, she sat
thoughtfully staring at the fire, saying little, and looking worried. As
the clock struck nine, and Jo proposed bed, Meg suddenly left her chair,
and, taking Beth\textquotesingle s stool, leaned her elbows on her
mother\textquotesingle s knee, saying bravely,---

"Marmee, I want to
\textquotesingle\textquotesingle fess.\textquotesingle"

"I thought so; what is it, dear?"

"Shall I go away?" asked Jo discreetly.

"Of course not; don\textquotesingle t I always tell you everything? I
was ashamed to speak of it before the children, but I want you to know
all the dreadful things I did at the Moffat\textquotesingle s."

"We are prepared," said Mrs. March, smiling, but looking a little
anxious.

"I told you they dressed me up, but I didn\textquotesingle t tell you
that they powdered and squeezed and frizzled, and made me look like a
fashion-plate. Laurie thought I wasn\textquotesingle t proper; I know he
did, though he didn\textquotesingle t say so, and one man called me
\textquotesingle a doll.\textquotesingle{} I knew it was silly, but they
flattered me, and said I was a beauty, and quantities of nonsense, so I
let them make a fool of me."

"Is that all?" asked Jo, as Mrs. March looked silently at the downcast
face of her pretty daughter, and could not find it in her heart to blame
her little follies.

"No; I drank champagne and romped and tried to flirt, and was altogether
abominable," said Meg self-reproachfully.

"There is something more, I think;" and Mrs. March smoothed the soft
cheek, which suddenly grew rosy, as Meg answered slowly,---

"Yes; it\textquotesingle s very silly, but I want to tell it, because I
hate to have people say and think such things about us and Laurie."

Then she told the various bits of gossip she had heard at the Moffats;
and, as she spoke, Jo saw her mother fold her lips tightly, as if ill
pleased that such ideas should be put into Meg\textquotesingle s
innocent mind.

"Well, if that isn\textquotesingle t the greatest rubbish I ever heard,"
cried Jo indignantly. "Why didn\textquotesingle t you pop out and tell
them so, on the spot?"

"I couldn\textquotesingle t, it was so embarrassing for me. I
couldn\textquotesingle t help hearing, at first, and then I was so angry
and ashamed, I didn\textquotesingle t remember that I ought to go away."

"Just wait till \emph{I} see Annie Moffat, and I\textquotesingle ll show
you how to settle such ridiculous stuff. The idea of having
\textquotesingle plans,\textquotesingle{} and being kind to Laurie,
because he\textquotesingle s rich, and may marry us by and by!
Won\textquotesingle t he shout, when I tell him what those silly things
say about us poor children?" and Jo laughed, as if, on second thoughts,
the thing struck her as a good joke.

"If you tell Laurie, I\textquotesingle ll never forgive you! She
mustn\textquotesingle t, must she, mother?" said Meg, looking
distressed.

"No; never repeat that foolish gossip, and forget it as soon as you
can," said Mrs. March gravely. "I was very unwise to let you go among
people of whom I know so little,---kind, I dare say, but worldly,
ill-bred, and full of these vulgar ideas about young people. I am more
sorry than I can express for the mischief this visit may have done you,
Meg."

"Don\textquotesingle t be sorry, I won\textquotesingle t let it hurt me;
I\textquotesingle ll forget all the bad, and remember only the good; for
I did enjoy a great deal, and thank you very much for letting me go.
I\textquotesingle ll not be sentimental or dissatisfied, mother; I know
I\textquotesingle m a silly little girl, and I\textquotesingle ll stay
with you till I\textquotesingle m fit to take care of myself. But it
\emph{is} nice to be praised and admired, and I can\textquotesingle t
help saying I like it," said Meg, looking half ashamed of the
confession.

"That is perfectly natural, and quite harmless, if the liking does not
become a passion, and lead one to do foolish or unmaidenly things. Learn
to know and value the praise which is worth having, and to excite the
admiration of excellent people by being modest as well as pretty, Meg."

Margaret sat thinking a moment, while Jo stood with her hands behind
her, looking both interested and a little perplexed; for it was a new
thing to see Meg blushing and talking about admiration, lovers, and
things of that sort; and Jo felt as if, during that fortnight, her
sister had grown up amazingly, and was drifting away from her into a
world where she could not follow.

"Mother, do you have \textquotesingle plans,\textquotesingle{} as Mrs.
Moffat said?" asked Meg bashfully.

"Yes, my dear, I have a great many; all mothers do, but mine differ
somewhat from Mrs. Moffat\textquotesingle s, I suspect. I will tell you
some of them, for the time has come when a word may set this romantic
little head and heart of yours right, on a very serious subject. You are
young, Meg, but not too young to understand me; and
mothers\textquotesingle{} lips are the fittest to speak of such things
to girls like you. Jo, your turn will come in time, perhaps, so listen
to my \textquotesingle plans,\textquotesingle{} and help me carry them
out, if they are good."

Jo went and sat on one arm of the chair, looking as if she thought they
were about to join in some very solemn affair. Holding a hand of each,
and watching the two young faces wistfully, Mrs. March said, in her
serious yet cheery way,---

\protect\phantomsection\label{6672479776654687619_37106-h-1.htm.xhtml_b053.png}{}
\pandocbounded{\includegraphics[keepaspectratio]{303483661336987339_b053.png}}

"I want my daughters to be beautiful, accomplished, and good; to be
admired, loved, and respected; to have a happy youth, to be well and
wisely married, and to lead useful, pleasant lives, with as little care
and sorrow to try them as God sees fit to send. To be loved and chosen
by a good man is the best and sweetest thing which can happen to a
woman; and I sincerely hope my girls may know this beautiful experience.
It is natural to think of it, Meg; right to hope and wait for it, and
wise to prepare for it; so that, when the happy time comes, you may feel
ready for the duties and worthy of the joy. My dear girls, I \emph{am}
ambitious for you, but not to have you make a dash in the world,---marry
rich men merely because they are rich, or have splendid houses, which
are not homes because love is wanting. Money is a needful and precious
thing,---and, when well used, a noble thing,---but I never want you to
think it is the first or only prize to strive for. I\textquotesingle d
rather see you poor men\textquotesingle s wives, if you were happy,
beloved, contented, than queens on thrones, without self-respect and
peace."

"Poor girls don\textquotesingle t stand any chance, Belle says, unless
they put themselves forward," sighed Meg.

"Then we\textquotesingle ll be old maids," said Jo stoutly.

"Right, Jo; better be happy old maids than unhappy wives, or unmaidenly
girls, running about to find husbands," said Mrs. March decidedly.
"Don\textquotesingle t be troubled, Meg; poverty seldom daunts a sincere
lover. Some of the best and most honored women I know were poor girls,
but so love-worthy that they were not allowed to be old maids. Leave
these things to time; make this home happy, so that you may be fit for
homes of your own, if they are offered you, and contented here if they
are not. One thing remember, my girls: mother is always ready to be your
confidant, father to be your friend; and both of us trust and hope that
our daughters, whether married or single, will be the pride and comfort
of our lives."

"We will, Marmee, we will!" cried both, with all their hearts, as she
bade them good-night.

\begin{center}\rule{0.5\linewidth}{0.5pt}\end{center}

\subsection{X. The P. C. and P.
O.}\label{6672479776654687619_37106-h-1.htm.xhtml_pgepubid00012}

\protect\phantomsection\label{6672479776654687619_37106-h-1.htm.xhtml_X}{}\hyperref[6672479776654687619_37106-h-0.htm.xhtml_contents]{X.}

THE P. C. AND P. O.

{As} spring came on, a new set of amusements became the fashion, and the
lengthening days gave long afternoons for work and play of all sorts.
The garden had to be put in order, and each sister had a quarter of the
little plot to do what she liked with. Hannah used to say,
"I\textquotesingle d know which each of them gardings belonged to, ef I
see \textquotesingle em in Chiny;" and so she might, for the
girls\textquotesingle{} tastes differed as much as their characters.
Meg\textquotesingle s had roses and heliotrope, myrtle, and a little
orange-tree in it. Jo\textquotesingle s bed was never alike two seasons,
for she was always trying experiments; this year it was to be a
plantation of sun-flowers, the seeds of which cheerful and aspiring
plant were to feed "Aunt Cockle-top" and her family of chicks. Beth had
old-fashioned, fragrant flowers in her garden,---sweet peas and
mignonette, larkspur, pinks, pansies, and southernwood, with chickweed
for the bird, and catnip for the pussies. Amy had a bower in
hers,---rather small and earwiggy, but very pretty to look at,---with
honeysuckles and morning-glories hanging their colored horns and bells
in graceful wreaths all over it; tall, white lilies, delicate ferns, and
as many brilliant, picturesque plants as would consent to blossom there.

Gardening, walks, rows on the river, and flower-hunts employed the fine
days; and for rainy ones, they had house diversions,---some old, some
new,---all more or less original. One of these was the "P. C."; for, as
secret societies were the fashion, it was thought proper to have one;
and, as all of the girls admired Dickens, they called themselves the
Pickwick Club. With a few interruptions, they had kept this up for a
year, and met every Saturday evening in the big garret, on which
occasions the ceremonies were as follows: Three chairs were arranged in
a row before a table, on which was a lamp, also four white badges, with
a big "P. C." in different colors on each, and the weekly newspaper,
called "The Pickwick Portfolio," to which all contributed something;
while Jo, who revelled in pens and ink, was the editor. At seven
o\textquotesingle clock, the four members ascended to the club-room,
tied their badges round their heads, and took their seats with great
solemnity. Meg, as the eldest, was Samuel Pickwick; Jo, being of a
literary turn, Augustus Snodgrass; Beth, because she was round and rosy,
Tracy Tupman, and Amy, who was always trying to do what she
couldn\textquotesingle t, was Nathaniel Winkle. Pickwick, the president,
read the paper, which was filled with original tales, poetry, local
news, funny advertisements, and hints, in which they good-naturedly
reminded each other of their faults and short-comings.

\protect\phantomsection\label{6672479776654687619_37106-h-1.htm.xhtml_b054.png}{}
\pandocbounded{\includegraphics[keepaspectratio]{303483661336987339_b054.png}}

On one occasion, Mr. Pickwick put on a pair of spectacles without any
glasses, rapped upon the table, hemmed, and, having stared hard at Mr.
Snodgrass, who was tilting back in his chair, till he arranged himself
properly, began to read:---

\begin{center}\rule{0.5\linewidth}{0.5pt}\end{center}

"The Pickwick Portfolio."

MAY 20, 18---

\begin{center}\rule{0.5\linewidth}{0.5pt}\end{center}

{Poet\textquotesingle s Corner.}

\begin{center}\rule{0.5\linewidth}{0.5pt}\end{center}

ANNIVERSARY ODE.

\begin{center}\rule{0.5\linewidth}{0.5pt}\end{center}

\protect\phantomsection\label{6672479776654687619_37106-h-2.htm.xhtml}{}

Again we meet to celebrate

With badge and solemn rite,

Our fifty-second anniversary,

In Pickwick Hall, to-night.

\hfill\break

We all are here in perfect health,

None gone from our small band;

Again we see each well-known face,

And press each friendly hand.

\hfill\break

Our Pickwick, always at his post,

With reverence we greet,

As, spectacles on nose, he reads

Our well-filled weekly sheet.

\hfill\break

Although he suffers from a cold,

We joy to hear him speak,

For words of wisdom from him fall,

In spite of croak or squeak.

\hfill\break

Old six-foot Snodgrass looms on high,

With elephantine grace,

And beams upon the company,

With brown and jovial face.

\hfill\break

Poetic fire lights up his eye,

He struggles \textquotesingle gainst his lot.

Behold ambition on his brow,

And on his nose a blot!

\hfill\break

Next our peaceful Tupman comes,

So rosy, plump, and sweet.

Who chokes with laughter at the puns,

And tumbles off his seat.

\hfill\break

Prim little Winkle too is here,

With every hair in place,

A model of propriety,

Though he hates to wash his face.

\hfill\break

The year is gone, we still unite

To joke and laugh and read,

And tread the path of literature

That doth to glory lead.

\hfill\break

Long may our paper prosper well,

Our club unbroken be,

And coming years their blessings pour

On the useful, gay "P. C."

A. Snodgrass.

\begin{center}\rule{0.5\linewidth}{0.5pt}\end{center}

THE MASKED MARRIAGE.

A TALE OF VENICE.

\begin{center}\rule{0.5\linewidth}{0.5pt}\end{center}

Gondola after gondola swept up to the marble steps, and left its lovely
load to swell the brilliant throng that filled the stately halls of
Count de Adelon. Knights and ladies, elves and pages, monks and
flower-girls, all mingled gayly in the dance. Sweet voices and rich
melody filled the air; and so with mirth and music the masquerade went
on.

"Has your Highness seen the Lady Viola to-night?" asked a gallant
troubadour of the fairy queen who floated down the hall upon his arm.

"Yes; is she not lovely, though so sad! Her dress is well chosen, too,
for in a week she weds Count Antonio, whom she passionately hates."

"By my faith, I envy him. Yonder he comes, arrayed like a bridegroom,
except the black mask. When that is off we shall see how he regards the
fair maid whose heart he cannot win, though her stern father bestows her
hand," returned the troubadour.

"\textquotesingle Tis whispered that she loves the young English artist
who haunts her steps, and is spurned by the old count," said the lady,
as they joined the dance.

The revel was at its height when a priest appeared, and, withdrawing the
young pair to an alcove hung with purple velvet, he motioned them to
kneel. Instant silence fell upon the gay throng; and not a sound, but
the dash of fountains or the rustle of orange-groves sleeping in the
moonlight, broke the hush, as Count de Adelon spoke thus:---

"My lords and ladies, pardon the ruse by which I have gathered you here
to witness the marriage of my daughter. Father, we wait your services."

All eyes turned toward the bridal party, and a low murmur of amazement
went through the throng, for neither bride nor groom removed their
masks. Curiosity and wonder possessed all hearts, but respect restrained
all tongues till the holy rite was over. Then the eager spectators
gathered round the count, demanding an explanation.

"Gladly would I give it if I could; but I only know that it was the whim
of my timid Viola, and I yielded to it. Now, my children, let the play
end. Unmask, and receive my blessing."

But neither bent the knee; for the young bridegroom replied, in a tone
that startled all listeners, as the mask fell, disclosing the noble face
of Ferdinand Devereux, the artist lover; and, leaning on the breast
where now flashed the star of an English earl, was the lovely Viola,
radiant with joy and beauty.

"My lord, you scornfully bade me claim your daughter when I could boast
as high a name and vast a fortune as the Count Antonio. I can do more;
for even your ambitious soul cannot refuse the Earl of Devereux and De
Vere, when he gives his ancient name and boundless wealth in return for
the beloved hand of this fair lady, now my wife."

The count stood like one changed to stone; and, turning to the
bewildered crowd, Ferdinand added, with a gay smile of triumph, "To you,
my gallant friends, I can only wish that your wooing may prosper as mine
has done; and that you may all win as fair a bride as I have, by this
masked marriage."

S. Pickwick.

\begin{center}\rule{0.5\linewidth}{0.5pt}\end{center}

Why is the P. C. like the Tower of Babel? It is full of unruly members.

\begin{center}\rule{0.5\linewidth}{0.5pt}\end{center}

THE HISTORY OF A SQUASH.

\begin{center}\rule{0.5\linewidth}{0.5pt}\end{center}

Once upon a time a farmer planted a little seed in his garden, and after
a while it sprouted and became a vine, and bore many squashes. One day
in October, when they were ripe, he picked one and took it to market. A
grocer-man bought and put it in his shop. That same morning, a little
girl, in a brown hat and blue dress, with a round face and snub nose,
went and bought it for her mother. She lugged it home, cut it up, and
boiled it in the big pot; mashed some of it, with salt and butter, for
dinner; and to the rest she added a pint of milk, two eggs, four spoons
of sugar, nutmeg, and some crackers; put it in a deep dish, and baked it
till it was brown and nice; and next day it was eaten by a family named
March.

{T. Tupman.}

\begin{center}\rule{0.5\linewidth}{0.5pt}\end{center}

{Mr. Pickwick}, \emph{Sir}:---

I address you upon the subject of sin the sinner I mean is a man named
Winkle who makes trouble in his club by laughing and sometimes
won\textquotesingle t write his piece in this fine paper I hope you will
pardon his badness and let him send a French fable because he
can\textquotesingle t write out of his head as he has so many lessons to
do and no brains in future I will try to take time by the fetlock and
prepare some work which will be all \emph{commy la fo} that means all
right I am in haste as it is nearly school time

Yours respectably,

N. Winkle.

{[}The above is a manly and handsome acknowledgment of past
misdemeanors. If our young friend studied punctuation, it would be
well.{]}

\begin{center}\rule{0.5\linewidth}{0.5pt}\end{center}

A SAD ACCIDENT.

\begin{center}\rule{0.5\linewidth}{0.5pt}\end{center}

On Friday last, we were startled by a violent shock in our basement,
followed by cries of distress. On rushing, in a body, to the cellar, we
discovered our beloved President prostrate upon the floor, having
tripped and fallen while getting wood for domestic purposes. A perfect
scene of ruin met our eyes; for in his fall Mr. Pickwick had plunged his
head and shoulders into a tub of water, upset a keg of soft soap upon
his manly form, and torn his garments badly. On being removed from this
perilous situation, it was discovered that he had suffered no injury but
several bruises; and, we are happy to add, is now doing well.

{Ed.}

\hfill\break

THE PUBLIC BEREAVEMENT.

\begin{center}\rule{0.5\linewidth}{0.5pt}\end{center}

It is our painful duty to record the sudden and mysterious disappearance
of our cherished friend, Mrs. Snowball Pat Paw. This lovely and beloved
cat was the pet of a large circle of warm and admiring friends; for her
beauty attracted all eyes, her graces and virtues endeared her to all
hearts, and her loss is deeply felt by the whole community.

When last seen, she was sitting at the gate, watching the
butcher\textquotesingle s cart; and it is feared that some villain,
tempted by her charms, basely stole her. Weeks have passed, but no trace
of her has been discovered; and we relinquish all hope, tie a black
ribbon to her basket, set aside her dish, and weep for her as one lost
to us forever.

\begin{center}\rule{0.5\linewidth}{0.5pt}\end{center}

A sympathizing friend sends the following gem:---

A LAMENT

FOR S.~B. PAT PAW.

\begin{center}\rule{0.5\linewidth}{0.5pt}\end{center}

We mourn the loss of our little pet,

And sigh o\textquotesingle er her hapless fate,

For never more by the fire she\textquotesingle ll sit,

Nor play by the old green gate.

\hfill\break

The little grave where her infant sleeps,

Is \textquotesingle neath the chestnut tree;

But o\textquotesingle er \emph{her} grave we may not weep,

We know not where it may be.

\hfill\break

Her empty bed, her idle ball,

Will never see her more;

No gentle tap, no loving purr

Is heard at the parlor-door.

\hfill\break

Another cat comes after her mice,

A cat with a dirty face;

But she does not hunt as our darling did,

Nor play with her airy grace.

\hfill\break

Her stealthy paws tread the very hall

Where Snowball used to play,

But she only spits at the dogs our pet

So gallantly drove away.

\hfill\break

She is useful and mild, and does her best,

But she is not fair to see;

And we cannot give her your place, dear,

Nor worship her as we worship thee.

A. S.

\begin{center}\rule{0.5\linewidth}{0.5pt}\end{center}

ADVERTISEMENTS.

\begin{center}\rule{0.5\linewidth}{0.5pt}\end{center}

{Miss Oranthy Bluggage}, the accomplished Strong-Minded Lecturer, will
deliver her famous Lecture on "{Woman and Her Position}," at Pickwick
Hall, next Saturday Evening, after the usual performances.

\begin{center}\rule{0.5\linewidth}{0.5pt}\end{center}

{A Weekly Meeting} will be held at Kitchen Place, to teach young ladies
how to cook. Hannah Brown will preside; and all are invited to attend.

\begin{center}\rule{0.5\linewidth}{0.5pt}\end{center}

{The Dustpan Society} will meet on Wednesday next, and parade in the
upper story of the Club House. All members to appear in uniform and
shoulder their brooms at nine precisely.

\begin{center}\rule{0.5\linewidth}{0.5pt}\end{center}

{Mrs. Beth Bouncer} will open her new assortment of
Doll\textquotesingle s Millinery next week. The latest Paris Fashions
have arrived, and orders are respectfully solicited.

\begin{center}\rule{0.5\linewidth}{0.5pt}\end{center}

{A New Play} will appear at the Barnville Theatre, in the course of a
few weeks, which will surpass anything ever seen on the American stage.
"{The Greek Slave}, or Constantine the Avenger," is the name of this
thrilling drama!!!

\begin{center}\rule{0.5\linewidth}{0.5pt}\end{center}

HINTS.

If S. P. didn\textquotesingle t use so much soap on his hands, he
wouldn\textquotesingle t always be late at breakfast. A. S. is requested
not to whistle in the street. T. T. please don\textquotesingle t forget
Amy\textquotesingle s napkin. N. W. must not fret because his dress has
not nine tucks.

\begin{center}\rule{0.5\linewidth}{0.5pt}\end{center}

WEEKLY REPORT.

Meg---Good.\\
Jo---Bad.\\
Beth---Very good.\\
Amy---Middling.

\begin{center}\rule{0.5\linewidth}{0.5pt}\end{center}

As the President finished reading the paper (which I beg leave to assure
my readers is a \emph{bona fide} copy of one written by \emph{bona fide}
girls once upon a time), a round of applause followed, and then Mr.
Snodgrass rose to make a proposition.

"Mr. President and gentlemen," he began, assuming a parliamentary
attitude and tone, "I wish to propose the admission of a new
member,---one who highly deserves the honor, would be deeply grateful
for it, and would add immensely to the spirit of the club, the literary
value of the paper, and be no end jolly and nice. I propose Mr. Theodore
Laurence as an honorary member of the P. C. Come now, do have him."

Jo\textquotesingle s sudden change of tone made the girls laugh; but all
looked rather anxious, and no one said a word, as Snodgrass took his
seat.

"We\textquotesingle ll put it to vote," said the President. "All in
favor of this motion please to manifest it by saying
\textquotesingle Ay.\textquotesingle"

A loud response from Snodgrass, followed, to everybody\textquotesingle s
surprise, by a timid one from Beth.

"Contrary minded say \textquotesingle No.\textquotesingle"

Meg and Amy were contrary minded; and Mr. Winkle rose to say, with great
elegance, "We don\textquotesingle t wish any boys; they only joke and
bounce about. This is a ladies\textquotesingle{} club, and we wish to be
private and proper."

"I\textquotesingle m afraid he\textquotesingle ll laugh at our paper,
and make fun of us afterward," observed Pickwick, pulling the little
curl on her forehead, as she always did when doubtful.

Up rose Snodgrass, very much in earnest. "Sir, I give you my word as a
gentleman, Laurie won\textquotesingle t do anything of the sort. He
likes to write, and he\textquotesingle ll give a tone to our
contributions, and keep us from being sentimental, don\textquotesingle t
you see? We can do so little for him, and he does so much for us, I
think the least we can do is to offer him a place here, and make him
welcome if he comes."

This artful allusion to benefits conferred brought Tupman to his feet,
looking as if he had quite made up his mind.

"Yes, we ought to do it, even if we \emph{are} afraid. I say he
\emph{may} come, and his grandpa, too, if he likes."

This spirited burst from Beth electrified the club, and Jo left her seat
to shake hands approvingly. "Now then, vote again. Everybody remember
it\textquotesingle s our Laurie, and say
\textquotesingle Ay!\textquotesingle" cried Snodgrass excitedly.

"Ay! ay! ay!" replied three voices at once.

"Good! Bless you! Now, as there\textquotesingle s nothing like
\textquotesingle taking time by the \emph{fetlock},\textquotesingle{} as
Winkle characteristically observes, allow me to present the new member;"
and, to the dismay of the rest of the club, Jo threw open the door of
the closet, and displayed Laurie sitting on a rag-bag, flushed and
twinkling with suppressed laughter.

\protect\phantomsection\label{6672479776654687619_37106-h-2.htm.xhtml_b055.png}{}
\pandocbounded{\includegraphics[keepaspectratio]{303483661336987339_b055.png}}

"You rogue! you traitor! Jo, how could you?" cried the three girls, as
Snodgrass led her friend triumphantly forth; and, producing both a chair
and a badge, installed him in a jiffy.

"The coolness of you two rascals is amazing," began Mr. Pickwick, trying
to get up an awful frown, and only succeeding in producing an amiable
smile. But the new member was equal to the occasion; and, rising, with a
grateful salutation to the Chair, said, in the most engaging manner,
"Mr. President and ladies,---I beg pardon, gentlemen,---allow me to
introduce myself as Sam Weller, the very humble servant of the club."

"Good! good!" cried Jo, pounding with the handle of the old warming-pan
on which she leaned.

"My faithful friend and noble patron," continued Laurie, with a wave of
the hand, "who has so flatteringly presented me, is not to be blamed for
the base stratagem of to-night. I planned it, and she only gave in after
lots of teasing."

"Come now, don\textquotesingle t lay it all on yourself; you know I
proposed the cupboard," broke in Snodgrass, who was enjoying the joke
amazingly.

"Never you mind what she says. I\textquotesingle m the wretch that did
it, sir," said the new member, with a Welleresque nod to Mr. Pickwick.
"But on my honor, I never will do so again, and henceforth \emph{dewote}
myself to the interest of this immortal club."

"Hear! hear!" cried Jo, clashing the lid of the warming-pan like a
cymbal.

"Go on, go on!" added Winkle and Tupman, while the President bowed
benignly.

"I merely wish to say, that as a slight token of my gratitude for the
honor done me, and as a means of promoting friendly relations between
adjoining nations, I have set up a post-office in the hedge in the lower
corner of the garden; a fine, spacious building, with padlocks on the
doors, and every convenience for the mails,---also the females, if I may
be allowed the expression. It\textquotesingle s the old martin-house;
but I\textquotesingle ve stopped up the door, and made the roof open, so
it will hold all sorts of things, and save our valuable time. Letters,
manuscripts, books, and bundles can be passed in there; and, as each
nation has a key, it will be uncommonly nice, I fancy. Allow me to
present the club key; and, with many thanks for your favor, take my
seat."

Great applause as Mr. Weller deposited a little key on the table, and
subsided; the warming-pan clashed and waved wildly, and it was some time
before order could be restored. A long discussion followed, and every
one came out surprising, for every one did her best; so it was an
unusually lively meeting, and did not adjourn till a late hour, when it
broke up with three shrill cheers for the new member.

No one ever regretted the admittance of Sam Weller, for a more devoted,
well-behaved, and jovial member no club could have. He certainly did add
"spirit" to the meetings, and "a tone" to the paper; for his orations
convulsed his hearers, and his contributions were excellent, being
patriotic, classical, comical, or dramatic, but never sentimental. Jo
regarded them as worthy of Bacon, Milton, or Shakespeare; and remodelled
her own works with good effect, she thought.

The P. O. was a capital little institution, and flourished wonderfully,
for nearly as many queer things passed through it as through the real
office. Tragedies and cravats, poetry and pickles, garden-seeds and long
letters, music and gingerbread, rubbers, invitations, scoldings and
puppies. The old gentleman liked the fun, and amused himself by sending
odd bundles, mysterious messages, and funny telegrams; and his gardener,
who was smitten with Hannah\textquotesingle s charms, actually sent a
love-letter to Jo\textquotesingle s care. How they laughed when the
secret came out, never dreaming how many love-letters that little
post-office would hold in the years to come!

\begin{center}\rule{0.5\linewidth}{0.5pt}\end{center}

\subsection{XI.
Experiments.}\label{6672479776654687619_37106-h-2.htm.xhtml_pgepubid00013}

\protect\phantomsection\label{6672479776654687619_37106-h-2.htm.xhtml_b056.png}{}
\pandocbounded{\includegraphics[keepaspectratio]{303483661336987339_b056.png}}

\protect\phantomsection\label{6672479776654687619_37106-h-2.htm.xhtml_XI}{}\hyperref[6672479776654687619_37106-h-0.htm.xhtml_contents]{XI.}

EXPERIMENTS.

"{The} first of June! The Kings are off to the seashore to-morrow, and
I\textquotesingle m free. Three months\textquotesingle{} vacation,---how
I shall enjoy it!" exclaimed Meg, coming home one warm day to find Jo
laid upon the sofa in an unusual state of exhaustion, while Beth took
off her dusty boots, and Amy made lemonade for the refreshment of the
whole party.

"Aunt March went to-day, for which, oh, be joyful!" said Jo. "I was
mortally afraid she\textquotesingle d ask me to go with her; if she had,
I should have felt as if I ought to do it; but Plumfield is about as gay
as a churchyard, you know, and I\textquotesingle d rather be excused. We
had a flurry getting the old lady off, and I had a fright every time she
spoke to me, for I was in such a hurry to be through that I was
uncommonly helpful and sweet, and feared she\textquotesingle d find it
impossible to part from me. I quaked till she was fairly in the
carriage, and had a final fright, for, as it drove off, she popped out
her head, saying, \textquotesingle Josy-phine, won\textquotesingle t
you---?\textquotesingle{} I didn\textquotesingle t hear any more, for I
basely turned and fled; I did actually run, and whisked round the
corner, where I felt safe."

"Poor old Jo! she came in looking as if bears were after her," said
Beth, as she cuddled her sister\textquotesingle s feet with a motherly
air.

"Aunt March is a regular samphire, is she not?" observed Amy, tasting
her mixture critically.

"She means \emph{vampire}, not sea-weed; but it doesn\textquotesingle t
matter; it\textquotesingle s too warm to be particular about
one\textquotesingle s parts of speech," murmured Jo.

"What shall you do all your vacation?" asked Amy, changing the subject,
with tact.

"I shall lie abed late, and do nothing," replied Meg, from the depths of
the rocking-chair. "I\textquotesingle ve been routed up early all
winter, and had to spend my days working for other people; so now
I\textquotesingle m going to rest and revel to my
heart\textquotesingle s content."

"No," said Jo; "that dozy way wouldn\textquotesingle t suit me.
I\textquotesingle ve laid in a heap of books, and I\textquotesingle m
going to improve my shining hours reading on my perch in the old
apple-tree, when I\textquotesingle m not having l---------"

"Don\textquotesingle t say \textquotesingle larks!\textquotesingle"
implored Amy, as a return snub for the "samphire" correction.

"I\textquotesingle ll say
\textquotesingle nightingales,\textquotesingle{} then, with Laurie;
that\textquotesingle s proper and appropriate, since
he\textquotesingle s a warbler."

"Don\textquotesingle t let us do any lessons, Beth, for a while, but
play all the time, and rest, as the girls mean to," proposed Amy.

"Well, I will, if mother doesn\textquotesingle t mind. I want to learn
some new songs, and my children need fitting up for the summer; they are
dreadfully out of order, and really suffering for clothes."

"May we, mother?" asked Meg, turning to Mrs. March, who sat sewing, in
what they called "Marmee\textquotesingle s corner."

"You may try your experiment for a week, and see how you like it. I
think by Saturday night you will find that all play and no work is as
bad as all work and no play."

\ul{"Oh, dear, no!} it will be delicious, I\textquotesingle m sure,"
said Meg complacently.

"I now propose a toast, as my \textquotesingle friend and pardner, Sairy
Gamp,\textquotesingle{} says. Fun forever, and no grubbing!" cried Jo,
rising, glass in hand, as the lemonade went round.

\protect\phantomsection\label{6672479776654687619_37106-h-2.htm.xhtml_b057.png}{}
\pandocbounded{\includegraphics[keepaspectratio]{303483661336987339_b057.png}}

They all drank it merrily, and began the experiment by lounging for the
rest of the day. Next morning, Meg did not appear till ten
o\textquotesingle clock; her solitary breakfast did not taste nice, and
the room seemed lonely and untidy; for Jo had not filled the vases, Beth
had not dusted, and Amy\textquotesingle s books lay scattered about.
Nothing was neat and pleasant but "Marmee\textquotesingle s corner,"
which looked as usual; and there Meg sat, to "rest and read," which
meant yawn, and imagine what pretty summer dresses she would get with
her salary. Jo spent the morning on the river, with Laurie, and the
afternoon reading and crying over "The Wide, Wide World," up in the
apple-tree. Beth began by rummaging everything out of the big closet,
where her family resided; but, getting tired before half done, she left
her establishment topsy-turvy, and went to her music, rejoicing that she
had no dishes to wash. Amy arranged her bower, put on her best white
frock, smoothed her curls, and sat down to draw, under the honeysuckles,
hoping some one would see and inquire who the young artist was. As no
one appeared but an inquisitive daddy-long-legs, who examined her work
with interest, she went to walk, got caught in a shower, and came home
dripping.

At tea-time they compared notes, and all agreed that it had been a
delightful, though unusually long day. Meg, who went shopping in the
afternoon, and got a "sweet blue muslin," had discovered, after she had
cut the breadths off, that it wouldn\textquotesingle t wash, which
mishap made her slightly cross. Jo had burnt the skin off her nose
boating, and got a raging headache by reading too long. Beth was worried
by the confusion of her closet, and the difficulty of learning three or
four songs at once; and Amy deeply regretted the damage done her frock,
for Katy Brown\textquotesingle s party was to be the next day; and now,
like Flora McFlimsey, she had "nothing to wear." But these were mere
trifles; and they assured their mother that the experiment was working
finely. She smiled, said nothing, and, with Hannah\textquotesingle s
help, did their neglected work, keeping home pleasant, and the domestic
machinery running smoothly. It was astonishing what a peculiar and
uncomfortable state of things was produced by the "resting and
revelling" process. The days kept getting longer and longer; the weather
was unusually variable, and so were tempers; an unsettled feeling
possessed every one, and Satan found plenty of mischief for the idle
hands to do. As the height of luxury, Meg put out some of her sewing,
and then found time hang so heavily that she fell to snipping and
spoiling her clothes, in her attempts to furbish them up à la Moffat. Jo
read till her eyes gave out, and she was sick of books; got so fidgety
that even good-natured Laurie had a quarrel with her, and so reduced in
spirits that she desperately wished she had gone with Aunt March. Beth
got on pretty well, for she was constantly forgetting that it was to be
\emph{all play, and no work}, and fell back into her old ways now and
then; but something in the air affected her, and, more than once, her
tranquillity was much disturbed; so much so, that, on one occasion, she
actually shook poor dear Joanna, and told her she was "a fright." Amy
fared worst of all, for her resources were small; and when her sisters
left her to amuse and care for herself, she soon found that accomplished
and important little self a great burden. She didn\textquotesingle t
like dolls, fairy-tales were childish, and one couldn\textquotesingle t
draw all the time; tea-parties didn\textquotesingle t amount to much,
neither did picnics, unless very well conducted. "If one could have a
fine house, full of nice girls, or go travelling, the summer would be
delightful; but to stay at home with three selfish sisters and a
grown-up boy was enough to try the patience of a Boaz," complained Miss
Malaprop, after several days devoted to pleasure, fretting, and
\emph{ennui}.

No one would own that they were tired of the experiment; but, by Friday
night, each acknowledged to herself that she was glad the week was
nearly done. Hoping to impress the lesson more deeply, Mrs. March, who
had a good deal of humor, resolved to finish off the trial in an
appropriate manner; so she gave Hannah a holiday, and let the girls
enjoy the full effect of the play system.

When they got up on Saturday morning, there was no fire in the kitchen,
no breakfast in the dining-room, and no mother anywhere to be seen.

"Mercy on us! what \emph{has} happened?" cried Jo, staring about her in
dismay.

Meg ran upstairs, and soon came back again, looking relieved, but rather
bewildered, and a little ashamed.

"Mother isn\textquotesingle t sick, only very tired, and she says she is
going to stay quietly in her room all day, and let us do the best we
can. It\textquotesingle s a very queer thing for her to do, she
doesn\textquotesingle t act a bit like herself; but she says it has been
a hard week for her, so we mustn\textquotesingle t grumble, but take
care of ourselves."

"That\textquotesingle s easy enough, and I like the idea;
I\textquotesingle m aching for something to do---that is, some new
amusement, you know," added Jo quickly.

In fact it \emph{was} an immense relief to them all to have a little
work, and they took hold with a will, but soon realized the truth of
Hannah\textquotesingle s saying, "Housekeeping ain\textquotesingle t no
joke." There was plenty of food in the larder, and, while Beth and Amy
set the table, Meg and Jo got breakfast, wondering, as they did so, why
servants ever talked about hard work.

"I shall take some up to mother, though she said we were not to think of
her, for she\textquotesingle d take care of herself," said Meg, who
presided, and felt quite matronly behind the teapot.

So a tray was fitted out before any one began, and taken up, with the
cook\textquotesingle s compliments. The boiled tea was very bitter, the
omelette scorched, and the biscuits speckled with saleratus; but Mrs.
March received her repast with thanks, and laughed heartily over it
after Jo was gone.

"Poor little souls, they will have a hard time, I\textquotesingle m
afraid; but they won\textquotesingle t suffer, and it will do them
good," she said, producing the more palatable viands with which she had
provided herself, and disposing of the bad breakfast, so that their
feelings might not be hurt,---a motherly little deception, for which
they were grateful.

Many were the complaints below, and great the chagrin of the head cook
at her failures. "Never mind, I\textquotesingle ll get the dinner, and
be servant; you be mistress, keep your hands nice, see company, and give
orders," said Jo, who knew still less than Meg about culinary affairs.

This obliging offer was gladly accepted; and Margaret retired to the
parlor, which she hastily put in order by whisking the litter under the
sofa, and shutting the blinds, to save the trouble of dusting. Jo, with
perfect faith in her own powers, and a friendly desire to make up the
quarrel, immediately put a note in the office, inviting Laurie to
dinner.

"You\textquotesingle d better see what you have got before you think of
having company," said Meg, when informed of the hospitable but rash act.

"Oh, there\textquotesingle s corned beef and plenty of potatoes; and I
shall get some asparagus, and a lobster, \textquotesingle for a
relish,\textquotesingle{} as Hannah says. We\textquotesingle ll have
lettuce, and make a salad. I don\textquotesingle t know how, but the
book tells. I\textquotesingle ll have blanc-mange and strawberries for
dessert; and coffee, too, if you want to be elegant."

"Don\textquotesingle t try too many messes, Jo, for you
can\textquotesingle t make anything but gingerbread and molasses candy,
fit to eat. I wash my hands of the dinner-party; and, since you have
asked Laurie on your own responsibility, you may just take care of him."

"I don\textquotesingle t want you to do anything but be civil to him,
and help to the pudding. You\textquotesingle ll give me your advice if I
get in a muddle, won\textquotesingle t you?" asked Jo, rather hurt.

"Yes; but I don\textquotesingle t know much, except about bread, and a
few trifles. You had better ask mother\textquotesingle s leave before
you order anything," returned Meg prudently.

"Of course I shall; I\textquotesingle m not a fool," and Jo went off in
a huff at the doubts expressed of her powers.

"Get what you like, and don\textquotesingle t disturb me;
I\textquotesingle m going out to dinner, and can\textquotesingle t worry
about things at home," said Mrs. March, when Jo spoke to her. "I never
enjoyed housekeeping, and I\textquotesingle m going to take a vacation
to-day, and read, write, go visiting, and amuse myself."

The unusual spectacle of her busy mother rocking comfortably, and
reading, early in the morning, made Jo feel as if some natural
phenomenon had occurred, for an eclipse, an earthquake, or a volcanic
eruption would hardly have seemed stranger.

"Everything is out of sorts, somehow," she said to herself, going down
stairs. "There\textquotesingle s Beth crying; that\textquotesingle s a
sure sign that something is wrong with this family. If Amy is bothering,
I\textquotesingle ll shake her."

Feeling very much out of sorts herself, Jo hurried into the parlor to
find Beth sobbing over Pip, the canary, who lay dead in the cage, with
his little claws pathetically extended, as if imploring the food for
want of which he had died.

"It\textquotesingle s all my fault---I forgot him---there
isn\textquotesingle t a seed or a drop left. O Pip! O Pip! how could I
be so cruel to you?" cried Beth, taking the poor thing in her hands, and
trying to restore him.

\protect\phantomsection\label{6672479776654687619_37106-h-2.htm.xhtml_b058.png}{}
\pandocbounded{\includegraphics[keepaspectratio]{303483661336987339_b058.png}}

Jo peeped into his half-open eye, felt his little heart, and finding him
stiff and cold, shook her head, and offered her domino-box for a coffin.

"Put him in the oven, and maybe he will get warm and revive," said Amy
hopefully.

"He\textquotesingle s been starved, and he
sha\textquotesingle n\textquotesingle t be baked, now
he\textquotesingle s dead. I\textquotesingle ll make him a shroud, and
he shall be buried in the garden; and I\textquotesingle ll never have
another bird, never, my Pip! for I am too bad to own one," murmured
Beth, sitting on the floor with her pet folded in her hands.

"The funeral shall be this afternoon, and we will all go. Now,
don\textquotesingle t cry, Bethy; it\textquotesingle s a pity, but
nothing goes right this week, and Pip has had the worst of the
experiment. Make the shroud, and lay him in my box; and, after the
dinner-party, we\textquotesingle ll have a nice little funeral," said
Jo, beginning to feel as if she had undertaken a good deal.

Leaving the others to console Beth, she departed to the kitchen, which
was in a most discouraging state of confusion. Putting on a big apron,
she fell to work, and got the dishes piled up ready for washing, when
she discovered that the fire was out.

"Here\textquotesingle s a sweet prospect!" muttered Jo, slamming the
stove-door open, and poking vigorously among the cinders.

Having rekindled the fire, she thought she would go to market while the
water heated. The walk revived her spirits; and, flattering herself that
she had made good bargains, she trudged home again, after buying a very
young lobster, some very old asparagus, and two boxes of acid
strawberries. By the time she got cleared up, the dinner arrived, and
the stove was red-hot. Hannah had left a pan of bread to rise, Meg had
worked it up early, set it on the hearth for a second rising, and
forgotten it. Meg was entertaining Sallie Gardiner in the parlor, when
the door flew open, and a floury, crocky, flushed, and dishevelled
figure appeared, demanding tartly,---

"I say, isn\textquotesingle t bread
\textquotesingle riz\textquotesingle{} enough when it runs over the
pans?"

Sallie began to laugh; but Meg nodded, and lifted her eyebrows as high
as they would go, which caused the apparition to vanish, and put the
sour bread into the oven without further delay. Mrs. March went out,
after peeping here and there to see how matters went, also saying a word
of comfort to Beth, who sat making a winding-sheet, while the dear
departed lay in state in the domino-box. A strange sense of helplessness
fell upon the girls as the gray bonnet vanished round the corner; and
despair seized them, when, a few minutes later, Miss Crocker appeared,
and said she\textquotesingle d come to dinner. Now, this lady was a
thin, yellow spinster, with a sharp nose and inquisitive eyes, who saw
everything, and gossiped about all she saw. They disliked her, but had
been taught to be kind to her, simply because she was old and poor, and
had few friends. So Meg gave her the easy-chair, and tried to entertain
her, while she asked questions, criticised everything, and told stories
of the people whom she knew.

Language cannot describe the anxieties, experiences, and exertions which
Jo underwent that morning; and the dinner she served up became a
standing joke. Fearing to ask any more advice, she did her best alone,
and discovered that something more than energy and good-will is
necessary to make a cook. She boiled the asparagus for an hour, and was
grieved to find the heads cooked off and the stalks harder than ever.
The bread burnt black; for the salad-dressing so aggravated her, that
she let everything else go till she had convinced herself that she could
not make it fit to eat. The lobster was a scarlet mystery to her, but
she hammered and poked, till it was unshelled, and its meagre
proportions concealed in a grove of lettuce-leaves. The potatoes had to
be hurried, not to keep the asparagus waiting, and were not done at
last. The blanc-mange was lumpy, and the strawberries not as ripe as
they looked, having been skilfully "deaconed."

"Well, they can eat beef, and bread and butter, if they are hungry; only
it\textquotesingle s mortifying to have to spend your whole morning for
nothing," thought Jo, as she rang the bell half an hour later than
usual, and stood, hot, tired, and dispirited, surveying the feast spread
for Laurie, accustomed to all sorts of elegance, and Miss Crocker, whose
curious eyes would mark all failures, and whose tattling tongue would
report them far and wide.

\protect\phantomsection\label{6672479776654687619_37106-h-2.htm.xhtml_b059.png}{}
\pandocbounded{\includegraphics[keepaspectratio]{303483661336987339_b059.png}}

Poor Jo would gladly have gone under the table, as one thing after
another was tasted and left; while Amy giggled, Meg looked distressed,
Miss Crocker pursed up her lips, and Laurie talked and laughed with all
his might, to give a cheerful tone to the festive scene.
Jo\textquotesingle s one strong point was the fruit, for she had sugared
it well, and had a pitcher of rich cream to eat with it. Her hot cheeks
cooled a trifle, and she drew a long breath, as the pretty glass plates
went round, and every one looked graciously at the little rosy islands
floating in a sea of cream. Miss Crocker tasted first, made a wry face,
and drank some water hastily. Jo, who had refused, thinking there might
not be enough, for they dwindled sadly after the picking over, glanced
at Laurie, but he was eating away manfully, though there was a slight
pucker about his mouth, and he kept his eye fixed on his plate. Amy, who
was fond of delicate fare, took a heaping spoonful, choked, hid her face
in her napkin, and left the table precipitately.

"Oh, what is it?" exclaimed Jo trembling.

"Salt instead of sugar, and the cream is sour," replied Meg, with a
tragic gesture.

Jo uttered a groan, and fell back in her chair; remembering that she had
given a last hasty powdering to the berries out of one of the two boxes
on the kitchen table, and had neglected to put the milk in the
refrigerator. She turned scarlet, and was on the verge of crying, when
she met Laurie\textquotesingle s eyes, which \emph{would} look merry in
spite of his heroic efforts; the comical side of the affair suddenly
struck her, and she laughed till the tears ran down her cheeks. So did
every one else, even "Croaker," as the girls called the old lady; and
the unfortunate dinner ended gayly, with bread and butter, olives and
fun.

"I haven\textquotesingle t strength of mind enough to clear up now, so
we will sober ourselves with a funeral," said Jo, as they rose; and Miss
Crocker made ready to go, being eager to tell the new story at another
friend\textquotesingle s dinner-table.

They did sober themselves, for Beth\textquotesingle s sake; Laurie dug a
grave under the ferns in the grove, little Pip was laid in, with many
tears, by his tender-hearted mistress, and covered with moss, while a
wreath of violets and chickweed was hung on the stone which bore his
epitaph, composed by Jo, while she struggled with the dinner:---

"Here lies Pip March,

Who died the 7th of June;

Loved and lamented sore,

And not forgotten soon."

At the conclusion of the ceremonies, Beth retired to her room, overcome
with emotion and lobster; but there was no place of repose, for the beds
were not made, and she found her grief much assuaged by beating up
pillows and putting things in order. Meg helped Jo clear away the
remains of the feast, which took half the afternoon, and left them so
tired that they agreed to be contented with tea and toast for supper.
Laurie took Amy to drive, which was a deed of charity, for the sour
cream seemed to have had a bad effect upon her temper. Mrs. March came
home to find the three older girls hard at work in the middle of the
afternoon; and a glance at the closet gave her an idea of the success of
one part of the experiment.

Before the housewives could rest, several people called, and there was a
scramble to get ready to see them; then tea must be got, errands done;
and one or two necessary bits of sewing neglected till the last minute.
As twilight fell, dewy and still, one by one they gathered in the porch
where the June roses were budding beautifully, and each groaned or
sighed as she sat down, as if tired or troubled.

"What a dreadful day this has been!" begun Jo, usually the first to
speak.

"It has seemed shorter than usual, but \emph{so} uncomfortable," said
Meg.

"Not a bit like home," added Amy.

"It can\textquotesingle t seem so without Marmee and little Pip," sighed
Beth, glancing, with full eyes, at the empty cage above her head.

"Here\textquotesingle s mother, dear, and you shall have another bird
to-morrow, if you want it."

As she spoke, Mrs. March came and took her place among them, looking as
if her holiday had not been much pleasanter than theirs.

"Are you satisfied with your experiment, girls, or do you want another
week of it?" she asked, as Beth nestled up to her, and the rest turned
toward her with brightening faces, as flowers turn toward the sun.

"I don\textquotesingle t!" cried Jo decidedly.

"Nor I," echoed the others.

"You think, then, that it is better to have a few duties, and live a
little for others, do you?"

"Lounging and larking doesn\textquotesingle t pay," observed Jo, shaking
her head. "I\textquotesingle m tired of it, and mean to go to work at
something right off."

"Suppose you learn plain cooking; that\textquotesingle s a useful
accomplishment, which no woman should be without," said Mrs. March,
laughing inaudibly at the recollection of Jo\textquotesingle s
dinner-party; for she had met Miss Crocker, and heard her account of it.

"Mother, did you go away and let everything be, just to see how
we\textquotesingle d get on?" cried Meg, who had had suspicions all day.

"Yes; I wanted you to see how the comfort of all depends on each doing
her share faithfully. While Hannah and I did your work, you got on
pretty well, though I don\textquotesingle t think you were very happy or
amiable; so I thought, as a little lesson, I would show you what happens
when every one thinks only of herself. Don\textquotesingle t you feel
that it is pleasanter to help one another, to have daily duties which
make leisure sweet when it comes, and to bear and forbear, that home may
be comfortable and lovely to us all?"

"We do, mother, we do!" cried the girls.

"Then let me advise you to take up your little burdens again; for though
they seem heavy sometimes, they are good for us, and lighten as we learn
to carry them. Work is wholesome, and there is plenty for every one; it
keeps us from \emph{ennui} and mischief, is good for health and spirits,
and gives us a sense of power and independence better than money or
fashion."

"We\textquotesingle ll work like bees, and love it too; see if we
don\textquotesingle t!" said Jo. "I\textquotesingle ll learn plain
cooking for my holiday task; and the next dinner-party I have shall be a
success."

"I\textquotesingle ll make the set of shirts for father, instead of
letting you do it, Marmee. I can and I will, though I\textquotesingle m
not fond of sewing; that will be better than fussing over my own things,
which are plenty nice enough as they are," said Meg.

"I\textquotesingle ll do my lessons every day, and not spend so much
time with my music and dolls. I am a stupid thing, and ought to be
studying, not playing," was Beth\textquotesingle s resolution; while Amy
followed their example by heroically declaring, "I shall learn to make
\ul{button-holes,} and attend to my parts of speech."

"Very good! then I am quite satisfied with the experiment, and fancy
that we shall not have to repeat it; only don\textquotesingle t go to
the other extreme, and delve like slaves. Have regular hours for work
and play; make each day both useful and pleasant, and prove that you
understand the worth of time by employing it well. Then youth will be
delightful, old age will bring few regrets, and life become a beautiful
success, in spite of poverty."

"We\textquotesingle ll remember, mother!" and they did.

\protect\phantomsection\label{6672479776654687619_37106-h-2.htm.xhtml_b060.png}{}
\pandocbounded{\includegraphics[keepaspectratio]{303483661336987339_b060.png}}

\begin{center}\rule{0.5\linewidth}{0.5pt}\end{center}

\subsection{XII. Camp
Laurence}\label{6672479776654687619_37106-h-2.htm.xhtml_pgepubid00014}

\protect\phantomsection\label{6672479776654687619_37106-h-2.htm.xhtml_XII}{}\hyperref[6672479776654687619_37106-h-0.htm.xhtml_contents1b]{XII.}

CAMP LAURENCE.

\protect\phantomsection\label{6672479776654687619_37106-h-2.htm.xhtml_b061.png}{}
\pandocbounded{\includegraphics[keepaspectratio]{303483661336987339_b061.png}}

{Beth} was post-mistress, for, being most at home, she could attend to
it regularly, and dearly liked the daily task of unlocking the little
door and distributing the mail. One July day she came in with her hands
full, and went about the house leaving letters and parcels, like the
penny post.

"Here\textquotesingle s your posy, mother! Laurie never forgets that,"
she said, putting the fresh nosegay in the vase that stood in
"Marmee\textquotesingle s corner," and was kept supplied by the
affectionate boy.

"Miss Meg March, one letter and a glove," continued Beth, delivering the
articles to her sister, who sat near her mother, stitching wristbands.

"Why, I left a pair over there, and here is only one," said Meg, looking
at the gray cotton glove.

"Didn\textquotesingle t you drop the other in the garden?"

"No, I\textquotesingle m sure I didn\textquotesingle t; for there was
only one in the office."

"I hate to have odd gloves! Never mind, the other may be found. My
letter is only a translation of the German song I wanted; I think Mr.
Brooke did it, for this isn\textquotesingle t Laurie\textquotesingle s
writing."

Mrs. March glanced at Meg, who was looking very pretty in her gingham
morning-gown, with the little curls blowing about her forehead, and very
womanly, as she sat sewing at her little work-table, full of tidy white
rolls; so unconscious of the thought in her mother\textquotesingle s
mind as she sewed and sung, while her fingers flew, and her thoughts
were busied with girlish fancies as innocent and fresh as the pansies in
her belt, that Mrs. March smiled, and was satisfied.

"Two letters for Doctor Jo, a book, and a funny old hat, which covered
the whole post-office, stuck outside," said Beth, laughing, as she went
into the study, where Jo sat writing.

"What a sly fellow Laurie is! I said I wished bigger hats were the
fashion, because I burn my face every hot day. He said,
\textquotesingle Why mind the fashion? Wear a big hat, and be
comfortable!\textquotesingle{} I said I would if I had one, and he has
sent me this, to try me. I\textquotesingle ll wear it, for fun, and show
him I \emph{don\textquotesingle t} care for the fashion;" and, hanging
the antique broad-brim on a bust of Plato, Jo read her letters.

One from her mother made her cheeks glow and her eyes fill, for it said
to her,---

\begin{quote}
"My dear:

"I write a little word to tell you with how much satisfaction I watch
your efforts to control your temper. You say nothing about your trials,
failures, or successes, and think, perhaps, that no one sees them but
the Friend whose help you daily ask, if I may trust the well-worn cover
of your guide-book. \emph{I}, too, have seen them all, and heartily
believe in the sincerity of your resolution, since it begins to bear
fruit. Go on, dear, patiently and bravely, and always believe that no
one sympathizes more tenderly with you than your loving

"{Mother.}"
\end{quote}

"That does me good! that\textquotesingle s worth millions of money and
pecks of praise. O Marmee, I do try! I will keep on trying, and not get
tired, since I have you to help me."

Laying her head on her arms, Jo wet her little romance with a few happy
tears, for she \emph{had} thought that no one saw and appreciated her
efforts to be good; and this assurance was doubly precious, doubly
encouraging, because unexpected, and from the person whose commendation
she most valued. Feeling stronger than ever to meet and subdue her
Apollyon, she pinned the note inside her frock, as a shield and a
reminder, lest she be taken unaware, and proceeded to open her other
letter, quite ready for either good or bad news. In a big, dashing hand,
Laurie wrote,---

"{Dear Jo},\\
What ho!

Some English girls and boys are coming to see me to-morrow and I want to
have a jolly time. If it\textquotesingle s fine, I\textquotesingle m
going to pitch my tent in Longmeadow, and row up the whole crew to lunch
and croquet,---have a fire, make messes, gypsy fashion, and all sorts of
larks. They are nice people, and like such things. Brooke will go, to
keep us boys steady, and Kate Vaughn will play propriety for the girls.
I want you all to come; can\textquotesingle t let Beth off, at any
price, and nobody shall worry her. Don\textquotesingle t bother about
rations,---I\textquotesingle ll see to that, and everything else,---only
do come, there\textquotesingle s a good fellow!

"In a tearing hurry,\\
Yours ever, {Laurie}."

"Here\textquotesingle s richness!" cried Jo, flying in to tell the news
to Meg.

"Of course we can go, mother? it will be such a help to Laurie, for I
can row, and Meg see to the lunch, and the children be useful in some
way."

"I hope the Vaughns are not fine, grown-up people. Do you know anything
about them, Jo?" asked Meg.

"Only that there are four of them. Kate is older than you, Fred and
Frank (twins) about my age, and a little girl (Grace), who is nine or
ten. Laurie knew them abroad, and liked the boys; I fancied, from the
way he primmed up his mouth in speaking of her, that he
didn\textquotesingle t admire Kate much."

"I\textquotesingle m so glad my French print is clean;
it\textquotesingle s just the thing, and so becoming!" observed Meg
complacently. "Have you anything decent, Jo?"

"Scarlet and gray boating suit, good enough for me. I shall row and
tramp about, so I don\textquotesingle t want any starch to think of.
You\textquotesingle ll come, Bethy?"

"If you won\textquotesingle t let any of the boys talk to me."

"Not a boy!"

"I like to please Laurie; and I\textquotesingle m not afraid of Mr.
Brooke, he is so kind; but I don\textquotesingle t want to play, or
sing, or say anything. I\textquotesingle ll work hard, and not trouble
any one; and you\textquotesingle ll take care of me, Jo, so
I\textquotesingle ll go."

"That\textquotesingle s my good girl; you do try to fight off your
shyness, and I love you for it. Fighting faults isn\textquotesingle t
easy, as I know; and a cheery word kind of gives a lift. Thank you,
mother," and Jo gave the thin cheek a grateful kiss, more precious to
Mrs. March than if it had given back the rosy roundness of her youth.

"I had a box of chocolate drops, and the picture I wanted to copy," said
Amy, showing her mail.

"And I got a note from Mr. Laurence, asking me to come over and play to
him to-night, before the lamps are lighted, and I shall go," added Beth,
whose friendship with the old gentleman prospered finely.

"Now let\textquotesingle s fly round, and do double duty to-day, so that
we can play to-morrow with free minds," said Jo, preparing to replace
her pen with a broom.

When the sun peeped into the girls\textquotesingle{} room early next
morning, to promise them a fine day, he saw a comical sight. Each had
made such preparation for the fête as seemed necessary and proper. Meg
had an extra row of little curl-papers across her forehead, Jo had
copiously anointed her afflicted face with cold cream, Beth had taken
Joanna to bed with her to atone for the approaching separation, and Amy
had capped the climax by putting a clothes-pin on her nose, to uplift
the offending feature. It was one of the kind artists use to hold the
paper on their drawing-boards, therefore quite appropriate and effective
for the purpose to which it was now put. This funny spectacle appeared
to amuse the sun, for he burst out with such radiance that Jo woke up,
and roused all her sisters by a hearty laugh at Amy\textquotesingle s
ornament.

\protect\phantomsection\label{6672479776654687619_37106-h-2.htm.xhtml_b062.png}{}
\pandocbounded{\includegraphics[keepaspectratio]{303483661336987339_b062.png}}

Sunshine and laughter were good omens for a pleasure party, and soon a
lively bustle began in both houses. Beth, who was ready first, kept
reporting what went on next door, and enlivened her
sisters\textquotesingle{} toilets by frequent telegrams from the window.

"There goes the man with the tent! I see Mrs. Barker doing up the lunch
in a hamper and a great basket. Now Mr. Laurence is looking up at the
sky, and the weathercock; I wish he would go, too.
There\textquotesingle s Laurie, looking like a sailor,---nice boy! Oh,
mercy me! here\textquotesingle s a carriage full of people---a tall
lady, a little girl, and two dreadful boys. One is lame; poor thing,
he\textquotesingle s got a crutch. Laurie didn\textquotesingle t tell us
that. Be quick, girls! it\textquotesingle s getting late. Why, there is
Ned Moffat, I do declare. Look, Meg, isn\textquotesingle t that the man
who bowed to you one day, when we were shopping?"

"So it is. How queer that he should come. I thought he was at the
Mountains. There is Sallie; I\textquotesingle m glad she got back in
time. Am I all right, Jo?" cried Meg, in a flutter.

"A regular daisy. Hold up your dress and put your hat straight; it looks
sentimental tipped that way, and will fly off at the first puff. Now,
then, come on!"

"O Jo, you are not going to wear that awful hat? It\textquotesingle s
too absurd! You shall \emph{not} make a guy of yourself," remonstrated
Meg, as Jo tied down, with a red ribbon, the broad-brimmed,
old-fashioned Leghorn Laurie had sent for a joke.

"I just will, though, for it\textquotesingle s capital,---so shady,
light, and big. It will make fun; and I don\textquotesingle t mind being
a guy if I\textquotesingle m comfortable." With that Jo marched straight
away, and the rest followed,---a bright little band of sisters, all
looking their best, in summer suits, with happy faces under the jaunty
hat-brims.

Laurie ran to meet, and present them to his friends, in the most cordial
manner. The lawn was the reception-room, and for several minutes a
lively scene was enacted there. Meg was grateful to see that Miss Kate,
though twenty, was dressed with a simplicity which American girls would
do well to imitate; and she was much flattered by Mr.
Ned\textquotesingle s assurances that he came especially to see her. Jo
understood why Laurie "primmed up his mouth" when speaking of Kate, for
that young lady had a stand-off-don\textquotesingle t-touch-me air,
which contrasted strongly with the free and easy demeanor of the other
girls. Beth took an observation of the new boys, and decided that the
lame one was not "dreadful," but gentle and feeble, and she would be
kind to him on that account. Amy found Grace a well-mannered, merry
little person; and after staring dumbly at one another for a few
minutes, they suddenly became very good friends.

Tents, lunch, and croquet utensils having been sent on beforehand, the
party was soon embarked, and the two boats pushed off together, leaving
Mr. Laurence waving his hat on the shore. Laurie and Jo rowed one boat;
Mr. Brooke and Ned the other; while Fred Vaughn, the riotous twin, did
his best to upset both by paddling about in a wherry like a disturbed
water-bug. Jo\textquotesingle s funny hat deserved a vote of thanks, for
it was of general utility; it broke the ice in the beginning, by
producing a laugh; it created quite a refreshing breeze, flapping to and
fro, as she rowed, and would make an excellent umbrella for the whole
party, if a shower came up, she said. Kate looked rather amazed at
Jo\textquotesingle s proceedings, especially as she exclaimed
"Christopher Columbus!" when she lost her oar; and Laurie said, "My dear
fellow, did I hurt you?" when he tripped over her feet in taking his
place. But after putting up her glass to examine the queer girl several
times, Miss Kate decided that she was "odd, but rather clever," and
smiled upon her from afar.

\protect\phantomsection\label{6672479776654687619_37106-h-2.htm.xhtml_b063.png}{}
\pandocbounded{\includegraphics[keepaspectratio]{303483661336987339_b063.png}}

Meg, in the other boat, was delightfully situated, face to face with the
rowers, who both admired the prospect, and feathered their oars with
uncommon "skill and dexterity." Mr. Brooke was a grave, silent young
man, with handsome brown eyes and a pleasant voice. Meg liked his quiet
manners, and considered him a walking encyclopædia of useful knowledge.
He never talked to her much; but he looked at her a good deal, and she
felt sure that he did not regard her with aversion. Ned, being in
college, of course put on all the airs which Freshmen think it their
bounden duty to assume; he was not very wise, but very good-natured, and
altogether an excellent person to carry on a picnic. Sallie Gardiner was
absorbed in keeping her white piqué dress clean, and chattering with the
ubiquitous Fred, who kept Beth in constant terror by his pranks.

It was not far to Longmeadow; but the tent was pitched and the wickets
down by the time they arrived. A pleasant green field, with three
wide-spreading oaks in the middle, and a smooth strip of turf for
croquet.

"Welcome to Camp Laurence!" said the young host, as they landed, with
exclamations of delight.

"Brooke is commander-in-chief; I am commissary-general; the other
fellows are staff-officers; and you, ladies, are company. The tent is
for your especial benefit, and that oak is your drawing-room; this is
the mess-room, and the third is the camp-kitchen. Now,
let\textquotesingle s have a game before it gets hot, and then
we\textquotesingle ll see about dinner."

Frank, Beth, Amy, and Grace sat down to watch the game played by the
other eight. Mr. Brooke chose Meg, Kate, and Fred; Laurie took Sallie,
Jo, and Ned. The Englishers played well; but the Americans played
better, and contested every inch of the ground as strongly as if the
spirit of \textquotesingle76 inspired them. Jo and Fred had several
skirmishes, and once narrowly escaped high words. Jo was through the
last wicket, and had missed the stroke, which failure ruffled her a good
deal. Fred was close behind her, and his turn came before hers; he gave
a stroke, his ball hit the wicket, and stopped an inch on the wrong
side. No one was very near; and running up to examine, he gave it a sly
nudge with his toe, which put it just an inch on the right side.

"I\textquotesingle m through! Now, Miss Jo, I\textquotesingle ll settle
you, and get in first," cried the young gentleman, swinging his mallet
for another blow.

\protect\phantomsection\label{6672479776654687619_37106-h-2.htm.xhtml_b064.png}{}
\pandocbounded{\includegraphics[keepaspectratio]{303483661336987339_b064.png}}

"You pushed it; I saw you; it\textquotesingle s my turn now," said Jo
sharply.

"Upon my word, I didn\textquotesingle t move it; it rolled a bit,
perhaps, but that is allowed; so stand off, please, and let me have a go
at the stake."

"We don\textquotesingle t cheat in America, but you can, if you choose,"
said Jo angrily.

"Yankees are a deal the most tricky, everybody knows. There you go!"
returned Fred, croqueting her ball far away.

Jo opened her lips to say something rude, but checked herself in time,
colored up to her forehead, and stood a minute, hammering down a wicket
with all her might, while Fred hit the stake, and declared himself out
with much exultation. She went off to get her ball, and was a long time
finding it, among the bushes; but she came back, looking cool and quiet,
and waited her turn patiently. It took several strokes to regain the
place she had lost; and, when she got there, the other side had nearly
won, for Kate\textquotesingle s ball was the last but one, and lay near
the stake.

"By George, it\textquotesingle s all up with us! Good-by, Kate. Miss Jo
owes me one, so you are finished," cried Fred excitedly, as they all
drew near to see the finish.

"Yankees have a trick of being generous to their enemies," said Jo, with
a look that made the lad redden, "especially when they beat them," she
added, as, leaving Kate\textquotesingle s ball untouched, she won the
game by a clever stroke.

Laurie threw up his hat; then remembered that it
wouldn\textquotesingle t do to exult over the defeat of his guests, and
stopped in the middle of a cheer to whisper to his friend,---

"Good for you, Jo! He did cheat, I saw him; we can\textquotesingle t
tell him so, but he won\textquotesingle t do it again, take my word for
it."

Meg drew her aside, under pretence of pinning up a loose braid, and said
approvingly,---

"It was dreadfully provoking; but you kept your temper, and
I\textquotesingle m so glad, Jo."

"Don\textquotesingle t praise me, Meg, for I could box his ears this
minute. I should certainly have boiled over if I hadn\textquotesingle t
stayed among the nettles till I got my rage under enough to hold my
tongue. It\textquotesingle s simmering now, so I hope
he\textquotesingle ll keep out of my way," returned Jo, biting her lips,
as she glowered at Fred from under her big hat.

"Time for lunch," said Mr. Brooke, looking at his watch.
"Commissary-general, will you make the fire and get water, while Miss
March, Miss Sallie, and I spread the table? Who can make good coffee?"

"Jo can," said Meg, glad to recommend her sister. So Jo, feeling that
her late lessons in cookery were to do her honor, went to preside over
the coffee-pot, while the children collected dry sticks, and the boys
made a fire, and got water from a spring near by. Miss Kate sketched,
and Frank talked to Beth, who was making little mats of braided rushes
to serve as plates.

The commander-in-chief and his aids soon spread the table-cloth with an
inviting array of eatables and drinkables, prettily decorated with green
leaves. Jo announced that the coffee was ready, and every one settled
themselves to a hearty meal; for youth is seldom dyspeptic, and exercise
develops wholesome appetites. A very merry lunch it was; for everything
seemed fresh and funny, and frequent peals of laughter startled a
venerable horse who fed near by. There was a pleasing inequality in the
table, which produced many mishaps to cups and plates; acorns dropped
into the milk, little black ants partook of the refreshments without
being invited, and fuzzy caterpillars swung down from the tree, to see
what was going on. Three white-headed children peeped over the fence,
and an objectionable dog barked at them from the other side of the river
with all his might and main.

\protect\phantomsection\label{6672479776654687619_37106-h-2.htm.xhtml_b065.png}{}
\pandocbounded{\includegraphics[keepaspectratio]{303483661336987339_b065.png}}\\
\protect\phantomsection\label{6672479776654687619_37106-h-2.htm.xhtml_ebm_caption1}{"A
very merry lunch it was."---Page 156.}

"There\textquotesingle s salt here, if you prefer it," said Laurie, as
he handed Jo a saucer of berries.

"Thank you, I prefer spiders," she replied, fishing up two unwary little
ones who had gone to a creamy death. "How dare you remind me of that
horrid dinner-party, when yours is so nice in every way?" added Jo, as
they both laughed, and ate out of one plate, the china having run short.

"I had an uncommonly good time that day, and haven\textquotesingle t got
over it yet. This is no credit to me, you know; I don\textquotesingle t
do anything; it\textquotesingle s you and Meg and Brooke who make it go,
and I\textquotesingle m no end obliged to you. What shall we do when we
can\textquotesingle t eat any more?" asked Laurie, feeling that his
trump card had been played when lunch was over.

"Have games, till it\textquotesingle s cooler. I brought
\textquotesingle Authors,\textquotesingle{} and I dare say Miss Kate
knows something new and nice. Go and ask her; she\textquotesingle s
company, and you ought to stay with her more."

"Aren\textquotesingle t you company too? I thought she\textquotesingle d
suit Brooke; but he keeps talking to Meg, and Kate just stares at them
through that ridiculous glass of hers. I\textquotesingle m going, so you
needn\textquotesingle t try to preach propriety, for you
can\textquotesingle t do it, Jo."

Miss Kate did know several new games; and as the girls would not, and
the boys could not, eat any more, they all adjourned to the drawing-room
to play "Rigmarole."

"One person begins a story, any nonsense you like, and tells as long as
he pleases, only taking care to stop short at some exciting point, when
the next takes it up and does the same. It\textquotesingle s very funny
when well done, and makes a perfect jumble of tragical comical stuff to
laugh over. Please start it, Mr. Brooke," said Kate, with a commanding
air, which surprised Meg, who treated the tutor with as much respect as
any other gentleman.

Lying on the grass at the feet of the two young ladies, Mr. Brooke
obediently began the story, with the handsome brown eyes steadily fixed
upon the sunshiny river.

\protect\phantomsection\label{6672479776654687619_37106-h-2.htm.xhtml_b066.png}{}
\pandocbounded{\includegraphics[keepaspectratio]{303483661336987339_b066.png}}

"Once on a time, a knight went out into the world to seek his fortune,
for he had nothing but his sword and his shield. He travelled a long
while, nearly eight-and-twenty years, and had a hard time of it, till he
came to the palace of a good old king, who had offered a reward to any
one who would tame and train a fine but unbroken colt, of which he was
very fond. The knight agreed to try, and got on slowly but surely; for
the colt was a gallant fellow, and soon learned to love his new master,
though he was freakish and wild. Every day, when he gave his lessons to
this pet of the king\textquotesingle s, the knight rode him through the
city; and, as he rode, he looked everywhere for a certain beautiful
face, which he had seen many times in his dreams, but never found. One
day, as he went prancing down a quiet street, he saw at the window of a
ruinous castle the lovely face. He was delighted, inquired who lived in
this old castle, and was told that several captive princesses were kept
there by a spell, and spun all day to lay up money to buy their liberty.
The knight wished intensely that he could free them; but he was poor,
and could only go by each day, watching for the sweet face, and longing
to see it out in the sunshine. At last, he resolved to get into the
castle and ask how he could help them. He went and knocked; the great
door flew open, and he beheld---"

"A ravishingly lovely lady, who exclaimed, with a cry of rapture,
\textquotesingle At last! at last!\textquotesingle" continued Kate, who
had read French novels, and admired the style.
"\textquotesingle\textquotesingle Tis she!\textquotesingle{} cried Count
Gustave, and fell at her feet in an ecstasy of joy. \textquotesingle Oh,
rise!\textquotesingle{} she said, extending a hand of marble fairness.
\textquotesingle Never! till you tell me how I may rescue
you,\textquotesingle{} swore the knight, \ul{still kneeling.}
\textquotesingle Alas, my cruel fate condemns me to remain here till my
tyrant is destroyed.\textquotesingle{} \textquotesingle Where is the
villain?\textquotesingle{} \textquotesingle In the mauve salon. Go,
brave heart, and save me from despair.\textquotesingle{}
\textquotesingle I obey, and return victorious or
dead!\textquotesingle{} With these thrilling words he rushed away, and
flinging open the door of the mauve salon, was about to enter, when he
received---"

\protect\phantomsection\label{6672479776654687619_37106-h-2.htm.xhtml_b067.png}{}
\pandocbounded{\includegraphics[keepaspectratio]{303483661336987339_b067.png}}

\protect\phantomsection\label{6672479776654687619_37106-h-2.htm.xhtml_b067a.png}{}
\pandocbounded{\includegraphics[keepaspectratio]{303483661336987339_b067a.png}}

"A stunning blow from the big Greek lexicon, which an old fellow in a
black gown fired at him," said Ned. "Instantly Sir
What\textquotesingle s-his-name recovered himself, pitched the tyrant
out of the window, and turned to join the lady, victorious, but with a
bump on his brow; found the door locked, tore up the curtains, made a
rope ladder, got half-way down when the ladder broke, and he went head
first into the moat, sixty feet below. Could swim like a duck, paddled
round the castle till he came to a little door guarded by two stout
fellows; knocked their heads together till they cracked like a couple of
nuts, then, by a trifling exertion of his prodigious strength, he
smashed in the door, went up a pair of stone steps covered with dust a
foot thick, toads as big as your fist, and spiders that would frighten
you into hysterics, Miss March. At the top of these steps he came plump
upon a sight that took his breath away and chilled his blood---"

"A tall figure, all in white with a veil over its face and a lamp in its
wasted hand," went on Meg. "It beckoned, gliding noiselessly before him
down a corridor as dark and cold as any tomb. Shadowy effigies in armor
stood on either side, a dead silence reigned, the lamp burned blue, and
the ghostly figure ever and anon turned its face toward him, showing the
glitter of awful eyes through its white veil. They reached a curtained
door, behind which sounded lovely music; he sprang forward to enter, but
the spectre plucked him back, and waved threateningly before him a---"

\protect\phantomsection\label{6672479776654687619_37106-h-2.htm.xhtml_b068.png}{}
\pandocbounded{\includegraphics[keepaspectratio]{303483661336987339_b068.png}}

"Snuff-box," said Jo, in a sepulchral tone, which convulsed the
audience. "\textquotesingle Thankee,\textquotesingle{} said the knight
politely, as he took a pinch, and sneezed seven times so violently that
his head fell off. \textquotesingle Ha! ha!\textquotesingle{} laughed
the ghost; and having peeped through the key-hole at the princesses
spinning away for dear life, the evil spirit picked up her victim and
put him in a large tin box, where there were eleven other knights packed
together without their heads, like sardines, who all rose and began
to---"

"Dance a hornpipe," cut in Fred, as Jo paused for breath; "and, as they
danced, the rubbishy old castle turned to a man-of-war in full sail.
\textquotesingle Up with the jib, reef the tops\textquotesingle l
halliards, helm hard a lee, and man the guns!\textquotesingle{} roared
the captain, as a Portuguese pirate hove in sight, with a flag black as
ink flying from her foremast. \textquotesingle Go in and win, my
hearties!\textquotesingle{} says the captain; and a tremendous fight
begun. Of course the British beat; they always do."

"No, they don\textquotesingle t!" cried Jo, aside.

\protect\phantomsection\label{6672479776654687619_37106-h-2.htm.xhtml_b069.png}{}
\pandocbounded{\includegraphics[keepaspectratio]{303483661336987339_b069.png}}

"Having taken the pirate captain prisoner, sailed slap over the
schooner, whose decks were piled with dead, and whose lee-scuppers ran
blood, for the order had been \textquotesingle Cutlasses, and die
hard!\textquotesingle{} \textquotesingle Bosen\textquotesingle s mate,
take a bight of the flying-jib sheet, and start this villain if he
don\textquotesingle t confess his sins double quick,\textquotesingle{}
said the British captain. The Portuguese held his tongue like a brick,
and walked the plank, while the jolly tars cheered like mad. But the sly
dog dived, came up under the man-of-war, scuttled her, and down she
went, with all sail set, \textquotesingle To the bottom of the sea, sea,
sea,\textquotesingle{} where---"

"Oh, gracious! what \emph{shall} I say?" cried Sallie, as Fred ended his
rigmarole, in which he had jumbled together, pell-mell, nautical phrases
and facts, out of one of his favorite books. "Well they went to the
bottom, and a nice mermaid welcomed them, but was much grieved on
finding the box of headless knights, and kindly pickled them in brine,
hoping to discover the mystery about them; for, being a woman, she was
curious. By and by a diver came down, and the mermaid said,
\textquotesingle I\textquotesingle ll give you this box of pearls if you
can take it up;\textquotesingle{} for she wanted to restore the poor
things to life, and couldn\textquotesingle t raise the heavy load
herself. So the diver hoisted it up, and was much disappointed, on
opening it, to find no pearls. He left it in a great lonely field, where
it was found by a---"

"Little goose-girl, who kept a hundred fat geese in the field," said
Amy, when Sallie\textquotesingle s invention gave out. "The little girl
was sorry for them, and asked an old woman what she should do to help
them. \textquotesingle Your geese will tell you, they know
everything,\textquotesingle{} said the old woman. So she asked what she
should use for new heads, since the old ones were lost, and all the
geese opened their hundred mouths and screamed---"

\protect\phantomsection\label{6672479776654687619_37106-h-2.htm.xhtml_b070.png}{}
\pandocbounded{\includegraphics[keepaspectratio]{303483661336987339_b070.png}}

"\textquotesingle Cabbages!\textquotesingle" continued Laurie promptly.
"\textquotesingle Just the thing,\textquotesingle{} said the girl, and
ran to get twelve fine ones from her garden. She put them on, the
knights revived at once, thanked her, and went on their way rejoicing,
never knowing the difference, for there were so many other heads like
them in the world that no one thought anything of it. The knight in whom
I\textquotesingle m interested went back to find the pretty face, and
learned that the princesses had spun themselves free, and all gone to be
married, but one. He was in a great state of mind at that; and mounting
the colt, who stood by him through thick and thin, rushed to the castle
to see which was left. Peeping over the hedge, he saw the queen of his
affections picking flowers in her garden. \textquotesingle Will you give
me a rose?\textquotesingle{} said he. \textquotesingle You must come and
get it. I can\textquotesingle t come to you; it isn\textquotesingle t
proper,\textquotesingle{} said she, as sweet as honey. He tried to climb
over the hedge, but it seemed to grow higher and higher; then he tried
to push through, but it grew thicker and thicker, and he was in despair.
So he patiently broke twig after twig, till he had made a little hole,
through which he peeped, saying imploringly, \textquotesingle Let me in!
let me in!\textquotesingle{} But the pretty princess did not seem to
understand, for she picked her roses quietly, and left him to fight his
way in. Whether he did or not, Frank will tell you."

"I can\textquotesingle t; I\textquotesingle m not playing, I never do,"
said Frank, dismayed at the sentimental predicament out of which he was
to rescue the absurd couple. Beth had disappeared behind Jo, and Grace
was asleep.

"So the poor knight is to be left sticking in the hedge, is he?" asked
Mr. Brooke, still watching the river, and playing with the wild rose in
his button-hole.

"I guess the princess gave him a posy, and opened the gate, after a
while," said Laurie, smiling to himself, as he threw acorns at his
tutor.

"What a piece of nonsense we have made! With practice we might do
something quite clever. Do you know
\textquotesingle Truth\textquotesingle?" asked Sallie, after they had
laughed over their story.

"I hope so," said Meg soberly.

"The game, I mean?"

"What is it?" said Fred.

"Why, you pile up your hands, choose a number, and draw out in turn, and
the person who draws at the number has to answer truly any questions put
by the rest. It\textquotesingle s great fun."

"Let\textquotesingle s try it," said Jo, who liked new experiments.

Miss Kate and Mr. Brooke, Meg, and Ned declined, but Fred, Sallie, Jo,
and Laurie piled and drew; and the lot fell to Laurie.

"Who are your heroes?" asked Jo.

"Grandfather and Napoleon."

"Which lady here do you think prettiest?" said Sallie.

"Margaret."

"Which do you like best?" from Fred.

"Jo, of course."

"What silly questions you ask!" and Jo gave a disdainful shrug as the
rest laughed at Laurie\textquotesingle s matter-of-fact tone.

"Try again; Truth isn\textquotesingle t a bad game," said Fred.

"It\textquotesingle s a very good one for you," retorted Jo, in a low
voice.

Her turn came next.

"What is your greatest fault?" asked Fred, by way of testing in her the
virtue he lacked himself.

"A quick temper."

"What do you most wish for?" said Laurie.

"A pair of boot-lacings," returned Jo, guessing and defeating his
purpose.

"Not a true answer; you must say what you really do want most."

"Genius; don\textquotesingle t you wish you could give it to me,
Laurie?" and she slyly smiled in his disappointed face.

"What virtues do you most admire in a man?" asked Sallie.

"Courage and honesty."

"Now my turn," said Fred, as his hand came last.

"Let\textquotesingle s give it to him," whispered Laurie to Jo, who
nodded, and asked at once,---

"Didn\textquotesingle t you cheat at croquet?"

"Well, yes, a little bit."

"Good! Didn\textquotesingle t you take your story out of
\textquotesingle The Sea-Lion?\textquotesingle" said Laurie.

"Rather."

"Don\textquotesingle t you think the English nation perfect in every
respect?" asked Sallie.

"I should be ashamed of myself if I didn\textquotesingle t."

"He\textquotesingle s a true John Bull. Now, Miss Sallie, you shall have
a chance without waiting to draw. I\textquotesingle ll harrow up your
feelings first, by asking if you don\textquotesingle t think you are
something of a flirt," said Laurie, as Jo nodded to Fred, as a sign that
peace was declared.

"You impertinent boy! of course I\textquotesingle m not," exclaimed
Sallie, with an air that proved the contrary.

"What do you hate most?" asked Fred.

"Spiders and rice-pudding."

"What do you like best?" asked Jo.

"Dancing and French gloves."

"Well, \emph{I} think Truth is a very silly play; let\textquotesingle s
have a sensible game of Authors, to refresh our minds," proposed Jo.

Ned, Frank, and the little girls joined in this, and, while it went on,
the three elders sat apart, talking. Miss Kate took out her sketch
again, and Margaret watched her, while Mr. Brooke lay on the grass, with
a book, which he did not read.

"How beautifully you do it! I wish I could draw," said Meg, with mingled
admiration and regret in her voice.

"Why don\textquotesingle t you learn? I should think you had taste and
talent for it," replied Miss Kate graciously.

"I haven\textquotesingle t time."

"Your mamma prefers other accomplishments, I fancy. So did mine; but I
proved to her that I had talent, by taking a few lessons privately, and
then she was quite willing I should go on. Can\textquotesingle t you do
the same with your governess?"

"I have none."

"I forgot; young ladies in America go to school more than with us. Very
fine schools they are, too, papa says. You go to a private one, I
suppose?"

"I don\textquotesingle t go at all; I am a governess myself."

"Oh, indeed!" said Miss Kate; but she might as well have said, "Dear me,
how dreadful!" for her tone implied it, and something in her face made
Meg color, and wish she had not been so frank.

Mr. Brooke looked up, and said quickly, "Young ladies in America love
independence as much as their ancestors did, and are admired and
respected for supporting themselves."

"Oh, yes; of course it\textquotesingle s very nice and proper in them to
do so. We have many most respectable and worthy young women, who do the
same and are employed by the nobility, because, being the daughters of
gentlemen, they are both well-bred and accomplished, you know," said
Miss Kate, in a patronizing tone, that hurt Meg\textquotesingle s pride,
and made her work seem not only more distasteful, but degrading.

"Did the German song suit, Miss March?" inquired Mr. Brooke, breaking an
awkward pause.

"Oh, yes! it was very sweet, and I\textquotesingle m much obliged to
whoever translated it for me;" and Meg\textquotesingle s downcast face
brightened as she spoke.

"Don\textquotesingle t you read German?" asked Miss Kate, with a look of
surprise.

"Not very well. My father, who taught me, is away, and I
don\textquotesingle t get on very fast alone, for I\textquotesingle ve
no one to correct my pronunciation."

"Try a little now; here is Schiller\textquotesingle s
\textquotesingle Mary Stuart,\textquotesingle{} and a tutor who loves to
teach," and Mr. Brooke laid his book on her lap, with an inviting smile.

"It\textquotesingle s so hard I\textquotesingle m afraid to try," said
Meg, grateful, but bashful in the presence of the accomplished young
lady beside her.

"I\textquotesingle ll read a bit to encourage you;" and Miss Kate read
one of the most beautiful passages, in a perfectly correct but perfectly
expressionless manner.

Mr. Brooke made no comment, as she returned the book to Meg, who said
innocently,---

"I thought it was poetry."

"Some of it is. Try this passage."

There was a queer smile about Mr. Brooke\textquotesingle s mouth as he
opened at poor Mary\textquotesingle s lament.

\ul{Meg, obediently following the long grass-blade} which her new tutor
used to point with, read slowly and timidly, unconsciously making poetry
of the hard words by the soft intonation of her musical voice. Down the
page went the green guide, and presently, forgetting her listener in the
beauty of the sad scene, Meg read as if alone, giving a little touch of
tragedy to the words of the unhappy queen. If she had seen the brown
eyes then, she would have stopped short; but she never looked up, and
the lesson was not spoiled for her.

"Very well indeed!" said Mr. Brooke, as she paused, quite ignoring her
many mistakes, and looking as if he did, indeed, "love to teach."

Miss Kate put up her glass, and, having taken a survey of the little
tableau before her, shut her sketch-book, saying, with condescension,---

\protect\phantomsection\label{6672479776654687619_37106-h-2.htm.xhtml_b071.png}{}
\pandocbounded{\includegraphics[keepaspectratio]{303483661336987339_b071.png}}

"You\textquotesingle ve a nice accent, and, in time, will be a clever
reader. I advise you to learn, for German is a valuable accomplishment
to teachers. I must look after Grace, she is romping;" and Miss Kate
strolled away, adding to herself, with a shrug, "I
didn\textquotesingle t come to chaperone a governess, though she
\emph{is} young and pretty. What odd people these Yankees are;
I\textquotesingle m afraid Laurie will be quite spoilt among them."

"I forgot that English people rather turn up their noses at governesses,
and don\textquotesingle t treat them as we do," said Meg, looking after
the retreating figure with an annoyed expression.

"Tutors, also, have rather a hard time of it there, as I know to my
sorrow. There\textquotesingle s no place like America for us workers,
Miss Margaret;" and Mr. Brooke looked so contented and cheerful, that
Meg was ashamed to lament her hard lot.

"I\textquotesingle m glad I live in it then. I don\textquotesingle t
like my work, but I get a good deal of satisfaction out of it after all,
so I won\textquotesingle t complain; I only wish I liked teaching as you
do."

"I think you would if you had Laurie for a pupil. I shall be very sorry
to lose him next year," said Mr. Brooke, busily punching holes in the
turf.

"Going to college, I suppose?" Meg\textquotesingle s lips asked that
question, but her eyes added, "And what becomes of you?"

"Yes; it\textquotesingle s high time he went, for he is ready; and as
soon as he is off, I shall turn soldier. I am needed."

"I am glad of that!" exclaimed Meg. "I should think every young man
would want to go; though it is hard for the mothers and sisters who stay
at home," she added sorrowfully.

"I have neither, and very few friends, to care whether I live or die,"
said Mr. Brooke, rather bitterly, as he absently put the dead rose in
the hole he had made and covered it up, like a little grave.

"Laurie and his grandfather would care a great deal, and we should all
be very sorry to have any harm happen to you," said Meg heartily.

"Thank you; that sounds pleasant," began Mr. Brooke, looking cheerful
again; but before he could finish his speech, Ned, mounted on the old
horse, came lumbering up to display his equestrian skill before the
young ladies, and there was no more quiet that day.

"Don\textquotesingle t you love to ride?" asked Grace of Amy, as they
stood resting, after a race round the field with the others, led by Ned.

"I dote upon it; my sister Meg used to ride when papa was rich, but we
don\textquotesingle t keep any horses now, except Ellen Tree," added
Amy, laughing.

"Tell me about Ellen Tree; is it a donkey?" asked Grace curiously.

\protect\phantomsection\label{6672479776654687619_37106-h-2.htm.xhtml_b072.png}{}
\pandocbounded{\includegraphics[keepaspectratio]{303483661336987339_b072.png}}

"Why, you see, Jo is crazy about horses, and so am I, but
we\textquotesingle ve only got an old side-saddle, and no horse. Out in
our garden is an apple-tree, that has a nice low branch; so Jo put the
saddle on it, fixed some reins on the part that turns up, and we bounce
away on Ellen Tree whenever we like."

"How funny!" laughed Grace. "I have a pony at home, and ride nearly
every day in the park, with Fred and Kate; it\textquotesingle s very
nice, for my friends go too, and the Row is full of ladies and
gentlemen."

"Dear, how charming! I hope I shall go abroad some day; but
I\textquotesingle d rather go to Rome than the Row," said Amy, who had
not the remotest idea what the Row was, and wouldn\textquotesingle t
have asked for the world.

Frank, sitting just behind the little girls, heard what they were
saying, and pushed his crutch away from him with an impatient gesture as
he watched the active lads going through all sorts of comical
gymnastics. Beth, who was collecting the scattered Author-cards, looked
up, and said, in her shy yet friendly way,---

"I\textquotesingle m afraid you are tired; can I do anything for you?"

"Talk to me, please; it\textquotesingle s dull, sitting by myself,"
answered Frank, who had evidently been used to being made much of at
home.

If he had asked her to deliver a Latin oration, it would not have seemed
a more impossible task to bashful Beth; but there was no place to run
to, no Jo to hide behind now, and the poor boy looked so wistfully at
her, that she bravely resolved to try.

"What do you like to talk about?" she asked, fumbling over the cards,
and dropping half as she tried to tie them up.

"Well, I like to hear about cricket and boating and hunting," said
Frank, who had not yet learned to suit his amusements to his strength.

"My heart! what shall I do? I don\textquotesingle t know anything about
them," thought Beth; and, forgetting the boy\textquotesingle s
misfortune in her flurry, she said, hoping to make him talk, "I never
saw any hunting, but I suppose you know all about it."

"I did once; but I can never hunt again, for I got hurt leaping a
confounded five-barred gate; so there are no more horses and hounds for
me," said Frank, with a sigh that made Beth hate herself for her
innocent blunder.

"Your deer are much prettier than our ugly buffaloes," she said, turning
to the prairies for help, and feeling glad that she had read one of the
boys\textquotesingle{} books in which Jo delighted.

Buffaloes proved soothing and satisfactory; and, in her eagerness to
amuse another, Beth forgot herself, and was quite unconscious of her
sisters\textquotesingle{} surprise and delight at the unusual spectacle
of Beth talking away to one of the dreadful boys, against whom she had
begged protection.

"Bless her heart! She pities him, so she is good to him," said Jo,
beaming at her from the croquet-ground.

"I always said she was a little saint," added Meg, as if there could be
no further doubt of it.

"I haven\textquotesingle t heard Frank laugh so much for ever so long,"
said Grace to Amy, as they sat discussing dolls, and making tea-sets out
of the acorn-cups.

"My sister Beth is a very \emph{fastidious} girl, when she likes to be,"
said Amy, well pleased at Beth\textquotesingle s success. She meant
"fascinating," but as Grace didn\textquotesingle t know the exact
meaning of either word, "fastidious" sounded well, and made a good
impression.

An impromptu circus, fox and geese, and an amicable game of croquet,
finished the afternoon. At sunset the tent was struck, hampers packed,
wickets pulled up, boats loaded, and the whole party floated down the
river, singing at the tops of their voices. Ned, getting sentimental,
warbled a serenade with the pensive refrain,---

"Alone, alone, ah! woe, alone,"

and at the lines---

"We each are young, we each have a heart,

Oh, why should we stand thus coldly apart?"

he looked at Meg with such a lackadaisical expression that she laughed
outright and spoilt his song.

"How can you be so cruel to me?" he whispered, under cover of a lively
chorus. "You\textquotesingle ve kept close to that starched-up
Englishwoman all day, and now you snub me."

"I didn\textquotesingle t mean to; but you looked so funny I really
couldn\textquotesingle t help it," replied Meg, passing over the first
part of his reproach; for it was quite true that she \emph{had} shunned
him, remembering the Moffat party and the talk after it.

Ned was offended, and turned to Sallie for consolation, saying to her
rather pettishly, "There isn\textquotesingle t a bit of flirt in that
girl, is there?"

"Not a particle; but she\textquotesingle s a dear," returned Sallie,
defending her friend even while confessing her short-comings.

"She\textquotesingle s not a stricken deer, any way," said Ned, trying
to be witty, and succeeding as well as very young gentlemen usually do.

On the lawn, where it had gathered, the little party separated with
cordial good-nights and good-byes, for the Vaughns were going to Canada.
As the four sisters went home through the garden, Miss Kate looked after
them, saying, without the patronizing tone in her voice, "In spite of
their demonstrative manners, American girls are very nice when one knows
them."

"I quite agree with you," said Mr. Brooke.

\protect\phantomsection\label{6672479776654687619_37106-h-2.htm.xhtml_b073.png}{}
\pandocbounded{\includegraphics[keepaspectratio]{303483661336987339_b073.png}}

\begin{center}\rule{0.5\linewidth}{0.5pt}\end{center}

\subsection{XIII. Castles in the
Air.}\label{6672479776654687619_37106-h-2.htm.xhtml_pgepubid00015}

\protect\phantomsection\label{6672479776654687619_37106-h-2.htm.xhtml_b074.png}{}
\pandocbounded{\includegraphics[keepaspectratio]{303483661336987339_b074.png}}

\protect\phantomsection\label{6672479776654687619_37106-h-2.htm.xhtml_XIII}{}\hyperref[6672479776654687619_37106-h-0.htm.xhtml_contents1b]{XIII.}

CASTLES IN THE AIR.

{Laurie} lay luxuriously swinging to and fro in his hammock, one warm
September afternoon, wondering what his neighbors were about, but too
lazy to go and find out. He was in one of his moods; for the day had
been both unprofitable and unsatisfactory, and he was wishing he could
live it over again. The hot weather made him indolent, and he had
shirked his studies, tried Mr. Brooke\textquotesingle s patience to the
utmost, displeased his grandfather by practising half the afternoon,
frightened the maid-servants half out of their wits, by mischievously
hinting that one of his dogs was going mad, and, after high words with
the stable-man about some fancied neglect of his horse, he had flung
himself into his hammock, to fume over the stupidity of the world in
general, till the peace of the lovely day quieted him in spite of
himself. Staring up into the green gloom of the horse-chestnut trees
above him, he dreamed dreams of all sorts, and was just imagining
himself tossing on the ocean, in a voyage round the world, when the
sound of voices brought him ashore in a flash. Peeping through the
meshes of the hammock, he saw the Marches coming out, as if bound on
some expedition.

"What in the world are those girls about now?" thought Laurie, opening
his sleepy eyes to take a good look, for there was something rather
peculiar in the appearance of his neighbors. Each wore a large, flapping
hat, a brown linen pouch slung over one shoulder, and carried a long
staff. Meg had a cushion, Jo a book, Beth a basket, and Amy a portfolio.
All walked quietly through the garden, out at the little back gate, and
began to climb the hill that lay between the house and river.

"Well, that\textquotesingle s cool!" said Laurie to himself, "to have a
picnic and never ask me. They can\textquotesingle t be going in the
boat, for they haven\textquotesingle t got the key. Perhaps they forgot
it; I\textquotesingle ll take it to them, and see what\textquotesingle s
going on."

Though possessed of half a dozen hats, it took him some time to find
one; then there was a hunt for the key, which was at last discovered in
his pocket; so that the girls were quite out of sight when he leaped the
fence and ran after them. Taking the shortest way to the boat-house, he
waited for them to appear: but no one came, and he went up the hill to
take an observation. A grove of pines covered one part of it, and from
the heart of this green spot came a clearer sound than the soft sigh of
the pines or the drowsy chirp of the crickets.

"Here\textquotesingle s a landscape!" thought Laurie, peeping through
the bushes, and looking wide-awake and good-natured already.

It \emph{was} rather a pretty little picture; for the sisters sat
together in the shady nook, with sun and shadow flickering over them,
the aromatic wind lifting their hair and cooling their hot cheeks, and
all the little wood-people going on with their affairs as if these were
no strangers, but old friends. Meg sat upon her cushion, sewing daintily
with her white hands, and looking as fresh and sweet as a rose, in her
pink dress, among the green. Beth was sorting the cones that lay thick
under the hemlock near by, for she made pretty things of them. Amy was
sketching a group of ferns, and Jo was knitting as she read aloud. A
shadow passed over the boy\textquotesingle s face as he watched them,
feeling that he ought to go away, because uninvited; yet lingering,
because home seemed very lonely, and this quiet party in the woods most
attractive to his restless spirit. He stood so still that a squirrel,
busy with its harvesting, ran down a pine close beside him, saw him
suddenly and skipped back, scolding so shrilly that Beth looked up,
espied the wistful face behind the birches, and beckoned with a
reassuring smile.

\protect\phantomsection\label{6672479776654687619_37106-h-2.htm.xhtml_b075.png}{}
\pandocbounded{\includegraphics[keepaspectratio]{303483661336987339_b075.png}}

"May I come in, please? or shall I be a bother?" he asked, advancing
slowly.

Meg lifted her eyebrows, but Jo scowled at her defiantly, and said, at
once, "Of course you may. We should have asked you before, only we
thought you wouldn\textquotesingle t care for such a
girl\textquotesingle s game as this."

"I always liked your games; but if Meg doesn\textquotesingle t want me,
I\textquotesingle ll go away."

"I\textquotesingle ve no objection, if you do something;
it\textquotesingle s against the rules to be idle here," replied Meg,
gravely but graciously.

"Much obliged; I\textquotesingle ll do anything if
you\textquotesingle ll let me stop a bit, for it\textquotesingle s as
dull as the Desert of Sahara down there. Shall I sew, read, cone, draw,
or do all at once? Bring on your bears; I\textquotesingle m ready," and
Laurie sat down, with a submissive expression delightful to behold.

"Finish this story while I set my heel," said Jo, handing him the book.

"Yes\textquotesingle m," was the meek answer, as he began, doing his
best to prove his gratitude for the favor of an admission into the "Busy
Bee Society."

The story was not a long one, and, when it was finished, he ventured to
ask a few questions as a reward of merit.

"Please, ma\textquotesingle am, could I inquire if this highly
instructive and charming institution is a new one?"

"Would you tell him?" asked Meg of her sisters.

"He\textquotesingle ll laugh," said Amy warningly.

"Who cares?" said Jo.

"I guess he\textquotesingle ll like it," added Beth.

"Of course I shall! I give you my word I won\textquotesingle t laugh.
Tell away, Jo, and don\textquotesingle t be afraid."

"The idea of being afraid of you! Well, you see we used to play
\textquotesingle Pilgrim\textquotesingle s Progress,\textquotesingle{}
and we have been going on with it in earnest, all winter and summer."

"Yes, I know," said Laurie, nodding wisely.

"Who told you?" demanded Jo.

"Spirits."

"No, I did; I wanted to amuse him one night when you were all away, and
he was rather dismal. He did like it, so don\textquotesingle t scold,
Jo," said Beth meekly.

"You can\textquotesingle t keep a secret. Never mind; it saves trouble
now."

"Go on, please," said Laurie, as Jo became absorbed in her work, looking
a trifle displeased.

"Oh, didn\textquotesingle t she tell you about this new plan of ours?
Well, we have tried not to waste our holiday, but each has had a task,
and worked at it with a will. The vacation is nearly over, the stints
are all done, and we are ever so glad that we didn\textquotesingle t
dawdle."

"Yes, I should think so;" and Laurie thought regretfully of his own idle
days.

"Mother likes to have us out of doors as much as possible; so we bring
our work here, and have nice times. For the fun of it we bring our
things in these bags, wear the old hats, use poles to climb the hill,
and play pilgrims, as we used to do years ago. We call this hill the
\textquotesingle Delectable Mountain,\textquotesingle{} for we can look
far away and see the country where we hope to live some time."

Jo pointed, and Laurie sat up to examine; for through an opening in the
wood one could look across the wide, blue river, the meadows on the
other side, far over the outskirts of the great city, to the green hills
that rose to meet the sky. The sun was low, and the heavens glowed with
the splendor of an autumn sunset. Gold and purple clouds lay on the
hill-tops; and rising high into the ruddy light were silvery white
peaks, that shone like the airy spires of some Celestial City.

"How beautiful that is!" said Laurie softly, for he was quick to see and
feel beauty of any kind.

"It\textquotesingle s often so; and we like to watch it, for it is never
the same, but always splendid," replied Amy, wishing she could paint it.

"Jo talks about the country where we hope to live some time,---the real
country, she means, with pigs and chickens, and haymaking. It would be
nice, but I wish the beautiful country up there was real, and we could
ever go to it," said Beth musingly.

"There is a lovelier country even than that, where we \emph{shall} go,
by and by, when we are good enough," answered Meg, with her sweet voice.

"It seems so long to wait, so hard to do; I want to fly away at once, as
those swallows fly, and go in at that splendid gate."

"You\textquotesingle ll get there, Beth, sooner or later; no fear of
that," said Jo; "I\textquotesingle m the one that will have to fight and
work, and climb and wait, and maybe never get in after all."

"You\textquotesingle ll have me for company, if that\textquotesingle s
any comfort. I shall have to do a deal of travelling before I come in
sight of your Celestial City. If I arrive late, you\textquotesingle ll
say a good word for me, won\textquotesingle t you, Beth?"

Something in the boy\textquotesingle s face troubled his little friend;
but she said cheerfully, with her quiet eyes on the changing clouds, "If
people really want to go, and really try all their lives, I think they
will get in; for I don\textquotesingle t believe there are any locks on
that door, or any guards at the gate. I always imagine it is as it is in
the picture, where the shining ones stretch out their hands to welcome
poor Christian as he comes up from the river."

"Wouldn\textquotesingle t it be fun if all the castles in the air which
we make could come true, and we could live in them?" said Jo, after a
little pause.

"I\textquotesingle ve made such quantities it would be hard to choose
which I\textquotesingle d have," said Laurie, lying flat, and throwing
cones at the squirrel who had betrayed him.

"You\textquotesingle d have to take your favorite one. What is it?"
asked Meg.

"If I tell mine, will you tell yours?"

"Yes, if the girls will too."

"We will. Now, Laurie."

"After I\textquotesingle d seen as much of the world as I want to,
I\textquotesingle d like to settle in Germany, and have just as much
music as I choose. I\textquotesingle m to be a famous musician myself,
and all creation is to rush to hear me; and I\textquotesingle m never to
be bothered about money or business, but just enjoy myself, and live for
what I like. That\textquotesingle s my favorite castle.
What\textquotesingle s yours, Meg?"

Margaret seemed to find it a little hard to tell hers, and waved a brake
before her face, as if to disperse imaginary gnats, while she said
slowly, "I should like a lovely house, full of all sorts of luxurious
things,---nice food, pretty clothes, handsome furniture, pleasant
people, and heaps of money. I am to be mistress of it, and manage it as
I like, with plenty of servants, so I never need work a bit. How I
should enjoy it! for I wouldn\textquotesingle t be idle, but do good,
and make every one love me dearly."

\protect\phantomsection\label{6672479776654687619_37106-h-2.htm.xhtml_b076.png}{}
\pandocbounded{\includegraphics[keepaspectratio]{303483661336987339_b076.png}}

"Wouldn\textquotesingle t you have a master for your castle in the air?"
asked Laurie slyly.

"I said \textquotesingle pleasant people,\textquotesingle{} you know;"
and Meg carefully tied up her shoe as she spoke, so that no one saw her
face.

"Why don\textquotesingle t you say you\textquotesingle d have a
splendid, wise, good husband, and some angelic little children? You know
your castle wouldn\textquotesingle t be perfect without," said blunt Jo,
who had no tender fancies yet, and rather scorned romance, except in
books.

"You\textquotesingle d have nothing but horses, inkstands, and novels in
yours," answered Meg petulantly.

"Wouldn\textquotesingle t I, though? I\textquotesingle d have a stable
full of Arabian steeds, rooms piled with books, and I\textquotesingle d
write out of a magic inkstand, so that my works should be as famous as
Laurie\textquotesingle s music. I want to do something splendid before I
go into my castle,---something heroic or wonderful, that
won\textquotesingle t be forgotten after I\textquotesingle m dead. I
don\textquotesingle t know what, but I\textquotesingle m on the watch
for it, and mean to astonish you all, some day. I think I shall write
books, and get rich and famous: that would suit me, so that is \emph{my}
favorite dream."

"Mine is to stay at home safe with father and mother, and help take care
of the family," said Beth contentedly.

"Don\textquotesingle t you wish for anything else?" asked Laurie.

"Since I had my little piano, I am perfectly satisfied. I only wish we
may all keep well and be together; nothing else."

"I have ever so many wishes; but the pet one is to be an artist, and go
to Rome, and do fine pictures, and be the best artist in the whole
world," was Amy\textquotesingle s modest desire.

"We\textquotesingle re an ambitious set, aren\textquotesingle t we?
Every one of us, but Beth, wants to be rich and famous, and gorgeous in
every respect. I do wonder if any of us will ever get our wishes," said
Laurie, chewing grass, like a meditative calf.

"I\textquotesingle ve got the key to my castle in the air; but whether I
can unlock the door remains to be seen," observed Jo mysteriously.

"I\textquotesingle ve got the key to mine, but I\textquotesingle m not
allowed to try it. Hang college!" muttered Laurie, with an impatient
sigh.

"Here\textquotesingle s mine!" and Amy waved her pencil.

"I haven\textquotesingle t got any," said Meg forlornly.

"Yes, you have," said Laurie at once.

"Where?"

"In your face."

"Nonsense; that\textquotesingle s of no use."

"Wait and see if it doesn\textquotesingle t bring you something worth
having," replied the boy, laughing at the thought of a charming little
secret which he fancied he knew.

Meg colored behind the brake, but asked no questions, and looked across
the river with the same expectant expression which Mr. Brooke had worn
when he told the story of the knight.

"If we are all alive ten years hence, let\textquotesingle s meet, and
see how many of us have got our wishes, or how much nearer we are then
than now," said Jo, always ready with a plan.

"Bless me! how old I shall be,---twenty-seven!" exclaimed Meg who felt
grown up already, having just reached seventeen.

"You and I shall be twenty-six, Teddy, Beth twenty-four, and Amy
twenty-two. What a venerable party!" said Jo.

"I hope I shall have done something to be proud of by that time; but
I\textquotesingle m such a lazy dog, I\textquotesingle m afraid I shall
\textquotesingle dawdle,\textquotesingle{} Jo."

"You need a motive, mother says; and when you get it, she is sure
you\textquotesingle ll work splendidly."

"Is she? By Jupiter I will, if I only get the chance!" cried Laurie,
sitting up with sudden energy. "I ought to be satisfied to please
grandfather, and I do try, but it\textquotesingle s working against the
grain, you see, and comes hard. He wants me to be an India merchant, as
he was, and I\textquotesingle d rather be shot. I hate tea and silk and
spices, and every sort of rubbish his old ships bring, and I
don\textquotesingle t care how soon they go to the bottom when I own
them. Going to college ought to satisfy him, for if I give him four
years he ought to let me off from the business; but he\textquotesingle s
set, and I \textquotesingle ve got to do just as he did, unless I break
away and please myself, as my father did. If there was any one left to
stay with the old gentleman, I\textquotesingle d do it to-morrow."

Laurie spoke excitedly, and looked ready to carry his threat into
execution on the slightest provocation; for he was growing up very fast,
and, in spite of his indolent ways, had a young man\textquotesingle s
hatred of subjection, a young man\textquotesingle s restless longing to
try the world for himself.

"I advise you to sail away in one of your ships, and never come home
again till you have tried your own way," said Jo, whose imagination was
fired by the thought of such a daring exploit, and whose sympathy was
excited by what she called "Teddy\textquotesingle s wrongs."

"That\textquotesingle s not right, Jo; you mustn\textquotesingle t talk
in that way, and Laurie mustn\textquotesingle t take your bad advice.
You should do just what your grandfather wishes, my dear boy," said Meg,
in her most maternal tone. "Do your best at college, and, when he sees
that you try to please him, I\textquotesingle m sure he
won\textquotesingle t be hard or unjust to you. As you say, there is no
one else to stay with and love him, and you\textquotesingle d never
forgive yourself if you left him without his permission.
Don\textquotesingle t be dismal or fret, but do your duty; and
you\textquotesingle ll get your reward, as good Mr. Brooke has, by being
respected and loved."

"What do you know about him?" asked Laurie, grateful for the good
advice, but objecting to the lecture, and glad to turn the conversation
from himself, after his unusual outbreak.

"Only what your grandpa told us about him,---how he took good care of
his own mother till she died, and wouldn\textquotesingle t go abroad as
tutor to some nice person, because he wouldn\textquotesingle t leave
her; and how he provides now for an old woman who nursed his mother; and
never tells any one, but is just as generous and patient and good as he
can be."

"So he is, dear old fellow!" said Laurie heartily, as Meg paused,
looking flushed and earnest with her story. "It\textquotesingle s like
grandpa to find out all about him, without letting him know, and to tell
all his goodness to others, so that they might like him. Brooke
couldn\textquotesingle t understand why your mother was so kind to him,
asking him over with me, and treating him in her beautiful friendly way.
He thought she was just perfect, and talked about it for days and days,
and went on about you all in flaming style. If ever I do get my wish,
you see what I\textquotesingle ll do for Brooke."

"Begin to do something now, by not plaguing his life out," said Meg
sharply.

"How do you know I do, miss?"

"I can always tell by his face, when he goes away. If you have been
good, he looks satisfied and walks briskly; if you have plagued him,
he\textquotesingle s sober and walks slowly, as if he wanted to go back
and do his work better."

\protect\phantomsection\label{6672479776654687619_37106-h-2.htm.xhtml_b077.png}{}
\pandocbounded{\includegraphics[keepaspectratio]{303483661336987339_b077.png}}

"Well, I like that! So you keep an account of my good and bad marks in
Brooke\textquotesingle s face, do you? I see him bow and smile as he
passes your window, but I didn\textquotesingle t know
you\textquotesingle d got up a telegraph."

"We haven\textquotesingle t; don\textquotesingle t be angry, and oh,
don\textquotesingle t tell him I said anything! It was only to show that
I cared how you get on, and what is said here is said in confidence, you
know," cried Meg, much alarmed at the thought of what might follow from
her careless speech.

"\emph{I} don\textquotesingle t tell tales," replied Laurie, with his
"high and mighty" air, as Jo called a certain expression which he
occasionally wore. "Only if Brooke is going to be a thermometer, I must
mind and have fair weather for him to report."

"Please don\textquotesingle t be offended. I didn\textquotesingle t mean
to preach or tell tales or be silly; I only thought Jo was encouraging
you in a feeling which you\textquotesingle d be sorry for, by and by.
You are so kind to us, we feel as if you were our brother, and say just
what we think. Forgive me, I meant it kindly." And Meg offered her hand
with a gesture both affectionate and timid.

Ashamed of his momentary pique, Laurie squeezed the kind little hand,
and said frankly, "I\textquotesingle m the one to be forgiven;
I\textquotesingle m cross, and have been out of sorts all day. I like to
have you tell me my faults and be sisterly, so don\textquotesingle t
mind if I am grumpy sometimes; I thank you all the same."

Bent on showing that he was not offended, he made himself as agreeable
as possible,---wound cotton for Meg, recited poetry to please Jo, shook
down cones for Beth, and helped Amy with her ferns, proving himself a
fit person to belong to the "Busy Bee Society." In the midst of an
animated discussion on the domestic habits of turtles (one of those
amiable creatures having strolled up from the river), the faint sound of
a bell warned them that Hannah had put the tea "to draw," and they would
just have time to get home to supper.

"May I come again?" asked Laurie.

"Yes, if you are good, and love your book, as the boys in the primer are
told to do," said Meg smiling.

"I\textquotesingle ll try."

"Then you may come, and I\textquotesingle ll teach you to knit as the
Scotchmen do; there\textquotesingle s a demand for socks just now,"
added Jo, waving hers, like a big blue worsted banner, as they parted at
the gate.

That night, when Beth played to Mr. Laurence in the twilight, Laurie,
standing in the shadow of the curtain, listened to the little David,
whose simple music always quieted his moody spirit, and watched the old
man, who sat with his gray head on his hand, thinking tender thoughts of
the dead child he had loved so much. Remembering the conversation of the
afternoon, the boy said to himself, with the resolve to make the
sacrifice cheerfully, "I\textquotesingle ll let my castle go, and stay
with the dear old gentleman while he needs me, for I am all he has."

\protect\phantomsection\label{6672479776654687619_37106-h-2.htm.xhtml_b078.png}{}
\pandocbounded{\includegraphics[keepaspectratio]{303483661336987339_b078.png}}

\begin{center}\rule{0.5\linewidth}{0.5pt}\end{center}

\subsection{XIV.
Secrets.}\label{6672479776654687619_37106-h-2.htm.xhtml_pgepubid00016}

\protect\phantomsection\label{6672479776654687619_37106-h-2.htm.xhtml_b079.png}{}
\pandocbounded{\includegraphics[keepaspectratio]{303483661336987339_b079.png}}

\protect\phantomsection\label{6672479776654687619_37106-h-2.htm.xhtml_XIV}{}\hyperref[6672479776654687619_37106-h-0.htm.xhtml_contents1b]{XIV.}

SECRETS.

{Jo} was very busy in the garret, for the October days began to grow
chilly, and the afternoons were short. For two or three hours the sun
lay warmly in the high window, showing Jo seated on the old sofa,
writing busily, with her papers spread out upon a trunk before her,
while Scrabble, the pet rat, promenaded the beams overhead, accompanied
by his oldest son, a fine young fellow, who was evidently very proud of
his whiskers. Quite absorbed in her work, Jo scribbled away till the
last page was filled, when she signed her name with a flourish, and
threw down her pen, exclaiming,---

"There, I\textquotesingle ve done my best! If this won\textquotesingle t
suit I shall have to wait till I can do better."

Lying back on the sofa, she read the manuscript carefully through,
making dashes here and there, and putting in many exclamation points,
which looked like little balloons; then she tied it up with a smart red
ribbon, and sat a minute looking at it with a sober, wistful expression,
which plainly showed how earnest her work had been. Jo\textquotesingle s
desk up here was an old tin kitchen, which hung against the wall. In it
she kept her papers and a few books, safely shut away from Scrabble,
who, being likewise of a literary turn, was fond of making a circulating
library of such books as were left in his way, by eating the leaves.
From this \ul{tin receptacle Jo produced another manuscript;} and,
putting both in her pocket, crept quietly down stairs, leaving her
friends to nibble her pens and taste her ink.

She put on her hat and jacket as noiselessly as possible, and, going to
the back entry window, got out upon the roof of a low porch, swung
herself down to the grassy bank, and took a roundabout way to the road.
Once there, she composed herself, hailed a passing omnibus, and rolled
away to town, looking very merry and mysterious.

If any one had been watching her, he would have thought her movements
decidedly peculiar; for, on alighting, she went off at a great pace till
she reached a certain number in a certain busy street; having found the
place with some difficulty, she went into the door-way, looked up the
dirty stairs, and, after standing stock still a minute, suddenly dived
into the street, and walked away as rapidly as she came. This manœuvre
she repeated several times, to the great amusement of a black-eyed young
gentleman lounging in the window of a building opposite. On returning
for the third time, Jo gave herself a shake, pulled her hat over her
eyes, and walked up the stairs, looking as if she were going to have all
her teeth out.

There was a dentist\textquotesingle s sign, among others, which adorned
the entrance, and, after staring a moment at the pair of artificial jaws
which slowly opened and shut to draw attention to a fine set of teeth,
the young gentleman put on his coat, took his hat, and went down to post
himself in the opposite \ul{door-way, saying,} with a smile and a
shiver,---

"It\textquotesingle s like her to come alone, but if she has a bad time
she\textquotesingle ll need some one to help her home."

In ten minutes Jo came running down stairs with a very red face, and the
general appearance of a person who had just passed through a trying
ordeal of some sort. When she saw the young gentleman she looked
anything but pleased, and passed him with a nod; but he followed, asking
with an air of sympathy,---

"Did you have a bad time?"

"Not very."

"You got through quickly."

"Yes, thank goodness!"

"Why did you go alone?"

"Didn\textquotesingle t want any one to know."

"You\textquotesingle re the oddest fellow I ever saw. How many did you
have out?"

Jo looked at her friend as if she did not understand him; then began to
laugh, as if mightily amused at something.

"There are two which I want to have come out, but I must wait a week."

"What are you laughing at? You are up to some mischief, Jo," said
Laurie, looking mystified.

"So are you. What were you doing, sir, up in that billiard saloon?"

"Begging your pardon, ma\textquotesingle am, it wasn\textquotesingle t a
billiard saloon, but a gymnasium, and I was taking a lesson in fencing."

"I\textquotesingle m glad of that."

"Why?"

"You can teach me, and then when we play Hamlet, you can be Laertes, and
we\textquotesingle ll make a fine thing of the fencing scene."

Laurie burst out with a hearty boy\textquotesingle s laugh, which made
several passers-by smile in spite of themselves.

"I\textquotesingle ll teach you whether we play Hamlet or not;
it\textquotesingle s grand fun, and will straighten you up capitally.
But I don\textquotesingle t believe that was your only reason for saying
\textquotesingle I\textquotesingle m glad,\textquotesingle{} in that
decided way; was it, now?"

"No, I was glad that you were not in the saloon, because I hope you
never go to such places. Do you?"

"Not often."

"I wish you wouldn\textquotesingle t."

"It\textquotesingle s no harm, Jo. I have billiards at home, but
it\textquotesingle s no fun unless you have good players; so, as
I\textquotesingle m fond of it, I come sometimes and have a game with
Ned Moffat or some of the other fellows."

"Oh dear, I\textquotesingle m so sorry, for you\textquotesingle ll get
to liking it better and better, and will waste time and money, and grow
like those dreadful boys. I did hope you\textquotesingle d stay
respectable, and be a satisfaction to your friends," said Jo, shaking
her head.

"Can\textquotesingle t a fellow take a little innocent amusement now and
then without losing his respectability?" asked Laurie, looking nettled.

"That depends upon how and where he takes it. I don\textquotesingle t
like Ned and his set, and wish you\textquotesingle d keep out of it.
Mother won\textquotesingle t let us have him at our house, though he
wants to come; and if you grow like him she won\textquotesingle t be
willing to have us frolic together as we do now."

"Won\textquotesingle t she?" asked Laurie anxiously.

"No, she can\textquotesingle t bear fashionable young men, and
she\textquotesingle d shut us all up in bandboxes rather than have us
associate with them."

"Well, she needn\textquotesingle t get out her bandboxes yet;
I\textquotesingle m not a fashionable party, and don\textquotesingle t
mean to be; but I do like harmless larks now and then,
don\textquotesingle t you?"

"Yes, nobody minds them, so lark away, but don\textquotesingle t get
wild, will you? or there will be an end of all our good times."

"I\textquotesingle ll be a double-distilled saint."

"I can\textquotesingle t bear saints: just be a simple, honest,
respectable boy, and we\textquotesingle ll never desert you. I
don\textquotesingle t know what I \emph{should} do if you acted like Mr.
King\textquotesingle s son; he had plenty of money, but
didn\textquotesingle t know how to spend it, and got tipsy and gambled,
and ran away, and forged his father\textquotesingle s name, I believe,
and was altogether horrid."

"You think I\textquotesingle m likely to do the same? Much obliged."

"No, I don\textquotesingle t---oh, \emph{dear}, no!---but I hear people
talking about money being such a temptation, and I sometimes wish you
were poor; I shouldn\textquotesingle t worry then."

"Do you worry about me, Jo?"

"A little, when you look moody or discontented, as you sometimes do; for
you\textquotesingle ve got such a strong will, if you once get started
wrong, I\textquotesingle m afraid it would be hard to stop you."

Laurie walked in silence a few minutes, and Jo watched him, wishing she
had held her tongue, for his eyes looked angry, though his lips still
smiled as if at her warnings.

"Are you going to deliver lectures all the way home?" he asked
presently.

"Of course not; why?"

"Because if you are, I\textquotesingle ll take a \textquotesingle bus;
if you are not, I\textquotesingle d like to walk with you, and tell you
something very interesting."

"I won\textquotesingle t preach any more, and I\textquotesingle d like
to hear the news immensely."

"Very well, then; come on. It\textquotesingle s a secret, and if I tell
you, you must tell me yours."

"I haven\textquotesingle t got any," began Jo, but stopped suddenly,
remembering that she had.

"You know you have,---you can\textquotesingle t hide anything; so up and
\textquotesingle fess, or I won\textquotesingle t tell," cried Laurie.

"Is your secret a nice one?"

"Oh, isn\textquotesingle t it! all about people you know, and such fun!
You ought to hear it, and I\textquotesingle ve been aching to tell it
this long time. Come, you begin."

"You\textquotesingle ll not say anything about it at home, will you?"

"Not a word."

"And you won\textquotesingle t tease me in private?"

"I never tease."

"Yes, you do; you get everything you want out of people. I
don\textquotesingle t know how you do it, but you are a born wheedler."

"Thank you; fire away."

"Well, I\textquotesingle ve left two stories with a newspaper man, and
he\textquotesingle s to give his answer next week," whispered Jo, in her
confidant\textquotesingle s ear.

\protect\phantomsection\label{6672479776654687619_37106-h-2.htm.xhtml_b080.png}{}
\pandocbounded{\includegraphics[keepaspectratio]{303483661336987339_b080.png}}

"Hurrah for Miss March, the celebrated American authoress!" cried
Laurie, throwing up his hat and catching it again, to the great delight
of two ducks, four cats, five hens, and half a dozen Irish children; for
they were out of the city now.

"Hush! It won\textquotesingle t come to anything, I dare say; but I
couldn\textquotesingle t rest till I had tried, and I said nothing about
it, because I didn\textquotesingle t want any one else to be
disappointed."

"It won\textquotesingle t fail. Why, Jo, your stories are works of
Shakespeare, compared to half the rubbish \ul{that is published every
day.} Won\textquotesingle t it be fun to see them in print; and
sha\textquotesingle n\textquotesingle t we feel proud of our authoress?"

Jo\textquotesingle s eyes sparkled, for it is always pleasant to be
believed in; and a friend\textquotesingle s praise is always sweeter
than a dozen newspaper puffs.

"Where\textquotesingle s \emph{your} secret? Play fair, Teddy, or
I\textquotesingle ll never believe you again," she said, trying to
extinguish the brilliant hopes that blazed up at a word of
encouragement.

"I may get into a scrape for telling; but I didn\textquotesingle t
promise not to, so I will, for I never feel easy in my mind till
I\textquotesingle ve told you any plummy bit of news I get. I know where
Meg\textquotesingle s glove is."

"Is that all?" said Jo, looking disappointed, as Laurie nodded and
twinkled, with a face full of mysterious intelligence.

"It\textquotesingle s quite enough for the present, as
you\textquotesingle ll agree when I tell you where it is."

"Tell, then."

Laurie bent, and whispered three words in Jo\textquotesingle s ear,
which produced a comical change. She stood and stared at him for a
minute, looking both surprised and displeased, then walked on, saying
sharply, "How do you know?"

"Saw it."

"Where?"

"Pocket."

"All this time?"

"Yes; isn\textquotesingle t that romantic?"

"No, it\textquotesingle s horrid."

"Don\textquotesingle t you like it?"

"Of course I don\textquotesingle t. It\textquotesingle s ridiculous; it
won\textquotesingle t be allowed. My patience! what would Meg say?"

"You are not to tell any one; mind that."

"I didn\textquotesingle t promise."

"That was understood, and I trusted you."

"Well, I won\textquotesingle t for the present, any way; but
I\textquotesingle m disgusted, and wish you hadn\textquotesingle t told
me."

"I thought you\textquotesingle d be pleased."

"At the idea of anybody coming to take Meg away? No, thank you."

"You\textquotesingle ll feel better about it when somebody comes to take
you away."

"I\textquotesingle d like to see any one try it," cried Jo fiercely.

"So should I!" and Laurie chuckled at the idea.

"I don\textquotesingle t think secrets agree with me; I feel rumpled up
in my mind since you told me that," said Jo, rather ungratefully.

"Race down this hill with me, and you\textquotesingle ll be all right,"
suggested Laurie.

\protect\phantomsection\label{6672479776654687619_37106-h-2.htm.xhtml_b081.png}{}
\pandocbounded{\includegraphics[keepaspectratio]{303483661336987339_b081.png}}

No one was in sight; the smooth road sloped invitingly before her; and
finding the temptation irresistible, Jo darted away, soon leaving hat
and comb behind her, and scattering hair-pins as she ran. Laurie reached
the goal first, and was quite satisfied with the success of his
treatment; for his Atalanta came panting up, with flying hair, bright
eyes, ruddy cheeks, and no signs of dissatisfaction in her face.

"I wish I was a horse; then I could run for miles in this splendid air,
and not lose my breath. It was capital; but see what a guy
it\textquotesingle s made me. Go, pick up my things, like a cherub as
you are," said Jo, dropping down under a maple-tree, which was carpeting
the bank with crimson leaves.

Laurie leisurely departed to recover the lost property, and Jo bundled
up her braids, hoping no one would pass by till she was tidy again. But
some one did pass, and who should it be but Meg, looking particularly
ladylike in her state and festival suit, for she had been making calls.

"What in the world are you doing here?" she asked, regarding her
dishevelled sister with well-bred surprise.

"Getting leaves," meekly answered Jo, sorting the rosy handful she had
just swept up.

"And hair-pins," added Laurie, throwing half a dozen into
Jo\textquotesingle s lap. "They grow on this road, Meg; so do combs and
brown straw hats."

"You have been running, Jo; how could you? When \emph{will} you stop
such romping ways?" said Meg reprovingly, as she settled her cuffs, and
smoothed her hair, with which the wind had taken liberties.

"Never till I\textquotesingle m stiff and old, and have to use a crutch.
Don\textquotesingle t try to make me grow up before my time, Meg:
it\textquotesingle s hard enough to have you change all of a sudden; let
me be a little girl as long as I can."

As she spoke, Jo bent over the leaves to hide the trembling of her lips;
for lately she had felt that Margaret was fast getting to be a woman,
and Laurie\textquotesingle s secret made her dread the separation which
must surely come some time, and now seemed very near. He saw the trouble
in her face, and drew Meg\textquotesingle s attention from it by asking
quickly, "Where have you been calling, all so fine?"

"At the Gardiners\textquotesingle, and Sallie has been telling me all
about Belle Moffat\textquotesingle s wedding. It was very splendid, and
they have gone to spend the winter in Paris. Just think how delightful
that must be!"

"Do you envy her, Meg?" said Laurie.

"I\textquotesingle m afraid I do."

"I\textquotesingle m glad of it!" muttered Jo, tying on her hat with a
jerk.

"Why?" asked Meg, looking surprised.

"Because if you care much about riches, you will never go and marry a
poor man," said Jo, frowning at Laurie, who was mutely warning her to
mind what she said.

"I shall never \textquotesingle{}\emph{go} and marry\textquotesingle{}
any one," observed Meg, walking on with great dignity, while the others
followed, laughing, whispering, skipping stones, and "behaving like
children," as Meg said to herself, though she might have been tempted to
join them if she had not had her best dress on.

For a week or two, Jo behaved so queerly that her sisters were quite
bewildered. She rushed to the door when the postman rang; was rude to
Mr. Brooke whenever they met; would sit looking at Meg with a woe-begone
face, occasionally jumping up to shake, and then to kiss her, in a very
mysterious manner; Laurie and she were always making signs to one
another, and talking about "Spread Eagles," till the girls declared they
had both lost their wits. On the second Saturday after Jo got out of the
window, Meg, as she sat sewing at her window, was scandalized by the
sight of Laurie chasing Jo all over the garden, and finally capturing
her in Amy\textquotesingle s bower. What went on there, Meg could not
see; but shrieks of laughter were heard, followed by the murmur of
voices and a great flapping of newspapers.

"What shall we do with that girl? She never \emph{will} behave like a
young lady," sighed Meg, as she watched the race with a disapproving
face.

"I hope she won\textquotesingle t; she is so funny and dear as she is,"
said Beth, who had never betrayed that she was a little hurt at
Jo\textquotesingle s having secrets with any one but her.

"It\textquotesingle s very trying, but we never can make her \emph{commy
la fo}," added Amy, who sat making some new frills for herself, with her
curls tied up in a very becoming way,---two agreeable things, which made
her feel unusually elegant and ladylike.

In a few minutes Jo bounced in, laid herself on the sofa, and affected
to read.

\protect\phantomsection\label{6672479776654687619_37106-h-2.htm.xhtml_b082.png}{}
\pandocbounded{\includegraphics[keepaspectratio]{303483661336987339_b082.png}}

"Have you anything interesting there?" asked Meg, with condescension.

"Nothing but a story; won\textquotesingle t amount to much, I guess,"
returned Jo, carefully keeping the name of the paper out of sight.

"You\textquotesingle d better read it aloud; that will amuse us and keep
you out of mischief," said Amy, in her most grown-up tone.

"What\textquotesingle s the name?" asked Beth, wondering why Jo kept her
face behind the sheet.

"The Rival Painters."

"That sounds well; read it," said Meg.

With a loud "Hem!" and a long breath, Jo began to read very fast. The
girls listened with interest, for the tale was romantic, and somewhat
pathetic, as most of the characters died in the end.

"I like that about the splendid picture," was Amy\textquotesingle s
approving remark, as Jo paused.

"I prefer the lovering part. Viola and Angelo are two of our favorite
names; isn\textquotesingle t that queer?" said Meg, wiping her eyes, for
the "lovering part" was tragical.

"Who wrote it?" asked Beth, who had caught a glimpse of
Jo\textquotesingle s face.

The reader suddenly sat up, cast away the paper, displaying a flushed
countenance, and, with a funny mixture of solemnity and excitement,
replied in a loud voice, "Your sister."

"You?" cried Meg, dropping her work.

"It\textquotesingle s very good," said Amy critically.

"I knew it! I knew it! O my Jo, I \emph{am} so proud!" and Beth ran to
hug her sister, and exult over this splendid success.

Dear me, how delighted they all were, to be sure! how Meg
wouldn\textquotesingle t believe it till she saw the words, "Miss
Josephine March," actually printed in the paper; how graciously Amy
criticised the artistic parts of the story, and offered hints for a
sequel, which unfortunately couldn\textquotesingle t be carried out, as
the hero and heroine were dead; how Beth got excited, and skipped and
sung with joy; how Hannah came in to exclaim "Sakes alive, well I
never!" in great astonishment at "that Jo\textquotesingle s
doin\textquotesingle s;" how proud Mrs. March was when she knew it; how
Jo laughed, with tears in her eyes, as she declared she might as well be
a peacock and done with it; and how the "Spread Eagle" might be said to
flap his wings triumphantly over the House of March, as the paper passed
from hand to hand.

"Tell us all about it." "When did it come?" "How much did you get for
it?" "What \emph{will} father say?" "Won\textquotesingle t Laurie
laugh?" cried the family, all in one breath, as they clustered about Jo;
for these foolish, affectionate people made a jubilee of every little
household joy.

"Stop jabbering, girls, and I\textquotesingle ll tell you everything,"
said Jo, wondering if Miss Burney felt any grander over her "Evelina"
than she did over her "Rival Painters." Having told how she disposed of
her tales, Jo added, "And when I went to get my answer, the man said he
liked them both, but didn\textquotesingle t pay beginners, only let them
print in his paper, and noticed the stories. It was good practice, he
said; and when the beginners improved, any one would pay. So I let him
have the two stories, and to-day this was sent to me, and Laurie caught
me with it, and insisted on seeing it, so I let him; and he said it was
good, and I shall write more, and he\textquotesingle s going to get the
next paid for, and I \emph{am} so happy, for in time I may be able to
support myself and help the girls."

Jo\textquotesingle s breath gave out here; and, wrapping her head in the
paper, she bedewed her little story with a few natural tears; for to be
independent, and earn the praise of those she loved were the dearest
wishes of her heart, and this seemed to be the first step toward that
happy end.

\begin{center}\rule{0.5\linewidth}{0.5pt}\end{center}

\subsection{XV. A
Telegram.}\label{6672479776654687619_37106-h-2.htm.xhtml_pgepubid00017}

\protect\phantomsection\label{6672479776654687619_37106-h-2.htm.xhtml_XV}{}\hyperref[6672479776654687619_37106-h-0.htm.xhtml_contents1b]{XV.}

A TELEGRAM.

\protect\phantomsection\label{6672479776654687619_37106-h-3.htm.xhtml}{}

\protect\phantomsection\label{6672479776654687619_37106-h-3.htm.xhtml_b083.png}{}
\pandocbounded{\includegraphics[keepaspectratio]{303483661336987339_b083.png}}

{"November} is the most disagreeable month in the whole year," said
Margaret, standing at the window one dull afternoon, looking out at the
frost-bitten garden.

"That\textquotesingle s the reason I was born in it," observed Jo
pensively, quite unconscious of the blot on her nose.

"If something very pleasant should happen now, we should think it a
delightful month," said Beth, who took a hopeful view of everything,
even November.

"I dare say; but nothing pleasant ever \emph{does} happen in this
family," said Meg, who was out of sorts. "We go grubbing along day after
day, without a bit of change, and very little fun. We might as well be
in a treadmill."

"My patience, how blue we are!" cried Jo. "I don\textquotesingle t much
wonder, poor dear, for you see other girls having splendid times, while
you grind, grind, year in and year out. Oh, don\textquotesingle t I wish
I could manage things for you as I do for my heroines!
You\textquotesingle re pretty enough and good enough already, so
I\textquotesingle d have some rich relation leave you a fortune
unexpectedly; then you\textquotesingle d dash out as an heiress, scorn
every one who has slighted you, go abroad, and come home my Lady
Something, in a blaze of splendor and elegance."

"People don\textquotesingle t have fortunes left them in that style
now-a-days; men have to work, and women to marry for money.
It\textquotesingle s a dreadfully unjust world," said Meg bitterly.

"Jo and I are going to make fortunes for you all; just wait ten years,
and see if we don\textquotesingle t," said Amy, who sat in a corner,
making mud pies, as Hannah called her little clay models of birds,
fruit, and faces.

"Can\textquotesingle t wait, and I\textquotesingle m afraid I
haven\textquotesingle t much faith in ink and dirt, though
I\textquotesingle m grateful for your good intentions."

Meg sighed, and turned to the frost-bitten garden again; Jo groaned, and
leaned both elbows on the table in a despondent attitude, but Amy
spatted away energetically; and Beth, who sat at the other window, said,
smiling, "Two pleasant things are going to happen right away: Marmee is
coming down the street, and Laurie is tramping through the garden as if
he had something nice to tell."

In they both came, Mrs. March with her usual question, "Any letter from
father, girls?" and Laurie to say in his persuasive way,
"Won\textquotesingle t some of you come for a drive?
I\textquotesingle ve been working away at mathematics till my head is in
a muddle, and I\textquotesingle m going to freshen my wits by a brisk
turn. It\textquotesingle s a dull day, but the air isn\textquotesingle t
bad, and I\textquotesingle m going to take Brooke home, so it will be
gay inside, if it isn\textquotesingle t out. Come, Jo, you and Beth will
go, won\textquotesingle t you?"

"Of course we will."

"Much obliged, but I\textquotesingle m busy;" and Meg whisked out her
work-basket, for she had agreed with her mother that it was best, for
her at least, not to drive often with the young gentleman.

"We three will be ready in a minute," cried Amy, running away to wash
her hands.

"Can I do anything for you, Madam Mother?" asked Laurie, leaning over
Mrs. March\textquotesingle s chair, with the affectionate look and tone
he always gave her.

"No, thank you, except call at the office, if you\textquotesingle ll be
so kind, dear. It\textquotesingle s our day for a letter, and the
postman hasn\textquotesingle t been. Father is as regular as the sun,
but there\textquotesingle s some delay on the way, perhaps."

A sharp ring interrupted her, and a minute after Hannah came in with a
letter.

"It\textquotesingle s one of them horrid telegraph things, mum," she
said, handing it as if she was afraid it would explode and do some
damage.

\protect\phantomsection\label{6672479776654687619_37106-h-3.htm.xhtml_b084.png}{}
\pandocbounded{\includegraphics[keepaspectratio]{303483661336987339_b084.png}}

At the word "telegraph," Mrs. March snatched it, read the two lines it
contained, and dropped back into her chair as white as if the little
paper had sent a bullet to her heart. Laurie dashed down stairs for
water, while Meg and Hannah supported her, and Jo read aloud, in a
frightened voice,---

\begin{quote}
"Mrs. March:

"Your husband is very ill. Come at once.

{S. Hale,}

"Blank Hospital, Washington"
\end{quote}

How still the room was as they listened breathlessly, how strangely the
day darkened outside, and how suddenly the whole world seemed to change,
as the girls gathered about their mother, feeling as if all the
happiness and support of their lives was about to be taken from them.
Mrs. March was herself again directly; read the message over, and
stretched out her arms to her daughters, saying, in a tone they never
forgot, "I shall go at once, but it may be too late. O children,
children, help me to bear it!"

For several minutes there was nothing but the sound of sobbing in the
room, mingled with broken words of comfort, tender assurances of help,
and hopeful whispers that died away in tears. Poor Hannah was the first
to recover, and with unconscious wisdom she set all the rest a good
example; for, with her, work was the panacea for most afflictions.

"The Lord keep the dear man! I won\textquotesingle t waste no time a
cryin\textquotesingle, but git your things ready right away, mum," she
said, heartily, as she wiped her face on her apron, gave her mistress a
warm shake of the hand with her own hard one, and went away, to work
like three women in one.

"She\textquotesingle s right; there\textquotesingle s no time for tears
now. Be calm, girls, and let me think."

They tried to be calm, poor things, as their mother sat up, looking
pale, but steady, and put away her grief to think and plan for them.

"Where\textquotesingle s Laurie?" she asked presently, when she had
collected her thoughts, and decided on the first duties to be done.

"Here, ma\textquotesingle am. Oh, let me do something!" cried the boy,
hurrying from the next room, whither he had withdrawn, feeling that
their first sorrow was too sacred for even his friendly eyes to see.

"Send a telegram saying I will come at once. The next train goes early
in the morning. I\textquotesingle ll take that."

"What else? The horses are ready; I can go anywhere, do anything," he
said, looking ready to fly to the ends of the earth.

"Leave a note at Aunt March\textquotesingle s. Jo, give me that pen and
paper."

Tearing off the blank side of one of her newly copied pages, Jo drew the
table before her mother, well knowing that money for the long, sad
journey must be borrowed, and feeling as if she could do anything to add
a little to the sum for her father.

"Now go, dear; but don\textquotesingle t kill yourself driving at a
desperate pace; there is no need of that."

Mrs. March\textquotesingle s warning was evidently thrown away; for five
minutes later Laurie tore by the window on his own fleet horse, riding
as if for his life.

"Jo, run to the rooms, and tell Mrs. King that I can\textquotesingle t
come. On the way get these things. I\textquotesingle ll put them down;
they\textquotesingle ll be needed, and I must go prepared for nursing.
Hospital stores are not always good. Beth, go and ask Mr. Laurence for a
couple of bottles of old wine: I\textquotesingle m not too proud to beg
for father; he shall have the best of everything. Amy, tell Hannah to
get down the black trunk; and, Meg, come and help me find my things, for
I\textquotesingle m half bewildered."

Writing, thinking, and directing, all at once, might well bewilder the
poor lady, and Meg begged her to sit quietly in her room for a little
while, and let them work. Every one scattered like leaves before a gust
of wind; and the quiet, happy household was broken up as suddenly as if
the paper had been an evil spell.

Mr. Laurence came hurrying back with Beth, bringing every comfort the
kind old gentleman could think of for the invalid, and friendliest
promises of protection for the girls during the mother\textquotesingle s
absence, which comforted her very much. There was nothing he
didn\textquotesingle t offer, from his own dressing-gown to himself as
escort. But that last was impossible. Mrs. March would not hear of the
old gentleman\textquotesingle s undertaking the long journey; yet an
expression of relief was visible when he spoke of it, for anxiety ill
fits one for travelling. He saw the look, knit his heavy eyebrows,
rubbed his hands, and marched abruptly away, saying he\textquotesingle d
be back directly. No one had time to think of him again till, as Meg ran
through the entry, with a pair of rubbers in one hand and a cup of tea
in the other, she came suddenly upon Mr. Brooke.

\protect\phantomsection\label{6672479776654687619_37106-h-3.htm.xhtml_b085.png}{}
\pandocbounded{\includegraphics[keepaspectratio]{303483661336987339_b085.png}}

"I\textquotesingle m very sorry to hear of this, Miss March," he said,
in the kind, quiet tone which sounded very pleasantly to her perturbed
spirit. "I came to offer myself as escort to your mother. Mr. Laurence
has commissions for me in Washington, and it will give me real
satisfaction to be of service to her there."

Down dropped the rubbers, and the tea was very near following, as Meg
put out her hand, with a face so full of gratitude, that Mr. Brooke
would have felt repaid for a much greater sacrifice than the trifling
one of time and comfort which he was about to make.

"How kind you all are! Mother will accept, I\textquotesingle m sure; and
it will be such a relief to know that she has some one to take care of
her. Thank you very, very much!"

Meg spoke earnestly, and forgot herself entirely till something in the
brown eyes looking down at her made her remember the cooling tea, and
lead the way into the parlor, saying she would call her mother.

Everything was arranged by the time Laurie returned with a note from
Aunt March, enclosing the desired sum, and a few lines repeating what
she had often said before,---that she had always told them it was absurd
for March to go into the army, always predicted that no good would come
of it, and she hoped they would take her advice next time. Mrs. March
put the note in the fire, the money in her purse, and went on with her
preparations, with her lips folded tightly, in a way which Jo would have
understood if she had been there.

The short afternoon wore away; all the other errands were done, and Meg
and her mother busy at some necessary \ul{needle-work, while} Beth and
Amy got tea, and Hannah finished her ironing with what she called a
"slap and a bang," but still Jo did not come. They began to get anxious;
and Laurie went off to find her, for no one ever knew what freak Jo
might take into her head. He missed her, however, and she came walking
in with a very queer expression of countenance, for there was a mixture
of fun and fear, satisfaction and regret, in it, which puzzled the
family as much as did the roll of bills she laid before her mother,
saying, with a little choke in her voice, "That\textquotesingle s my
contribution towards making father comfortable and bringing him home!"

"My dear, where did you get it? Twenty-five dollars! Jo, I hope you
haven\textquotesingle t done anything rash?

"No, it\textquotesingle s mine honestly; I didn\textquotesingle t beg,
borrow, or steal it. I earned it; and I don\textquotesingle t think
you\textquotesingle ll blame me, for I only sold what was my own."

As she spoke, Jo took off her bonnet, and a general outcry arose, for
all her abundant hair was cut short.

"Your hair! Your beautiful hair!" "O Jo, how could you? Your one
beauty." "My dear girl, there was no need of this." "She
doesn\textquotesingle t look like my Jo any more, but I love her dearly
for it!"

As every one exclaimed, and Beth hugged the cropped head tenderly, Jo
assumed an indifferent air, which did not deceive any one a particle,
and said, rumpling up the brown bush, and trying to look as if she liked
it, "It doesn\textquotesingle t affect the fate of the nation, so
don\textquotesingle t wail, Beth. It will be good for my vanity; I was
getting too proud of my wig. It will do my brains good to have that mop
taken off; my head feels deliciously light and cool, and the barber said
I could soon have a curly crop, which will be boyish, becoming, and easy
to keep in order. I\textquotesingle m satisfied; so please take the
money, and let\textquotesingle s have supper."

"Tell me all about it, Jo. \emph{I} am not quite satisfied, but I
can\textquotesingle t blame you, for I know how willingly you sacrificed
your vanity, as you call it, to your love. But, my dear, it was not
necessary, and I\textquotesingle m afraid you will regret it, one of
these days," said Mrs. March.

"No, I won\textquotesingle t!" returned Jo stoutly, feeling much
relieved that her prank was not entirely condemned.

"What made you do it?" asked Amy, who would as soon have thought of
cutting off her head as her pretty hair.

"Well, I was wild to do something for father," replied Jo, as they
gathered about the table, for healthy young people can eat even in the
midst of trouble. "I hate to borrow as much as mother does, and I knew
Aunt March would croak; she always does, if you ask for a ninepence. Meg
gave all her quarterly salary toward the rent, and I only got some
clothes with mine, so I felt wicked, and was bound to have some money,
if I sold the nose off my face to get it."

"You needn\textquotesingle t feel wicked, my child: you had no winter
things, and got the simplest with your own hard earnings," said Mrs.
March, with a look that warmed Jo\textquotesingle s heart.

"I hadn\textquotesingle t the least idea of selling my hair at first,
but as I went along I kept thinking what I could do, and feeling as if
I\textquotesingle d like to dive into some of the rich stores and help
myself. In a barber\textquotesingle s window I saw tails of hair with
the prices marked; and one black tail, not so thick as mine, was forty
dollars. It came over me all of a sudden that I had one thing to make
money out of, and without stopping to think, I walked in, asked if they
bought hair, and what they would give for mine."

"I don\textquotesingle t see how you dared to do it," said Beth, in a
tone of awe.

"Oh, he was a little man who looked as if he merely lived to oil his
hair. He rather stared, at first, as if he wasn\textquotesingle t used
to having girls bounce into his shop and ask him to buy their hair. He
said he didn\textquotesingle t care about mine, it
wasn\textquotesingle t the fashionable color, and he never paid much for
it in the first place; the work put into it made it dear, and so on. It
was getting late, and I was afraid, if it wasn\textquotesingle t done
right away, that I shouldn\textquotesingle t have it done at all, and
you know when I start to do a thing, I hate to give it up; so I begged
him to take it, and told him why I was in such a hurry. It was silly, I
dare say, but it changed his mind, for I got rather excited, and told
the story in my topsy-turvy way, and his wife heard, and said so
kindly,---

"\textquotesingle Take it, Thomas, and oblige the young lady;
I\textquotesingle d do as much for our Jimmy any day if I had a spire of
hair worth selling.\textquotesingle"

"Who was Jimmy?" asked Amy, who liked to have things explained as they
went along.

"Her son, she said, who was in the army. How friendly such things make
strangers feel, don\textquotesingle t they? She talked away all the time
the man clipped, and diverted my mind nicely."

\protect\phantomsection\label{6672479776654687619_37106-h-3.htm.xhtml_b086.png}{}
\pandocbounded{\includegraphics[keepaspectratio]{303483661336987339_b086.png}}

"Didn\textquotesingle t you feel dreadfully when the first cut came?"
asked Meg, with a shiver.

"I took a last look at my hair while the man got his things, and that
was the end of it. I never snivel over trifles like that; I will
confess, though, I felt queer when I saw the dear old hair laid out on
the table, and felt only the short, rough ends on my head. It almost
seemed as if I\textquotesingle d an arm or a leg off. The woman saw me
look at it, and picked out a long lock for me to keep.
I\textquotesingle ll give it to you, Marmee, just to remember past
glories by; for a crop is so comfortable I don\textquotesingle t think I
shall ever have a mane again."

Mrs. March folded the wavy, chestnut lock, and laid it away with a short
gray one in her desk. She only said "Thank you, deary," but something in
her face made the girls change the subject, and talk as cheerfully as
they could about Mr. Brooke\textquotesingle s kindness, the prospect of
a fine day to-morrow, and the happy times they would have when father
came home to be nursed.

No one wanted to go to bed, when, at ten o\textquotesingle clock, Mrs.
March put by the last finished job, and said, "Come, girls." Beth went
to the piano and played the father\textquotesingle s favorite hymn; all
began bravely, but broke down one by one, till Beth was left alone,
singing with all her heart, for to her music was always a sweet
consoler.

"Go to bed and don\textquotesingle t talk, for we must be up early, and
shall need all the sleep we can get. Good-night, my darlings," said Mrs.
March, as the hymn ended, for no one cared to try another.

They kissed her quietly, and went to bed as silently as if the dear
invalid lay in the next room. Beth and Amy soon fell asleep in spite of
the great trouble, but Meg lay awake, thinking the most serious thoughts
she had ever known in her short life. Jo lay motionless, and her sister
fancied that she was asleep, till a stifled sob made her exclaim, as she
touched a wet cheek,---

"Jo, dear, what is it? Are you crying about father?"

"No, not now."

"What then?"

"My---my hair!" burst out poor Jo, trying vainly to smother her emotion
in the pillow.

It did not sound at all comical to Meg, who kissed and caressed the
afflicted heroine in the tenderest manner.

"I\textquotesingle m not sorry," protested Jo, with a choke.
"I\textquotesingle d do it again to-morrow, if I could.
It\textquotesingle s only the vain, selfish part of me that goes and
cries in this silly way. Don\textquotesingle t tell any one,
it\textquotesingle s all over now. I thought you were asleep, so I just
made a little private moan for my one beauty. How came you to be awake?"

"I can\textquotesingle t sleep, I\textquotesingle m so anxious," said
Meg.

"Think about something pleasant, and you\textquotesingle ll soon drop
off."

"I tried it, but felt wider awake than ever."

"What did you think of?"

"Handsome faces,---eyes particularly," answered Meg, smiling to herself,
in the dark.

"What color do you like best?"

"Brown---that is, sometimes; blue are lovely."

Jo laughed, and Meg sharply ordered her not to talk, then amiably
promised to make her hair curl, and fell asleep to dream of living in
her castle in the air.

The clocks were striking midnight, and the rooms were very still, as a
figure glided quietly from bed to bed, smoothing a coverlid here,
settling a pillow there, and pausing to look long and tenderly at each
unconscious face, to kiss each with lips that mutely blessed, and to
pray the fervent prayers which only mothers utter. As she lifted the
curtain to look out into the dreary night, the moon broke suddenly from
behind the clouds, and shone upon her like a bright, benignant face,
which seemed to whisper in the silence, "Be comforted, dear soul! There
is always light behind the clouds."

\protect\phantomsection\label{6672479776654687619_37106-h-3.htm.xhtml_b087.png}{}
\pandocbounded{\includegraphics[keepaspectratio]{303483661336987339_b087.png}}

\begin{center}\rule{0.5\linewidth}{0.5pt}\end{center}

\subsection{XVI.
Letters.}\label{6672479776654687619_37106-h-3.htm.xhtml_pgepubid00018}

\protect\phantomsection\label{6672479776654687619_37106-h-3.htm.xhtml_b088.png}{}
\pandocbounded{\includegraphics[keepaspectratio]{303483661336987339_b088.png}}

\protect\phantomsection\label{6672479776654687619_37106-h-3.htm.xhtml_XVI}{}\hyperref[6672479776654687619_37106-h-0.htm.xhtml_contents1b]{XVI.}

LETTERS.

{In} the cold gray dawn the sisters lit their lamp, and read their
chapter with an earnestness never felt before; for now the shadow of a
real trouble had come, the little books were full of help and comfort;
and, as they dressed, they agreed to say good-by cheerfully and
hopefully, and send their mother on her anxious journey unsaddened by
tears or complaints from them. Everything seemed very strange when they
went down,---so dim and still outside, so full of light and bustle
within. Breakfast at that early hour seemed odd, and even
Hannah\textquotesingle s familiar face looked unnatural as she flew
about her kitchen with her night-cap on. The big trunk stood ready in
the hall, mother\textquotesingle s cloak and bonnet lay on the sofa, and
mother herself sat trying to eat, but looking so pale and worn with
sleeplessness and anxiety that the girls found it very hard to keep
their resolution. Meg\textquotesingle s eyes kept filling in spite of
herself; Jo was obliged to hide her face in the kitchen roller more than
once; and the little girls\textquotesingle{} wore a grave, troubled
expression, as if sorrow was a new experience to them.

Nobody talked much, but as the time drew very near, and they sat waiting
for the carriage, Mrs. March said to the girls, who were all busied
about her, one folding her shawl, another smoothing out the strings of
her bonnet, a third putting on her overshoes, and a fourth fastening up
her travelling bag,---

"Children, I leave you to Hannah\textquotesingle s care and Mr.
Laurence\textquotesingle s protection. Hannah is faithfulness itself,
and our good neighbor will guard you as if you were his own. I have no
fears for you, yet I am anxious that you should take this trouble
rightly. Don\textquotesingle t grieve and fret when I am gone, or think
that you can comfort yourselves by being idle and trying to forget. Go
on with your work as usual, for work is a blessed solace. Hope and keep
busy; and whatever happens, remember that you never can be fatherless."

"Yes, mother."

"Meg, dear, be prudent, watch over your sisters, consult Hannah, and, in
any perplexity, go to Mr. Laurence. Be patient, Jo,
don\textquotesingle t get despondent or do rash things; write to me
often, and be my brave girl, ready to help and cheer us all. Beth,
comfort yourself with your music, and be faithful to the little home
duties; and you, Amy, help all you can, be obedient, and keep happy safe
at home."

"We will, mother! we will!"

The rattle of an approaching carriage made them all start and listen.
That was the hard minute, but the girls stood it well: no one cried, no
one ran away or uttered a lamentation, though their hearts were very
heavy as they sent loving messages to father, remembering, as they
spoke, that it might be too late to deliver them. They kissed their
mother quietly, clung about her tenderly, and tried to wave their hands
cheerfully when she drove away.

Laurie and his grandfather came over to see her off, and Mr. Brooke
looked so strong and sensible and kind that the girls christened him
"Mr. Greatheart" on the spot.

"Good-by, my darlings! God bless and keep us all!" whispered Mrs. March,
as she kissed one dear little face after the other, and hurried into the
carriage.

As she rolled away, the sun came out, and, looking back, she saw it
shining on the group at the gate, like a good omen. They saw it also,
and smiled and waved their hands; and the last thing she beheld, as she
turned the corner, was the four bright faces, and behind them, like a
body-guard, old Mr. Laurence, faithful Hannah, and devoted Laurie.

\protect\phantomsection\label{6672479776654687619_37106-h-3.htm.xhtml_b089.png}{}
\pandocbounded{\includegraphics[keepaspectratio]{303483661336987339_b089.png}}

"How kind every one is to us!" she said, turning to find fresh proof of
it in the respectful sympathy of the young man\textquotesingle s face.

"I don\textquotesingle t see how they can help it," returned Mr. Brooke,
laughing so infectiously that Mrs. March could not help smiling; and so
the long journey began with the good omens of sunshine, smiles, and
cheerful words.

"I feel as if there had been an earthquake," said Jo, as their neighbors
went home to breakfast, leaving them to rest and refresh themselves.

"It seems as if half the house was gone," added Meg forlornly.

Beth opened her lips to say something, but could only point to the pile
of nicely-mended hose which lay on mother\textquotesingle s table,
showing that even in her last hurried moments she had thought and worked
for them. It was a little thing, but it went straight to their hearts;
and, in spite of their brave resolutions, they all broke down, and cried
bitterly.

Hannah wisely allowed them to relieve their feelings, and, when the
shower showed signs of clearing up, she came to the rescue, armed with a
coffee-pot.

"Now, my dear young ladies, remember what your ma said, and
don\textquotesingle t fret. Come and have a cup of coffee all round, and
then let\textquotesingle s fall to work, and be a credit to the family."

Coffee was a treat, and Hannah showed great tact in making it that
morning. No one could resist her persuasive nods, or the fragrant
invitation issuing from the nose of the coffee-pot. They drew up to the
table, exchanged their handkerchiefs for napkins, and in ten minutes
were all right again.

"\textquotesingle Hope and keep busy;\textquotesingle{}
that\textquotesingle s the motto for us, so let\textquotesingle s see
who will remember it best. I shall go to Aunt March, as usual. Oh,
won\textquotesingle t she lecture though!" said Jo, as she sipped with
returning spirit.

"I shall go to my Kings, though I\textquotesingle d much rather stay at
home and attend to things here," said Meg, wishing she
hadn\textquotesingle t made her eyes so red.

"No need of that; Beth and I can keep house perfectly well," put in Amy,
with an important air.

"Hannah will tell us what to do; and we\textquotesingle ll have
everything nice when you come home," added Beth, getting out her mop and
dish-tub without delay.

"I think anxiety is very interesting," observed Amy, eating sugar,
pensively.

The girls couldn\textquotesingle t help laughing, and felt better for
it, though Meg shook her head at the young lady who could find
consolation in a sugar-bowl.

The sight of the \ul{turn-overs made Jo} sober again; and when the two
went out to their daily tasks, they looked sorrowfully back at the
window where they were accustomed to see their mother\textquotesingle s
face. It was gone; but Beth had remembered the little household
ceremony, and there she was, nodding away at them like a rosy-faced
mandarin.

"That\textquotesingle s so like my Beth!" said Jo, waving her hat, with
a grateful face. "Good-by, Meggy; I hope the Kings won\textquotesingle t
train to-day. Don\textquotesingle t fret about father, dear," she added,
as they parted.

"And I hope Aunt March won\textquotesingle t croak. Your hair \emph{is}
becoming, and it looks very boyish and nice," returned Meg, trying not
to smile at the curly head, which looked comically small on her tall
sister\textquotesingle s shoulders.

"That\textquotesingle s my only comfort;" and, touching her hat \emph{à
la} Laurie, away went Jo, feeling like a shorn sheep on a wintry day.

News from their father comforted the girls very much; for, though
dangerously ill, the presence of the best and tenderest of nurses had
already done him good. Mr. Brooke sent a bulletin every day, and, as the
head of the family, Meg insisted on reading the despatches, which grew
more and more cheering as the week passed. At first, every one was eager
to write, and plump envelopes were carefully poked into the letter-box
by one or other of the sisters, who felt rather important with their
Washington correspondence. As one of these packets contained
characteristic notes from the party, we will rob an imaginary mail, and
read them:---

\begin{quote}
"My Dearest Mother,---

"It is impossible to tell you how happy your last letter made us, for
the news was so good we couldn\textquotesingle t help laughing and
crying over it. How very kind Mr. Brooke is, and how fortunate that Mr.
Laurence\textquotesingle s business detains him near you so long, since
he is so useful to you and father. The girls are all as good as gold. Jo
helps me with the sewing, and insists on doing all sorts of hard jobs. I
should be afraid she might overdo, if I didn\textquotesingle t know that
her \textquotesingle moral fit\textquotesingle{}
wouldn\textquotesingle t last long. Beth is as regular about her tasks
as a clock, and never forgets what you told her. She grieves about
father, and looks sober except when she is at her little piano. Amy
minds me nicely, and I take great care of her. She does her own hair,
and I am teaching her to make button-holes and mend her stockings. She
tries very hard, and I know you will be pleased with her improvement
when you come. Mr. Laurence watches over us like a motherly old hen, as
Jo says; and Laurie is very kind and neighborly. He and Jo keep us
merry, for we get pretty blue sometimes, and feel like orphans, with you
so far away. Hannah is a perfect saint; she does not scold at all, and
always calls me Miss \textquotesingle Margaret,\textquotesingle{} which
is quite proper, you know, and treats me with respect. We are all well
and busy; but we long, day and night, to have you back. Give my dearest
love to father, and believe me, ever your own

"Meg."
\end{quote}

This note, prettily written on scented paper, was a great contrast to
the next, which was scribbled on a big sheet of thin foreign paper,
ornamented with blots and all manner of flourishes and curly-tailed
letters:---

\begin{quote}
"My Precious Marmee,---

"Three cheers for dear father! Brooke was a trump to telegraph right
off, and let us know the minute he was better. I rushed up garret when
the letter came, and tried to thank God for being so good to us; but I
could only cry, and say, \textquotesingle I\textquotesingle m glad!
I\textquotesingle m glad!\textquotesingle{} Didn\textquotesingle t that
do as well as a regular prayer? for I felt a great many in my heart. We
have such funny times; and now I can enjoy them, for every one is so
desperately good, it\textquotesingle s like living in a nest of
turtle-doves. You\textquotesingle d laugh to see Meg head the table and
try to be motherish. She gets prettier every day, and
I\textquotesingle m in love with her sometimes. The children are regular
archangels, and I---well, I\textquotesingle m Jo, and never shall be
anything else. Oh, I must tell you that I came near having a quarrel
with Laurie. I freed my mind about a silly little thing, and he was
offended. I was right, but didn\textquotesingle t speak as I ought, and
he marched home, saying he wouldn\textquotesingle t come again till I
begged pardon. I declared I wouldn\textquotesingle t, and got mad. It
lasted all day; I felt bad, and wanted you very much. Laurie and I are
both so proud, it\textquotesingle s hard to beg pardon; but I thought
he\textquotesingle d come to it, for I \emph{was} in the right. He
didn\textquotesingle t come; and just at night I remembered what you
said when Amy fell into the river. I read my little book, felt better,
resolved not to let the sun set on \emph{my} anger, and ran over to tell
Laurie I was sorry. I met him at the gate, coming for the same thing. We
both laughed, begged each other\textquotesingle s pardon, and felt all
good and comfortable again.

"I made a \textquotesingle pome\textquotesingle{} yesterday, when I was
helping Hannah wash; and, as father likes my silly little things, I put
it in to amuse him. Give him the lovingest hug that ever was, and kiss
yourself a dozen times for your

"Topsy-Turvy Jo."
\end{quote}

\begin{center}\rule{0.5\linewidth}{0.5pt}\end{center}

"A SONG FROM THE SUDS.

"Queen of my tub, I merrily sing,

While the white foam rises high;

And sturdily wash and rinse and wring,

And fasten the clothes to dry;

Then out in the free fresh air they swing,

Under the sunny sky.

\hfill\break

"I wish we could wash from our hearts and souls

The stains of the week away,

And let water and air by their magic make

Ourselves as pure as they;

Then on the earth there would be indeed

A glorious washing-day!

\hfill\break

"Along the path of a useful life,

Will heart\textquotesingle s-ease ever bloom;

The busy mind has no time to think

Of sorrow or care or gloom;

And anxious thoughts may be swept away,

As we bravely wield a broom.

\hfill\break

"I am glad a task to me is given,

To labor at day by day;

For it brings me health and strength and hope,

And I cheerfully learn to say,---

\textquotesingle Head, you may think, Heart, you may feel,

But, Hand, you shall work alway!\textquotesingle"

\protect\phantomsection\label{6672479776654687619_37106-h-3.htm.xhtml_b090.png}{}
\pandocbounded{\includegraphics[keepaspectratio]{303483661336987339_b090.png}}

\begin{quote}
"Dear Mother,---

"There is only room for me to send my love, and some pressed pansies
from the root I have been keeping safe in the house for father to see. I
read every morning, try to be good all day, and sing myself to sleep
with father\textquotesingle s tune. I can\textquotesingle t sing
\textquotesingle Land of the Leal\textquotesingle{} now; it makes me
cry. Every one is very kind, and we are as happy as we can be without
you. Amy wants the rest of the page, so I must stop. I
didn\textquotesingle t forget to cover the holders, and I wind the clock
and air the rooms every day.

"Kiss dear father on the cheek he calls mine. Oh, do come soon to your
loving

"Little Beth."
\end{quote}

\begin{center}\rule{0.5\linewidth}{0.5pt}\end{center}

\begin{quote}
"Ma Chere Mamma,---

"We are all well I do my lessons always and never corroberate the
girls---Meg says I mean contradick so I put in both words and you can
take the properest. Meg is a great comfort to me and lets me have jelly
every night at tea its so good for me Jo says because it keeps me sweet
tempered. Laurie is not as respeckful as he ought to be now I am almost
in my teens, he calls me Chick and hurts my feelings by talking French
to me very fast when I say Merci or Bon jour as Hattie King does. The
sleeves of my blue dress were all worn out, and Meg put in new ones, but
the full front came wrong and they are more blue than the dress. I felt
bad but did not fret I bear my troubles well but I do wish Hannah would
put more starch in my aprons and have buckwheats every day.
Can\textquotesingle t she? Didn\textquotesingle t I make that
interrigation point nice? Meg says my punchtuation and spelling are
disgraceful and I am mortyfied but dear me I have so many things to do,
I can\textquotesingle t stop. Adieu, I send heaps of love to Papa.

"Your affectionate daughter,

"Amy Curtis March."
\end{quote}

\protect\phantomsection\label{6672479776654687619_37106-h-3.htm.xhtml_b091.png}{}
\pandocbounded{\includegraphics[keepaspectratio]{303483661336987339_b091.png}}

\begin{quote}
"Dear Mis March,---

"I jes drop a line to say we git on fust rate. The girls is clever and
fly round right smart. Miss Meg is going to make a proper good
housekeeper; she hes the liking for it, and gits the hang of things
surprisin quick. Jo doos beat all for goin ahead, but she
don\textquotesingle t stop to cal\textquotesingle k\textquotesingle late
fust, and you never know where she\textquotesingle s like to bring up.
She done out a tub of clothes on Monday, but she starched
\textquotesingle em afore they was wrenched, and blued a pink calico
dress till I thought I should a died a laughin. Beth is the best of
little creeters, and a sight of help to me, bein so forehanded and
dependable. She tries to learn everything, and really goes to market
beyond her years; likewise keeps accounts, with my help, quite
wonderful. We have got on very economical so fur; I
don\textquotesingle t let the girls hev coffee only once a week,
accordin to your wish, and keep em on plain wholesome vittles. Amy does
well about frettin, wearin her best clothes and eatin sweet stuff. Mr.
Laurie is as full of didoes as usual, and turns the house upside down
frequent; but he heartens up the girls, and so I let em hev full swing.
The old gentleman sends heaps of things, and is rather wearin, but means
wal, and it aint my place to say nothin. My bread is riz, so no more at
this time. I send my duty to Mr. March, and hope he\textquotesingle s
seen the last of his Pewmonia.

{"Yours Respectful,}

"Hannah Mullet."
\end{quote}

\begin{center}\rule{0.5\linewidth}{0.5pt}\end{center}

\begin{quote}
\ul{"Head Nurse of Ward} No. 2,---

"All serene on the Rappahannock, troops in fine condition, commissary
department well conducted, the Home Guard under Colonel Teddy always on
duty, Commander-in-chief General Laurence reviews the army daily,
Quartermaster Mullett keeps order in camp, and Major Lion does picket
duty at night. A salute of twenty-four guns was fired on receipt of good
news from Washington, and a dress parade took place at head-quarters.
Commander-in-chief sends best wishes, in which he is heartily joined by

"{Colonel Teddy.}"
\end{quote}

\begin{center}\rule{0.5\linewidth}{0.5pt}\end{center}

\begin{quote}
"{Dear Madam},---

"The little girls are all well; Beth and my boy report daily; Hannah is
a model servant, and guards pretty Meg like a dragon. Glad the fine
weather holds; pray make Brooke useful, and draw on me for funds if
expenses exceed your estimate. Don\textquotesingle t let your husband
want anything. Thank God he is mending.

{"Your sincere friend and servant,}

"{James Laurence.}"
\end{quote}

\protect\phantomsection\label{6672479776654687619_37106-h-3.htm.xhtml_b092.png}{}
\pandocbounded{\includegraphics[keepaspectratio]{303483661336987339_b092.png}}

\begin{center}\rule{0.5\linewidth}{0.5pt}\end{center}

\subsection{XVII. Little
Faithful.}\label{6672479776654687619_37106-h-3.htm.xhtml_pgepubid00019}

\protect\phantomsection\label{6672479776654687619_37106-h-3.htm.xhtml_XVII}{}\hyperref[6672479776654687619_37106-h-0.htm.xhtml_contents1b]{XVII.}

LITTLE FAITHFUL.

{For} a week the amount of virtue in the old house would have supplied
the neighborhood. It was really amazing, for every one seemed in a
heavenly frame of mind, and self-denial was all the fashion. Relieved of
their first anxiety about their father, the girls insensibly relaxed
their praiseworthy efforts a little, and began to fall back into the old
ways. They did not forget their motto, but hoping and keeping busy
seemed to grow easier; and after such tremendous exertions, they felt
that Endeavor deserved a holiday, and gave it a good many.

Jo caught a bad cold through neglect to cover the shorn head enough, and
was ordered to stay at home till she was better, for Aunt March
didn\textquotesingle t like to hear people read with colds in their
heads. Jo liked this, and after an energetic rummage from garret to
cellar, subsided on the sofa to nurse her cold with arsenicum and books.
Amy found that housework and art did not go well together, and returned
to her mud pies. Meg went daily to her pupils, and sewed, or thought she
did, at home, but much time was spent in writing long letters to her
mother, or reading the Washington despatches over and over. Beth kept
on, with only slight relapses into idleness or grieving. All the little
duties were faithfully done each day, and many of her
sisters\textquotesingle{} also, for they were forgetful, and the house
seemed like a clock whose pendulum was gone a-visiting. When her heart
got heavy with longings for mother or fears for father, she went away
into a certain closet, hid her face in the folds of a certain dear old
gown, and made her little moan and prayed her little prayer quietly by
herself. Nobody knew what cheered her up after a sober fit, but every
one felt how sweet and helpful Beth was, and fell into a way of going to
her for comfort or advice in their small affairs.

All were unconscious that this experience was a test of character; and,
when the first excitement was over, felt that they had done well, and
deserved praise. So they did; but their mistake was in ceasing to do
well, and they learned this lesson through much anxiety and regret.

"Meg, I wish you\textquotesingle d go and see the Hummels; you know
mother told us not to forget them," said Beth, ten days after Mrs.
March\textquotesingle s departure.

"I\textquotesingle m too tired to go this afternoon," replied Meg,
rocking comfortably as she sewed.

"Can\textquotesingle t you, Jo?" asked Beth.

"Too stormy for me with my cold."

"I thought it was almost well."

"It\textquotesingle s well enough for me to go out with Laurie, but not
well enough to go to the Hummels\textquotesingle," said Jo, laughing,
but looking a little ashamed of her inconsistency.

"Why don\textquotesingle t you go yourself?" asked Meg.

"I \emph{have} been every day, but the baby is sick, and I
don\textquotesingle t know what to do for it. Mrs. Hummel goes away to
work, and Lottchen takes care of it; but it gets sicker and sicker, and
I think you or Hannah ought to go."

Beth spoke earnestly, and Meg promised she would go to-morrow.

"Ask Hannah for some nice little mess, and take it round, Beth; the air
will do you good," said Jo, adding apologetically, "I\textquotesingle d
go, but I want to finish my writing."

"My head aches and I\textquotesingle m tired, so I thought may be some
of you would go," said Beth.

"Amy will be in presently, and she will run down for us," suggested Meg.

"Well, I\textquotesingle ll rest a little and wait for her."

So Beth lay down on the sofa, the others returned to their work, and the
Hummels were forgotten. An hour passed: Amy did not come; Meg went to
her room to try on a new dress; Jo was absorbed in her story, and Hannah
was sound asleep before the kitchen fire, when Beth quietly put on her
hood, filled her basket with odds and ends for the poor children, and
went out into the chilly air, with a heavy head, and a grieved look in
her patient eyes. It was late when she came back, and no one saw her
creep upstairs and shut herself into her mother\textquotesingle s room.
Half an hour after Jo went to "mother\textquotesingle s closet" for
something, and there found Beth sitting on the medicine chest, looking
very grave, with red eyes, and a camphor-bottle in her hand.

"Christopher Columbus! What\textquotesingle s the matter?" cried Jo, as
Beth put out her hand as if to warn her off, and asked quickly,---

"You\textquotesingle ve had the scarlet fever, haven\textquotesingle t
you?"

\ul{"Years ago, when Meg did.} Why?"

"Then I\textquotesingle ll tell you. Oh, Jo, the baby\textquotesingle s
dead!"

"What baby?"

"Mrs. Hummel\textquotesingle s; it died in my lap before she got home,"
cried Beth, with a sob.

"My poor dear, how dreadful for you! I ought to have gone," said Jo,
taking her sister in her arms as she sat down in her
mother\textquotesingle s big chair, with a remorseful face.

"It wasn\textquotesingle t dreadful, Jo, only so sad! I saw in a minute
that it was sicker, but Lottchen said her mother had gone for a doctor,
so I took baby and let Lotty rest. It seemed asleep, but all of a sudden
it gave a little cry, and trembled, and then lay very still. I tried to
warm its feet, and Lotty gave it some milk, but it
didn\textquotesingle t stir, and I knew it was dead."

\protect\phantomsection\label{6672479776654687619_37106-h-3.htm.xhtml_b093.png}{}
\pandocbounded{\includegraphics[keepaspectratio]{303483661336987339_b093.png}}

"Don\textquotesingle t cry, dear! What did you do?"

"I just sat and held it softly till Mrs. Hummel came with the doctor. He
said it was dead, and looked at Heinrich and Minna, who have got sore
throats. \textquotesingle Scarlet fever, ma\textquotesingle am. Ought to
have called me before,\textquotesingle{} he said crossly. Mrs. Hummel
told him she was poor, and had tried to cure baby herself, but now it
was too late, and she could only \ul{ask him to help the others,} and
trust to charity for his pay. He smiled then, and was kinder; but it was
very sad, and I cried with them till he turned round, all of a sudden,
and told me to go home and take belladonna right away, or
I\textquotesingle d have the fever."

"No, you won\textquotesingle t!" cried Jo, hugging her close, with a
frightened look. "O Beth, if you should be sick I never could forgive
myself! What \emph{shall} we do?"

"Don\textquotesingle t be frightened, I guess I shan\textquotesingle t
have it badly. I looked in mother\textquotesingle s book, and saw that
it begins with headache, sore throat, and queer feelings like mine, so I
did take some belladonna, and I feel better," said Beth, laying her cold
hands on her hot forehead, and trying to look well.

"If mother was only at home!" exclaimed Jo, seizing the book, and
feeling that Washington was an immense way off. She read a page, looked
at Beth, felt her head, peeped into her throat, and then said gravely;
"You\textquotesingle ve been over the baby every day for more than a
week, and among the others who are going to have it; so
I\textquotesingle m afraid \emph{you} are going to have it, Beth.
I\textquotesingle ll call Hannah, she knows all about sickness."

"Don\textquotesingle t let Amy come; she never had it, and I should hate
to give \ul{it to her.} Can\textquotesingle t you and Meg have it over
again?" asked Beth, anxiously.

"I guess not; don\textquotesingle t care if I do; serve me right,
selfish pig, to let you go, and stay writing rubbish myself!" muttered
Jo, as she went to consult Hannah.

The good soul was wide awake in a minute, and took the lead at once,
assuring Jo that there was no need to worry; every one had scarlet
fever, and, if rightly treated, nobody died,---all of which Jo believed,
and felt much relieved as they went up to call Meg.

"Now I\textquotesingle ll tell you what we\textquotesingle ll do," said
Hannah, when she had examined and questioned Beth; "we will have Dr.
Bangs, just to take a look at you, dear, and see that we start right;
then we\textquotesingle ll send Amy off to Aunt March\textquotesingle s,
for a spell, to keep her out of harm\textquotesingle s way, and one of
you girls can stay at home and amuse Beth for a day or two."

"I shall stay, of course; I\textquotesingle m oldest," began Meg,
looking anxious and self-reproachful.

"\emph{I} shall, because it\textquotesingle s my fault she is sick; I
told mother I\textquotesingle d do the errands, and I
haven\textquotesingle t," said Jo decidedly.

"Which will you have, Beth? there ain\textquotesingle t no need of but
one," said Hannah.

"Jo, please;" and Beth leaned her head against her sister, with a
contented look, which effectually settled that point.

"I\textquotesingle ll go and tell Amy," said Meg, feeling a little hurt,
yet rather relieved, on the whole, for she did not like nursing, and Jo
did.

Amy rebelled outright, and passionately declared that she had rather
have the fever than go to Aunt March. Meg reasoned, pleaded, and
commanded: all in vain. Amy protested that she would \emph{not} go; and
Meg left her in despair, to ask Hannah what should be done. Before she
came back, Laurie walked into the parlor to find Amy sobbing, with her
head in the sofa-cushions. She told her story, expecting to be consoled;
but Laurie only put his hands in his pockets and walked about the room,
whistling softly, as he knit his brows in deep thought. Presently he sat
down beside her, and said, in his most wheedlesome tone, "Now be a
sensible little woman, and do as they say. No, don\textquotesingle t
cry, but hear what a jolly plan I\textquotesingle ve got. You go to Aunt
March\textquotesingle s, and I\textquotesingle ll come and take you out
every day, driving or walking, and we\textquotesingle ll have capital
times. Won\textquotesingle t that be better than moping here?"

\protect\phantomsection\label{6672479776654687619_37106-h-3.htm.xhtml_b094.png}{}
\pandocbounded{\includegraphics[keepaspectratio]{303483661336987339_b094.png}}

"I don\textquotesingle t wish to be sent off as if I was in the way,"
began Amy, in an injured voice.

"Bless your heart, child, it\textquotesingle s to keep you well. You
don\textquotesingle t want to be sick, do you?"

"No, I\textquotesingle m sure I don\textquotesingle t; but I dare say I
shall be, for I\textquotesingle ve been with Beth all the time."

"That\textquotesingle s the very reason you ought to go away at once, so
that you may escape it. Change of air and care will keep you well, I
dare say; or, if it does not entirely, you will have the fever more
lightly. I advise you to be off as soon as you can, for scarlet fever is
no joke, miss."

"But it\textquotesingle s dull at Aunt March\textquotesingle s, and she
is so cross," said Amy, looking rather frightened.

"It won\textquotesingle t be dull with me popping in every day to tell
you how Beth is, and take you out gallivanting. The old lady likes me,
and I\textquotesingle ll be as sweet as possible to her, so she
won\textquotesingle t peck at us, whatever we do."

"Will you take me out in the trotting wagon with Puck?"

"On my honor as a gentleman."

"And come every single day?"

"See if I don\textquotesingle t."

"And bring me back the minute Beth is well?"

"The identical minute."

"And go to the theatre, truly?"

"A dozen theatres, if we may."

"Well---I guess---I will," said Amy slowly.

"Good girl! Call Meg, and tell her you\textquotesingle ll give in," said
Laurie, with an approving pat, which annoyed Amy more than the "giving
in."

Meg and Jo came running down to behold the miracle which had been
wrought; and Amy, feeling very precious and self-sacrificing, promised
to go, if the doctor said Beth was going to be ill.

"How is the little dear?" asked Laurie; for Beth was his especial pet,
and he felt more anxious about her than he liked to show.

"She is lying down on mother\textquotesingle s bed, and feels better.
The baby\textquotesingle s death troubled her, but I dare say she has
only got cold. Hannah \emph{says} she thinks so; but she \emph{looks}
worried, and that makes me fidgety," answered Meg.

"What a trying world it is!" said Jo, rumpling up her hair in a fretful
sort of way. "No sooner do we get out of one trouble than down comes
another. There doesn\textquotesingle t seem to be anything to hold on to
when mother\textquotesingle s gone; so I\textquotesingle m all at sea."

"Well, don\textquotesingle t make a porcupine of yourself, it
isn\textquotesingle t becoming. Settle your wig, Jo, and tell me if I
shall telegraph to your mother, or do anything?" asked Laurie, who never
had been reconciled to the loss of his friend\textquotesingle s one
beauty.

"That is what troubles me," said Meg. "I think we ought to tell her if
Beth is really ill, but Hannah says we mustn\textquotesingle t, for
mother can\textquotesingle t leave father, and it will only make them
anxious. Beth won\textquotesingle t be sick long, and Hannah knows just
what to do, and mother said we were to mind her, so I suppose we must,
but it doesn\textquotesingle t seem quite right to me."

"Hum, well, I can\textquotesingle t say; suppose you ask grandfather
after the doctor has been."

"We will. Jo, go and get Dr. Bangs at once," commanded Meg; "we
can\textquotesingle t decide anything till he has been."

"Stay where you are, Jo; I\textquotesingle m errand-boy to this
establishment," said Laurie, taking up his cap.

"I\textquotesingle m afraid you are busy," began Meg.

"No, I\textquotesingle ve done my lessons for the day."

"Do you study in vacation time?" asked Jo.

"I follow the good example my neighbors set me," was
Laurie\textquotesingle s answer, as he swung himself out of the room.

"I have great hopes of my boy," observed Jo, watching him fly over the
fence with an approving smile.

"He does very well---for a boy," was Meg\textquotesingle s somewhat
ungracious answer, for the subject did not interest her.

Dr. Bangs came, said Beth had symptoms of the fever, but thought she
would have it lightly, though he looked sober over the Hummel story. Amy
was ordered off at once, and provided with something to ward off danger,
she departed in great state, with Jo and Laurie as escort.

Aunt March received them with her usual hospitality.

"What do you want now?" she asked, looking sharply over her spectacles,
while the parrot, sitting on the back of her chair, called out,---

\protect\phantomsection\label{6672479776654687619_37106-h-3.htm.xhtml_b095.png}{}
\pandocbounded{\includegraphics[keepaspectratio]{303483661336987339_b095.png}}

"Go away. No boys allowed here."

Laurie retired to the window, and Jo told her story.

"No more than I expected, if you are allowed to go poking about among
poor folks. Amy can stay and make herself useful if she
isn\textquotesingle t sick, which I\textquotesingle ve no doubt she will
be,---looks like it now. Don\textquotesingle t cry, child, it worries me
to hear people sniff."

Amy \emph{was} on the point of crying, but Laurie slyly pulled the
parrot\textquotesingle s tail, which caused Polly to utter an astonished
croak, and call out,---

"Bless my boots!" in such a funny way, that she laughed instead.

"What do you hear from your mother?" asked the old lady gruffly.

"Father is much better," replied Jo, trying to keep sober.

"Oh, is he? Well, that won\textquotesingle t last long, I fancy; March
never had any stamina," was the cheerful reply.

"Ha, ha! never say die, take a pinch of snuff, good by, good by!"
squalled Polly, dancing on her perch, and clawing at the old
lady\textquotesingle s cap as Laurie tweaked him in the rear.

"Hold your tongue, you disrespectful old bird! and, Jo,
you\textquotesingle d better go at once; it isn\textquotesingle t proper
to be gadding about so late with a rattle-pated boy like---"

"Hold your tongue, you disrespectful old bird!" cried Polly, tumbling
off the chair with a bounce, and running to peck the "rattle-pated" boy,
who was shaking with laughter at the last speech.

"I don\textquotesingle t think I \emph{can} bear it, but
I\textquotesingle ll try," thought Amy, as she was left alone with Aunt
March.

"Get along, you fright!" screamed Polly; and at that rude speech Amy
could not restrain a sniff.

\begin{center}\rule{0.5\linewidth}{0.5pt}\end{center}

\subsection{XVIII. Dark
Days.}\label{6672479776654687619_37106-h-3.htm.xhtml_pgepubid00020}

\protect\phantomsection\label{6672479776654687619_37106-h-3.htm.xhtml_XVIII}{}\hyperref[6672479776654687619_37106-h-0.htm.xhtml_contents1b]{XVIII.}

DARK DAYS.

\protect\phantomsection\label{6672479776654687619_37106-h-3.htm.xhtml_b096.png}{}
\pandocbounded{\includegraphics[keepaspectratio]{303483661336987339_b096.png}}

{Beth} did have the fever, and was much sicker than any one but Hannah
and the doctor suspected. The girls knew nothing about illness, and Mr.
Laurence was not allowed to see her, so Hannah had everything all her
own way, and busy Dr. Bangs did his best, but left a good deal to the
excellent nurse. Meg stayed at home, lest she should infect the Kings,
and kept house, feeling very anxious and a little guilty when she wrote
letters in which no mention was made of Beth\textquotesingle s illness.
She could not think it right to deceive her mother, but she had been
bidden to mind Hannah, and Hannah wouldn\textquotesingle t hear of "Mrs.
March bein\textquotesingle{} told, and worried just for sech a trifle."
Jo devoted herself to Beth day and night; not a hard task, for Beth was
very patient, and bore her pain uncomplainingly as long as she could
control herself. But there came a time when during the fever fits she
began to talk in a hoarse, broken voice, to play on the coverlet, as if
on her beloved little piano, and try to sing with a throat so swollen
that there was no music left; a time when she did not know the familiar
faces round her, but addressed them by wrong names, and called
imploringly for her mother. Then Jo grew frightened, Meg begged to be
allowed to write the truth, and even Hannah said she "would think of it,
though there was no danger \emph{yet}." A letter from Washington added
to their trouble, for Mr. March had had a relapse, and could not think
of coming home for a long while.

How dark the days seemed now, how sad and lonely the house, and how
heavy were the hearts of the sisters as they worked and waited, while
the shadow of death hovered over the once happy home! Then it was that
Margaret, sitting alone with tears dropping often on her work, felt how
rich she had been in things more precious than any luxuries money could
buy,---in love, protection, peace, and health, the real blessings of
life. Then it was that Jo, living in the darkened room, with that
suffering little sister always before her eyes, and that pathetic voice
sounding in her ears, learned to see the beauty and the sweetness of
Beth\textquotesingle s nature, to feel how deep and tender a place she
filled in all hearts, and to acknowledge the worth of
Beth\textquotesingle s unselfish ambition, to live for others, and make
home happy by the exercise of those simple virtues which all may
possess, and which all should love and value more than talent, wealth,
or beauty. And Amy, in her exile, longed eagerly to be at home, that she
might work for Beth, feeling now that no service would be hard or
irksome, and remembering, with regretful grief, how many neglected tasks
those willing hands had done for her. Laurie haunted the house like a
restless ghost, and Mr. Laurence locked the grand piano, because he
could not bear to be reminded of the young neighbor who used to make the
twilight pleasant for him. Every one missed Beth. The milkman, baker,
grocer, and butcher inquired how she did; poor Mrs. Hummel came to beg
pardon for her thoughtlessness, and to get a shroud for Minna; the
neighbors sent all sorts of comforts and good wishes, and even those who
knew her best were surprised to find how many friends shy little Beth
had made.

Meanwhile she lay on her bed with old Joanna at her side, for even in
her wanderings she did not forget her forlorn \emph{protégé}. She longed
for her cats, but would not have them brought, lest they should get
sick; and, in her quiet hours, she was full of anxiety about Jo. She
sent loving messages to Amy, bade them tell her mother that she would
write soon; and often begged for pencil and paper to try to say a word,
that father might not think she had neglected him. But soon even these
intervals of consciousness ended, and she lay hour after hour, tossing
to and fro, with incoherent words on her lips, or sank into a heavy
sleep which brought her no refreshment. Dr. Bangs came twice a day,
Hannah sat up at night, Meg kept a telegram in her desk all ready to
send off at any minute, and Jo never stirred from Beth\textquotesingle s
side.

The first of December was a wintry day indeed to them, for a bitter wind
blew, snow fell fast, and the year seemed getting ready for its death.
When Dr. Bangs came that morning, he looked long at Beth, held the hot
hand in both his own a minute, and laid it gently down, saying, in a low
tone, to Hannah,---

"If Mrs. March \emph{can} leave her husband, she\textquotesingle d
better be sent for."

Hannah nodded without speaking, for her lips twitched nervously; Meg
dropped down into a chair as the strength seemed to go out of her limbs
at the sound of those words; and Jo, after standing with a pale face for
a minute, ran to the parlor, snatched up the telegram, and, throwing on
her things, rushed out into the storm. She was soon back, and, while
noiselessly taking off her cloak, Laurie came in with a letter, saying
that Mr. March was mending again. Jo read it thankfully, but the heavy
weight did not seem lifted off her heart, and her face was so full of
misery that Laurie asked quickly,---

"What is it? is Beth worse?"

"I\textquotesingle ve sent for mother," said Jo, tugging at her rubber
boots with a tragical expression.

"Good for you, Jo! Did you do it on your own responsibility?" asked
Laurie, as he seated her in the hall chair, and took off the rebellious
boots, seeing how her hands shook.

"No, the doctor told us to."

"O Jo, it\textquotesingle s not so bad as that?" cried Laurie, with a
startled face.

"Yes, it is; she doesn\textquotesingle t know us, she
doesn\textquotesingle t even talk about the flocks of green doves, as
she calls the vine-leaves on the wall; she doesn\textquotesingle t look
like my Beth, and there\textquotesingle s nobody to help us bear it;
mother and father both gone, and God seems so far away I
can\textquotesingle t find Him."

As the tears streamed fast down poor Jo\textquotesingle s cheeks, she
stretched out her hand in a helpless sort of way, as if groping in the
dark, and Laurie took it in his, whispering, as well as he could, with a
lump in his throat,---

"I\textquotesingle m here. Hold on to me, Jo, dear!"

\protect\phantomsection\label{6672479776654687619_37106-h-3.htm.xhtml_b097.png}{}
\pandocbounded{\includegraphics[keepaspectratio]{303483661336987339_b097.png}}

She could not speak, but she did "hold on," and the warm grasp of the
friendly human hand comforted her sore heart, and seemed to lead her
nearer to the Divine arm which alone could uphold her in her trouble.
Laurie longed to say something tender and comfortable, but no fitting
words came to him, so he stood silent, gently stroking her bent head as
her mother used to do. It was the best thing he could have done; far
more soothing than the most eloquent words, for Jo felt the unspoken
sympathy, and, in the silence, learned the sweet solace which affection
administers to sorrow. Soon she dried the tears which had relieved her,
and looked up with a grateful face.

"Thank you, Teddy, I\textquotesingle m better now; I
don\textquotesingle t feel so forlorn, and will try to bear it if it
comes."

"Keep hoping for the best; that will help you, Jo. Soon your mother will
be here, and then everything will be right."

"I\textquotesingle m so glad father is better; now she
won\textquotesingle t feel so bad about leaving him. Oh, me! it does
seem as if all the troubles came in a heap, and I got the heaviest part
on my shoulders," sighed Jo, spreading her wet handkerchief over her
knees to dry.

"Doesn\textquotesingle t Meg pull fair?" asked Laurie, looking
indignant.

"Oh, yes; she tries to, but she can\textquotesingle t love Bethy as I
do; and she won\textquotesingle t miss her as I shall. Beth is my
conscience, and I \emph{can\textquotesingle t} give her up. I
can\textquotesingle t! I can\textquotesingle t!"

Down went Jo\textquotesingle s face into the wet handkerchief, and she
cried despairingly; for she had kept up bravely till now, and never shed
a tear. Laurie drew his hand across his eyes, but could not speak till
he had subdued the choky feeling in his throat and steadied his lips. It
might be unmanly, but he couldn\textquotesingle t help it, and I am glad
of it. Presently, as Jo\textquotesingle s sobs quieted, he said
hopefully, "I don\textquotesingle t think she will die;
she\textquotesingle s so good, and we all love her so much, I
don\textquotesingle t believe God will take her away yet."

"The good and dear people always do die," groaned Jo, but she stopped
crying, for her friend\textquotesingle s words cheered her up, in spite
of her own doubts and fears.

"Poor girl, you\textquotesingle re worn out. It isn\textquotesingle t
like you to be forlorn. Stop a bit; I\textquotesingle ll hearten you up
in a jiffy."

Laurie went off two stairs at a time, and Jo laid her wearied head down
on Beth\textquotesingle s little brown hood, which no one had thought of
moving from the table where she left it. It must have possessed some
magic, for the submissive spirit of its gentle owner seemed to enter
into Jo; and, when Laurie came running down with a glass of wine, she
took it with a smile, and said bravely, "I drink---Health to my Beth!
You are a good doctor, Teddy, and \emph{such} a comfortable friend; how
can I ever pay you?" she added, as the wine refreshed her body, as the
kind words had done her troubled mind.

"I\textquotesingle ll send in my bill, by and by; and to-night
I\textquotesingle ll give you something that will warm the cockles of
your heart better than quarts of wine," said Laurie, beaming at her with
a face of suppressed satisfaction at something.

"What is it?" cried Jo, forgetting her woes for a minute, in her wonder.

"I telegraphed to your mother yesterday, and Brooke answered
she\textquotesingle d come at once, and she\textquotesingle ll be here
to-night, and everything will be all right. Aren\textquotesingle t you
glad I did it?"

Laurie spoke very fast, and turned red and excited all in a minute, for
he had kept his plot a secret, for fear of disappointing the girls or
harming Beth. Jo grew quite white, flew out of her chair, and the moment
he stopped speaking she electrified him by throwing her arms round his
neck, and crying out, with a joyful cry, "O Laurie! O mother! I
\emph{am} so glad!" She did not weep again, but laughed hysterically,
and trembled and clung to her friend as if she was a little bewildered
by the sudden news. Laurie, though decidedly amazed, behaved with great
presence of mind; he patted her back soothingly, and, finding that she
was recovering, followed it up by a bashful kiss or two, which brought
Jo round at once. Holding on to the banisters, she put him gently away,
saying breathlessly, "Oh, don\textquotesingle t! I
didn\textquotesingle t mean to; it was dreadful of me; but you were such
a dear to go and do it in spite of Hannah that I
couldn\textquotesingle t help flying at you. Tell me all about it, and
don\textquotesingle t give me wine again; it makes me act so."

"I don\textquotesingle t mind," laughed Laurie, as he settled his tie.
"Why, you see I got fidgety, and so did grandpa. We thought Hannah was
overdoing the authority business, and your mother ought to know.
She\textquotesingle d never forgive us if Beth---well, if anything
happened, you know. So I got grandpa to say it was high time we did
something, and off I pelted to the office yesterday, for the doctor
looked sober, and Hannah most took my head off when I proposed a
telegram. I never \emph{can} bear to be \textquotesingle lorded
over;\textquotesingle{} so that settled my mind, and I did it. Your
mother will come, I know, and the late train is in at two, {a.m.} I
shall go for her; and you\textquotesingle ve only got to bottle up your
rapture, and keep Beth quiet, till that blessed lady gets here."

"Laurie, you\textquotesingle re an angel! How shall I ever thank you?"

"Fly at me again; I rather like it," said Laurie, looking
mischievous,---a thing he had not done for a fortnight.

"No, thank you. I\textquotesingle ll do it by proxy, when your grandpa
comes. Don\textquotesingle t tease, but go home and rest, for
you\textquotesingle ll be up half the night. Bless you, Teddy, bless
you!"

Jo had backed into a corner; and, as she finished her speech, she
vanished precipitately into the kitchen, where she sat down upon a
dresser, and told the assembled cats that she was "happy, oh, \emph{so}
happy!" while Laurie departed, feeling that he had made rather a neat
thing of it.

"That\textquotesingle s the interferingest chap I ever see; but I
forgive him, and do hope Mrs. March is coming on right away," said
Hannah, with an air of relief, when Jo told the good news.

Meg had a quiet rapture, and then brooded over the letter, while Jo set
the sick-room in order, and Hannah "knocked up a couple of pies in case
of company unexpected." A breath of fresh air seemed to blow through the
house, and something better than sunshine brightened the quiet rooms.
Everything appeared to feel the hopeful change; Beth\textquotesingle s
bird began to chirp again, and a half-blown rose was discovered on
Amy\textquotesingle s bush in the window; the fires seemed to burn with
unusual cheeriness; and every time the girls met, their pale faces broke
into smiles as they hugged one another, whispering encouragingly,
"Mother\textquotesingle s coming, dear! mother\textquotesingle s
coming!" Every one rejoiced but Beth; she lay in that heavy stupor,
alike unconscious of hope and joy, doubt and danger. It was a piteous
sight,---the once rosy face so changed and vacant, the once busy hands
so weak and wasted, the once smiling lips quite dumb, and the once
pretty, well-kept hair scattered rough and tangled on the pillow. All
day she lay so, only rousing now and then to mutter, "Water!" with lips
so parched they could hardly shape the word; all day Jo and Meg hovered
over her, watching, waiting, hoping, and trusting in God and mother; and
all day the snow fell, the bitter wind raged, and the hours dragged
slowly by. But night came at last; and every time the clock struck, the
sisters, still sitting on either side the bed, looked at each other with
brightening eyes, for each hour brought help nearer. The doctor had been
in to say that some change, for better or worse, would probably take
place about midnight, at which time he would return.

Hannah, quite worn out, lay down on the sofa at the
bed\textquotesingle s foot, and fell fast asleep; Mr. Laurence marched
to and fro in the parlor, feeling that he would rather face a rebel
battery than Mrs. March\textquotesingle s anxious countenance as she
entered; Laurie lay on the rug, pretending to rest, but staring into the
fire with the thoughtful look which made his black eyes beautifully soft
and clear.

The girls never forgot that night, for no sleep came to them as they
kept their watch, with that dreadful sense of powerlessness which comes
to us in hours like those.

"If God spares Beth I never will complain again," whispered Meg
earnestly.

"If God spares Beth I\textquotesingle ll try to love and serve Him all
my life," answered Jo, with equal fervor.

"I wish I had no heart, it aches so," sighed Meg, after a pause.

"If life is often as hard as this, I don\textquotesingle t see how we
ever shall get through it," added her sister despondently.

Here the clock struck twelve, and both forgot themselves in watching
Beth, for they fancied a change passed over her wan face. The house was
still as death, and nothing but the wailing of the wind broke the deep
hush. Weary Hannah slept on, and no one but the sisters saw the pale
shadow which seemed to fall upon the little bed. An hour went by, and
nothing happened except Laurie\textquotesingle s quiet departure for the
station. Another hour,---still no one came; and anxious fears of delay
in the storm, or accidents by the way, or, worst of all, a great grief
at Washington, haunted the poor girls.

It was past two, when Jo, who stood at the window thinking how dreary
the world looked in its winding-sheet of snow, heard a movement by the
bed, and, turning quickly, saw Meg kneeling before their
mother\textquotesingle s easy-chair, with her face hidden. A dreadful
fear passed coldly over Jo, as she thought, "Beth is dead, and Meg is
afraid to tell me."

She was back at her post in an instant, and to her excited eyes a great
change seemed to have taken place. The fever flush and the look of pain
were gone, and the beloved little face looked so pale and peaceful in
its utter repose, that Jo felt no desire to weep or to lament. Leaning
low over this dearest of her sisters, she kissed the damp forehead with
her heart on her lips, and softly whispered, "Good-by, my Beth;
good-by!"

As if waked by the stir, Hannah started out of her sleep, hurried to the
bed, looked at Beth, felt her hands, listened at her lips, and then,
throwing her apron over her head, sat down to rock to and fro,
exclaiming, under her breath, "The fever\textquotesingle s turned;
she\textquotesingle s sleepin\textquotesingle{} nat\textquotesingle ral;
her skin\textquotesingle s damp, and she breathes easy. Praise be given!
Oh, my goodness me!"

Before the girls could believe the happy truth, the doctor came to
confirm it. He was a homely man, but they thought his face quite
heavenly when he smiled, and said, with a fatherly look at them, "Yes,
my dears, I think the little girl will pull through this time. Keep the
house quiet; let her sleep, and when she wakes, give her---"

What they were to give, neither heard; for both crept into the dark
hall, and, sitting on the stairs, held each other close, rejoicing with
hearts too full for words. When they went back to be kissed and cuddled
by faithful Hannah, they found Beth lying, as she used to do, with her
cheek pillowed on her hand, the dreadful pallor gone, and breathing
quietly, as if just fallen asleep.

"If mother would only come now!" said Jo, as the winter night began to
wane.

"See," said Meg, coming up with a white, half-opened rose, "I thought
this would hardly be ready to lay in Beth\textquotesingle s hand
to-morrow if she---went away from us. But it has blossomed in the night,
and now I mean to put it in my vase here, so that when the darling
wakes, the first thing she sees will be the little rose, and
mother\textquotesingle s face."

Never had the sun risen so beautifully, and never had the world seemed
so lovely, as it did to the heavy eyes of Meg and Jo, as they looked out
in the early morning, when their long, sad vigil was done.

"It looks like a fairy world," said Meg, smiling to herself, as she
stood behind the curtain, watching the dazzling sight.

"Hark!" cried Jo, starting to her feet.

Yes, there was a sound of bells at the door below, a cry from Hannah,
and then Laurie\textquotesingle s voice saying, in a joyful whisper,
"Girls, she\textquotesingle s come! she\textquotesingle s come!"

\begin{center}\rule{0.5\linewidth}{0.5pt}\end{center}

\subsection{XIX. Amy\textquotesingle s
Will.}\label{6672479776654687619_37106-h-3.htm.xhtml_pgepubid00021}

\protect\phantomsection\label{6672479776654687619_37106-h-3.htm.xhtml_b098.png}{}
\pandocbounded{\includegraphics[keepaspectratio]{303483661336987339_b098.png}}

\protect\phantomsection\label{6672479776654687619_37106-h-3.htm.xhtml_XIX}{}\hyperref[6672479776654687619_37106-h-0.htm.xhtml_contents1b]{XIX.}

AMY\textquotesingle S WILL.

{While} these things were happening at home, Amy was having hard times
at Aunt March\textquotesingle s. She felt her exile deeply, and, for the
first time in her life, realized how much she was beloved and petted at
home. Aunt March never petted any one; she did not approve of it; but
she meant to be kind, for the well-behaved little girl pleased her very
much, and Aunt March had a soft place in her old heart for her
nephew\textquotesingle s children, though she didn\textquotesingle t
think proper to confess it. She really did her best to make Amy happy,
but, dear me, what mistakes she made! Some old people keep young at
heart in spite of wrinkles and gray hairs, can sympathize with
children\textquotesingle s little cares and joys, make them feel at
home, and can hide wise lessons under pleasant plays, giving and
receiving friendship in the sweetest way. But Aunt March had not this
gift, and she worried Amy very much with her rules and orders, her prim
ways, and long, prosy talks. Finding the child more docile and amiable
than her sister, the old lady felt it her duty to try and counteract, as
far as possible, the bad effects of home freedom and indulgence. So she
took Amy in hand, and taught her as she herself had been taught sixty
years ago,---a process which carried dismay to Amy\textquotesingle s
soul, and made her feel like a fly in the web of a very strict spider.

\protect\phantomsection\label{6672479776654687619_37106-h-3.htm.xhtml_b099.png}{}
\pandocbounded{\includegraphics[keepaspectratio]{303483661336987339_b099.png}}

She had to wash the cups every morning, and polish up the old-fashioned
spoons, the fat silver teapot, and the glasses, till they shone. Then
she must dust the room, and what a trying job that was! Not a speck
escaped Aunt March\textquotesingle s eye, and all the furniture had claw
legs, and much carving, which was never dusted to suit. Then Polly must
be fed, the lap-dog combed, and a dozen trips upstairs and down, to get
things, or deliver orders, for the old lady was very lame, and seldom
left her big chair. After these tiresome labors, she must do her
lessons, which was a daily trial of every virtue she possessed. Then she
was allowed one hour for exercise or play, and didn\textquotesingle t
she enjoy it? Laurie came every day, and wheedled Aunt March, till Amy
was allowed to go out with him, when they walked and rode, and had
capital times. After dinner, she had to read aloud, and sit still while
the old lady slept, which she usually did for an hour, as she dropped
off over the first page. Then patchwork or towels appeared, and Amy
sewed with outward meekness and inward rebellion till dusk, when she was
allowed to amuse herself as she liked till tea-time. The evenings were
the worst of all, for Aunt March fell to telling long stories about her
youth, which were so unutterably dull that Amy was always ready to go to
bed, intending to cry over her hard fate, but usually going to sleep
before she had squeezed out more than a tear or two.

If it had not been for Laurie, and old Esther, the maid, she felt that
she never could have got through that dreadful time. The parrot alone
was enough to drive her distracted, for he soon felt that she did not
admire him, and revenged himself by being as mischievous as possible. He
pulled her hair whenever she came near him, upset his bread and milk to
plague her when she had newly cleaned his cage, made Mop bark by pecking
at him while Madam dozed; called her names before company, and behaved
in all respects like a reprehensible old bird. Then she could not endure
the dog,---a fat, cross beast, who snarled and yelped at her when she
made his toilet, and who lay on his back, with all his legs in the air
and a most idiotic expression of countenance when he wanted something to
eat, which was about a dozen times a day. The cook was bad-tempered, the
old coachman deaf, and Esther the only one who ever took any notice of
the young lady.

\protect\phantomsection\label{6672479776654687619_37106-h-3.htm.xhtml_b100.png}{}
\pandocbounded{\includegraphics[keepaspectratio]{303483661336987339_b100.png}}

Esther was a Frenchwoman, who had lived with "Madame," as she called her
mistress, for many years, and who rather tyrannized over the old lady,
who could not get along without her. Her real name was Estelle, but Aunt
March ordered her to change it, and she obeyed, on condition that she
was never asked to change her religion. She took a fancy to
Mademoiselle, and amused her very much, with odd stories of her life in
France, when Amy sat with her while she got up Madame\textquotesingle s
laces. She also allowed her to roam about the great house, and examine
the curious and pretty things stored away in the big wardrobes and the
ancient chests; for Aunt March hoarded like a magpie.
Amy\textquotesingle s chief delight was an Indian cabinet, full of queer
drawers, little pigeon-holes, and secret places, in which were kept all
sorts of ornaments, some precious, some merely curious, all more or less
antique. To examine and arrange these things gave Amy great
satisfaction, especially the jewel-cases, in which, on velvet cushions,
reposed the ornaments which had adorned a belle forty years ago. There
was the garnet set which Aunt March wore when she came out, the pearls
her father gave her on her wedding-day, her lover\textquotesingle s
diamonds, the jet mourning rings and pins, the queer lockets, with
portraits of dead friends, and weeping willows made of hair inside; the
baby bracelets her one little daughter had worn; Uncle
March\textquotesingle s big watch, with the red seal so many childish
hands had played with, and in a box, all by itself, lay Aunt
March\textquotesingle s wedding-ring, too small now for her fat finger,
but put carefully away, like the most precious jewel of them all.

\protect\phantomsection\label{6672479776654687619_37106-h-3.htm.xhtml_b101.png}{}
\pandocbounded{\includegraphics[keepaspectratio]{303483661336987339_b101.png}}

"Which would Mademoiselle choose if she had her will?" asked Esther, who
always sat near to watch over and lock up the valuables.

"I like the diamonds best, but there is no necklace among them, and
I\textquotesingle m fond of necklaces, they are so becoming. I should
choose this if I might," replied Amy, looking with great admiration at a
string of gold and ebony beads, from which hung a heavy cross of the
same.

"I, too, covet that, but not as a necklace; ah, no! to me it is a
rosary, and as such I should use it like a good Catholic," said Esther,
eying the handsome thing wistfully.

"Is it meant to use as you use the string of good-smelling wooden beads
hanging over your glass?" asked Amy.

"Truly, yes, to pray with. It would be pleasing to the saints if one
used so fine a rosary as this, instead of wearing it as a vain bijou."

"You seem to take a great deal of comfort in your prayers, Esther, and
always come down looking quiet and satisfied. I wish I could."

"If Mademoiselle was a Catholic, she would find true comfort; but, as
that is not to be, it would be well if you went apart each day, to
meditate and pray, as did the good mistress whom I served before Madame.
She had a little chapel, and in it found solacement for much trouble."

"Would it be right for me to do so too?" asked Amy, who, in her
loneliness, felt the need of help of some sort, and found that she was
apt to forget her little book, now that Beth was not there to remind her
of it.

"It would be excellent and charming; and I shall gladly arrange the
little dressing-room for you if you like it. Say nothing to Madame, but
when she sleeps go you and sit alone a while to think good thoughts, and
pray the dear God to preserve your sister."

Esther was truly pious, and quite sincere in her advice; for she had an
affectionate heart, and felt much for the sisters in their anxiety. Amy
liked the idea, and gave her leave to arrange the light closet next her
room, hoping it would do her good.

"I wish I knew where all these pretty things would go when Aunt March
dies," she said, as she slowly replaced the shining rosary, and shut the
jewel-cases one by one.

"To you and your sisters. I know it; Madame confides in me; I witnessed
her will, and it is to be so," whispered Esther, smiling.

"How nice! but I wish she\textquotesingle d let us have them now.
Pro-cras-ti-nation is not agreeable," observed Amy, taking a last look
at the diamonds.

"It is too soon yet for the young ladies to wear these things. The first
one who is affianced will have the pearls---Madame has said it; and I
have a fancy that the little turquoise ring will be given to you when
you go, for Madame approves your good behavior and charming manners."

"Do you think so? Oh, I\textquotesingle ll be a lamb, if I can only have
that lovely ring! It\textquotesingle s ever so much prettier than Kitty
Bryant\textquotesingle s. I do like Aunt March, after all;" and Amy
tried on the blue ring with a delighted face, and a firm resolve to earn
it.

From that day she was a model of obedience, and the old lady
complacently admired the success of her training. Esther fitted up the
closet with a little table, placed a footstool before it, and over it a
picture taken from one of the shut-up rooms. She thought it was of no
great value, but, being appropriate, she borrowed it, well knowing that
Madame would never know it, nor care if she did. It was, however, a very
valuable copy of one of the famous pictures of the world, and
Amy\textquotesingle s beauty-loving eyes were never tired of looking up
at the sweet face of the divine mother, while tender thoughts of her own
were busy at her heart. On the table she laid her little Testament and
hymn-book, kept a vase always full of the best flowers Laurie brought
her, and came every day to "sit alone, thinking good thoughts, and
praying the dear God to preserve her sister." Esther had given her a
rosary of black beads, with a silver cross, but Amy hung it up and did
not use it, feeling doubtful as to its fitness for Protestant prayers.

The little girl was very sincere in all this, for, being left alone
outside the safe home-nest, she felt the need of some kind hand to hold
by so sorely, that she instinctively turned to the strong and tender
Friend, whose fatherly love most closely surrounds his little children.
She missed her mother\textquotesingle s help to understand and rule
herself, but having been taught where to look, she did her best to find
the way, and walk in it confidingly. But Amy was a young pilgrim, and
just now her burden seemed very heavy. She tried to forget herself, to
keep cheerful, and be satisfied with doing right, though no one saw or
praised her for it. In her first effort at being very, very good, she
decided to make her will, as Aunt March had done; so that if she
\emph{did} fall ill and die, her possessions might be justly and
generously divided. It cost her a pang even to think of giving up the
little treasures which in her eyes were as precious as the old
lady\textquotesingle s jewels.

During one of her play-hours she wrote out the important document as
well as she could, with some help from Esther as to certain legal terms,
and, when the good-natured Frenchwoman had signed her name, Amy felt
relieved, and laid it by to show Laurie, whom she wanted as a second
witness. As it was a rainy day, she went upstairs to amuse herself in
one of the large chambers, and took Polly with her for company. In this
room there was a wardrobe full of old-fashioned costumes, with which
Esther allowed her to play, and it was her favorite amusement to array
herself in the faded brocades, and parade up and down before the long
mirror, making stately courtesies, and sweeping her train about, with a
rustle which delighted her ears. So busy was she on this day that she
did not hear Laurie\textquotesingle s ring, nor see his face peeping in
at her, as she gravely promenaded to and fro, flirting her fan and
tossing her head, on which she wore a great pink turban, contrasting
oddly with her blue brocade dress and yellow quilted petticoat. She was
obliged to walk carefully, for she had on high-heeled shoes, and, as
Laurie told Jo afterward, it was a comical sight to see her mince along
in her gay suit, with Polly sidling and bridling just behind her,
imitating her as well as he could, and occasionally stopping to laugh or
exclaim, "Ain\textquotesingle t we fine? Get along, you fright! Hold
your tongue! Kiss me, dear! Ha! ha!"

\protect\phantomsection\label{6672479776654687619_37106-h-3.htm.xhtml_b102.png}{}
\pandocbounded{\includegraphics[keepaspectratio]{303483661336987339_b102.png}}

Having with difficulty restrained an explosion of merriment, lest it
should offend her majesty, Laurie tapped, and was graciously received.

"Sit down and rest while I put these things away; then I want to consult
you about a very serious matter," said Amy, when she had shown her
splendor, and driven Polly into a corner. "That bird is the trial of my
life," she continued, removing the pink mountain from her head, while
Laurie seated himself astride of a chair. "Yesterday, when aunt was
asleep, and I was trying to be as still as a mouse, Polly began to
squall and flap about in his cage; so I went to let him out, and found a
big spider there. I poked it out, and it ran under the bookcase; Polly
marched straight after it, stooped down and peeped under the bookcase,
saying, in his funny way, with a cock of his eye, \textquotesingle Come
out and take a walk, my dear.\textquotesingle{} I
\emph{couldn\textquotesingle t} help laughing, which made Poll swear,
and aunt woke up and scolded us both."

"Did the spider accept the old fellow\textquotesingle s invitation?"
asked Laurie, yawning.

"Yes; out it came, and away ran Polly, frightened to death, and
scrambled up on aunt\textquotesingle s chair, calling out,
\textquotesingle Catch her! catch her! catch her!\textquotesingle{} as I
chased the spider.

"That\textquotesingle s a lie! Oh lor!" cried the parrot, pecking at
Laurie\textquotesingle s toes.

"I\textquotesingle d wring your neck if you were mine, you old torment,"
cried Laurie, shaking his fist at the bird, who put his head on one
side, and gravely croaked, "Allyluyer! bless your buttons, dear!"

"Now I\textquotesingle m ready," said Amy, shutting the wardrobe, and
taking a paper out of her pocket. "I want you to read that, please, and
tell me if it is legal and right. I felt that I ought to do it, for life
is uncertain and I don\textquotesingle t want any ill-feeling over my
tomb."

\protect\phantomsection\label{6672479776654687619_37106-h-3.htm.xhtml_b103.png}{}
\pandocbounded{\includegraphics[keepaspectratio]{303483661336987339_b103.png}}

Laurie bit his lips, and turning a little from the pensive speaker, read
the following document, with praiseworthy gravity, considering the
spelling:---

\begin{quote}
"MY LAST WILL AND TESTIMENT.

"I, Amy Curtis March, being in my sane mind, do give and bequeethe all
my earthly property---viz. to wit:---namely

"To my father, my best pictures, sketches, maps, and works of art,
including frames. Also my \$100, to do what he likes with.

"To my mother, all my clothes, except the blue apron with
pockets,---also my likeness, and my medal, with much love.

"To my dear sister Margaret, I give my turkquoise ring (if I get it),
also my green box with the doves on it, also my piece of real lace for
her neck, and my sketch of her as a memorial of her
\textquotesingle little girl.\textquotesingle{}

"To Jo I leave my breast-pin, the one mended with sealing wax, also my
bronze inkstand---she lost the cover---and my most precious plaster
rabbit, because I am sorry I burnt up her story.

"To Beth (if she lives after me) I give my dolls and the little bureau,
my fan, my linen collars and my new slippers if she can wear them being
thin when she gets well. And I herewith also leave her my regret that I
ever made fun of old Joanna.

"To my friend and neighbor Theodore Laurence I bequeethe my paper
marshay portfolio, my clay model of a horse though he did say it
hadn\textquotesingle t any neck. Also in return for his great kindness
in the hour of affliction any one of my artistic works he likes, Noter
Dame is the best.

"To our venerable benefactor Mr. Laurence I leave my purple box with a
looking glass in the cover which will be nice for his pens and remind
him of the departed girl who thanks him for his favors to her family,
specially Beth.

"I wish my favorite playmate Kitty Bryant to have the blue silk apron
and my gold-bead ring with a kiss.

"To Hannah I give the bandbox she wanted and all the patch work I leave
hoping she \textquotesingle will remember me, when it you
see.\textquotesingle{}

"And now having disposed of my most valuable property I hope all will be
satisfied and not blame the dead. I forgive every one, and trust we may
all meet when the trump shall sound. Amen.

"To this will and testiment I set my hand and seal on this 20th day of
Nov. Anni Domino 1861.

"{Amy Curtis March.}
\end{quote}

\begin{longtable}[]{@{}
  >{\raggedright\arraybackslash}p{(\linewidth - 4\tabcolsep) * \real{0.3333}}
  >{\raggedright\arraybackslash}p{(\linewidth - 4\tabcolsep) * \real{0.3333}}
  >{\raggedright\arraybackslash}p{(\linewidth - 4\tabcolsep) * \real{0.3333}}@{}}
\toprule\noalign{}
\endhead
\bottomrule\noalign{}
\endlastfoot
"Witnesses: & \{ & \begin{minipage}[t]{\linewidth}\raggedright
Estelle Valnor,\\
Theodore Laurence."\strut
\end{minipage} \\
\end{longtable}

The last name was written in pencil, and Amy explained that he was to
rewrite it in ink, and seal it up for her properly.

"What put it into your head? Did any one tell you about
Beth\textquotesingle s giving away her things?" asked Laurie soberly, as
Amy laid a bit of red tape, with sealing-wax, a taper, and a standish
before him.

She explained; and then asked anxiously, "What about Beth?"

"I\textquotesingle m sorry I spoke; but as I did, I\textquotesingle ll
tell you. She felt so ill one day that she told Jo she wanted to give
her piano to Meg, her cats to you, and the poor old doll to Jo, who
would love it for her sake. She was sorry she had so little to give, and
left locks of hair to the rest of us, and her best love to grandpa.
\emph{She} never thought of a will."

Laurie was signing and sealing as he spoke, and did not look up till a
great tear dropped on the paper. Amy\textquotesingle s face was full of
trouble; but she only said, "Don\textquotesingle t people put sort of
\ul{postscripts to their wills,} sometimes?"

"Yes; \textquotesingle codicils,\textquotesingle{} they call them."

"Put one in mine then---that I wish \emph{all} my curls cut off, and
given round to my friends. I forgot it; but I want it done, though it
will spoil my looks."

Laurie added it, smiling at Amy\textquotesingle s last and greatest
sacrifice. Then he amused her for an hour, and was much interested in
all her trials. But when he came to go, Amy held him back to whisper,
with trembling lips, "Is there really any danger about Beth?"

"I\textquotesingle m afraid there is; but we must hope for the best, so
don\textquotesingle t cry, dear;" and Laurie put his arm about her with
a brotherly gesture which was very comforting.

When he had gone, she went to her little chapel, and, sitting in the
twilight, prayed for Beth, with streaming tears and an aching heart,
feeling that a million turquoise rings would not console her for the
loss of her gentle little sister.

\protect\phantomsection\label{6672479776654687619_37106-h-3.htm.xhtml_b104.png}{}
\pandocbounded{\includegraphics[keepaspectratio]{303483661336987339_b104.png}}

\begin{center}\rule{0.5\linewidth}{0.5pt}\end{center}

\subsection{XX.
Confidential.}\label{6672479776654687619_37106-h-3.htm.xhtml_pgepubid00022}

\protect\phantomsection\label{6672479776654687619_37106-h-3.htm.xhtml_b105.png}{}
\pandocbounded{\includegraphics[keepaspectratio]{303483661336987339_b105.png}}

\protect\phantomsection\label{6672479776654687619_37106-h-3.htm.xhtml_XX}{}\hyperref[6672479776654687619_37106-h-0.htm.xhtml_contents1b]{XX.}

CONFIDENTIAL.

I {don\textquotesingle t} think I have any words in which to tell the
meeting of the mother and daughters; such hours are beautiful to live,
but very hard to describe, so I will leave it to the imagination of my
readers, merely saying that the house was full of genuine happiness, and
that Meg\textquotesingle s tender hope was realized; for when Beth woke
from that long, healing sleep, the first objects on which her eyes fell
\emph{were} the little rose and mother\textquotesingle s face. Too weak
to wonder at anything, she only smiled, and nestled close into the
loving arms about her, feeling that the hungry longing was satisfied at
last. Then she slept again, and the girls waited upon their mother, for
she would not unclasp the thin hand which clung to hers even in sleep.
Hannah had "dished up" an astonishing breakfast for the traveller,
finding it impossible to vent her excitement in any other way; and Meg
and Jo fed their mother like dutiful young storks, while they listened
to her whispered account of father\textquotesingle s state, Mr.
Brooke\textquotesingle s promise to stay and nurse him, the delays which
the storm occasioned on the homeward journey, and the unspeakable
comfort Laurie\textquotesingle s hopeful face had given her when she
arrived, worn out with fatigue, anxiety, and cold.

What a strange, yet pleasant day that was! so brilliant and gay without,
for all the world seemed abroad to welcome the first snow; so quiet and
reposeful within, for every one slept, spent with watching, and a
Sabbath stillness reigned through the house, while nodding Hannah
mounted guard at the door. With a blissful sense of burdens lifted off,
Meg and Jo closed their weary eyes, and lay at rest, like storm-beaten
boats, safe at anchor in a quiet harbor. Mrs. March would not leave
Beth\textquotesingle s side, but rested in the big chair, waking often
to look at, touch, and brood over her child, like a miser over some
recovered treasure.

Laurie, meanwhile, posted off to comfort Amy, and told his story so well
that Aunt March actually "sniffed" herself, and never once said, "I told
you so." Amy came out so strong on this occasion that I think the good
thoughts in the little chapel really began to bear fruit. She dried her
tears quickly, restrained her impatience to see her mother, and never
even thought of the turquoise ring, when the old lady heartily agreed in
Laurie\textquotesingle s opinion, that she behaved "like a capital
little woman." Even Polly seemed impressed, for he called her "good
girl," blessed her buttons, and begged her to "come and take a walk,
dear," in his most affable tone. She would very gladly have gone out to
enjoy the bright wintry weather; but, discovering that Laurie was
dropping with sleep in spite of manful efforts to conceal the fact, she
persuaded him to rest on the sofa, while she wrote a note to her mother.
She was a long time about it; and, when she returned, he was stretched
out, with both arms under his head, sound asleep, while Aunt March had
pulled down the curtains, and sat doing nothing in an unusual fit of
benignity.

After a while, they began to think he was not going to wake till night,
and I\textquotesingle m not sure that he would, had he not been
effectually roused by Amy\textquotesingle s cry of joy at sight of her
mother. There probably were a good many happy little girls in and about
the city that day, but it is my private opinion that Amy was the
happiest of all, when she sat in her mother\textquotesingle s lap and
told her trials, receiving consolation and compensation in the shape of
approving smiles and fond caresses. They were alone together in the
chapel, to which her mother did not object when its purpose was
explained to her.

"On the contrary, I like it very much, dear," looking from the dusty
rosary to the well-worn little book, and the lovely picture with its
garland of evergreen. "It is an excellent plan to have some place where
we can go to be quiet, when things vex or grieve us. There are a good
many hard times in this life of ours, but we can always bear them if we
ask help in the right way. I think my little girl is learning this?"

"Yes, mother; and when I go home I mean to have a corner in the big
closet to put my books, and the copy of that picture which
I\textquotesingle ve tried to make. The woman\textquotesingle s face is
not good,---it\textquotesingle s too beautiful for me to draw,---but the
baby is done better, and I love it very much. I like to think He was a
little child once, for then I don\textquotesingle t seem so far away,
and that helps me."

As Amy pointed to the smiling Christ-child on his
mother\textquotesingle s knee, Mrs. March saw something on the lifted
hand that made her smile. She said nothing, but Amy understood the look,
and, after a minute\textquotesingle s pause, she added gravely,---

"I wanted to speak to you about this, but I forgot it. Aunt gave me the
ring to-day; she called me to her and kissed me, and put it on my
finger, and said I was a credit to her, and she\textquotesingle d like
to keep me always. She gave that funny guard to keep the turquoise on,
as it\textquotesingle s too big. I\textquotesingle d like to wear them,
mother; can I?"

"They are very pretty, but I think you\textquotesingle re rather too
young for such ornaments, Amy," said Mrs. March, looking at the plump
little hand, with the band of sky-blue stones on the forefinger, and the
quaint guard, formed of two tiny, golden hands clasped together.

"I\textquotesingle ll try not to be vain," said Amy. "I
don\textquotesingle t think I like it only because it\textquotesingle s
so pretty; but I want to wear it as the girl in the story wore her
bracelet, to remind me of something."

"Do you mean Aunt March?" asked her mother, laughing.

"No, to remind me not to be selfish." Amy looked so earnest and sincere
about it, that her mother stopped laughing, and listened respectfully to
the little plan.

"I\textquotesingle ve thought a great deal lately about my
\textquotesingle bundle of naughties,\textquotesingle{} and being
selfish is the largest one in it; so I\textquotesingle m going to try
hard to cure it, if I can. Beth isn\textquotesingle t selfish, and
that\textquotesingle s the reason every one loves her and feels so bad
at the thoughts of losing her. People wouldn\textquotesingle t feel half
so bad about me if I was sick, and I don\textquotesingle t deserve to
have them; but I\textquotesingle d like to be loved and missed by a
great many friends, so I\textquotesingle m going to try and be like Beth
all I can. I\textquotesingle m apt to forget my resolutions; but if I
had something always about me to remind me, I guess I should do better.
May I try this way?"

"Yes; but I have more faith in the corner of the big closet. Wear your
ring, dear, and do your best; I think you will prosper, for the sincere
wish to be good is half the battle. Now I must go back to Beth. Keep up
your heart, little daughter, and we will soon have you home again."

That evening, while Meg was writing to her father, to report the
traveller\textquotesingle s safe arrival, Jo slipped up stairs into
Beth\textquotesingle s room, and, finding her mother in her usual place,
stood a minute twisting her fingers in her hair, with a worried gesture
and an undecided look.

"What is it, deary?" asked Mrs. March, holding out her hand, with a face
which invited confidence.

"I want to tell you something, mother."

"About Meg?"

"How quickly you guessed! Yes, it\textquotesingle s about her, and
though it\textquotesingle s a little thing, it fidgets me."

"Beth is asleep; speak low, and tell me all about it. That Moffat
hasn\textquotesingle t been here, I hope?" asked Mrs. March rather
sharply.

"No, I should have shut the door in his face if he had," said Jo,
settling herself on the floor at her mother\textquotesingle s feet.
"Last summer Meg left a pair of gloves over at the
Laurences\textquotesingle, and only one was returned. We forgot all
about it, till Teddy told me that Mr. Brooke had it. He kept it in his
waistcoat pocket, and once it fell out, and Teddy joked him about it,
and Mr. Brooke owned that he liked Meg, but didn\textquotesingle t dare
say so, she was so young and he so poor. Now, isn\textquotesingle t it a
\emph{dread}ful state of things?"

"Do you think Meg cares for him?" asked Mrs. March, with an anxious
look.

"Mercy me! I don\textquotesingle t know anything about love and such
nonsense!" cried Jo, with a funny mixture of interest and contempt. "In
novels, the girls show it by starting and blushing, fainting away,
growing thin, and acting like fools. Now Meg does not do anything of the
sort: she eats and drinks and sleeps, like a sensible creature: she
looks straight in my face when I talk about that man, and only blushes a
little bit when Teddy jokes about lovers. I forbid him to do it, but he
doesn\textquotesingle t mind me as he ought."

"Then you fancy that Meg is \emph{not} interested in John?"

"Who?" cried Jo, staring.

"Mr. Brooke. I call him \textquotesingle John\textquotesingle{} now; we
fell into the way of doing so at the hospital, and he likes it."

"Oh, dear! I know you\textquotesingle ll take his part:
he\textquotesingle s been good to father, and you won\textquotesingle t
send him away, but let Meg marry him, if she wants to. Mean thing! to go
petting papa and helping you, just to wheedle you into liking him;" and
Jo pulled her hair again with a wrathful tweak.

"My dear, don\textquotesingle t get angry about it, and I will tell you
how it happened. John went with me at Mr. Laurence\textquotesingle s
request, and was so devoted to poor father that we
couldn\textquotesingle t help getting fond of him. He was perfectly open
and honorable about Meg, for he told us he loved her, but would earn a
comfortable home before he asked her to marry him. He only wanted our
leave to love her and work for her, and the right to make her love him
if he could. He is a truly excellent young man, and we could not refuse
to listen to him; but I will not consent to Meg\textquotesingle s
engaging herself so young."

"Of course not; it would be idiotic! I knew there was mischief brewing;
I felt it; and now it\textquotesingle s worse than I imagined. I just
wish I could marry Meg myself, and keep her safe in the family."

This odd arrangement made Mrs. March smile; but she said gravely, "Jo, I
confide in you, and don\textquotesingle t wish you to say anything to
Meg yet. When John comes back, and I see them together, I can judge
better of her feelings toward him."

"She\textquotesingle ll see his in those handsome eyes that she talks
about, and then it will be all up with her. She\textquotesingle s got
such a soft heart, it will melt like butter in the sun if any one looks
sentimentally at her. She read the short reports he sent more than she
did your letters, and pinched me when I spoke of it, and likes brown
eyes, and doesn\textquotesingle t think John an ugly name, and
she\textquotesingle ll go and fall in love, and there\textquotesingle s
an end of peace and fun, and cosy times together. I see it all!
they\textquotesingle ll go lovering around the house, and we shall have
to dodge; Meg will be absorbed, and no good to me any more; Brooke will
scratch up a fortune somehow, carry her off, and make a hole in the
family; and I shall break my heart, and everything will be abominably
uncomfortable. Oh, dear me! why weren\textquotesingle t we all boys,
then there wouldn\textquotesingle t be any bother."

Jo leaned her chin on her knees, in a disconsolate attitude, and shook
her fist at the reprehensible John. Mrs. March sighed, and Jo looked up
with an air of relief.

"You don\textquotesingle t like it, mother? I\textquotesingle m glad of
it. Let\textquotesingle s send him about his business, and not tell Meg
a word of it, but all be happy together as we always have been."

"I did wrong to sigh, Jo. It is natural and right you should all go to
homes of your own, in time; but I do want to keep my girls as long as I
can; and I am sorry that this happened so soon, for Meg is only
seventeen, and it will be some years before John can make a home for
her. Your father and I have agreed that she shall not bind herself in
any way, nor be married, before twenty. If she and John love one
another, they can wait, and test the love by doing so. She is
conscientious, and I have no fear of her treating him unkindly. My
pretty, tender-hearted girl! I hope things will go happily with her."

"Hadn\textquotesingle t you rather have her marry a rich man?" asked Jo,
as her mother\textquotesingle s voice faltered a little over the last
words.

"Money is a good and useful thing, Jo; and I hope my girls will never
feel the need of it too bitterly, nor be tempted by too much. I should
like to know that John was firmly established in some good business,
which gave him an income large enough to keep free from debt and make
Meg comfortable. I\textquotesingle m not ambitious for a splendid
fortune, a fashionable position, or a great name for my girls. If rank
and money come with love and virtue, also, I should accept them
gratefully, and enjoy your good fortune; but I know, by experience, how
much genuine happiness can be had in a plain little house, where the
daily bread is earned, and some privations give sweetness to the few
pleasures. I am content to see Meg begin humbly, for, if I am not
mistaken, she will be rich in the possession of a good
man\textquotesingle s heart, and that is better than a fortune."

"I understand, mother, and quite agree; but I\textquotesingle m
disappointed about Meg, for I\textquotesingle d planned to have her
marry Teddy by and by, and sit in the lap of luxury all her days.
Wouldn\textquotesingle t it be nice?" asked Jo, looking up, with a
brighter face.

"He is younger than she, you know," began Mrs. March; but Jo broke
in,---

"Only a little; he\textquotesingle s old for his age, and tall; and can
be quite grown-up in his manners if he likes. Then he\textquotesingle s
rich and generous and good, and loves us all; and \emph{I} say
it\textquotesingle s a pity my plan is spoilt."

"I\textquotesingle m afraid Laurie is hardly grown up enough for Meg,
and altogether too much of a weathercock, just now, for any one to
depend on. Don\textquotesingle t make plans, Jo; but let time and their
own hearts mate your friends. We can\textquotesingle t meddle safely in
such matters, and had better not get \textquotesingle romantic
rubbish,\textquotesingle{} as you call it, into our heads, lest it spoil
our friendship."

"Well, I won\textquotesingle t; but I hate to see things going all
criss-cross and getting snarled up, when a pull here and a snip there
would straighten it out. I wish wearing flat-irons on our heads would
keep us from growing up. But buds will be roses, and kittens,
cats,---more\textquotesingle s the pity!"

"What\textquotesingle s that about flat-irons and cats?" asked Meg, as
she crept into the room, with the finished letter in her hand.

"Only one of my stupid speeches. I\textquotesingle m going to bed; come,
Peggy," said Jo, unfolding herself, like an animated puzzle.

"Quite right, and beautifully written. Please add that I send my love to
John," said Mrs. March, as she glanced over the letter, and gave it
back.

"Do you call him \textquotesingle John\textquotesingle?" asked Meg,
smiling, with her innocent eyes looking down into her
mother\textquotesingle s.

"Yes; he has been like a son to us, and we are very fond of him,"
replied Mrs. March, returning the look with a keen one.

"I\textquotesingle m glad of that, he is so lonely. Good-night, mother,
dear. It is so inexpressibly comfortable to have you here," was
Meg\textquotesingle s quiet answer.

The kiss her mother gave her was a very tender one; and, as she went
away, Mrs. March said, with a mixture of satisfaction and regret, "She
does not love John yet, but will soon learn to."

\protect\phantomsection\label{6672479776654687619_37106-h-3.htm.xhtml_b106.png}{}
\pandocbounded{\includegraphics[keepaspectratio]{303483661336987339_b106.png}}

\begin{center}\rule{0.5\linewidth}{0.5pt}\end{center}

\subsection{XXI. Laurie makes Mischief, and Jo makes
Peace.}\label{6672479776654687619_37106-h-3.htm.xhtml_pgepubid00023}

\protect\phantomsection\label{6672479776654687619_37106-h-3.htm.xhtml_b107.png}{}
\pandocbounded{\includegraphics[keepaspectratio]{303483661336987339_b107.png}}

\protect\phantomsection\label{6672479776654687619_37106-h-3.htm.xhtml_XXI}{}\hyperref[6672479776654687619_37106-h-0.htm.xhtml_contents1b]{XXI.}

LAURIE MAKES MISCHIEF, AND JO MAKES PEACE.

{Jo\textquotesingle s} face was a study next day, for the secret rather
weighed upon her, and she found it hard not to look mysterious and
important. Meg observed it, but did not trouble herself to make
inquiries, for she had learned that the best way to manage Jo was by the
law of contraries, so she felt sure of being told everything if she did
not ask. She was rather surprised, therefore, when the silence remained
unbroken, and Jo assumed a patronizing air, which decidedly aggravated
Meg, who in her turn assumed an air of dignified reserve, and devoted
herself to her mother. This left Jo to her own devices; for Mrs. March
had taken her place as nurse, and bade her rest, exercise, and amuse
herself after her long confinement. Amy being gone, Laurie was her only
refuge; and, much as she enjoyed his society, she rather dreaded him
just then, for he was an incorrigible tease, and she feared he would
coax her secret from her.

She was quite right, for the mischief-loving lad no sooner suspected a
mystery than he set himself to find it out, and led Jo a trying life of
it. He wheedled, bribed, ridiculed, threatened, and scolded; affected
indifference, that he might surprise the truth from her; declared he
knew, then that he didn\textquotesingle t care; and, at last, by dint of
perseverance, he satisfied himself that it concerned Meg and Mr. Brooke.
Feeling indignant that he was not taken into his tutor\textquotesingle s
confidence, he set his wits to work to devise some proper retaliation
for the slight.

Meg meanwhile had apparently forgotten the matter, and was absorbed in
preparations for her father\textquotesingle s return; but all of a
sudden a change seemed to come over her, and, for a day or two, she was
quite unlike herself. She started when spoken to, blushed when looked
at, was very quiet, and sat over her sewing, with a timid, troubled look
on her face. To her mother\textquotesingle s inquiries she answered that
she was quite well, and Jo\textquotesingle s she silenced by begging to
be let alone.

"She feels it in the air---love, I mean---and she\textquotesingle s
going very fast. She\textquotesingle s got most of the symptoms,---is
twittery and cross, doesn\textquotesingle t eat, lies awake, and mopes
in corners. I caught her singing that song he gave her, and once she
said \textquotesingle John,\textquotesingle{} as you do, and then turned
as red as a poppy. Whatever shall we do?" said Jo, looking ready for any
measures, however violent.

"Nothing but wait. Let her alone, be kind and patient, and
father\textquotesingle s coming will settle everything," replied her
mother.

"Here\textquotesingle s a note to you, Meg, all sealed up. How odd!
Teddy never seals mine," said Jo, next day, as she distributed the
contents of the little post-office.

Mrs. March and Jo were deep in their own affairs, when a sound from Meg
made them look up to see her staring at her note, with a frightened
face.

"My child, what is it?" cried her mother, running to her, while Jo tried
to take the paper which had done the mischief.

"It\textquotesingle s all a mistake---he didn\textquotesingle t send it.
O Jo, how could you do it?" and Meg hid her face in her hands, crying as
if her heart was quite broken.

"Me! I\textquotesingle ve done nothing! What\textquotesingle s she
talking about?" cried Jo, bewildered.

Meg\textquotesingle s mild eyes kindled with anger as she pulled a
crumpled note from her pocket, and threw it at Jo, saying
reproachfully,---

"You wrote it, and that bad boy helped you. How could you be so rude, so
mean, and cruel to us both?"

Jo hardly heard her, for she and her mother were reading the note, which
was written in a peculiar hand.

\protect\phantomsection\label{6672479776654687619_37106-h-3.htm.xhtml_b108.png}{}
\pandocbounded{\includegraphics[keepaspectratio]{303483661336987339_b108.png}}

\begin{quote}
"{My Dearest Margaret},---

"I can no longer restrain my passion, and must know my fate before I
return. I dare not tell your parents yet, but I think they would consent
if they knew that we adored one another. Mr. Laurence will help me to
some good place, and then, my sweet girl, you will make me happy. I
implore you to say nothing to your family yet, but to send one word of
hope through Laurie to

"Your devoted {John}."
\end{quote}

"Oh, the little villain! that\textquotesingle s the way he meant to pay
me for keeping my word to mother. I\textquotesingle ll give him a hearty
scolding, and bring him over to beg pardon," cried Jo, burning to
execute immediate justice. But her mother held her back, saying, with a
look she seldom wore,---

"Stop, Jo, you must clear yourself first. You have played so many
pranks, that I am afraid you have had a hand in this."

"On my word, mother, I haven\textquotesingle t! I never saw that note
before, and don\textquotesingle t know anything about it, as true as I
live!" said Jo, so earnestly that they believed her. "If I \emph{had}
taken a part in it I\textquotesingle d have done it better than this,
and have written a sensible note. I should think you\textquotesingle d
have known Mr. Brooke wouldn\textquotesingle t write such stuff as
that," she added, scornfully tossing down the paper.

"It\textquotesingle s like his writing," faltered Meg, comparing it with
the note in her hand.

"O Meg, you didn\textquotesingle t answer it?" cried Mrs. March quickly.

"Yes, I did!" and Meg hid her face again, overcome with shame.

"Here\textquotesingle s a scrape! \emph{Do} let me bring that wicked boy
over to explain, and be lectured. I can\textquotesingle t rest till I
get hold of him;" and Jo made for the door again.

"Hush! let me manage this, for it is worse than I thought. Margaret,
tell me the whole story," commanded Mrs. March, sitting down by Meg, yet
keeping hold of Jo, lest she should fly off.

"I received the first letter from Laurie, who didn\textquotesingle t
look as if he knew anything about it," began Meg, without looking up. "I
was worried at first, and meant to tell you; then I remembered how you
liked Mr. Brooke, so I thought you wouldn\textquotesingle t mind if I
kept my little secret for a few days. I\textquotesingle m so silly that
I liked to think no one knew; and, while I was deciding what to say, I
felt like the girls in books, who have such things to do. Forgive me,
mother, I\textquotesingle m paid for my silliness now; I never can look
him in the face again."

"What did you say to him?" asked Mrs. March.

"I only said I was too young to do anything about it yet; that I
didn\textquotesingle t wish to have secrets from you, and he must speak
to father. I was very grateful for his kindness, and would be his
friend, but nothing more, for a long while."

Mrs. March smiled, as if well pleased, and Jo clapped her hands,
exclaiming, with a laugh,---

"You are almost equal to Caroline Percy, who was a pattern of prudence!
Tell on, Meg. What did he say to that?"

"He writes in a different way entirely, telling me that he never sent
any love-letter at all, and is very sorry that my roguish sister, Jo,
should take such liberties with our names. It\textquotesingle s very
kind and respectful, but think how dreadful for me!"

Meg leaned against her mother, looking the image of despair, and Jo
tramped about the room, calling Laurie names. All of a sudden she
stopped, caught up the two notes, and, after looking at them closely,
said decidedly, "I don\textquotesingle t believe Brooke ever saw either
of these letters. Teddy wrote both, and keeps yours to crow over me
with, because I wouldn\textquotesingle t tell him my secret."

"Don\textquotesingle t have any secrets, Jo; tell it to mother, and keep
out of trouble, as I should have done," said Meg warningly.

"Bless you, child! Mother told me."

"That will do, Jo. I\textquotesingle ll comfort Meg while you go and get
Laurie. I shall sift the matter to the bottom, and put a stop to such
pranks at once."

Away ran Jo, and Mrs. March gently told Meg Mr. Brooke\textquotesingle s
real feelings. "Now, dear, what are your own? Do you love him enough to
wait till he can make a home for you, or will you keep yourself quite
free for the present?"

"I\textquotesingle ve been so scared and worried, I
don\textquotesingle t want to have anything to do with lovers for a long
while,---perhaps never," answered Meg petulantly. "If John
\emph{doesn\textquotesingle t} know anything about this nonsense,
don\textquotesingle t tell him, and make Jo and Laurie hold their
tongues. I won\textquotesingle t be deceived and plagued and made a fool
of,---it\textquotesingle s a shame!"

Seeing that Meg\textquotesingle s usually gentle temper was roused and
her pride hurt by this mischievous joke, Mrs. March soothed her by
promises of entire silence, and great discretion for the future. The
instant Laurie\textquotesingle s step was heard in the hall, Meg fled
into the study, and Mrs. March received the culprit alone. Jo had not
told him why he was wanted, fearing he wouldn\textquotesingle t come;
but he knew the minute he saw Mrs. March\textquotesingle s face, and
stood twirling his hat, with a guilty air which convicted him at once.
Jo was dismissed, but chose to march up and down the hall like a
sentinel, having some fear that the prisoner might bolt. The sound of
voices in the parlor rose and fell for half an hour; but what happened
during that interview the girls never knew.

When they were called in, Laurie was standing by their mother, with such
a penitent face that Jo forgave him on the spot, but did not think it
wise to betray the fact. Meg received his humble apology, and was much
comforted by the assurance that Brooke knew nothing of the joke.

"I\textquotesingle ll never tell him to my dying day,---wild horses
sha\textquotesingle n\textquotesingle t drag it out of me; so
you\textquotesingle ll forgive me, Meg, and I\textquotesingle ll do
anything to show how out-and-out sorry I am," he added, looking very
much ashamed of himself.

"I\textquotesingle ll try; but it was a very ungentlemanly thing to do.
I didn\textquotesingle t think you could be so sly and malicious,
Laurie," replied Meg, trying to hide her maidenly confusion under a
gravely reproachful air.

"It was altogether abominable, and I don\textquotesingle t deserve to be
spoken to for a month; but you will, though, won\textquotesingle t you?"
and Laurie folded his hands together with such an imploring gesture, as
he spoke in his irresistibly persuasive tone, that it was impossible to
frown upon him, in spite of his scandalous behavior. Meg pardoned him,
and Mrs. March\textquotesingle s grave face relaxed, in spite of her
efforts to keep sober, when she heard him declare that he would atone
for his sins by all sorts of penances, and abase himself like a worm
before the injured damsel.

Jo stood aloof, meanwhile, trying to harden her heart against him, and
succeeding only in primming up her face into an expression of entire
disapprobation. Laurie looked at her once or twice, but, as she showed
no sign of relenting, he felt injured, and turned his back on her till
the others were done with him, when he made her a low bow, and walked
off without a word.

As soon as he had gone, she wished she had been more forgiving; and when
Meg and her mother went upstairs, she felt lonely, and longed for Teddy.
After resisting for some time, she yielded to the impulse, and, armed
with a book to return, went over to the big house.

"Is Mr. Laurence in?" asked Jo, of a housemaid, who was coming down
stairs.

"Yes, miss; but I don\textquotesingle t believe he\textquotesingle s
seeable just yet."

"Why not? is he ill?"

"La, no, miss, but he\textquotesingle s had a scene with Mr. Laurie, who
is in one of his tantrums about something, which vexes the old
gentleman, so I dursn\textquotesingle t go nigh him."

"Where is Laurie?"

"Shut up in his room, and he won\textquotesingle t answer, though
I\textquotesingle ve been a-tapping. I don\textquotesingle t know
what\textquotesingle s to become of the dinner, for it\textquotesingle s
ready, and there\textquotesingle s no one to eat it."

"I\textquotesingle ll go and see what the matter is. I\textquotesingle m
not afraid of either of them."

Up went Jo, and knocked smartly on the door of Laurie\textquotesingle s
little study.

"Stop that, or I\textquotesingle ll open the door and make you!" called
out the young gentleman, in a threatening tone.

Jo immediately knocked again; the door flew open, and in she bounced,
before Laurie could recover from his surprise. Seeing that he really
\emph{was} out of temper, Jo, who knew how to manage him, assumed a
contrite expression, and going artistically down upon her knees, said
meekly, "Please forgive me for being so cross. I came to make it up, and
can\textquotesingle t go away till I have."

"It\textquotesingle s all right. Get up, and don\textquotesingle t be a
goose, Jo," was the cavalier reply to her petition.

\protect\phantomsection\label{6672479776654687619_37106-h-3.htm.xhtml_b109.png}{}
\pandocbounded{\includegraphics[keepaspectratio]{303483661336987339_b109.png}}

"Thank you; I will. Could I ask what\textquotesingle s the matter? You
don\textquotesingle t look exactly easy in your mind."

"I\textquotesingle ve been shaken, and I won\textquotesingle t bear it!"
growled Laurie indignantly.

"Who did it?" demanded Jo.

"Grandfather; if it had been any one else I\textquotesingle d have---"
and the injured youth finished his sentence by an energetic gesture of
the right arm.

"That\textquotesingle s nothing; I often shake you, and you
don\textquotesingle t mind," said Jo soothingly.

"Pooh! you\textquotesingle re a girl, and it\textquotesingle s fun; but
I\textquotesingle ll allow no man to shake \emph{me}."

"I don\textquotesingle t think any one would care to try it, if you
looked as much like a thunder-cloud as you do now. Why were you treated
so?"

"Just because I wouldn\textquotesingle t say what your mother wanted me
for. I\textquotesingle d promised not to tell, and of course I
wasn\textquotesingle t going to break my word."

"Couldn\textquotesingle t you satisfy your grandpa in any other way?"

"No; he \emph{would} have the truth, the whole truth, and nothing but
the truth. I\textquotesingle d have told my part of the scrape, if I
could without bringing Meg in. As I couldn\textquotesingle t, I held my
tongue, and bore the scolding till the old gentleman collared me. Then I
got angry, and bolted, for fear I should forget myself."

"It wasn\textquotesingle t nice, but he\textquotesingle s sorry, I know;
so go down and make up. I\textquotesingle ll help you."

"Hanged if I do! I\textquotesingle m not going to be lectured and
pummelled by every one, just for a bit of a frolic. I \emph{was} sorry
about Meg, and begged pardon like a man; but I won\textquotesingle t do
it again, when I wasn\textquotesingle t in the wrong."

"He didn\textquotesingle t know that."

"He ought to trust me, and not act as if I was a baby.
It\textquotesingle s no use, Jo; he\textquotesingle s got to learn that
I\textquotesingle m able to take care of myself, and
don\textquotesingle t need any one\textquotesingle s apron-string to
hold on by."

"What pepper-pots you are!" sighed Jo. "How do you mean to settle this
affair?"

"Well, he ought to beg pardon, and believe me when I say I
can\textquotesingle t tell him what the fuss\textquotesingle s about."

"Bless you! he won\textquotesingle t do that."

"I won\textquotesingle t go down till he does."

"Now, Teddy, be sensible; let it pass, and I\textquotesingle ll explain
what I can. You can\textquotesingle t stay here, so
what\textquotesingle s the use of being melodramatic?"

"I don\textquotesingle t intend to stay here long, any way.
I\textquotesingle ll slip off and take a journey somewhere, and when
grandpa misses me he\textquotesingle ll come round fast enough."

"I dare say; but you ought not to go and worry him."

"Don\textquotesingle t preach. I\textquotesingle ll go to Washington and
see Brooke; it\textquotesingle s gay there, and I\textquotesingle ll
enjoy myself after the troubles."

"What fun you\textquotesingle d have! I wish I could run off too," said
Jo, forgetting her part of Mentor in lively visions of martial life at
the capital.

"Come on, then! Why not? You go and surprise your father, and
I\textquotesingle ll stir up old Brooke. It would be a glorious joke;
let\textquotesingle s do it, Jo. We\textquotesingle ll leave a letter
saying we are all right, and trot off at once. I\textquotesingle ve got
money enough; it will do you good, and be no harm, as you go to your
father."

For a moment Jo looked as if she would agree; for, wild as the plan was,
it just suited her. She was tired of care and confinement, longed for
change, and thoughts of her father blended temptingly with the novel
charms of camps and hospitals, liberty and fun. Her eyes kindled as they
turned wistfully toward the window, but they fell on the old house
opposite, and she shook her head with sorrowful decision.

"If I was a boy, we\textquotesingle d run away together, and have a
capital time; but as I\textquotesingle m a miserable girl, I must be
proper, and stop at home. Don\textquotesingle t tempt me, Teddy,
it\textquotesingle s a crazy plan."

"That\textquotesingle s the fun of it," began Laurie, who had got a
wilful fit on him, and was possessed to break out of bounds in some way.

"Hold your tongue!" cried Jo, covering her ears.
"\textquotesingle Prunes and prisms\textquotesingle{} are my doom, and I
may as well make up my mind to it. I came here to moralize, not to hear
about things that make me skip to think of."

\protect\phantomsection\label{6672479776654687619_37106-h-3.htm.xhtml_b110.png}{}
\pandocbounded{\includegraphics[keepaspectratio]{303483661336987339_b110.png}}

"I know Meg would wet-blanket such a proposal, but I thought you had
more spirit," began Laurie insinuatingly.

"Bad boy, be quiet! Sit down and think of your own sins,
don\textquotesingle t go making me add to mine. If I get your grandpa to
apologize for the shaking, will you give up running away?" asked Jo
seriously.

"Yes, but you won\textquotesingle t do it," answered Laurie, who wished
"to make up," but felt that his outraged dignity must be appeased first.

"If I can manage the young one I can the old one," muttered Jo, as she
walked away, leaving Laurie bent over a railroad map, with his head
propped up on both hands.

"Come in!" and Mr. Laurence\textquotesingle s gruff voice sounded
gruffer than ever, as Jo tapped at his door.

"It\textquotesingle s only me, sir, come to return a book," she said
blandly, as she entered.

"Want any more?" asked the old gentleman, looking grim and vexed, but
trying not to show it.

"Yes, please. I like old Sam so well, I think I\textquotesingle ll try
the second volume," returned Jo, hoping to propitiate him by accepting a
second dose of Boswell\textquotesingle s "Johnson," as he had
recommended that lively work.

The shaggy eyebrows unbent a little, as he rolled the steps toward the
shelf where the Johnsonian literature was placed. Jo skipped up, and,
sitting on the top step, affected to be searching for her book, but was
really wondering how best to introduce the dangerous object of her
visit. Mr. Laurence seemed to suspect that something was brewing in her
mind; for, after taking several brisk turns about the room, he faced
round on her, speaking so abruptly that "Rasselas" tumbled face downward
on the floor.

"What has that boy been about? Don\textquotesingle t try to shield him.
I know he has been in mischief by the way he acted when he came home. I
can\textquotesingle t get a word from him; and when I threatened to
shake the truth out of him he bolted upstairs, and locked himself into
his room."

"He did do wrong, but we forgave him, and all promised not to say a word
to any one," began Jo reluctantly.

"That won\textquotesingle t do; he shall not shelter himself behind a
promise from you soft-hearted girls. If he\textquotesingle s done
anything amiss, he shall confess, beg pardon, and be punished. Out with
it, Jo, I won\textquotesingle t be kept in the dark."

Mr. Laurence looked so alarming and spoke so sharply that Jo would have
gladly run away, if she could, but she was perched aloft on the steps,
and he stood at the foot, a lion in the path, so she had to stay and
brave it out.

\protect\phantomsection\label{6672479776654687619_37106-h-3.htm.xhtml_b111.png}{}
\pandocbounded{\includegraphics[keepaspectratio]{303483661336987339_b111.png}}

"Indeed, sir, I cannot tell; mother forbade it. Laurie has confessed,
asked pardon, and been punished quite enough. We don\textquotesingle t
keep silence to shield him, but some one else, and it will make more
trouble if you interfere. Please don\textquotesingle t; it was partly my
fault, but it\textquotesingle s all right now; so let\textquotesingle s
forget it, and talk about the
\textquotesingle Rambler,\textquotesingle{} or something pleasant."

"Hang the \textquotesingle Rambler!\textquotesingle{} come down and give
me your word that this harum-scarum boy of mine hasn\textquotesingle t
done anything ungrateful or impertinent. If he has, after all your
kindness to him, I\textquotesingle ll thrash him with my own hands."

The threat sounded awful, but did not alarm Jo, for she knew the
irascible old gentleman would never lift a finger against his grandson,
whatever he might say to the contrary. She obediently descended, and
made as light of the prank as she could without betraying Meg or
forgetting the truth.

"Hum---ha---well, if the boy held his tongue because he promised, and
not from obstinacy, I\textquotesingle ll forgive him.
He\textquotesingle s a stubborn fellow, and hard to manage," said Mr.
Laurence, rubbing up his hair till it looked as if he had been out in a
gale, and smoothing the frown from his brow with an air of relief.

"So am I; but a kind word will govern me when all the
king\textquotesingle s horses and all the king\textquotesingle s men
couldn\textquotesingle t," said Jo, trying to say a kind word for her
friend, who seemed to get out of one scrape only to fall into another.

"You think I\textquotesingle m not kind to him, hey?" was the sharp
answer.

"Oh, dear, no, sir; you are rather too kind sometimes, and then just a
trifle hasty when he tries your patience. Don\textquotesingle t you
think you are?"

Jo was determined to have it out now, and tried to look quite placid,
though she quaked a little after her bold speech. To her great relief
and surprise, the old gentleman only threw his spectacles on to the
table with a rattle, and exclaimed frankly,---

"You\textquotesingle re right, girl, I am! I love the boy, but he tries
my patience past bearing, and I don\textquotesingle t know how it will
end, if we go on so."

"I\textquotesingle ll tell you, he\textquotesingle ll run away." Jo was
sorry for that speech the minute it was made; she meant to warn him that
Laurie would not bear much restraint, and hoped he would be more
forbearing with the lad.

Mr. Laurence\textquotesingle s ruddy face changed suddenly, and he sat
down, with a troubled glance at the picture of a handsome man, which
hung over his table. It was Laurie\textquotesingle s father, who
\emph{had} run away in his youth, and married against the imperious old
man\textquotesingle s will. Jo fancied he remembered and regretted the
past, and she wished she had held her tongue.

"He won\textquotesingle t do it unless he is very much worried, and only
threatens it sometimes, when he gets tired of studying. I often think I
should like to, especially since my hair was cut; so, if you ever miss
us, you may advertise for two boys, and look among the ships bound for
India."

She laughed as she spoke, and Mr. Laurence looked relieved, evidently
taking the whole as a joke.

"You hussy, how dare you talk in that way? Where\textquotesingle s your
respect for me, and your proper bringing up? Bless the boys and girls!
What torments they are; yet we can\textquotesingle t do without them,"
he said, pinching her cheeks good-humoredly. "Go and bring that boy down
to his dinner, tell him it\textquotesingle s all right, and advise him
not to put on tragedy airs with his grandfather. I won\textquotesingle t
bear it."

"He won\textquotesingle t come, sir; he feels badly because you
didn\textquotesingle t believe him when he said he
couldn\textquotesingle t tell. I think the shaking hurt his feelings
very much."

Jo tried to look pathetic, but must have failed, for Mr. Laurence began
to laugh, and she knew the day was won.

"I\textquotesingle m sorry for that, and ought to thank him for not
shaking \emph{me,} I suppose. What the dickens does the fellow expect?"
and the old gentleman looked a trifle ashamed of his own testiness.

"If I were you, I\textquotesingle d write him an apology, sir. He says
he won\textquotesingle t come down till he has one, and talks about
Washington, and goes on in an absurd way. A formal apology will make him
see how foolish he is, and bring him down quite amiable. Try it; he
likes fun, and this way is better than talking. I\textquotesingle ll
carry it up, and teach him his duty."

Mr. Laurence gave her a sharp look, and put on his spectacles, saying
slowly, "You\textquotesingle re a sly puss, but I don\textquotesingle t
mind being managed by you and Beth. Here, give me a bit of paper, and
let us have done with this nonsense."

The note was written in the terms which one gentleman would use to
another after offering some deep insult. Jo dropped a kiss on the top of
Mr. Laurence\textquotesingle s bald head, and ran up to slip the apology
under Laurie\textquotesingle s door, advising him, through the key-hole,
to be submissive, decorous, and a few other agreeable impossibilities.
Finding the door locked again, she left the note to do its work, and was
going quietly away, when the young gentleman slid down the banisters,
and waited for her at the bottom, saying, with his most virtuous
expression of countenance, "What a good fellow you are, Jo! Did you get
blown up?" he added, laughing.

"No; he was pretty mild, on the whole."

"Ah! I got it all round; even you cast me off over there, and I felt
just ready to go to the deuce," he began apologetically.

"Don\textquotesingle t talk in that way; turn over a new leaf and begin
again, Teddy, my son."

"I keep turning over new leaves, and spoiling them, as I used to spoil
my copy-books; and I make so many beginnings there never will be an
end," he said dolefully.

"Go and eat your dinner; you\textquotesingle ll feel better after it.
Men always croak when they are hungry," and Jo whisked out at the front
door after that.

"That\textquotesingle s a \textquotesingle label\textquotesingle{} on my
\textquotesingle sect,\textquotesingle" answered Laurie, quoting Amy, as
he went to partake of humble-pie dutifully with his grandfather, who was
quite saintly in temper and overwhelmingly respectful in manner all the
rest of the day.

Every one thought the matter ended and the little cloud blown over; but
the mischief was done, for, though others forgot it, Meg remembered. She
never alluded to a certain person, but she thought of him a good deal,
dreamed dreams more than ever; and once Jo, rummaging her
sister\textquotesingle s desk for stamps, found a bit of paper scribbled
over with the words, "Mrs. John Brooke;" whereat she groaned tragically,
and cast it into the fire, feeling that Laurie\textquotesingle s prank
had hastened the evil day for her.

\begin{center}\rule{0.5\linewidth}{0.5pt}\end{center}

\subsection{XXII. Pleasant
Meadows.}\label{6672479776654687619_37106-h-3.htm.xhtml_pgepubid00024}

\protect\phantomsection\label{6672479776654687619_37106-h-4.htm.xhtml}{}

\protect\phantomsection\label{6672479776654687619_37106-h-4.htm.xhtml_b112.png}{}
\pandocbounded{\includegraphics[keepaspectratio]{303483661336987339_b112.png}}

\protect\phantomsection\label{6672479776654687619_37106-h-4.htm.xhtml_XXII}{}\hyperref[6672479776654687619_37106-h-0.htm.xhtml_contents1b]{XXII.}

PLEASANT MEADOWS.

{Like} sunshine after storm were the peaceful weeks which followed. The
invalids improved rapidly, and Mr. March began to talk of returning
early in the new year. Beth was soon able to lie on the study sofa all
day, amusing herself with the well-beloved cats, at first, and, in time,
with doll\textquotesingle s sewing, which had fallen sadly behindhand.
Her once active limbs were so stiff and feeble that Jo took her a daily
airing about the house in her strong arms. Meg cheerfully blackened and
burnt her white hands cooking delicate messes for "the dear;" while Amy,
a loyal slave of the ring, celebrated her return by giving away as many
of her treasures as she could prevail on her sisters to accept.

As Christmas approached, the usual mysteries began to haunt the house,
and Jo frequently convulsed the family by proposing utterly impossible
or magnificently absurd ceremonies, in honor of this unusually merry
Christmas. Laurie was equally impracticable, and would have had
bonfires, sky-rockets, and triumphal arches, if he had had his own way.
After many skirmishes and snubbings, the ambitious pair were considered
effectually quenched, and went about with forlorn faces, which were
rather belied by explosions of laughter when the two got together.

Several days of unusually mild weather fitly ushered in a splendid
Christmas Day. Hannah "felt in her bones" that it was going to be an
unusually fine day, and she proved herself a true prophetess, for
everybody and everything seemed bound to produce a grand success. To
begin with, Mr. March wrote that he should soon be with them; then Beth
felt uncommonly well that morning, and, being dressed in her
mother\textquotesingle s gift,---a soft crimson merino wrapper,---was
borne in triumph to the window to behold the offering of Jo and Laurie.
The Unquenchables had done their best to be worthy of the name, for,
like elves, they had worked by night, and conjured up a comical
surprise. Out in the garden stood a stately snow-maiden, crowned with
holly, bearing a basket of fruit and flowers in one hand, a great roll
of new music in the other, a perfect rainbow of an Afghan round her
chilly shoulders, and a Christmas carol issuing from her lips, on a pink
paper streamer:---

\protect\phantomsection\label{6672479776654687619_37106-h-4.htm.xhtml_b113.png}{}
\pandocbounded{\includegraphics[keepaspectratio]{303483661336987339_b113.png}}

"THE JUNGFRAU TO BETH.

"God bless you, dear Queen Bess!

May nothing you dismay,

But health and peace and happiness

Be yours, this Christmas Day.

\hfill\break

"Here\textquotesingle s fruit to feed our busy bee,

And flowers for her nose;

Here\textquotesingle s music for her pianee,

An Afghan for her toes.

\hfill\break

"A portrait of Joanna, see,

By Raphael No. 2,

Who labored with great industry

To make it fair and true.

\hfill\break

"Accept a ribbon red, I beg,

For Madam Purrer\textquotesingle s tail;

And ice-cream made by lovely Peg,---

A Mont Blanc in a pail.

\hfill\break

"Their dearest love my makers laid

Within my breast of snow:

Accept it, and the Alpine maid,

From Laurie and from Jo."

How Beth laughed when she saw it, how Laurie ran up and down to bring in
the gifts, and what ridiculous speeches Jo made as she presented them!

"I\textquotesingle m so full of happiness, that, if father was only
here, I couldn\textquotesingle t hold one drop more," said Beth, quite
sighing with contentment as Jo carried her off to the study to rest
after the excitement, and to refresh herself with some of the delicious
grapes the "Jungfrau" had sent her.

"So am I," added Jo, slapping the pocket wherein reposed the
long-desired Undine and Sintram.

"I\textquotesingle m sure I am," echoed Amy, poring over the engraved
copy of the Madonna and Child, which her mother had given her, in a
pretty frame.

"Of course I am!" cried Meg, smoothing the silvery folds of her first
silk dress; for Mr. Laurence had insisted on giving it.

"How can \emph{I} be otherwise?" said Mrs. March gratefully, as her eyes
went from her husband\textquotesingle s letter to Beth\textquotesingle s
smiling face, and her hand caressed the brooch made of gray and golden,
chestnut and dark brown hair, which the girls had just fastened on her
breast.

Now and then, in this work-a-day world, things do happen in the
delightful story-book fashion, and what a comfort that is. Half an hour
after every one had said they were so happy they could only hold one
drop more, the drop came. Laurie opened the parlor door, and popped his
head in very quietly. He might just as well have turned a somersault and
uttered an Indian war-whoop; for his face was so full of suppressed
excitement and his voice so treacherously joyful, that every one jumped
up, though he only said, in a queer, breathless voice,
"Here\textquotesingle s another Christmas present for the March family."

Before the words were well out of his mouth, he was whisked away
somehow, and in his place appeared a tall man, muffled up to the eyes,
leaning on the arm of another tall man, who tried to say something and
couldn\textquotesingle t. Of course there was a general stampede; and
for several minutes everybody seemed to lose their wits, for the
strangest things were done, and no one said a word. Mr. March became
invisible in the embrace of four pairs of loving arms; Jo disgraced
herself by nearly fainting away, and had to be doctored by Laurie in the
china-closet; Mr. Brooke kissed Meg entirely by mistake, as he somewhat
incoherently explained; and Amy, the dignified, tumbled over a stool,
and, never stopping to get up, hugged and cried over her
father\textquotesingle s boots in the most touching manner. Mrs. March
was the first to recover herself, and held up her hand with a warning,
"Hush! remember Beth!"

But it was too late; the study door flew open, the little red wrapper
appeared on the threshold,---joy put strength into the feeble
limbs,---and Beth ran straight into her father\textquotesingle s arms.
Never mind what happened just after that; for the full hearts
overflowed, washing away the bitterness of the past, and leaving only
the sweetness of the present.

It was not at all romantic, but a hearty laugh set everybody straight
again, for Hannah was discovered behind the door, sobbing over the fat
turkey, which she had forgotten to put down when she rushed up from the
kitchen. As the laugh subsided, Mrs. March began to thank Mr. Brooke for
his faithful care of her husband, at which Mr. Brooke suddenly
remembered that Mr. March needed rest, and, seizing Laurie, he
precipitately retired. Then the two invalids were ordered to repose,
which they did, by both sitting in one big chair, and talking hard.

Mr. March told how he had longed to surprise them, and how, when the
fine weather came, he had been allowed by his doctor to take advantage
of it; how devoted Brooke had been, and how he was altogether a most
estimable and upright young man. Why Mr. March paused a minute just
there, and, after a glance at Meg, who was violently poking the fire,
looked at his wife with an inquiring lift of the eyebrows, I leave you
to imagine; also why Mrs. March gently nodded her head, and asked,
rather abruptly, if he wouldn\textquotesingle t have something to eat.
Jo saw and understood the look; and she stalked grimly away to get wine
and beef-tea, muttering to herself, as she slammed the door, "I hate
estimable young men with brown eyes!"

There never \emph{was} such a Christmas dinner as they had that day. The
fat turkey was a sight to behold, when Hannah sent him up, stuffed,
browned, and decorated; so was the plum-pudding, which quite melted in
one\textquotesingle s mouth; likewise the jellies, in which Amy revelled
like a fly in a honey-pot. Everything turned out well, which was a
mercy, Hannah said, "For my mind was that flustered, mum, that
it\textquotesingle s a merrycle I didn\textquotesingle t roast the
pudding, and stuff the turkey with raisins, let alone
bilin\textquotesingle{} of it in a cloth."

Mr. Laurence and his grandson dined with them, also Mr. Brooke,---at
whom Jo glowered darkly, to Laurie\textquotesingle s infinite amusement.
Two easy-chairs stood side by side at the head of the table, in which
sat Beth and her father, feasting modestly on chicken and a little
fruit. They drank healths, told stories, sung songs, "reminisced," as
the old folks say, and had a thoroughly good time. A sleigh-ride had
been planned, but the girls would not leave their father; so the guests
departed early, and, as twilight gathered, the happy family sat together
round the fire.

"Just a year ago we were groaning over the dismal Christmas we expected
to have. Do you remember?" asked Jo, breaking a short pause which had
followed a long conversation about many things.

"Rather a pleasant year on the whole!" said Meg, smiling at the fire,
and congratulating herself on having treated Mr. Brooke with dignity.

"I think it\textquotesingle s been a pretty hard one," observed Amy,
watching the light shine on her ring, with thoughtful eyes.

"I\textquotesingle m glad it\textquotesingle s over, because
we\textquotesingle ve got you back," whispered Beth, who sat on her
father\textquotesingle s knee.

"Rather a rough road for you to travel, my little pilgrims, especially
the latter part of it. But you have got on bravely; and I think the
burdens are in a fair way to tumble off very soon," said Mr. March,
looking with fatherly satisfaction at the four young faces gathered
round him.

"How do you know? Did mother tell you?" asked Jo.

"Not much; straws show which way the wind blows, and
I\textquotesingle ve made several discoveries to-day."

"Oh, tell us what they are!" cried Meg, who sat beside him.

"Here is one;" and taking up the hand which lay on the arm of his chair,
he pointed to the roughened forefinger, a burn on the back, and two or
three little hard spots on the palm. "I remember a time when this hand
was white and smooth, and your first care was to keep it so. It was very
pretty then, but to me it is much prettier now,---for in these seeming
blemishes I read a little history. A burnt-offering has been made of
vanity; this hardened palm has earned something better than blisters;
and I\textquotesingle m sure the sewing done by these pricked fingers
will last a long time, so much good-will went into the stitches. Meg, my
dear, I value the womanly skill which keeps home happy more than white
hands or fashionable accomplishments. I\textquotesingle m proud to shake
this good, industrious little hand, and hope I shall not soon be asked
to give it away."

If Meg had wanted a reward for hours of patient labor, she received it
in the hearty pressure of her father\textquotesingle s hand and the
approving smile he gave her.

"What about Jo? Please say something nice; for she has tried so hard,
and been so very, very good to me," said Beth, in her
father\textquotesingle s ear.

He laughed, and looked across at the tall girl who sat opposite, with an
unusually mild expression in her brown face.

"In spite of the curly crop, I don\textquotesingle t see the
\textquotesingle son Jo\textquotesingle{} whom I left a year ago," said
Mr. March. "I see a young lady who pins her collar straight, laces her
boots neatly, and neither whistles, talks slang, nor lies on the rug as
she used to do. Her face is rather thin and pale, just now, with
watching and anxiety; but I like to look at it, for it has grown
gentler, and her voice is lower; she doesn\textquotesingle t bounce, but
moves quietly, and takes care of a certain little person in a motherly
way which delights me. I rather miss my wild girl; but if I get a
strong, helpful, tender-hearted woman in her place, I shall feel quite
satisfied. I don\textquotesingle t know whether the shearing sobered our
black sheep, but I do know that in all Washington I
couldn\textquotesingle t find anything beautiful enough to be bought
with the five-and-twenty dollars which my good girl sent me."

Jo\textquotesingle s keen eyes were rather dim for a minute, and her
thin face grew rosy in the firelight, as she received her
father\textquotesingle s praise, feeling that she did deserve a portion
of it.

"Now Beth," said Amy, longing for her turn, but ready to wait.

"There\textquotesingle s so little of her, I\textquotesingle m afraid to
say much, for fear she will slip away altogether, though she is not so
shy as she used to be," began their father cheerfully; but recollecting
how nearly he \emph{had} lost her, he held her close, saying tenderly,
with her cheek against his own, "I\textquotesingle ve got you safe, my
Beth, and I\textquotesingle ll keep you so, please God."

After a minute\textquotesingle s silence, he looked down at Amy, who sat
on the cricket at his feet, and said, with a caress of the shining
hair,---

"I observed that Amy took drumsticks at dinner, ran errands for her
mother all the afternoon, gave Meg her place to-night, and has waited on
every one with patience and good-humor. I also observe that she does not
fret much nor look in the glass, and has not even mentioned a very
pretty ring which she wears; so I conclude that she has learned to think
of other people more and of herself less, and has decided to try and
mould her character as carefully as she moulds her little clay figures.
I am glad of this; for though I should be very proud of a graceful
statue made by her, I shall be infinitely prouder of a lovable daughter,
with a talent for making life beautiful to herself and others."

"What are you thinking of, Beth?" asked Jo, when Amy had thanked her
father and told about her ring.

"I read in \textquotesingle Pilgrim\textquotesingle s
Progress\textquotesingle{} to-day, how, after many troubles, Christian
and Hopeful came to a pleasant green meadow, where lilies bloomed all
the year round, and there they rested happily, as we do now, before they
went on to their journey\textquotesingle s end," answered Beth; adding,
as she slipped out of her father\textquotesingle s arms, and went slowly
to the instrument, "It\textquotesingle s singing time now, and I want to
be in my old place. I\textquotesingle ll try to sing the song of the
shepherd-boy which the Pilgrims heard. I made the music for father,
because he likes the verses."

So, sitting at the dear little piano, Beth softly touched the keys, and,
in the sweet voice they had never thought to hear again, sung to her own
accompaniment the quaint hymn, which was a singularly fitting song for
her:---

"He that is down need fear no fall,

He that is low no pride;

He that is humble ever shall

Have God to be his guide.

\hfill\break

"I am content with what I have,

Little be it or much;

And, Lord! contentment still I crave,

Because Thou savest such.

\hfill\break

"Fulness to them a burden is,

That go on pilgrimage;

Here little, and hereafter bliss,

Is best from age to age!"

\begin{center}\rule{0.5\linewidth}{0.5pt}\end{center}

\protect\phantomsection\label{6672479776654687619_37106-h-4.htm.xhtml_b114.png}{}
\pandocbounded{\includegraphics[keepaspectratio]{303483661336987339_b114.png}}\\
\protect\phantomsection\label{6672479776654687619_37106-h-4.htm.xhtml_ebm_caption2}{
"He sat in the big chair by Beth\textquotesingle s sofa with the other
three close by."---Page 277.}

\begin{center}\rule{0.5\linewidth}{0.5pt}\end{center}

\subsection{XXIII. Aunt March settles the
Question.}\label{6672479776654687619_37106-h-4.htm.xhtml_pgepubid00025}

\protect\phantomsection\label{6672479776654687619_37106-h-4.htm.xhtml_XXIII}{}\hyperref[6672479776654687619_37106-h-0.htm.xhtml_contents1b]{XXIII.}

AUNT MARCH SETTLES THE QUESTION.

\protect\phantomsection\label{6672479776654687619_37106-h-4.htm.xhtml_b115.png}{}
\pandocbounded{\includegraphics[keepaspectratio]{303483661336987339_b115.png}}

{Like} bees swarming after their queen, mother and daughters hovered
about Mr. March the next day, neglecting everything to look at, wait
upon, and listen to the new invalid, who was in a fair way to be killed
by kindness. As he sat propped up in a big chair by
Beth\textquotesingle s sofa, with the other three close by, and Hannah
popping in her head now and then, "to peek at the dear man," nothing
seemed needed to complete their happiness. But something \emph{was}
needed, and the elder ones felt it, though none confessed the fact. Mr.
and Mrs. March looked at one another with an anxious expression, as
their eyes followed Meg. Jo had sudden fits of sobriety, and was seen to
shake her fist at Mr. Brooke\textquotesingle s umbrella, which had been
left in the hall; Meg was absent-minded, shy, and silent, started when
the bell rang, and colored when John\textquotesingle s name was
mentioned; Amy said "Every one seemed waiting for something, and
couldn\textquotesingle t settle down, which was queer, since father was
safe at home," and Beth innocently wondered why their neighbors
didn\textquotesingle t run over as usual.

Laurie went by in the afternoon, and, seeing Meg at the window, seemed
suddenly possessed with a melodramatic fit, for he fell down upon one
knee in the snow, beat his breast, tore his hair, and clasped his hands
imploringly, as if begging some boon; and when Meg told him to behave
himself and go away, he wrung imaginary tears out of his handkerchief,
and staggered round the corner as if in utter despair.

"What does the goose mean?" said Meg, laughing, and trying to look
unconscious.

"He\textquotesingle s showing you how your John will go on by and by.
Touching, isn\textquotesingle t it?" answered Jo scornfully.

"Don\textquotesingle t say \emph{my John}, it isn\textquotesingle t
proper or true;" but Meg\textquotesingle s voice lingered over the words
as if they sounded pleasant to her. "Please don\textquotesingle t plague
me, Jo; I\textquotesingle ve told you I don\textquotesingle t care
\emph{much} about him, and there isn\textquotesingle t to be anything
said, but we are all to be friendly, and go on as before."

"We can\textquotesingle t, for something \emph{has} been said, and
Laurie\textquotesingle s mischief has spoilt you for me. I see it, and
so does mother; you are not like your old self a bit, and seem ever so
far away from me. I don\textquotesingle t mean to plague you, and will
bear it like a man, but I do wish it was all settled. I hate to wait; so
if you mean ever to do it, make haste and have it over quickly," said Jo
pettishly.

"\emph{I} can\textquotesingle t say or do anything till he speaks, and
he won\textquotesingle t, because father said I was too young," began
Meg, bending over her work, with a queer little smile, which suggested
that she did not quite agree with her father on that point.

"If he did speak, you wouldn\textquotesingle t know what to say, but
would cry or blush, or let him have his own way, instead of giving a
good, decided, No."

"I\textquotesingle m not so silly and weak as you think. I know just
what I should say, for I\textquotesingle ve planned it all, so I
needn\textquotesingle t be taken unawares; there\textquotesingle s no
knowing what may happen, and I wished to be prepared."

Jo couldn\textquotesingle t help smiling at the important air which Meg
had unconsciously assumed, and which was as becoming as the pretty color
varying in her cheeks.

"Would you mind telling me what you\textquotesingle d say?" asked Jo
more respectfully.

"Not at all; you are sixteen now, quite old enough to be my confidant,
and my experience will be useful to you by and by, perhaps, in your own
affairs of this sort."

"Don\textquotesingle t mean to have any; it\textquotesingle s fun to
watch other people philander, but I should feel like a fool doing it
myself," said Jo, looking alarmed at the thought.

"I think not, if you liked any one very much, and he liked you." Meg
spoke as if to herself, and glanced out at the lane, where she had often
seen lovers walking together in the summer twilight.

"I thought you were going to tell your speech to that man," said Jo,
rudely shortening her sister\textquotesingle s little reverie.

"Oh, I should merely say, quite calmly and decidedly,
\textquotesingle Thank you, Mr. Brooke, you are very kind, but I agree
with father that I am too young to enter into any engagement at present;
so please say no more, but let us be friends as we
were.\textquotesingle"

"Hum! that\textquotesingle s stiff and cool enough. I
don\textquotesingle t believe you\textquotesingle ll ever say it, and I
know he won\textquotesingle t be satisfied if you do. If he goes on like
the rejected lovers in books, you\textquotesingle ll give in, rather
than hurt his feelings."

\ul{"No, I won\textquotesingle t!} I shall tell him I\textquotesingle ve
made up my mind, and shall walk out of the room with dignity."

Meg rose as she spoke, and was just going to rehearse the dignified
exit, when a step in the hall made her fly into her seat, and begin to
sew as if her life depended on finishing that particular seam in a given
time. Jo smothered a laugh at the sudden change, and, when some one gave
a modest tap, opened the door with a grim aspect, which was anything but
hospitable.

"Good afternoon. I came to get my umbrella,---that is, to see how your
father finds himself to-day," said Mr. Brooke, getting a trifle confused
as his eye went from one tell-tale face to the other.

"It\textquotesingle s very well, he\textquotesingle s in the rack,
I\textquotesingle ll get him, and tell it you are here," and having
jumbled her father and the umbrella well together in her reply, Jo
slipped out of the room to give Meg a chance to make her speech and air
her dignity. But the instant she vanished, Meg began to sidle towards
the door, murmuring,---

"Mother will like to see you. Pray sit down, I\textquotesingle ll call
her."

"Don\textquotesingle t go; are you afraid of me, Margaret?" and Mr.
Brooke looked so hurt that Meg thought she must have done something very
rude. She blushed up to the little curls on her forehead, for he had
never called her Margaret before, and she was surprised to find how
natural and sweet it seemed to hear him say it. Anxious to appear
friendly and at her ease, she put out her hand with a confiding gesture,
and said gratefully,---

"How can I be afraid when you have been so kind to father? I only wish I
could thank you for it."

\protect\phantomsection\label{6672479776654687619_37106-h-4.htm.xhtml_b116.png}{}
\pandocbounded{\includegraphics[keepaspectratio]{303483661336987339_b116.png}}

"Shall I tell you how?" asked Mr. Brooke, holding the small hand fast in
both his own, and looking down at Meg with so much love in the brown
eyes, that her heart began to flutter, and she both longed to run away
and to stop and listen.

"Oh no, please don\textquotesingle t---I\textquotesingle d rather not,"
she said, trying to withdraw her hand, and looking frightened in spite
of her denial.

"I won\textquotesingle t trouble you, I only want to know if you care
for me a little, Meg. I love you so much, dear," added Mr. Brooke
tenderly.

This was the moment for the calm, proper speech, but Meg
didn\textquotesingle t make it; she forgot every word of it, hung her
head, and answered, "I don\textquotesingle t know," so softly, that John
had to stoop down to catch the foolish little reply.

He seemed to think it was worth the trouble, for he smiled to himself as
if quite satisfied, pressed the plump hand gratefully, and said, in his
most persuasive tone, "Will you try and find out? I want to know
\emph{so} much; for I can\textquotesingle t go to work with any heart
until I learn whether I am to have my reward in the end or not."

"I\textquotesingle m too young," faltered Meg, wondering why she was so
fluttered, yet rather enjoying it.

"I\textquotesingle ll wait; and in the meantime, you could be learning
to like me. Would it be a very hard lesson, dear?"

"Not if I chose to learn it, but---"

"Please choose to learn, Meg. I love to teach, and this is easier than
German," broke in John, getting possession of the other hand, so that
she had no way of hiding her face, as he bent to look into it.

His tone was properly beseeching; but, stealing a shy look at him, Meg
saw that his eyes were merry as well as tender, and that he wore the
satisfied smile of one who had no doubt of his success. This nettled
her; Annie Moffat\textquotesingle s foolish lessons in coquetry came
into her mind, and the love of power, which sleeps in the bosoms of the
best of little women, woke up all of a sudden and took possession of
her. She felt excited and strange, and, not knowing what else to do,
followed a capricious impulse, and, withdrawing her hands, said
petulantly, "I \emph{don\textquotesingle t} choose. Please go away and
let me be!"

Poor Mr. Brooke looked as if his lovely castle in the air was tumbling
about his ears, for he had never seen Meg in such a mood before, and it
rather bewildered him.

"Do you really mean that?" he asked anxiously, following her as she
walked away.

"Yes, I do; I don\textquotesingle t want to be worried about such
things. Father says I needn\textquotesingle t; it\textquotesingle s too
soon and I\textquotesingle d rather not."

"Mayn\textquotesingle t I hope you\textquotesingle ll change your mind
by and by? I\textquotesingle ll wait, and say nothing till you have had
more time. Don\textquotesingle t play with me, Meg. I
didn\textquotesingle t think that of you."

"Don\textquotesingle t think of me at all. I\textquotesingle d rather
you wouldn\textquotesingle t," said Meg, taking a naughty satisfaction
in trying her lover\textquotesingle s patience and her own power.

He was grave and pale now, and looked decidedly more like the novel
heroes whom she admired; but he neither slapped his forehead nor tramped
about the room, as they did; he just stood looking at her so wistfully,
so tenderly, that she found her heart relenting in spite of her. What
would have happened next I cannot say, if Aunt March had not come
hobbling in at this interesting minute.

The old lady couldn\textquotesingle t resist her longing to see her
nephew; for she had met Laurie as she took her airing, and, hearing of
Mr. March\textquotesingle s arrival, drove straight out to see him. The
family were all busy in the back part of the house, and she had made her
way quietly in, hoping to surprise them. She did surprise two of them so
much that Meg started as if she had seen a ghost, and Mr. Brooke
vanished into the study.

\protect\phantomsection\label{6672479776654687619_37106-h-4.htm.xhtml_b117.png}{}
\pandocbounded{\includegraphics[keepaspectratio]{303483661336987339_b117.png}}

"Bless me, what\textquotesingle s all this?" cried the old lady, with a
rap of her cane, as she glanced from the pale young gentleman to the
scarlet young lady.

"It\textquotesingle s father\textquotesingle s friend.
I\textquotesingle m \emph{so} surprised to see you!" stammered Meg,
feeling that she was in for a lecture now.

"That\textquotesingle s evident," returned Aunt March, sitting down.
"But what is father\textquotesingle s friend saying to make you look
like a peony? There\textquotesingle s mischief going on, and I insist
upon knowing what it is," with another rap.

"We were merely talking. Mr. Brooke came for his umbrella," began Meg,
wishing that Mr. Brooke and the umbrella were safely out of the house.

"Brooke? That boy\textquotesingle s tutor? Ah! I understand now. I know
all about it. Jo blundered into a wrong message in one of your
father\textquotesingle s letters, and I made her tell me. You
haven\textquotesingle t gone and accepted him, child?" cried Aunt March,
looking scandalized.

"Hush! he\textquotesingle ll hear.
Sha\textquotesingle n\textquotesingle t I call mother?" said Meg, much
troubled.

"Not yet. I\textquotesingle ve something to say to you, and I must free
my mind at once. Tell me, do you mean to marry this Cook? If you do, not
one penny of my money ever goes to you. Remember that, and be a sensible
girl," said the old lady impressively.

Now Aunt March possessed in perfection the art of rousing the spirit of
opposition in the gentlest people, and enjoyed doing it. The best of us
have a spice of perversity in us, especially when we are young and in
love. If Aunt March had begged Meg to accept John Brooke, she would
probably have declared she couldn\textquotesingle t think of it; but as
she was peremptorily ordered \emph{not} to like him, she immediately
made up her mind that she would. Inclination as well as perversity made
the decision easy, and, being already much excited, Meg opposed the old
lady with unusual spirit.

"I shall marry whom I please, Aunt March, and you can leave your money
to any one you like," she said, nodding her head with a resolute air.

"Highty tighty! Is that the way you take my advice, miss?
You\textquotesingle ll be sorry for it, by and by, when
you\textquotesingle ve tried love in a cottage, and found it a failure."

"It can\textquotesingle t be a worse one than some people find in big
houses," retorted Meg.

Aunt March put on her glasses and took a look at the girl, for she did
not know her in this new mood. Meg hardly knew herself, she felt so
brave and independent,---so glad to defend John, and assert her right to
love him, if she liked. Aunt March saw that she had begun wrong, and,
after a little pause, made a fresh start, saying, as mildly as she
could, "Now, Meg, my dear, be reasonable, and take my advice. I mean it
kindly, and don\textquotesingle t want you to spoil your whole life by
making a mistake at the beginning. You ought to marry well, and help
your family; it\textquotesingle s your duty to make a rich match, and it
ought to be impressed upon you."

"Father and mother don\textquotesingle t think so; they like John,
though he \emph{is} poor."

"Your parents, my dear, have no more worldly wisdom than two babies."

"I\textquotesingle m glad of it," cried Meg stoutly.

Aunt March took no notice, but went on with her lecture. "This Rook is
poor, and hasn\textquotesingle t got any rich relations, has he?"

"No; but he has many warm friends."

"You can\textquotesingle t live on friends; try it, and see how cool
they\textquotesingle ll grow. He hasn\textquotesingle t any business,
has he?"

"Not yet; Mr. Laurence is going to help him."

"That won\textquotesingle t last long. James Laurence is a crotchety old
fellow, and not to be depended on. So you intend to marry a man without
money, position, or business, and go on working harder than you do now,
when you might be comfortable all your days by minding me and doing
better? I thought you had more sense, Meg."

"I couldn\textquotesingle t do better if I waited half my life! John is
good and wise; he\textquotesingle s got heaps of talent;
he\textquotesingle s willing to work, and sure to get on,
he\textquotesingle s so energetic and brave. Every one likes and
respects him, and I\textquotesingle m proud to think he cares for me,
though I\textquotesingle m so poor and young and silly," said Meg,
looking prettier than ever in her earnestness.

"He knows \emph{you} have got rich relations, child;
that\textquotesingle s the secret of his liking, I suspect."

"Aunt March, how dare you say such a thing? John is above such meanness,
and I won\textquotesingle t listen to you a minute if you talk so,"
cried Meg indignantly, forgetting everything but the injustice of the
old lady\textquotesingle s suspicions. "My John wouldn\textquotesingle t
marry for money, anymore than I would. We are willing to work, and we
mean to wait. I\textquotesingle m not afraid of being poor, for
I\textquotesingle ve been happy so far, and I know I shall be with him,
because he loves me, and I---"

Meg stopped there, remembering all of a sudden that she
hadn\textquotesingle t made up her mind; that she had told "her John" to
go away, and that he might be overhearing her inconsistent remarks.

Aunt March was very angry, for she had set her heart on having her
pretty niece make a fine match, and something in the
girl\textquotesingle s happy young face made the lonely old woman feel
both sad and sour.

"Well, I wash my hands of the whole affair! You are a wilful child, and
you\textquotesingle ve lost more than you know by this piece of folly.
No, I won\textquotesingle t stop; I\textquotesingle m disappointed in
you, and haven\textquotesingle t spirits to see your father now.
Don\textquotesingle t expect anything from me when you are married; your
Mr. Book\textquotesingle s friends must take care of you.
I\textquotesingle m done with you forever."

And, slamming the door in Meg\textquotesingle s face, Aunt March drove
off in high dudgeon. She seemed to take all the girl\textquotesingle s
courage with her; for, when left alone, Meg stood a moment, undecided
whether to laugh or cry. Before she could make up her mind, she was
taken possession of by Mr. Brooke, who said, all in one breath, "I
couldn\textquotesingle t help hearing, Meg. Thank you for defending me,
and Aunt March for proving that you \emph{do} care for me a little bit."

"I didn\textquotesingle t know how much, till she abused you," began
Meg.

"And I needn\textquotesingle t go away, but may stay and be happy, may
I, dear?"

Here was another fine chance to make the crushing speech and the stately
exit, but Meg never thought of doing either, and disgraced herself
forever in Jo\textquotesingle s eyes by meekly whispering, "Yes, John,"
and hiding her face on Mr. Brooke\textquotesingle s waistcoat.

Fifteen minutes after Aunt March\textquotesingle s departure, Jo came
softly down stairs, paused an instant at the parlor door, and, hearing
no sound within, nodded and smiled, with a satisfied expression, saying
to herself, "She has sent him away as we planned, and that affair is
settled. I\textquotesingle ll go and hear the fun, and have a good laugh
over it."

But poor Jo never got her laugh, for she was transfixed upon the
threshold by a spectacle which held her there, staring with her mouth
nearly as wide open as her eyes. Going in to exult over a fallen enemy,
and to praise a strong-minded sister for the banishment of an
objectionable lover, it certainly \emph{was} a shock to behold the
aforesaid enemy serenely sitting on the sofa, with the strong-minded
sister enthroned upon his knee, and wearing an expression of the most
abject submission. Jo gave a sort of gasp, as if a cold shower-bath had
suddenly fallen upon her,---for such an unexpected turning of the tables
actually took her breath away. At the odd sound, the lovers turned and
saw her. Meg jumped up, looking both proud and shy; but "that man," as
Jo called him, actually laughed, and said coolly, as he kissed the
astonished new-comer, "Sister Jo, congratulate us!"

That was adding insult to injury,---it was altogether too much,---and,
making some wild demonstration with her hands, Jo vanished without a
word. Rushing upstairs, she startled the invalids by exclaiming
tragically, as she burst into the room, "Oh, \emph{do} somebody go down
quick; \ul{John Brooke is acting dreadfully,} and Meg likes it!"

Mr. and Mrs. March left the room with speed; and, casting herself upon
the bed, Jo cried and scolded tempestuously as she told the awful news
to Beth and Amy. The little girls, however, considered it a most
agreeable and interesting event, and Jo got little comfort from them; so
she went up to her refuge in the garret, and confided her troubles to
the rats.

Nobody ever knew what went on in the parlor that afternoon; but a great
deal of talking was done, and quiet Mr. Brooke astonished his friends by
the eloquence and spirit with which he pleaded his suit, told his plans,
and persuaded them to arrange everything just as he wanted it.

The tea-bell rang before he had finished describing the paradise which
he meant to earn for Meg, and he proudly took her in to supper, both
looking so happy that Jo hadn\textquotesingle t the heart to be jealous
or dismal. Amy was very much impressed by John\textquotesingle s
devotion and Meg\textquotesingle s dignity. Beth beamed at them from a
distance, while Mr. and Mrs. March surveyed the young couple with such
tender satisfaction that it was perfectly evident Aunt March was right
in calling them as "unworldly as a pair of babies." No one ate much, but
every one looked very happy, and the old room seemed to brighten up
amazingly when the first romance of the family began there.

"You can\textquotesingle t say nothing pleasant ever happens now, can
you, Meg?" said Amy, trying to decide how she would group the lovers in
the sketch she was planning to take.

"No, I\textquotesingle m sure I can\textquotesingle t. How much has
happened since I said that! It seems a year ago," answered Meg, who was
in a blissful dream, lifted far above such common things as bread and
butter.

"The joys come close upon the sorrows this time, and I rather think the
changes have begun," said Mrs. March. "In most families there comes, now
and then, a year full of events; this has been such an one, but it ends
well, after all."

"Hope the next will end better," muttered Jo, who found it very hard to
see Meg absorbed in a stranger before her face; for Jo loved a few
persons very dearly, and dreaded to have their affection lost or
lessened in any way.

"I hope the third year from this \emph{will} end better; I mean it
shall, if I live to work out my plans," said Mr. Brooke, smiling at Meg,
as if everything had become possible to him now.

"Doesn\textquotesingle t it seem very long to wait?" asked Amy, who was
in a hurry for the wedding.

"I\textquotesingle ve got so much to learn before I shall be ready, it
seems a short time to me," answered Meg, with a sweet gravity in her
face, never seen there before.

"You have only to wait; \emph{I} am to do the work," said John,
beginning his labors by picking up Meg\textquotesingle s napkin, with an
expression which caused Jo to shake her head, and then say to herself,
with an air of relief, as the front door banged, "Here comes Laurie. Now
we shall have a little sensible conversation."

\protect\phantomsection\label{6672479776654687619_37106-h-4.htm.xhtml_b118.png}{}
\pandocbounded{\includegraphics[keepaspectratio]{303483661336987339_b118.png}}

But Jo was mistaken; for Laurie came prancing in, overflowing with
spirits, bearing a great bridal-looking bouquet for "Mrs. John Brooke,"
and evidently laboring under the delusion that the whole affair had been
brought about by his excellent management.

"I knew Brooke would have it all his own way, he always does; for when
he makes up his mind to accomplish anything, it\textquotesingle s done,
though the sky falls," said Laurie, when he had presented his offering
and his congratulations.

"Much obliged for that recommendation. I take it as a good omen for the
future, and invite you to my wedding on the spot," answered Mr. Brooke,
who felt at \ul{peace with all mankind,} even his mischievous pupil.

"I\textquotesingle ll come if I\textquotesingle m at the ends of the
earth; for the sight of Jo\textquotesingle s face alone, on that
occasion, would be worth a long journey. You don\textquotesingle t look
festive, ma\textquotesingle am; what\textquotesingle s the matter?"
asked Laurie, following her into a corner of the parlor, whither all had
adjourned to greet Mr. Laurence.

"I don\textquotesingle t approve of the match, but I\textquotesingle ve
made up my mind to bear it, and shall not say a word against it," said
Jo solemnly. "You can\textquotesingle t know how hard it is for me to
give up Meg," she continued, with a little quiver in her voice.

"You don\textquotesingle t give her up. You only go halves," said Laurie
consolingly.

"It never can be the same again. I\textquotesingle ve lost my dearest
friend," sighed Jo.

"You\textquotesingle ve got me, anyhow. I\textquotesingle m not good for
much, I know; but I\textquotesingle ll stand by you, Jo, all the days of
my life; upon my word I will!" and Laurie meant what he said.

"I know you will, and I\textquotesingle m ever so much obliged; you are
always a great comfort to me, Teddy," returned Jo, gratefully shaking
hands.

"Well, now, don\textquotesingle t be dismal, there\textquotesingle s a
good fellow. It\textquotesingle s all right, you see. Meg is happy;
Brooke will fly round and get settled immediately; grandpa will attend
to him, and it will be very jolly to see Meg in her own little house.
We\textquotesingle ll have capital times after she is gone, for I shall
be through college before long, and then we\textquotesingle ll go
abroad, or some nice trip or other. Wouldn\textquotesingle t that
console you?"

"I rather think it would; but there\textquotesingle s no knowing what
may happen in three years," said Jo thoughtfully.

"That\textquotesingle s true. Don\textquotesingle t you wish you could
take a look forward, and see where we shall all be then? I do," returned
Laurie.

"I think not, for I might see something sad; and every one looks so
happy now, I don\textquotesingle t believe they could be much improved,"
and Jo\textquotesingle s eyes went slowly round the room, brightening as
they looked, for the prospect was a pleasant one.

Father and mother sat together, quietly re-living the first chapter of
the romance which for them began some twenty years ago. Amy was drawing
the lovers, who sat apart in a beautiful world of their own, the light
of which touched their faces with a grace the little artist could not
copy. Beth lay on her sofa, talking cheerily with her old friend, who
held her little hand as if he felt that it possessed the power to lead
him along the peaceful way she walked. Jo lounged in her favorite low
seat, with the grave, quiet look which best became her; and Laurie,
leaning on the back of her chair, his chin on a level with her curly
head, smiled with his friendliest aspect, and nodded at her in the long
glass which reflected them both.

\begin{center}\rule{0.5\linewidth}{0.5pt}\end{center}

So grouped, the curtain falls upon Meg, Jo, Beth, and Amy. Whether it
ever rises again, depends upon the reception given to the first act of
the domestic drama called "{Little Women}."

\begin{center}\rule{0.5\linewidth}{0.5pt}\end{center}

\protect\phantomsection\label{6672479776654687619_37106-h-4.htm.xhtml_b118a.png}{}
\pandocbounded{\includegraphics[keepaspectratio]{303483661336987339_b118a.png}}\\
\protect\phantomsection\label{6672479776654687619_37106-h-4.htm.xhtml_ebm_caption3}{Home
of the Little Women}

\begin{center}\rule{0.5\linewidth}{0.5pt}\end{center}

\subsection{XXIV.
Gossip.}\label{6672479776654687619_37106-h-4.htm.xhtml_pgepubid00026}

The Second Part

\begin{center}\rule{0.5\linewidth}{0.5pt}\end{center}

\protect\phantomsection\label{6672479776654687619_37106-h-4.htm.xhtml_b119.png}{}
\pandocbounded{\includegraphics[keepaspectratio]{303483661336987339_b119.png}}\\
\protect\phantomsection\label{6672479776654687619_37106-h-4.htm.xhtml_ebm_caption4}{The
Dove-Cote}

\protect\phantomsection\label{6672479776654687619_37106-h-4.htm.xhtml_XXIV}{}\hyperref[6672479776654687619_37106-h-0.htm.xhtml_contents2]{XXIV.}

GOSSIP.

{In} order that we may start afresh, and go to Meg\textquotesingle s
wedding with free minds, it will be well to begin with a little gossip
about the Marches. And here let me premise, that if any of the elders
think there is too much "lovering" in the story, as I fear they may
(I\textquotesingle m not afraid the young folks will make that
objection), I can only say with Mrs. March, "What \emph{can} you expect
when I have four gay girls in the house, and a dashing young neighbor
over the way?"

The three years that have passed have brought but few changes to the
quiet family. The war is over, and Mr. March safely at home, busy with
his books and the small parish which found in him a minister by nature
as by grace,---a quiet, studious man, rich in the wisdom that is better
than learning, the charity which calls all mankind "brother," the piety
that blossoms into character, making it august and lovely.

These attributes, in spite of poverty and the strict integrity which
shut him out from the more worldly successes, attracted to him many
admirable persons, as naturally as sweet herbs draw bees, and as
naturally he gave them the honey into which fifty years of hard
experience had distilled no bitter drop. Earnest young men found the
gray-headed scholar as young at heart as they; thoughtful or troubled
women instinctively brought their doubts and sorrows to him, sure of
finding the gentlest sympathy, the wisest counsel; sinners told their
sins to the pure-hearted old man, and were both rebuked and saved;
gifted men found a companion in him; ambitious men caught glimpses of
nobler ambitions than their own; and even worldlings confessed that his
beliefs were beautiful and true, although "they wouldn\textquotesingle t
pay."

To outsiders, the five energetic women seemed to rule the house, and so
they did in many things; but the quiet scholar, sitting among his books,
was still the head of the family, the household conscience, anchor, and
comforter; for to him the busy, anxious women always turned in troublous
times, finding him, in the truest sense of those sacred words, husband
and father.

The girls gave their hearts into their mother\textquotesingle s keeping,
their souls into their father\textquotesingle s; and to both parents,
who lived and labored so faithfully for them, they gave a love that grew
with their growth, and bound them tenderly together by the sweetest tie
which blesses life and outlives death.

Mrs. March is as brisk and cheery, though rather grayer, than when we
saw her last, and just now so absorbed in Meg\textquotesingle s affairs
that the hospitals and homes, still full of wounded "boys" and
soldiers\textquotesingle{} widows, decidedly miss the motherly
missionary\textquotesingle s visits.

John Brooke did his duty manfully for a year, got wounded, was sent
home, and not allowed to return. He received no stars or bars, but he
deserved them, for he cheerfully risked all he had; and life and love
are very precious when both are in full bloom. Perfectly resigned to his
discharge, he devoted himself to getting well, preparing for business,
and earning a home for Meg. With the good sense and sturdy independence
that characterized him, he refused Mr. Laurence\textquotesingle s more
generous offers, and accepted the place of book-keeper feeling better
satisfied to begin with \ul{an honestly-earned salary, than by running}
any risks with borrowed money.

Meg had spent the time in working as well as waiting, growing womanly in
character, wise in housewifely arts, and prettier than ever; for love is
a great beautifier. She had her girlish ambitions and hopes, and felt
some disappointment at the humble way in which the new life must begin.
Ned Moffat had just married Sallie Gardiner, and Meg
couldn\textquotesingle t help contrasting their fine house and carriage,
many gifts, and splendid outfit, with her own, and secretly wishing she
could have the same. But somehow envy and discontent soon vanished when
she thought of all the patient love and labor John had put into the
little home awaiting her; and when they sat together in the twilight,
talking over their small plans, the future always grew so beautiful and
bright that she forgot Sallie\textquotesingle s splendor, and felt
herself the richest, happiest girl in Christendom.

Jo never went back to Aunt March, for the old lady took such a fancy to
Amy that she bribed her with the offer of drawing lessons from one of
the best teachers going; and for the sake of this advantage, Amy would
have served a far harder mistress. So she gave her mornings to duty, her
afternoons to pleasure, and prospered finely. Jo, meantime, devoted
herself to literature and Beth, who remained delicate long after the
fever was a thing of the past. Not an invalid exactly, but never again
the rosy, healthy creature she had been; yet always hopeful, happy, and
serene, busy with the quiet duties she loved, every
one\textquotesingle s friend, and an angel in the house, long before
those who loved her most had learned to know it.

As long as "The Spread Eagle" paid her a dollar a column for her
"rubbish," as she called it, Jo felt herself a woman of means, and spun
her little romances diligently. But great plans fermented in her busy
brain and ambitious mind, and the old tin kitchen in the garret held a
slowly increasing pile of blotted manuscript, which was one day to place
the name of March upon the roll of fame.

Laurie, having dutifully gone to college to please his grandfather, was
now getting through it in the easiest possible manner to please himself.
A universal favorite, thanks to money, manners, much talent, and the
kindest heart that ever got its owner into scrapes by trying to get
other people out of them, he stood in great danger of being spoilt, and
probably would have been, like many another promising boy, if he had not
possessed a talisman against evil in the memory of the kind old man who
was bound up in his success, the motherly friend who watched over him as
if he were her son, and last, but not least by any means, the knowledge
that four innocent girls loved, admired, and believed in him with all
their hearts.

Being only "a glorious human boy," of course he frolicked and flirted,
grew dandified, aquatic, sentimental, or gymnastic, as college fashions
ordained; hazed and was hazed, talked slang, and more than once came
perilously near suspension and expulsion. But as high spirits and the
love of fun were the causes of these pranks, he always managed to save
himself by frank confession, honorable atonement, or the irresistible
power of persuasion which he possessed in perfection. In fact, he rather
prided himself on his narrow escapes, and liked to thrill the girls with
graphic accounts of his triumphs over wrathful tutors, dignified
professors, and vanquished enemies. The "men of my class" were heroes in
the eyes of the girls, who never wearied of the exploits of "our
fellows," and were frequently allowed to bask in the smiles of these
great creatures, when Laurie brought them home with him.

Amy especially enjoyed this high honor, and became quite a belle among
them; for her ladyship early felt and learned to use the gift of
fascination with which she was endowed. Meg was too much absorbed in her
private and particular John to care for any other lords of creation, and
Beth too shy to do more than peep at them, and wonder how Amy dared to
order them about so; but Jo felt quite in her element, and found it very
difficult to refrain from imitating the gentlemanly attitudes, phrases,
and feats, which seemed more natural to her than the decorums prescribed
for young ladies. They all liked Jo immensely, but never fell in love
with her, though very few escaped without paying the tribute of a
sentimental sigh or two at Amy\textquotesingle s shrine. And speaking of
sentiment brings us very naturally to the "Dove-cote."

That was the name of the little brown house which Mr. Brooke had
prepared for Meg\textquotesingle s first home. Laurie had christened it,
saying it was highly appropriate to the gentle lovers, who "went on
together like a pair of turtle-doves, with first a bill and then a coo."
It was a tiny house, with a little garden behind, and a lawn about as
big as a pocket-handkerchief in front. Here Meg meant to have a
fountain, shrubbery, and a profusion of lovely flowers; though just at
present, the fountain was represented by a weather-beaten urn, very like
a dilapidated slop-bowl; the shrubbery consisted of several young
larches, undecided whether to live or die; and the profusion of flowers
was merely hinted by regiments of sticks, to show where seeds were
planted. But inside, it was altogether charming, and the happy bride saw
no fault from garret to cellar. To be sure, the hall was so narrow, it
was fortunate that they had no piano, for one never could have been got
in whole; the dining-room was so small that six people were a tight fit;
and the kitchen stairs seemed built for the express purpose of
precipitating both servants and china pell-mell into the coal-bin. But
once get used to these slight blemishes, and nothing could be more
complete, for good sense and good taste had presided over the
furnishing, and the result was highly satisfactory. There were no
marble-topped tables, long mirrors, or lace curtains in the little
parlor, but simple furniture, plenty of books, a fine picture or two, a
stand of flowers in the bay-window, and, scattered all about, the pretty
gifts which came from friendly hands, and were the fairer for the loving
messages they brought.

I don\textquotesingle t think the Parian Psyche Laurie gave lost any of
its beauty because John put up the bracket it stood upon; that any
upholsterer could have draped the plain muslin curtains more gracefully
than Amy\textquotesingle s artistic hand; or that any store-room was
ever better provided with good wishes, merry words, and happy hopes,
than that in which Jo and her mother put away Meg\textquotesingle s few
boxes, barrels, and bundles; and I am morally certain that the
spandy-new kitchen never \emph{could} have looked so cosey and neat if
Hannah had not arranged every pot and pan a dozen times over, and laid
the fire all ready for lighting, the minute "Mis. Brooke came home." I
also doubt if any young matron ever began life with so rich a supply of
dusters, holders, and piece-bags; for Beth made enough to last till the
silver wedding came round, and invented three different kinds of
dishcloths for the express service of the bridal china.

People who hire all these things done for them never know what they
lose; for the homeliest tasks get beautified if loving hands do them,
and Meg found so many proofs of this, that everything in her small nest,
from the kitchen roller to the silver vase on her parlor table, was
eloquent of home love and tender forethought.

What happy times they had planning together, what solemn shopping
excursions; what funny mistakes they made, and what shouts of laughter
arose over Laurie\textquotesingle s ridiculous bargains. In his love of
jokes, this young gentleman, though nearly through college, was as much
of a boy as ever. His last whim had been to bring with him, on his
weekly visits, some new, useful, and ingenious article for the young
housekeeper. Now a bag of remarkable clothes-pins; next, a wonderful
nutmeg-grater, which fell to pieces at the first trial; a knife-cleaner
that spoilt all the knives; or a sweeper that picked the nap neatly off
the carpet, and left the dirt; labor-saving soap that took the skin off
one\textquotesingle s hands; infallible cements which stuck firmly to
nothing but the fingers of the deluded buyer; and every kind of
tin-ware, from a toy savings-bank for odd pennies, to a wonderful boiler
which would wash articles in its own steam, with every prospect of
exploding in the process.

In vain Meg begged him to stop. John laughed at him, and Jo called him
"Mr. Toodles." He was possessed with a mania for patronizing Yankee
ingenuity, and seeing his friends fitly furnished forth. So each week
beheld some fresh absurdity.

Everything was done at last, even to Amy\textquotesingle s arranging
different colored soaps to match the different colored rooms, and
Beth\textquotesingle s setting the table for the first meal.

"Are you satisfied? Does it seem like home, and do you feel as if you
should be happy here?" asked Mrs. March, as she and her daughter went
through the new kingdom, arm-in-arm; for just then they seemed to cling
together more tenderly than ever.

"Yes, mother, perfectly satisfied, thanks to you all, and \emph{so}
happy that I can\textquotesingle t talk about it," answered Meg, with a
look that was better than words.

"If she only had a servant or two it would be all right," said Amy,
coming out of the parlor, where she had been trying to decide whether
the bronze Mercury looked best on the whatnot or the mantle-piece.

"Mother and I have talked that over, and I have made up my mind to try
her way first. There will be so little to do, that, with Lotty to run my
errands and help me here and there, I shall only have enough work to
keep me from getting lazy or homesick," answered Meg tranquilly.

"Sallie Moffat has four," began Amy.

"If Meg had four the house wouldn\textquotesingle t hold them, and
master and missis would have to camp in the garden," broke in Jo, who,
enveloped in a big blue pinafore, was giving the last polish to the
door-handles.

"Sallie isn\textquotesingle t a poor man\textquotesingle s wife, and
many maids are in keeping with her fine establishment. Meg and John
begin humbly, but I have a feeling that there will be quite as much
happiness in the little house as in the big one. It\textquotesingle s a
great mistake for young girls like Meg to leave themselves nothing to do
but dress, give orders, and gossip. When I was first married, I used to
long for my new clothes to wear out or get torn, so that I might have
the pleasure of mending them; for I got heartily sick of doing fancy
work and tending my pocket handkerchief."

"Why didn\textquotesingle t you go into the kitchen and make messes, as
Sallie says she does, to amuse herself, though they never turn out well,
and the servants laugh at her," said Meg.

"I did, after a while; not to \textquotesingle mess,\textquotesingle{}
but to learn of Hannah how things should be done, that my servants need
\emph{not} laugh at me. It was play then; but there came a time when I
was truly grateful that I not only possessed the will but the power to
cook wholesome food for my little girls, and help myself when I could no
longer afford to hire help. You begin at the other end, Meg, dear; but
the lessons you learn now will be of use to you by and by, when John is
a richer man, for the mistress of a house, however splendid, should know
how work ought to be done, if she wishes to be well and honestly
served."

"Yes, mother, I\textquotesingle m sure of that," said Meg, listening
respectfully to the little lecture; for the best of women will hold
forth upon the all-absorbing subject of housekeeping. "Do you know I
like this room most of all in my baby-house," added Meg, a minute after,
as they went upstairs, and she looked into her well-stored linen-closet.

Beth was there, laying the snowy piles smoothly on the shelves, and
exulting over the goodly array. All three laughed as Meg spoke; for that
linen-closet was a joke. You see, having said that if Meg married "that
Brooke" she shouldn\textquotesingle t have a cent of her money, Aunt
March was rather in a quandary, when time had appeased her wrath and
made her repent her vow. She never broke her word, and was much
exercised in her mind how to get round it, and at last devised a plan
whereby she could satisfy herself. Mrs. Carrol,
Florence\textquotesingle s mamma, was ordered to buy, have made, and
marked, a generous supply of house and table linen, and send it as
\emph{her} present, all of which was faithfully done; but the secret
leaked out, and was greatly enjoyed by the family; for Aunt March tried
to look utterly unconscious, and insisted that she could give nothing
but the old-fashioned pearls, long promised to the first bride.

"That\textquotesingle s a housewifely taste which I am glad to see. I
had a young friend who set up housekeeping with six sheets, but she had
finger bowls for company, and that satisfied her," said Mrs. March,
patting the damask table-cloths, with a truly feminine appreciation of
their fineness.

"I haven\textquotesingle t a single finger-bowl, but this is a
\textquotesingle set out\textquotesingle{} that will last me all my
days, Hannah says;" and Meg looked quite contented, as well she might.

"Toodles is coming," cried Jo from below; and they all went down to meet
Laurie, whose weekly visit was an important event in their quiet lives.

A tall, broad-shouldered young fellow, with a cropped head, a felt-basin
of a hat, and a fly-away coat, came tramping down the road at a great
pace, walked over the low fence without stopping to open the gate,
straight up to Mrs. March, with both hands out, and a hearty---

"Here I am, mother! Yes, it\textquotesingle s all right."

The last words were in answer to the look the elder lady gave him; a
kindly questioning look, which the handsome eyes met so frankly that the
little ceremony closed, as usual, with a motherly kiss.

"For Mrs. John Brooke, with the maker\textquotesingle s congratulations
and compliments. Bless you, Beth! What a refreshing spectacle you are,
Jo. Amy, you are getting altogether too handsome for a single lady."

As Laurie spoke, he delivered a brown paper parcel to Meg, pulled
Beth\textquotesingle s hair-ribbon, stared at Jo\textquotesingle s big
pinafore, and fell into an attitude of mock rapture before Amy, then
shook hands all round, and every one began to talk.

"Where is John?" asked Meg anxiously.

"Stopped to get the license for to-morrow, ma\textquotesingle am."

"Which side won the last match, Teddy?" inquired Jo, who persisted in
feeling an interest in manly sports, despite her nineteen years.

"Ours, of course. Wish you\textquotesingle d been there to see."

"How is the lovely Miss Randal?" asked Amy, with a significant smile.

"More cruel than ever; don\textquotesingle t you see how
I\textquotesingle m pining away?" and Laurie gave his broad chest a
sounding slap and heaved a melodramatic sigh.

"What\textquotesingle s the last joke? Undo the bundle and see, Meg,"
said Beth, eying the knobby parcel with curiosity.

"It\textquotesingle s a useful thing to have in the house in case of
fire or thieves," observed Laurie, as a watchman\textquotesingle s
rattle appeared, amid the laughter of the girls.

\protect\phantomsection\label{6672479776654687619_37106-h-4.htm.xhtml_b120.png}{}
\pandocbounded{\includegraphics[keepaspectratio]{303483661336987339_b120.png}}

"Any time when John is away, and you get frightened, Mrs. Meg, just
swing that out of the front window, and it will rouse the neighborhood
in a jiffy. Nice thing, isn\textquotesingle t it?" and Laurie gave them
a sample of its powers that made them cover up their ears.

"There\textquotesingle s gratitude for you! and speaking of gratitude
reminds me to mention that you may thank Hannah for saving your
wedding-cake from destruction. I saw it going into your house as I came
by, and if she hadn\textquotesingle t defended it manfully
I\textquotesingle d have had a pick at it, for it looked like a
remarkably plummy one."

"I wonder if you will ever grow up, Laurie," said Meg, in a matronly
tone.

"I\textquotesingle m doing my best, ma\textquotesingle am, but
can\textquotesingle t get much higher, I\textquotesingle m afraid, as
six feet is about all men can do in these degenerate days," responded
the young gentleman, whose head was about level with the little
chandelier. "I suppose it would be profanation to eat anything in this
spick and span new bower, so, as I\textquotesingle m tremendously
hungry, I propose an adjournment," he added presently.

"Mother and I are going to wait for John. There are some last things to
settle," said Meg, bustling away.

"Beth and I are going over to Kitty Bryant\textquotesingle s to get more
flowers for to-morrow," added Amy, tying a picturesque hat over her
picturesque curls, and enjoying the effect as much as anybody.

"Come, Jo, don\textquotesingle t desert a fellow. I\textquotesingle m in
such a state of exhaustion I can\textquotesingle t get home without
help. Don\textquotesingle t take off your apron, whatever you do;
it\textquotesingle s peculiarly becoming," said Laurie, as Jo bestowed
his especial aversion in her capacious pocket, and offered him her arm
to support his feeble steps.

"Now, Teddy, I want to talk seriously to you about to-morrow," began Jo,
as they strolled away together. "You \emph{must} promise to behave well,
and not cut up any pranks, and spoil our plans."

"Not a prank."

"And don\textquotesingle t say funny things when we ought to be sober."

"I never do; you are the one for that."

"And I implore you not to look at me during the ceremony; I shall
certainly laugh if you do."

"You won\textquotesingle t see me; you\textquotesingle ll be crying so
hard that the thick fog round you will obscure the prospect."

"I never cry unless for some great affliction."

"Such as fellows going to college, hey?" cut in Laurie, with a
suggestive laugh.

"Don\textquotesingle t be a peacock. I only moaned a trifle to keep the
girls company."

"Exactly. I say, Jo, how is grandpa this week; pretty amiable?"

"Very; why, have you got into a scrape, and want to know how
he\textquotesingle ll take it?" asked Jo rather sharply.

"Now, Jo, do you think I\textquotesingle d look your mother in the face,
and say \textquotesingle All right,\textquotesingle{} if it
wasn\textquotesingle t?" and Laurie stopped short, with an injured air.

"No, I don\textquotesingle t."

"Then don\textquotesingle t go and be suspicious; I only want some
money," said Laurie, walking on again, appeased by her hearty tone.

"You spend a great deal, Teddy."

"Bless you, \emph{I} don\textquotesingle t spend it; it spends itself,
somehow, and is gone before I know it."

"You are so generous and kind-hearted that you let people borrow, and
can\textquotesingle t say \textquotesingle No\textquotesingle{} to any
one. We heard about Henshaw, and all you did for him. If you always
spent money in that way, no one would blame you," said Jo warmly.

"Oh, he made a mountain out of a mole-hill. You wouldn\textquotesingle t
have me let that fine fellow work himself to death, just for the want of
a little help, when he is worth a dozen of us lazy chaps, would you?"

"Of course not; but I don\textquotesingle t see the use of your having
seventeen waistcoats, endless neckties, and a new hat every time you
come home. I thought you\textquotesingle d got over the dandy period;
but every now and then it breaks out in a new spot. Just now
it\textquotesingle s the fashion to be hideous,---to make your head look
like a scrubbing-brush, wear a strait-jacket, orange gloves, and
clumping, square-toed boots. If it was cheap ugliness,
I\textquotesingle d say nothing; but it costs as much as the other, and
I don\textquotesingle t get any satisfaction out of it."

Laurie threw back his head, and laughed so heartily at this attack, that
the felt-basin fell off, and Jo walked on it, which insult only afforded
him an opportunity for expatiating on the advantages of a
rough-and-ready costume, as he folded up the maltreated hat, and stuffed
it into his pocket.

"Don\textquotesingle t lecture any more, there\textquotesingle s a good
soul! I have enough all through the week, and like to enjoy myself when
I come home. I\textquotesingle ll get myself up regardless of expense,
to-morrow, and be a satisfaction to my friends."

"I\textquotesingle ll leave you in peace if you\textquotesingle ll
\emph{only} let your hair grow. I\textquotesingle m not aristocratic,
but I do object to being seen with a person who looks like a young
prize-fighter," observed Jo severely.

"This unassuming style promotes study; that\textquotesingle s why we
adopt it," returned Laurie, who certainly could not be accused of
vanity, having voluntarily sacrificed a handsome curly crop to the
demand for quarter-of-an-inch-long stubble.

"By the way, Jo, I think that little Parker is really getting desperate
about Amy. He talks of her constantly, writes poetry, and moons about in
a most suspicious manner. He\textquotesingle d better nip his little
passion in the bud, hadn\textquotesingle t he?" added Laurie, in a
confidential, elder-brotherly tone, after a minute\textquotesingle s
silence.

"Of course he had; we don\textquotesingle t want any more marrying in
this family for years to come. Mercy on us, what \emph{are} the children
thinking of?" and Jo looked as much scandalized as if Amy and little
Parker were not yet in their teens.

"It\textquotesingle s a fast age, and I don\textquotesingle t know what
we are coming to, ma\textquotesingle am. You are a mere infant, but
you\textquotesingle ll go next, Jo, and we\textquotesingle ll be left
lamenting," said Laurie, shaking his head over the degeneracy of the
times.

"Don\textquotesingle t be alarmed; I\textquotesingle m not one of the
agreeable sort. Nobody will want me, and it\textquotesingle s a mercy,
for there should always be one old maid in a family."

"You won\textquotesingle t give any one a chance," said Laurie, with a
sidelong glance, and a little more color than before in his sunburnt
face. "You won\textquotesingle t show the soft side of your character;
and if a fellow gets a peep at it by accident, and can\textquotesingle t
help showing that he likes it, you treat him as Mrs. Gummidge did her
sweetheart,---throw cold water over him,---and get so thorny no one
dares touch or look at you."

"I don\textquotesingle t like that sort of thing; I\textquotesingle m
too busy to be worried with nonsense, and I think it\textquotesingle s
dreadful to break up families so. Now don\textquotesingle t say any more
about it; Meg\textquotesingle s wedding has turned all our heads, and we
talk of nothing but lovers and such absurdities. I don\textquotesingle t
wish to get cross, so let\textquotesingle s change the subject;" and Jo
looked quite ready to fling cold water on the slightest provocation.

Whatever his feelings might have been, Laurie found a vent for them in a
long low whistle, and the fearful prediction, as they parted at the
gate, "Mark my words, Jo, you\textquotesingle ll go next."

\protect\phantomsection\label{6672479776654687619_37106-h-4.htm.xhtml_b121.png}{}
\pandocbounded{\includegraphics[keepaspectratio]{303483661336987339_b121.png}}

\begin{center}\rule{0.5\linewidth}{0.5pt}\end{center}

\subsection{XXV. The First
Wedding.}\label{6672479776654687619_37106-h-4.htm.xhtml_pgepubid00027}

\protect\phantomsection\label{6672479776654687619_37106-h-4.htm.xhtml_b122.png}{}
\pandocbounded{\includegraphics[keepaspectratio]{303483661336987339_b122.png}}

\protect\phantomsection\label{6672479776654687619_37106-h-4.htm.xhtml_XXV}{}\hyperref[6672479776654687619_37106-h-0.htm.xhtml_contents2]{XXV.}

THE FIRST WEDDING.

{The} June roses over the porch were awake bright and early on that
morning, rejoicing with all their hearts in the cloudless sunshine, like
friendly little neighbors, as they were. Quite flushed with excitement
were their ruddy faces, as they swung in the wind, whispering to one
another what they had seen; for some peeped in at the dining-room
windows, where the feast was spread, some climbed up to nod and smile at
the sisters as they dressed the bride, others waved a welcome to those
who came and went on various errands in garden, porch, and hall, and
all, from the rosiest full-blown flower to the palest baby-bud, offered
their tribute of beauty and fragrance to the gentle mistress who had
loved and tended them so long.

Meg looked very like a rose herself; for all that was best and sweetest
in heart and soul seemed to bloom into her face that day, making it fair
and tender, with a charm more beautiful than beauty. Neither silk, lace,
nor orange-flowers would she have. "I don\textquotesingle t want to look
strange or fixed up to-day," she said. "I don\textquotesingle t want a
fashionable wedding, but only those about me whom I love, and to them I
wish to look and be my familiar self."

So she made her wedding gown herself, sewing into it the tender hopes
and innocent romances of a girlish heart. Her sisters braided up her
pretty hair, and the only ornaments she wore were the lilies of the
valley, which "her John" liked best of all the flowers that grew.

"You \emph{do} look just like our own dear Meg, only so very sweet and
lovely that I should hug you if it wouldn\textquotesingle t crumple your
dress," cried Amy, surveying her with delight, when all was done.

"Then I am satisfied. But please hug and kiss me, every one, and
don\textquotesingle t mind my dress; I want a great many crumples of
this sort put into it to-day;" and Meg opened her arms to her sisters,
who clung about her with April faces for a minute, feeling that the new
love had not changed the old.

"Now I\textquotesingle m going to tie John\textquotesingle s cravat for
him, and then to stay a few minutes with father quietly in the study;"
and Meg ran down to perform these little ceremonies, and then to follow
her mother wherever she went, conscious that, in spite of the smiles on
the motherly face, there was a secret sorrow hid in the motherly heart
at the flight of the first bird from the nest.

As the younger girls stand together, giving the last touches to their
simple toilet, it may be a good time to tell of a few changes which
three years have wrought in their appearance; for all are looking their
best just now.

Jo\textquotesingle s angles are much softened; she has learned to carry
herself with ease, if not grace. The curly crop has lengthened into a
thick coil, more becoming to the small head atop of the tall figure.
There is a fresh color in her brown cheeks, a soft shine in her eyes,
and only gentle words fall from her sharp tongue to-day.

Beth has grown slender, pale, and more quiet than ever; the beautiful,
kind eyes are larger, and in them lies an expression that saddens one,
although it is not sad itself. It is the shadow of pain which touches
the young face with such pathetic patience; but Beth seldom complains,
and always speaks hopefully of "being better soon."

Amy is with truth considered "the flower of the family;" for at sixteen
she has the air and bearing of a full-grown woman---not beautiful, but
possessed of that indescribable charm called grace. One saw it in the
lines of her figure, the make and motion of her hands, the flow of her
dress, the droop of her hair,---unconscious, yet harmonious, and as
attractive to many as beauty itself. Amy\textquotesingle s nose still
afflicted her, for it never \emph{would} grow Grecian; so did her mouth,
being too wide, and having a decided chin. These offending features gave
character to her whole face, but she never could see it, and consoled
herself with her wonderfully fair complexion, keen blue eyes, and curls,
more golden and abundant than ever.

All three wore suits of thin silver gray (their best gowns for the
summer), with blush-roses in hair and bosom; and all three looked just
what they were,---fresh-faced, happy-hearted girls, pausing a moment in
their busy lives to read with wistful eyes the sweetest chapter in the
romance of womanhood.

There were to be no ceremonious performances, everything was to be as
natural and homelike as possible; so when Aunt March arrived, she was
scandalized to see the bride come running to welcome and lead her in, to
find the bridegroom fastening up a garland that had fallen down, and to
catch a glimpse of the paternal minister marching upstairs with a grave
countenance, and a wine-bottle under each arm.

"Upon my word, here\textquotesingle s a state of things!" cried the old
lady, taking the seat of honor prepared for her, and settling the folds
of her lavender \emph{moire} with a great rustle. "You
oughtn\textquotesingle t to be seen till the last minute, child."

"I\textquotesingle m not a show, aunty, and no one is coming to stare at
me, to criticise my dress, or count the cost of my luncheon.
I\textquotesingle m too happy to care what any one says or thinks, and
I\textquotesingle m going to have my little wedding just as I like it.
John, dear, here\textquotesingle s your hammer;" and away went Meg to
help "that man" in his highly improper employment.

Mr. Brooke didn\textquotesingle t even say "Thank you," but as he
stooped for the unromantic tool, he kissed his little bride behind the
folding-door, with a look that made Aunt March whisk out her
pocket-handkerchief, with a sudden dew in her sharp old eyes.

A crash, a cry, and a laugh from Laurie, accompanied by the indecorous
exclamation, "Jupiter Ammon! Jo\textquotesingle s upset the cake again!"
caused a momentary flurry, which was hardly over when a flock of cousins
arrived, and "the party came in," as Beth used to say when a child.

"Don\textquotesingle t let that young giant come near me; he worries me
worse than mosquitoes," whispered the old lady to Amy, as the rooms
filled, and Laurie\textquotesingle s black head towered above the rest.

"He has promised to be very good to-day, and he \emph{can} be perfectly
elegant if he likes," returned Amy, gliding away to warn Hercules to
beware of the dragon, which warning caused him to haunt the old lady
with a devotion that nearly distracted her.

There was no bridal procession, but a sudden silence fell upon the room
as Mr. March and the young pair took their places under the green arch.
Mother and sisters gathered close, as if loath to give Meg up; the
fatherly voice broke more than once, which only seemed to make the
service more beautiful and solemn; the bridegroom\textquotesingle s hand
trembled visibly, and no one heard his replies; but Meg looked straight
up in her husband\textquotesingle s eyes, and said, "I will!" with such
tender trust in her own face and voice that her mother\textquotesingle s
heart rejoiced, and Aunt March sniffed audibly.

Jo did \emph{not} cry, though she was very near it once, and was only
saved from a demonstration by the consciousness that Laurie was staring
fixedly at her, with a comical mixture of merriment and emotion in his
wicked black eyes. Beth kept her face hidden on her
mother\textquotesingle s shoulder, but Amy stood like a graceful statue,
with a most becoming ray of sunshine touching her white forehead and the
flower in her hair.

It wasn\textquotesingle t at all the thing, I\textquotesingle m afraid,
but the minute she was fairly married, Meg cried, "The first kiss for
Marmee!" and, turning, gave it with her heart on her lips. During the
next fifteen minutes she looked more like a rose than ever, for every
one availed themselves of their privileges to the fullest extent, from
Mr. Laurence to old Hannah, who, adorned with a head-dress fearfully and
wonderfully made, fell upon her in the hall, crying, with a sob and a
chuckle, "Bless you, deary, a hundred times! The cake
ain\textquotesingle t hurt a mite, and everything looks lovely."

Everybody cleared up after that, and said something brilliant, or tried
to, which did just as well, for laughter is ready when hearts are light.
There was no display of gifts, for they were already in the little
house, nor was there an elaborate breakfast, but a plentiful lunch of
cake and fruit, dressed with flowers. Mr. Laurence and Aunt March
shrugged and smiled at one another when water, lemonade, and coffee were
found to be the only sorts of nectar which the three Hebes carried
round. No one said anything, however, till Laurie, who insisted on
serving the bride, appeared before her, with a loaded salver in his hand
and a puzzled expression on his face.

"Has Jo smashed all the bottles by accident?" he whispered, "or am I
merely laboring under a delusion that I saw some lying about loose this
morning?"

"No; your grandfather kindly offered us his best, and Aunt March
actually sent some, but father put away a little for Beth, and
despatched the rest to the Soldiers\textquotesingle{} Home. You know he
thinks that wine should be used only in illness, and mother says that
neither she nor her daughters will ever offer it to any young man under
her roof."

Meg spoke seriously, and expected to see Laurie frown or laugh; but he
did neither, for after a quick look at her, he said, in his impetuous
way, "I like that! for I\textquotesingle ve seen enough harm done to
wish other women would think as you do."

"You are not made wise by experience, I hope?" and there was an anxious
accent in Meg\textquotesingle s voice.

"No; I give you my word for it. Don\textquotesingle t think too well of
me, either; this is not one of my temptations. Being brought up where
wine is as common as water, and almost as harmless, I
don\textquotesingle t care for it; but when a pretty girl offers it, one
doesn\textquotesingle t like to refuse, you see."

"But you will, for the sake of others, if not for your own. Come,
Laurie, promise, and give me one more reason to call this the happiest
day of my life."

A demand so sudden and so serious made the young man hesitate a moment,
for ridicule is often harder to bear than self-denial. Meg knew that if
he gave the promise he would keep it at all costs; and, feeling her
power, used it as a woman may for her friend\textquotesingle s good. She
did not speak, but she looked up at him with a face made very eloquent
by happiness, and a smile which said, "No one can refuse me anything
to-day." Laurie certainly could not; and, with an answering smile, he
gave her his hand, saying heartily, "I promise, Mrs. Brooke!"

"I thank you, very, very much."

"And I drink \textquotesingle long life to your
resolution,\textquotesingle{} Teddy," cried Jo, baptizing him with a
splash of lemonade, as she waved her glass, and beamed approvingly upon
him.

So the toast was drunk, the pledge made, and loyally kept, in spite of
many temptations; for, with instinctive wisdom, the girls had seized a
happy moment to do their friend a service, for which he thanked them all
his life.

After lunch, people strolled about, by twos and threes, through house
and garden, enjoying the sunshine without and within. Meg and John
happened to be standing together in the middle of the grass-plot, when
Laurie was seized with an inspiration which put the finishing touch to
this unfashionable wedding.

"All the married people take hands and dance round the new-made husband
and wife, as the Germans do, while we bachelors and spinsters prance in
couples outside!" cried Laurie, promenading down the path with Amy, with
such infectious spirit and skill that every one else followed their
example without a murmur. Mr. and Mrs. March, Aunt and Uncle Carrol,
began it; others rapidly joined in; even Sallie Moffat, after a
moment\textquotesingle s hesitation, threw her train over her arm, and
whisked Ned into the ring. But the crowning joke was Mr. Laurence and
Aunt March; for when the stately old gentleman \emph{chasséed} solemnly
up to the old lady, she just tucked her cane under her arm, and hopped
briskly away to join hands with the rest, and dance about the bridal
pair, while the young folks pervaded the garden, like butterflies on a
midsummer day.

Want of breath brought the impromptu ball to a close, and then people
began to go.

"I wish you well, my dear, I heartily wish you well; but I think
you\textquotesingle ll be sorry for it," said Aunt March to Meg, adding
to the bridegroom, as he led her to the carriage,
"You\textquotesingle ve got a treasure, young man, see that you deserve
it."

"That is the prettiest wedding I\textquotesingle ve been to for an age,
Ned, and I don\textquotesingle t see why, for there
wasn\textquotesingle t a bit of style about it," observed Mrs. Moffat to
her husband, as they drove away.

"Laurie, my lad, if you ever want to indulge in this sort of thing, get
one of those little girls to help you, and I shall be perfectly
satisfied," said Mr. Laurence, settling himself in his easy-chair to
rest, after the excitement of the morning.

"I\textquotesingle ll do my best to gratify you, sir," was
Laurie\textquotesingle s unusually dutiful reply, as he carefully
unpinned the posy Jo had put in his button-hole.

The little house was not far away, and the only bridal journey Meg had
was the quiet walk with John, from the old home to the new. When she
came down, looking like a pretty Quakeress in her dove-colored suit and
straw bonnet tied with white, they all gathered about her to say
"good-by," as tenderly as if she had been going to make the grand tour.

"Don\textquotesingle t feel that I am separated from you, Marmee dear,
or that I love you any the less for loving John so much," she said,
clinging to her mother, with full eyes, for a moment. "I shall come
every day, father, and expect to keep my old place in all your hearts,
though I \emph{am} married. Beth is going to be with me a great deal,
and the other girls will drop in now and then to laugh at my
housekeeping struggles. Thank you all for my happy wedding-day. Good-by,
good-by!"

They stood watching her, with faces full of love and hope and tender
pride, as she walked away, leaning on her husband\textquotesingle s arm,
with her hands full of flowers, and the June sunshine brightening her
happy face,---and so Meg\textquotesingle s married life began.

\begin{center}\rule{0.5\linewidth}{0.5pt}\end{center}

\subsection{XXVI. Artistic
Attempts.}\label{6672479776654687619_37106-h-4.htm.xhtml_pgepubid00028}

\protect\phantomsection\label{6672479776654687619_37106-h-4.htm.xhtml_b123.png}{}
\pandocbounded{\includegraphics[keepaspectratio]{303483661336987339_b123.png}}

\protect\phantomsection\label{6672479776654687619_37106-h-4.htm.xhtml_XXVI}{}\hyperref[6672479776654687619_37106-h-0.htm.xhtml_contents2]{XXVI.}

ARTISTIC ATTEMPTS.

{It} takes people a long time to learn the difference between talent and
genius, especially ambitious young men and women. Amy was learning this
distinction through much tribulation; for, mistaking enthusiasm for
inspiration, she attempted every branch of art with youthful audacity.
For a long time there was a lull in the "mud-pie" business, and she
devoted herself to the finest pen-and-ink drawing, in which she showed
such taste and skill that her graceful handiwork proved both pleasant
and profitable. But overstrained eyes soon caused pen and ink to be laid
aside for a bold attempt at poker-sketching. While this attack lasted,
the family lived in constant fear of a conflagration; for the odor of
burning wood pervaded the house at all hours; smoke issued from attic
and shed with alarming frequency, red-hot pokers lay about
promiscuously, and Hannah never went to bed without a pail of water and
the dinner-bell at her door, in case of fire. Raphael\textquotesingle s
face was found boldly executed on the under side of the moulding-board,
and Bacchus on the head of a beer-barrel; a chanting cherub adorned the
cover of the sugar-bucket, and attempts to portray Romeo and Juliet
supplied kindlings for some time.

From fire to oil was a natural transition for burnt fingers, and Amy
fell to painting with undiminished ardor. An artist friend fitted her
out with his cast-off palettes, brushes, and colors; and she daubed
away, producing pastoral and marine views such as were never seen on
land or sea. Her monstrosities in the way of cattle would have taken
prizes at an agricultural fair; and the perilous pitching of her vessels
would have produced sea-sickness in the most nautical observer, if the
utter disregard to all known rules of shipbuilding and rigging had not
convulsed him with laughter at the first glance. Swarthy boys and
dark-eyed Madonnas, staring at you from one corner of the studio,
suggested Murillo; oily-brown shadows of faces, with a lurid streak in
the wrong place, meant Rembrandt; buxom ladies and dropsical infants,
Rubens; and Turner appeared in tempests of blue thunder, orange
lightning, brown rain, and purple clouds, with a tomato-colored splash
in the middle, which might be the sun or a buoy, a
sailor\textquotesingle s shirt or a king\textquotesingle s robe, as the
spectator pleased.

\protect\phantomsection\label{6672479776654687619_37106-h-4.htm.xhtml_b124.png}{}
\pandocbounded{\includegraphics[keepaspectratio]{303483661336987339_b124.png}}

Charcoal portraits came next; and the entire family hung in a row,
looking as wild and crocky as if just evoked from a coal-bin. Softened
into crayon sketches, they did better; for the likenesses were good, and
Amy\textquotesingle s hair, Jo\textquotesingle s nose,
Meg\textquotesingle s mouth, and Laurie\textquotesingle s eyes were
pronounced "wonderfully fine." A return to clay and plaster followed,
and ghostly casts of her acquaintances haunted corners of the house, or
tumbled off closet-shelves on to people\textquotesingle s heads.
Children were enticed in as models, till their incoherent accounts of
her mysterious doings caused Miss Amy to be regarded in the light of a
young ogress. Her efforts in this line, however, were brought to an
abrupt close by an untoward accident, which quenched her ardor. Other
models failing her for a time, she undertook to cast her own pretty
foot, and the family were one day alarmed by an unearthly bumping and
screaming, and running to the rescue, found the young enthusiast hopping
wildly about the shed, with her foot held fast in a pan-full of plaster,
which had hardened with unexpected rapidity. With much difficulty and
some danger she was dug out; for Jo was so overcome with laughter while
she excavated, that her knife went too far, cut the poor foot, and left
a lasting memorial of one artistic attempt, at least.

After this Amy subsided, till a mania for sketching from nature set her
to haunting river, field, and wood, for picturesque studies, and sighing
for ruins to copy. She caught endless colds sitting on damp grass to
book "a delicious bit," composed of a stone, a stump, one mushroom, and
a broken mullein-stalk, or "a heavenly mass of clouds," that looked like
a choice display of feather-beds when done. She sacrificed her
complexion floating on the river in the midsummer sun, to study light
and shade, and got a wrinkle over her nose, trying after "points of
sight," or whatever the squint-and-string performance is called.

If "genius is eternal patience," as Michael Angelo affirms, Amy
certainly had some claim to the divine attribute, for she persevered in
spite of all obstacles, failures, and discouragements, firmly believing
that in time she should do something worthy to be called "high art."

She was learning, doing, and enjoying other things, meanwhile, for she
had resolved to be an attractive and accomplished woman, even if she
never became a great artist. Here she succeeded better; for she was one
of those happily created beings who please without effort, make friends
everywhere, and take life so gracefully and easily that less fortunate
souls are tempted to believe that such are born under a lucky star.
Everybody liked her, for among her good gifts was tact. She had an
instinctive sense of what was pleasing and proper, always said the right
thing to the right person, did just what suited the time and place, and
was so self-possessed that her sisters used to say, "If Amy went to
court without any rehearsal beforehand, she\textquotesingle d know
exactly what to do."

One of her weaknesses was a desire to move in "our best society,"
without being quite sure what the \emph{best} really was. Money,
position, fashionable accomplishments, and elegant manners were most
desirable things in her eyes, and she liked to associate with those who
possessed them, often mistaking the false for the true, and admiring
what was not admirable. Never forgetting that by birth she was a
gentlewoman, she cultivated her aristocratic tastes and feelings, so
that when the opportunity came she might be ready to take the place from
which poverty now excluded her.

"My lady," as her friends called her, sincerely desired to be a genuine
lady, and was so at heart, but had yet to learn that money cannot buy
refinement of nature, that rank does not always confer nobility, and
that true breeding makes itself felt in spite of external drawbacks.

"I want to ask a favor of you, mamma," Amy said, coming in, with an
important air, one day.

"Well, little girl, what is it?" replied her mother, in whose eyes the
stately young lady still remained "the baby."

"Our drawing class breaks up next week, and before the girls separate
for the summer, I want to ask them out here for a day. They are wild to
see the river, sketch the broken bridge, and copy some of the things
they admire in my book. They have been very kind to me in many ways, and
I am grateful, for they are all rich, and know I am poor, yet they never
made any difference."

"Why should they?" and Mrs. March put the question with what the girls
called her "Maria Theresa air."

"You know as well as I that it \emph{does} make a difference with nearly
every one, so don\textquotesingle t ruffle up, like a dear, motherly
hen, when your chickens get pecked by smarter birds; the ugly duckling
turned out a swan, you know;" and Amy smiled without bitterness, for she
possessed a happy temper and hopeful spirit.

Mrs. March laughed, and smoothed down her maternal pride as she
asked,---

"Well, my swan, what is your plan?"

"I should like to ask the girls out to lunch next week, to take them a
drive to the places they want to see, a row on the river, perhaps, and
make a little artistic \emph{fête} for them."

"That looks feasible. What do you want for lunch? Cake, sandwiches,
fruit, and coffee will be all that is necessary, I suppose?"

"Oh dear, no! we must have cold tongue and chicken, French chocolate and
ice-cream, besides. The girls are used to such things, and I want my
lunch to be proper and elegant, though I \emph{do} work for my living."

"How many young ladies are there?" asked her mother, beginning to look
sober.

"Twelve or fourteen in the class, but I dare say they
won\textquotesingle t all come."

"Bless me, child, you will have to charter an omnibus to carry them
about."

"Why, mother, how \emph{can} you think of such a thing? Not more than
six or eight will probably come, so I shall hire a beach-wagon, and
borrow Mr. Laurence\textquotesingle s cherry-bounce."
(Hannah\textquotesingle s pronunciation of \emph{char-à-banc}.)

"All this will be expensive, Amy."

"Not very; I\textquotesingle ve calculated the cost, and
I\textquotesingle ll pay for it myself."

"Don\textquotesingle t you think, dear, that as these girls are used to
such things, and the best we can do will be nothing new, that some
simpler plan would be pleasanter to them, as a change, if nothing more,
and much better for us than buying or borrowing what we
don\textquotesingle t need, and attempting a style not in keeping with
our circumstances?"

"If I can\textquotesingle t have it as I like, I don\textquotesingle t
care to have it at all. I know that I can carry it out perfectly well,
if you and the girls will help a little; and I don\textquotesingle t see
why I can\textquotesingle t if I\textquotesingle m willing to pay for
it," said Amy, with the decision which opposition was apt to change into
obstinacy.

Mrs. March knew that experience was an excellent teacher, and when it
was possible she left her children to learn alone the lessons which she
would gladly have made easier, if they had not objected to taking advice
as much as they did salts and senna.

"Very well, Amy; if your heart is set upon it, and you see your way
through without too great an outlay of money, time, and temper,
I\textquotesingle ll say no more. Talk it over with the girls, and
whichever way you decide, I\textquotesingle ll do my best to help you."

"Thanks, mother; you are always \emph{so} kind;" and away went Amy to
lay her plan before her sisters.

Meg agreed at once, and promised her aid, gladly offering anything she
possessed, from her little house itself to her very best salt-spoons.
But Jo frowned upon the whole project, and would have nothing to do with
it at first.

"Why in the world should you spend your money, worry your family, and
turn the house upside down for a parcel of girls who
don\textquotesingle t care a sixpence for you? I thought you had too
much pride and sense to truckle to any mortal woman just because she
wears French boots and rides in a \emph{coupé}," said Jo, who, being
called from the tragical climax of her novel, was not in the best mood
for social enterprises.

"I \emph{don\textquotesingle t} truckle, and I hate being patronized as
much as you do!" returned Amy indignantly, for the two still jangled
when such questions arose. "The girls do care for me, and I for them,
and there\textquotesingle s a great deal of kindness and sense and
talent among them, in spite of what you call fashionable nonsense. You
don\textquotesingle t care to make people like you, to go into good
society, and cultivate your manners and tastes. I do, and I mean to make
the most of every chance that comes. \emph{You} can go through the world
with your elbows out and your nose in the air, and call it independence,
if you like. That\textquotesingle s not my way."

When Amy whetted her tongue and freed her mind she usually got the best
of it, for she seldom failed to have common sense on her side, while Jo
carried her love of liberty and hate of conventionalities to such an
unlimited extent that she naturally found herself worsted in an
argument. Amy\textquotesingle s definition of Jo\textquotesingle s idea
of independence was such a good hit that both burst out laughing, and
the discussion took a more amiable turn. Much against her will, Jo at
length consented to sacrifice a day to Mrs. Grundy, and help her sister
through what she regarded as "a nonsensical business."

The invitations were sent, nearly all accepted, and the following Monday
was set apart for the grand event. Hannah was out of humor because her
week\textquotesingle s work was deranged, and prophesied that "ef the
washin\textquotesingle{} and ironin\textquotesingle{}
warn\textquotesingle t done reg\textquotesingle lar
nothin\textquotesingle{} would go well anywheres." This hitch in the
mainspring of the domestic machinery had a bad effect upon the whole
concern; but Amy\textquotesingle s motto was "Nil desperandum," and
having made up her mind what to do, she proceeded to do it in spite of
all obstacles. To begin with, Hannah\textquotesingle s cooking
didn\textquotesingle t turn out well: the chicken was tough, the tongue
too salt, and the chocolate wouldn\textquotesingle t froth properly.
Then the cake and ice cost more than Amy expected, so did the wagon; and
various other expenses, which seemed trifling at the outset, counted up
rather alarmingly afterward. Beth got cold and took to her bed, Meg had
an unusual number of callers to keep her at home, and Jo was in such a
divided state of mind that her breakages, accidents, and mistakes were
uncommonly numerous, serious, and trying.

"If it hadn\textquotesingle t been for mother I never should have got
through," as Amy declared afterward, and gratefully remembered when "the
best joke of the season" was entirely forgotten by everybody else.

If it was not fair on Monday, the young ladies were to come on
Tuesday,---an arrangement which aggravated Jo and Hannah to the last
degree. On Monday morning the weather was in that undecided state which
is more exasperating than a steady pour. It drizzled a little, shone a
little, blew a little, and didn\textquotesingle t make up its mind till
it was too late for any one else to make up theirs. Amy was up at dawn,
hustling people out of their beds and through their breakfasts, that the
house might be got in order. The parlor struck her as looking uncommonly
shabby; but without stopping to sigh for what she had not, she skilfully
made the best of what she had, arranging chairs over the worn places in
the carpet, covering stains on the walls with pictures framed in ivy,
and filling up empty corners with home-made statuary, which gave an
artistic air to the room, as did the lovely vases of flowers Jo
scattered about.

The lunch looked charmingly; and as she surveyed it, she sincerely hoped
it would taste well, and that the borrowed glass, china, and silver
would get safely home again. The carriages were promised, Meg and mother
were all ready to do the honors, Beth was able to help Hannah behind the
scenes, Jo had engaged to be as lively and amiable as an absent mind, an
aching head, and a very decided disapproval of everybody and everything
would allow, and, as she wearily dressed, Amy cheered herself with
anticipations of the happy moment, when, lunch safely over, she should
drive away with her friends for an afternoon of artistic delights; for
the "cherry-bounce" and the broken bridge were her strong points.

Then came two hours of suspense, during which she vibrated from parlor
to porch, while public opinion varied like the weathercock. A smart
shower at eleven had evidently quenched the enthusiasm of the young
ladies who were to arrive at twelve, for nobody came; and at two the
exhausted family sat down in a blaze of sunshine to consume the
perishable portions of the feast, that nothing might be lost.

"No doubt about the weather to-day; they will certainly come, so we must
fly round and be ready for them," said Amy, as the sun woke her next
morning. She spoke briskly, but in her secret soul she wished she had
said nothing about Tuesday, for her interest, like her cake, was getting
a little stale.

"I can\textquotesingle t get any lobsters, so you will have to do
without salad to-day," said Mr. March, coming in half an hour later,
with an expression of placid despair.

"Use the chicken, then; the toughness won\textquotesingle t matter in a
salad," advised his wife.

"Hannah left it on the kitchen-table a minute, and the kittens got at
it. I\textquotesingle m very sorry, Amy," added Beth, who was still a
patroness of cats.

"Then I \emph{must} have a lobster, for tongue alone
won\textquotesingle t do," said Amy decidedly.

"Shall I rush into town and demand one?" asked Jo, with the magnanimity
of a martyr.

"You\textquotesingle d come bringing it home under your arm, without any
paper, just to try me. I\textquotesingle ll go myself," answered Amy,
whose temper was beginning to fail.

Shrouded in a thick veil and armed with a genteel travelling-basket, she
departed, feeling that a cool drive would soothe her ruffled spirit, and
fit her for the labors of the day. After some delay, the object of her
desire was procured, likewise a bottle of dressing, to prevent further
loss of time at home, and off she drove again, well pleased with her own
forethought.

As the omnibus contained only one other passenger, a sleepy old lady,
Amy pocketed her veil, and beguiled the tedium of the way by trying to
find out where all her money had gone to. So busy was she with her card
full of refractory figures that she did not observe a new-comer, who
entered without stopping the vehicle, till a masculine voice said,
"Good-morning, Miss March," and, looking up, she beheld one of
Laurie\textquotesingle s most elegant college friends. Fervently hoping
that he would get out before she did, Amy utterly ignored the basket at
her feet, and, congratulating herself that she had on her new travelling
dress, returned the young man\textquotesingle s greeting with her usual
suavity and spirit.

They got on excellently; for Amy\textquotesingle s chief care was soon
set at rest by learning that the gentleman would leave first, and she
was chatting away in a peculiarly lofty strain, when the old lady got
out. In stumbling to the door, she upset the basket, and---oh,
horror!---the lobster, in all its vulgar size and brilliancy, was
revealed to the highborn eyes of a Tudor.

"By Jove, she\textquotesingle s forgotten her dinner!" cried the
unconscious youth, poking the scarlet monster into its place with his
cane, and preparing to hand out the basket after the old lady.

"Please
don\textquotesingle t---it\textquotesingle s---it\textquotesingle s
mine," murmured Amy, with a face nearly as red as her fish.

\protect\phantomsection\label{6672479776654687619_37106-h-4.htm.xhtml_b125.png}{}
\pandocbounded{\includegraphics[keepaspectratio]{303483661336987339_b125.png}}

"Oh, really, I beg pardon; it\textquotesingle s an uncommonly fine one,
isn\textquotesingle t it?" said Tudor, with great presence of mind, and
an air of sober interest that did credit to his breeding.

Amy recovered herself in a breath, set her basket boldly on the seat,
and said, laughing,---

"Don\textquotesingle t you wish you were to have some of the salad
he\textquotesingle s to make, and to see the charming young ladies who
are to eat it?"

Now that was tact, for two of the ruling foibles of the masculine mind
were touched: the lobster was instantly surrounded by a halo of pleasing
reminiscences, and curiosity about "the charming young ladies" diverted
his mind from the comical mishap.

"I suppose he\textquotesingle ll laugh and joke over it with Laurie, but
I sha\textquotesingle n\textquotesingle t see them;
that\textquotesingle s a comfort," thought Amy, as Tudor bowed and
departed.

She did not mention this meeting at home (though she discovered that,
thanks to the upset, her new dress was much damaged by the rivulets of
dressing that meandered down the skirt), but went through with the
preparations which now seemed more irksome than before; and at twelve
o\textquotesingle clock all was ready again. Feeling that the neighbors
were interested in her movements, she wished to efface the memory of
yesterday\textquotesingle s failure by a grand success to-day; so she
ordered the "cherry-bounce," and drove away in state to meet and escort
her guests to the banquet.

"There\textquotesingle s the rumble, they\textquotesingle re coming!
I\textquotesingle ll go into the porch to meet them; it looks
hospitable, and I want the poor child to have a good time after all her
trouble," said Mrs. March, suiting the action to the word. But after one
glance, she retired, with an indescribable expression, for, looking
quite lost in the big carriage, sat Amy and one young lady.

"Run, Beth, and help Hannah clear half the things off the table; it will
be too absurd to put a luncheon for twelve before a single girl," cried
Jo, hurrying away to the lower regions, too excited to stop even for a
laugh.

In came Amy, quite calm, and delightfully cordial to the one guest who
had kept her promise; the rest of the family, being of a dramatic turn,
played their parts equally well, and Miss Eliott found them a most
hilarious set; for it was impossible to entirely control the merriment
which possessed them. The remodelled lunch being gayly partaken of, the
studio and garden visited, and art discussed with enthusiasm, Amy
ordered a buggy (alas for the elegant cherry-bounce!) and drove her
friend quietly about the neighborhood till sunset, when "the party went
out."

As she came walking in, looking very tired, but as composed as ever, she
observed that every vestige of the unfortunate \emph{fête} had
disappeared, except a suspicious pucker about the corners of
Jo\textquotesingle s mouth.

"You\textquotesingle ve had a lovely afternoon for your drive, dear,"
said her mother, as respectfully as if the whole twelve had come.

"Miss Eliott is a very sweet girl, and seemed to enjoy herself, I
thought," observed Beth, with unusual warmth.

"Could you spare me some of your cake? I really need some, I have so
much company, and I can\textquotesingle t make such delicious stuff as
yours," asked Meg soberly.

"Take it all; I\textquotesingle m the only one here who likes sweet
things, and it will mould before I can dispose of it," answered Amy,
thinking with a sigh of the generous store she had laid in for such an
end as this.

"It\textquotesingle s a pity Laurie isn\textquotesingle t here to help
us," began Jo, as they sat down to ice-cream and salad for the second
time in two days.

A warning look from her mother checked any further remarks, and the
whole family ate in heroic silence, till Mr. March mildly observed,
"Salad was one of the favorite dishes of the ancients, and
Evelyn"---here a general explosion of laughter cut short the "history of
sallets," to the great surprise of the learned gentleman.

"Bundle everything into a basket and send it to the Hummels: Germans
like messes. I\textquotesingle m sick of the sight of this; and
there\textquotesingle s no reason you should all die of a surfeit
because I\textquotesingle ve been a fool," cried Amy, wiping her eyes.

"I thought I \emph{should} have died when I saw you two girls rattling
about in the what-you-call-it, like two little kernels in a very big
nutshell, and mother waiting in state to receive the throng," sighed Jo,
quite spent with laughter.

"I\textquotesingle m very sorry you were disappointed, dear, but we all
did our best to satisfy you," said Mrs. March, in a tone full of
motherly regret.

"I \emph{am} satisfied; I\textquotesingle ve done what I undertook, and
it\textquotesingle s not my fault that it failed; I comfort myself with
that," said Amy, with a little quiver in her voice. "I thank you all
very much for helping me, and I\textquotesingle ll thank you still more
if you won\textquotesingle t allude to it for a month, at least."

No one did for several months; but the word "\emph{fête}" always
produced a general smile, and Laurie\textquotesingle s birthday gift to
Amy was a tiny coral lobster in the shape of a charm for her
watch-guard.

\protect\phantomsection\label{6672479776654687619_37106-h-4.htm.xhtml_b126.png}{}
\pandocbounded{\includegraphics[keepaspectratio]{303483661336987339_b126.png}}

\begin{center}\rule{0.5\linewidth}{0.5pt}\end{center}

\subsection{XXVII. Literary
Lessons.}\label{6672479776654687619_37106-h-4.htm.xhtml_pgepubid00029}

\protect\phantomsection\label{6672479776654687619_37106-h-4.htm.xhtml_b127.png}{}
\pandocbounded{\includegraphics[keepaspectratio]{303483661336987339_b127.png}}

\protect\phantomsection\label{6672479776654687619_37106-h-4.htm.xhtml_XXVII}{}\hyperref[6672479776654687619_37106-h-0.htm.xhtml_contents2]{XXVII.}

LITERARY LESSONS.

{Fortune} suddenly smiled upon Jo, and dropped a good-luck penny in her
path. Not a golden penny, exactly, but I doubt if half a million would
have given more real happiness than did the little sum that came to her
in this wise.

Every few weeks she would shut herself up in her room, put on her
scribbling suit, and "fall into a vortex," as she expressed it, writing
away at her novel with all her heart and soul, for till that was
finished she could find no peace. Her "scribbling suit" consisted of a
black woollen pinafore on which she could wipe her pen at will, and a
cap of the same material, adorned with a cheerful red bow, into which
she bundled her hair when the decks were cleared for action. This cap
was a beacon to the inquiring eyes of her family, who during these
periods kept their distance, merely popping in their heads
semi-occasionally, to ask, with interest, "Does genius burn, Jo?" They
did not always venture even to ask this question, but took an
observation of the cap, and judged accordingly. If this expressive
article of dress was drawn low upon the forehead, it was a sign that
hard work was going on; in exciting moments it was pushed rakishly
askew; and when despair seized the author it was plucked wholly off, and
cast upon the floor. At such times the intruder silently withdrew; and
not until the red bow was seen gayly erect upon the gifted brow, did any
one dare address Jo.

She did not think herself a genius by any means; but when the writing
fit came on, she gave herself up to it with entire abandon, and led a
blissful life, unconscious of want, care, or bad weather, while she sat
safe and happy in an imaginary world, full of friends almost as real and
dear to her as any in the flesh. Sleep forsook her eyes, meals stood
untasted, day and night were all too short to enjoy the happiness which
blessed her only at such times, and made these hours worth living, even
if they bore no other fruit. The divine afflatus usually lasted a week
or two, and then she emerged from her "vortex," hungry, sleepy, cross,
or despondent.

She was just recovering from one of these attacks when she was prevailed
upon to escort Miss Crocker to a lecture, and in return for her virtue
was rewarded with a new idea. It was a People\textquotesingle s Course,
the lecture on the Pyramids, and Jo rather wondered at the choice of
such a subject for such an audience, but took it for granted that some
great social evil would be remedied or some great want supplied by
unfolding the glories of the Pharaohs to an audience whose thoughts were
busy with the price of coal and flour, and whose lives were spent in
trying to solve harder riddles than that of the Sphinx.

They were early; and while Miss Crocker set the heel of her stocking, Jo
amused herself by examining the faces of the people who occupied the
seat with them. On her left were two matrons, with massive foreheads,
and bonnets to match, discussing Woman\textquotesingle s Rights and
making tatting. Beyond sat a pair of humble lovers, artlessly holding
each other by the hand, a sombre spinster eating peppermints out of a
paper bag, and an old gentleman taking his preparatory nap behind a
yellow bandanna. On her right, her only neighbor was a studious-looking
lad absorbed in a newspaper.

It was a pictorial sheet, and Jo examined the work of art nearest her,
idly wondering what unfortuitous concatenation of circumstances needed
the melodramatic illustration of an Indian in full war costume, tumbling
over a precipice with a wolf at his throat, while two infuriated young
gentlemen, with unnaturally small feet and big eyes, were stabbing each
other close by, and a dishevelled female was flying away in the
background with her mouth wide open. Pausing to turn a page, the lad saw
her looking, and, with boyish good-nature, offered half his paper,
saying bluntly, "Want to read it? That\textquotesingle s a first-rate
story."

Jo accepted it with a smile, for she had never outgrown her liking for
lads, and soon found herself involved in the usual labyrinth of love,
mystery, and murder, for the story belonged to that class of light
literature in which the passions have a holiday, and when the
author\textquotesingle s invention fails, a grand catastrophe clears the
stage of one half the \emph{dramatis personæ}, leaving the other half to
exult over their downfall.

"Prime, isn\textquotesingle t it?" asked the boy, as her eye went down
the last paragraph of her portion.

"I think you and I could do as well as that if we tried," returned Jo,
amused at his admiration of the trash.

"I should think I was a pretty lucky chap if I could. She makes a good
living out of such stories, they say;" and he pointed to the name of
Mrs. S. L. A. N. G. Northbury, under the title of the tale.

"Do you know her?" asked Jo, with sudden interest.

"No; but I read all her pieces, and I know a fellow who works in the
office where this paper is printed."

"Do you say she makes a good living out of stories like this?" and Jo
looked more respectfully at the agitated group and thickly-sprinkled
exclamation-points that adorned the page.

"Guess she does! She knows just what folks like, and gets paid well for
writing it."

Here the lecture began, but Jo heard very little of it, for while Prof.
Sands was prosing away about Belzoni, Cheops, scarabei, and
hieroglyphics, she was covertly taking down the address of the paper,
and boldly resolving to try for the hundred-dollar prize offered in its
columns for a sensational story. By the time the lecture ended and the
audience awoke, she had built up a splendid fortune for herself (not the
first founded upon paper), and was already deep in the concoction of her
story, being unable to decide whether the duel should come before the
elopement or after the murder.

She said nothing of her plan at home, but fell to work next day, much to
the disquiet of her mother, who always looked a little anxious when
"genius took to burning." Jo had never tried this style before,
contenting herself with very mild romances for the "Spread Eagle." Her
theatrical experience and miscellaneous reading were of service now, for
they gave her some idea of dramatic effect, and supplied plot, language,
and costumes. Her story was as full of desperation and despair as her
limited acquaintance with those uncomfortable emotions enabled her to
make it, and, having located it in Lisbon, she wound up with an
earthquake, as a striking and appropriate \emph{dénouement}. The
manuscript was privately despatched, accompanied by a note, modestly
saying that if the tale didn\textquotesingle t get the prize, which the
writer hardly dared expect, she would be very glad to receive any sum it
might be considered worth.

\protect\phantomsection\label{6672479776654687619_37106-h-4.htm.xhtml_b128.png}{}
\pandocbounded{\includegraphics[keepaspectratio]{303483661336987339_b128.png}}

Six weeks is a long time to wait, and a still longer time for a girl to
keep a secret; but Jo did both, and was just beginning to give up all
hope of ever seeing her manuscript again, when a letter arrived which
almost took her breath away; for on opening it, a check for a hundred
dollars fell into her lap. For a minute she stared at it as if it had
been a snake, then she read her letter and began to cry. If the amiable
gentleman who wrote that kindly note could have known what intense
happiness he was giving a fellow-creature, I think he would devote his
leisure hours, if he has any, to that amusement; for Jo valued the
letter more than the money, because it was encouraging; and after years
of effort it was \emph{so} pleasant to find that she had learned to do
something, though it was only to write a sensation story.

A prouder young woman was seldom seen than she, when, having composed
herself, she electrified the family by appearing before them with the
letter in one hand, the check in the other, announcing that she had won
the prize. Of course there was a great jubilee, and when the story came
every one read and praised it; though after her father had told her that
the language was good, the romance fresh and hearty, and the tragedy
quite thrilling, he shook his head, and said in his unworldly way,---

"You can do better than this, Jo. Aim at the highest, and never mind the
money."

"\emph{I} think the money is the best part of it. What \emph{will} you
do with such a fortune?" asked Amy, regarding the magic slip of paper
with a reverential eye.

"Send Beth and mother to the seaside for a month or two," answered Jo
promptly.

"Oh, how splendid! No, I can\textquotesingle t do it, dear, it would be
so selfish," cried Beth, who had clapped her thin hands, and taken a
long breath, as if pining for fresh ocean-breezes; then stopped herself,
and motioned away the check which her sister waved before her.

"Ah, but you shall go, I\textquotesingle ve set my heart on it;
that\textquotesingle s what I tried for, and that\textquotesingle s why
I succeeded. I never get on when I think of myself alone, so it will
help me to work for you, don\textquotesingle t you see? Besides, Marmee
needs the change, and she won\textquotesingle t leave you, so you
\emph{must} go. Won\textquotesingle t it be fun to see you come home
plump and rosy again? Hurrah for Dr. Jo, who always cures her patients!"

To the sea side they went, after much discussion; and though Beth
didn\textquotesingle t come home as plump and rosy as could be desired,
she was much better, while Mrs. March declared she felt ten years
younger; so Jo was satisfied with the investment of her prize money, and
fell to work with a cheery spirit, bent on earning more of those
delightful checks. She did earn several that year, and began to feel
herself a power in the house; for by the magic of a pen, her "rubbish"
turned into comforts for them all. "The Duke\textquotesingle s Daughter"
paid the butcher\textquotesingle s bill, "A Phantom Hand" put down a new
carpet, and the "Curse of the Coventrys" proved the blessing of the
Marches in the way of groceries and gowns.

Wealth is certainly a most desirable thing, but poverty has its sunny
side, and one of the sweet uses of adversity is the genuine satisfaction
which comes from hearty work of head or hand; and to the inspiration of
necessity, we owe half the wise, beautiful, and useful blessings of the
world. Jo enjoyed a taste of this satisfaction, and ceased to envy
richer girls, taking great comfort in the knowledge that she could
supply her own wants, and need ask no one for a penny.

Little notice was taken of her stories, but they found a market; and,
encouraged by this fact, she resolved to make a bold stroke for fame and
fortune. Having copied her novel for the fourth time, read it to all her
confidential friends, and submitted it with fear and trembling to three
publishers, she at last disposed of it, on condition that she would cut
it down one third, and omit all the parts which she particularly
admired.

"Now I must either bundle it back into my tin-kitchen to mould, pay for
printing it myself, or chop it up to suit purchasers, and get what I can
for it. Fame is a very good thing to have in the house, but cash is more
convenient; so I wish to take the sense of the meeting on this important
subject," said Jo, calling a family council.

"Don\textquotesingle t spoil your book, my girl, for there is more in it
than you know, and the idea is well worked out. Let it wait and ripen,"
was her father\textquotesingle s advice; and he practised as he
preached, having waited patiently thirty years for fruit of his own to
ripen, and being in no haste to gather it, even now, when it was sweet
and mellow.

"It seems to me that Jo will profit more by making the trial than by
waiting," said Mrs. March. "Criticism is the best test of such work, for
it will show her both unsuspected merits and faults, and help her to do
better next time. We are too partial; but the praise and blame of
outsiders will prove useful, even if she gets but little money."

"Yes," said Jo, knitting her brows, "that\textquotesingle s just it;
I\textquotesingle ve been fussing over the thing so long, I really
don\textquotesingle t know whether it\textquotesingle s good, bad, or
indifferent. It will be a great help to have cool, impartial persons
take a look at it, and tell me what they think of it."

"I wouldn\textquotesingle t leave out a word of it;
you\textquotesingle ll spoil it if you do, for the interest of the story
is more in the minds than in the actions of the people, and it will be
all a muddle if you don\textquotesingle t explain as you go on," said
Meg, who firmly believed that this book was the most remarkable novel
ever written.

"But Mr. Allen says, \textquotesingle Leave out the explanations, make
it brief and dramatic, and let the characters tell the
story,\textquotesingle" interrupted Jo, turning to the
publisher\textquotesingle s note.

"Do as he tells you; he knows what will sell, and we
don\textquotesingle t. Make a good, popular book, and get as much money
as you can. By and by, when, you\textquotesingle ve got a name, you can
afford to digress, and have philosophical and metaphysical people in
your novels," said Amy, who took a strictly practical view of the
subject.

"Well," said Jo, laughing, "if my people \emph{are}
\textquotesingle philosophical and metaphysical,\textquotesingle{} it
isn\textquotesingle t my fault, for I know nothing about such things,
except what I hear father say, sometimes. If I\textquotesingle ve got
some of his wise ideas jumbled up with my romance, so much the better
for me. Now, Beth, what do you say?"

"I should so like to see it printed \emph{soon}," was all Beth said, and
smiled in saying it; but there was an unconscious emphasis on the last
word, and a wistful look in the eyes that never lost their childlike
candor, which chilled Jo\textquotesingle s heart, for a minute, with a
foreboding fear, and decided her to make her little venture "soon."

So, with Spartan firmness, the young authoress laid her first-born on
her table, and chopped it up as ruthlessly as any ogre. In the hope of
pleasing every one, she took every one\textquotesingle s advice; and,
like the old man and his donkey in the fable, suited nobody.

Her father liked the metaphysical streak which had unconsciously got
into it; so that was allowed to remain, though she had her doubts about
it. Her mother thought that there \emph{was} a trifle too much
description; out, therefore, it nearly all came, and with it many
necessary links in the story. Meg admired the tragedy; so Jo piled up
the agony to suit her, while Amy objected to the fun, and, with the best
intentions in life, Jo quenched the sprightly scenes which relieved the
sombre character of the story. Then, to complete the ruin, she cut it
down one third, and confidingly sent the poor little romance, like a
picked robin, out into the big, busy world, to try its fate.

Well, it was printed, and she got three hundred dollars for it; likewise
plenty of praise and blame, both so much greater than she expected that
she was thrown into a state of bewilderment, from which it took her some
time to recover.

"You said, mother, that criticism would help me; but how can it, when
it\textquotesingle s so contradictory that I don\textquotesingle t know
whether I\textquotesingle ve written a promising book or broken all the
ten commandments?" cried poor Jo, turning over a heap of notices, the
perusal of which filled her with pride and joy one minute, wrath and
dire dismay the next. "This man says \textquotesingle An exquisite book,
full of truth, beauty, and earnestness; all is sweet, pure, and
healthy,\textquotesingle" continued the perplexed authoress. "The next,
\textquotesingle The theory of the book is bad, full of morbid fancies,
spiritualistic ideas, and unnatural characters.\textquotesingle{} Now,
as I had no theory of any kind, don\textquotesingle t believe in
Spiritualism, and copied my characters from life, I
don\textquotesingle t see how this critic \emph{can} be right. Another
says, \textquotesingle It\textquotesingle s one of the best American
novels which has appeared for years\textquotesingle{} (I know better
than that); and the next asserts that \textquotesingle though it is
original, and written with great force and feeling, it is a dangerous
book.\textquotesingle{} \textquotesingle Tisn\textquotesingle t! Some
make fun of it, some over-praise, and nearly all insist that I had a
deep theory to expound, when I only wrote it for the pleasure and the
money. I wish I\textquotesingle d printed it whole or not at all, for I
do hate to be so misjudged."

Her family and friends administered comfort and commendation liberally;
yet it was a hard time for sensitive, high-spirited Jo, who meant so
well, and had apparently done so ill. But it did her good, for those
whose opinion had real value gave her the criticism which is an
author\textquotesingle s best education; and when the first soreness was
over, she could laugh at her poor little book, yet believe in it still,
and feel herself the wiser and stronger for the buffeting she had
received.

"Not being a genius, like Keats, it won\textquotesingle t kill me," she
said stoutly; "and I\textquotesingle ve got the joke on my side, after
all; for the parts that were taken straight out of real life are
denounced as impossible and absurd, and the scenes that I made up out of
my own silly head are pronounced \textquotesingle charmingly natural,
tender, and true.\textquotesingle{} So I\textquotesingle ll comfort
myself with that; and when I\textquotesingle m ready,
I\textquotesingle ll up again and take another."

\protect\phantomsection\label{6672479776654687619_37106-h-4.htm.xhtml_b129.png}{}
\pandocbounded{\includegraphics[keepaspectratio]{303483661336987339_b129.png}}

\begin{center}\rule{0.5\linewidth}{0.5pt}\end{center}

\subsection{XXVIII. Domestic
Experiences.}\label{6672479776654687619_37106-h-4.htm.xhtml_pgepubid00030}

\protect\phantomsection\label{6672479776654687619_37106-h-4.htm.xhtml_b130.png}{}
\pandocbounded{\includegraphics[keepaspectratio]{303483661336987339_b130.png}}

\protect\phantomsection\label{6672479776654687619_37106-h-4.htm.xhtml_XXVIII}{}\hyperref[6672479776654687619_37106-h-0.htm.xhtml_contents2]{XXVIII.}

DOMESTIC EXPERIENCES.

{Like} most other young matrons, Meg began her married life with the
determination to be a model housekeeper. John should find home a
paradise; he should always see a smiling face, should fare sumptuously
every day, and never know the loss of a button. She brought so much
love, energy, and cheerfulness to the work that she could not but
succeed, in spite of some obstacles. Her paradise was not a tranquil
one; for the little woman fussed, was over-anxious to please, and
bustled about like a true Martha, cumbered with many cares. She was too
tired, sometimes, even to smile; John grew dyspeptic after a course of
dainty dishes, and ungratefully demanded plain fare. As for buttons, she
soon learned to wonder where they went, to shake her head over the
carelessness of men, and to threaten to make him sew them on himself,
and then see if \emph{his} work would stand impatient tugs and clumsy
fingers any better than hers.

They were very happy, even after they discovered that they
couldn\textquotesingle t live on love alone. John did not find
Meg\textquotesingle s beauty diminished, though she beamed at him from
behind the familiar coffee-pot; nor did Meg miss any of the romance from
the daily parting, when her husband followed up his kiss with the tender
inquiry, "Shall I send home veal or mutton for dinner, darling?" The
little house ceased to be a glorified bower, but it became a home, and
the young couple soon felt that it was a change for the better. At first
they played keep-house, and frolicked over it like children; then John
took steadily to business, feeling the cares of the head of a family
upon his shoulders; and Meg laid by her cambric wrappers, put on a big
apron, and fell to work, as before said, with more energy than
discretion.

While the cooking mania lasted she went through Mrs.
Cornelius\textquotesingle s Receipt Book as if it were a mathematical
exercise, working out the problems with patience and care. Sometimes her
family were invited in to help eat up a too bounteous feast of
successes, or Lotty would be privately despatched with a batch of
failures, which were to be concealed from all eyes in the convenient
stomachs of the little Hummels. An evening with John over the
account-books usually produced a temporary lull in the culinary
enthusiasm, and a frugal fit would ensue, during which the poor man was
put through a course of bread-pudding, hash, and warmed-over coffee,
which tried his soul, although he bore it with praiseworthy fortitude.
Before the golden mean was found, however, Meg added to her domestic
possessions what young couples seldom get on long without,---a family
jar.

Fired with a housewifely wish to see her store-room stocked with
home-made preserves, she undertook to put up her own currant jelly. John
was requested to order home a dozen or so of little pots, and an extra
quantity of sugar, for their own currants were ripe, and were to be
attended to at once. As John firmly believed that "my wife" was equal to
anything, and took a natural pride in her skill, he resolved that she
should be gratified, and their only crop of fruit laid by in a most
pleasing form for winter use. Home came four dozen delightful little
pots, half a barrel of sugar, and a small boy to pick the currants for
her. With her pretty hair tucked into a little cap, arms bared to the
elbow, and a checked apron which had a coquettish look in spite of the
bib, the young housewife fell to work, feeling no doubts about her
success; for hadn\textquotesingle t she seen Hannah do it hundreds of
times? The array of pots rather amazed her at first, but John was so
fond of jelly, and the nice little jars would look so well on the top
shelf, that Meg resolved to fill them all, and spent a long day picking,
boiling, straining, and fussing over her jelly. She did her best; she
asked advice of Mrs. Cornelius; she racked her brain to remember what
Hannah did that she had left undone; she reboiled, resugared, and
restrained, but that dreadful stuff wouldn\textquotesingle t
"\emph{jell}."

She longed to run home, bib and all, and ask mother to lend a hand, but
John and she had agreed that they would never annoy any one with their
private worries, experiments, or quarrels. They had laughed over that
last word as if the idea it suggested was a most preposterous one; but
they had held to their resolve, and whenever they could get on without
help they did so, and no one interfered, for Mrs. March had advised the
plan. So Meg wrestled alone with the refractory sweetmeats all that hot
summer day, and at five o\textquotesingle clock sat down in her
topsy-turvy kitchen, wrung her bedaubed hands, lifted up her voice and
wept.

Now, in the first flush of the new life, she had often said,---

"My husband shall always feel free to bring a friend home whenever he
likes. I shall always be prepared; there shall be no flurry, no
scolding, no discomfort, but a neat house, a cheerful wife, and a good
dinner. John, dear, never stop to ask my leave, invite whom you please,
and be sure of a welcome from me."

How charming that was, to be sure! John quite glowed with pride to hear
her say it, and felt what a blessed thing it was to have a superior
wife. But, although they had had company from time to time, it never
happened to be unexpected, and Meg had never had an opportunity to
distinguish herself till now. It always happens so in this vale of
tears; there is an inevitability about such things which we can only
wonder at, deplore, and bear as we best can.

If John had not forgotten all about the jelly, it really would have been
unpardonable in him to choose that day, of all the days in the year, to
bring a friend home to dinner unexpectedly. Congratulating himself that
a handsome repast had been ordered that morning, feeling sure that it
would be ready to the minute, and indulging in pleasant anticipations of
the charming effect it would produce, when his pretty wife came running
out to meet him, he escorted his friend to his mansion, with the
irrepressible satisfaction of a young host and husband.

It is a world of disappointments, as John discovered when he reached the
Dove-cote. The front door usually stood hospitably open; now it was not
only shut, but locked, and yesterday\textquotesingle s mud still adorned
the steps. The parlor-windows were closed and curtained, no picture of
the pretty wife sewing on the piazza, in white, with a distracting
little bow in her hair, or a bright-eyed hostess, smiling a shy welcome
as she greeted her guest. Nothing of the sort, for not a soul appeared,
but a sanguinary-looking boy asleep under the currant-bushes.

"I\textquotesingle m afraid something has happened. Step into the
garden, Scott, while I look up Mrs. Brooke," said John, alarmed at the
silence and solitude.

Round the house he hurried, led by a pungent smell of burnt sugar, and
Mr. Scott strolled after him, with a queer look on his face. He paused
discreetly at a distance when Brooke disappeared; but he could both see
and hear, and, being a bachelor, enjoyed the prospect mightily.

In the kitchen reigned confusion and despair; one edition of jelly was
trickled from pot to pot, another lay upon the floor, and a third was
burning gayly on the stove. Lotty, with Teutonic phlegm, was calmly
eating bread and currant wine, for the jelly was still in a hopelessly
liquid state, while Mrs. Brooke, with her apron over her head, sat
sobbing dismally.

"My dearest girl, what is the matter?" cried John, rushing in, with
awful visions of scalded hands, sudden news of affliction, and secret
consternation at the thought of the guest in the garden.

"O John, I \emph{am} so tired and hot and cross and worried!
I\textquotesingle ve been at it till I\textquotesingle m all worn out.
Do come and help me or I \emph{shall} die!" and the exhausted housewife
cast herself upon his breast, giving him a sweet welcome in every sense
of the word, for her pinafore had been baptized at the same time as the
floor.

"What worries you, dear? Has anything dreadful happened?" asked the
anxious John, tenderly kissing the crown of the little cap, which was
all askew.

"Yes," sobbed Meg despairingly.

"Tell me quick, then. Don\textquotesingle t cry, I can bear anything
better than that. Out with it, love."

"The---the jelly won\textquotesingle t jell and I don\textquotesingle t
know what to do!"

John Brooke laughed then as he never dared to laugh afterward; and the
derisive Scott smiled involuntarily as he heard the hearty peal, which
put the finishing stroke to poor Meg\textquotesingle s woe.

"Is that all? Fling it out of window, and don\textquotesingle t bother
any more about it. I\textquotesingle ll buy you quarts if you want it;
but for heaven\textquotesingle s sake don\textquotesingle t have
hysterics, for I\textquotesingle ve brought Jack Scott home to dinner,
and---"

John got no further, for Meg cast him off, and clasped her hands with a
tragic gesture as she fell into a chair, exclaiming in a tone of mingled
indignation, reproach, and dismay,---

"A man to dinner, and everything in a mess! John Brooke, how
\emph{could} you do such a thing?"

"Hush, he\textquotesingle s in the garden! I forgot the confounded
jelly, but it can\textquotesingle t be helped now," said John, surveying
the prospect with an anxious eye.

"You ought to have sent word, or told me this morning, and you ought to
have remembered how busy I was," continued Meg petulantly; for even
turtle-doves will peck when ruffled.

"I didn\textquotesingle t know it this morning, and there was no time to
send word, for I met him on the way out. I never thought of asking
leave, when you have always told me to do as I liked. I never tried it
before, and hang me if I ever do again!" added John, with an aggrieved
air.

"I should hope not! Take him away at once; I can\textquotesingle t see
him, and there isn\textquotesingle t any dinner."

"Well, I like that! Where\textquotesingle s the beef and vegetables I
sent home, and the pudding you promised?" cried John, rushing to the
larder.

"I hadn\textquotesingle t time to cook anything; I meant to dine at
mother\textquotesingle s. I\textquotesingle m sorry, but I was \emph{so}
busy;" and Meg\textquotesingle s tears began again.

John was a mild man, but he was human; and after a long
day\textquotesingle s work, to come home tired, hungry, and hopeful, to
find a chaotic house, an empty table, and a cross wife was not exactly
conducive to repose of mind or manner. He restrained himself, however,
and the little squall would have blown over, but for one unlucky word.

"It\textquotesingle s a scrape, I acknowledge; but if you will lend a
hand, we\textquotesingle ll pull through, and have a good time yet.
Don\textquotesingle t cry, dear, but just exert yourself a bit, and
knock us up something to eat. We\textquotesingle re both as hungry as
hunters, so we sha\textquotesingle n\textquotesingle t mind what it is.
Give us the cold meat, and bread and cheese; we won\textquotesingle t
ask for jelly."

He meant it for a good-natured joke; but that one word sealed his fate.
Meg thought it was \emph{too} cruel to hint about her sad failure, and
the last atom of patience vanished as he spoke.

"You must get yourself out of the scrape as you can; I\textquotesingle m
too used up to \textquotesingle exert\textquotesingle{} myself for any
one. It\textquotesingle s like a man to propose a bone and vulgar bread
and cheese for company. I won\textquotesingle t have anything of the
sort in my house. Take that Scott up to mother\textquotesingle s, and
tell him I\textquotesingle m away, sick, dead,---anything. I
won\textquotesingle t see him, and you two can laugh at me and my jelly
as much as you like: you won\textquotesingle t have anything else here;"
and having delivered her defiance all in one breath, Meg cast away her
pinafore, and precipitately left the field to bemoan herself in her own
room.

What those two creatures did in her absence, she never knew; but Mr.
Scott was not taken "up to mother\textquotesingle s," and when Meg
descended, after they had strolled away together, she found traces of a
promiscuous lunch which filled her with horror. Lotty reported that they
had eaten "a much, and greatly laughed, and the master bid her throw
away all the sweet stuff, and hide the pots."

Meg longed to go and tell mother; but a sense of shame at her \ul{own
short-comings,} of loyalty to John, "who might be cruel, but nobody
should know it," restrained her; and after a summary clearing up, she
dressed herself prettily, and sat down to wait for John to come and be
forgiven.

Unfortunately, John didn\textquotesingle t come, not seeing the matter
in that light. He had carried it off as a good joke with Scott, excused
his little wife as well as he could, and played the host so hospitably
that his friend enjoyed the impromptu dinner, and promised to come
again. But John was angry, though he did not show it; he felt that Meg
had got him into a scrape, and then deserted him in his hour of need.
"It wasn\textquotesingle t fair to tell a man to bring folks home any
time, with perfect freedom, and when he took you at your word, to flame
up and blame him, and leave him in the lurch, to be laughed at or
pitied. No, by George, it wasn\textquotesingle t! and Meg must know it."
He had fumed inwardly during the feast, but when the flurry was over,
and he strolled home, after seeing Scott off, a milder mood came over
him. "Poor little thing! it was hard upon her when she tried so heartily
to please me. She was wrong, of course, but then she was young. I must
be patient and teach her." He hoped she had not gone home---he hated
gossip and interference. For a minute he was ruffled again at the mere
thought of it; and then the fear that Meg would cry herself sick
softened his heart, and sent him on at a quicker pace, resolving to be
calm and kind, but firm, quite firm, and show her where she had failed
in her duty to her spouse.

Meg likewise resolved to be "calm and kind, but firm," and show
\emph{him} his duty. She longed to run to meet him, and beg pardon, and
be kissed and comforted, as she was sure of being; but, of course, she
did nothing of the sort, and when she saw John coming, began to hum
quite naturally, as she rocked and sewed, like a lady of leisure in her
best parlor.

John was a little disappointed not to find a tender Niobe; but, feeling
that his dignity demanded the first apology, he made none, only came
leisurely in, and laid himself upon the sofa, with the singularly
relevant remark,---

"We are going to have a new moon, my dear."

"I\textquotesingle ve no objection," was Meg\textquotesingle s equally
soothing remark.

A few other topics of general interest were introduced by Mr. Brooke,
and wet-blanketed by Mrs. Brooke, and conversation languished. John went
to one window, unfolded his paper, and wrapped himself in it,
figuratively speaking. Meg went to the other window, and sewed as if new
rosettes for her slippers were among the necessaries of life. Neither
spoke; both looked quite "calm and firm," and both felt desperately
uncomfortable.

\protect\phantomsection\label{6672479776654687619_37106-h-5.htm.xhtml}{}

\protect\phantomsection\label{6672479776654687619_37106-h-5.htm.xhtml_b131.png}{}
\pandocbounded{\includegraphics[keepaspectratio]{303483661336987339_b131.png}}

"Oh dear," thought Meg, "married life is very trying, and does need
infinite patience, as well as love, as mother says." The word "mother"
suggested other maternal counsels, given long ago, and received with
unbelieving protests.

"John is a good man, but he has his faults, and you must learn to see
and bear with them, remembering your own. He is very decided, but never
will be obstinate, if you reason kindly, not oppose impatiently. He is
very accurate, and particular about the truth---a good trait, though you
call him \textquotesingle fussy.\textquotesingle{} Never deceive him by
look or word, Meg, and he will give you the confidence you deserve, the
support you need. He has a temper, not like ours,---one flash, and then
all over,---but the white, still anger, that is seldom stirred, but once
kindled, is hard to quench. Be careful, very careful, not to wake this
anger against yourself, for peace and happiness depend on keeping his
respect. Watch yourself, be the first to ask pardon if you both err, and
guard against the little piques, misunderstandings, and hasty words that
often pave the way for bitter sorrow and regret."

These words came back to Meg, as she sat sewing in the sunset,
especially the last. This was the first serious disagreement; her own
hasty speeches sounded both silly and unkind, as she recalled them, her
own anger looked childish now, and thoughts of poor John coming home to
such a scene quite melted her heart. She glanced at him with tears in
her eyes, but he did not see them; she put down her work and got up,
thinking, "I \emph{will} be the first to say, \textquotesingle Forgive
me,\textquotesingle" but he did not seem to hear her; she went very
slowly across the room, for pride was hard to swallow, and stood by him,
but he did not turn his head. For a minute she felt as if she really
couldn\textquotesingle t do it; then came the thought, "This is the
beginning, I\textquotesingle ll do my part, and have nothing to reproach
myself with," and stooping down, she softly kissed her husband on the
forehead. Of course that settled it; the penitent kiss was better than a
world of words, and John had her on his knee in a minute, saying
tenderly,---

"It was too bad to laugh at the poor little jelly-pots. Forgive me,
dear, I never will again!"

But he did, oh bless you, yes, hundreds of times, and so did Meg, both
declaring that it was the sweetest jelly they ever made; for family
peace was preserved in that little family jar.

After this, Meg had Mr. Scott to dinner by special invitation, and
served him up a pleasant feast without a cooked wife for the first
course; on which occasion she was so gay and gracious, and made
everything go off so charmingly, that Mr. Scott told John he was a happy
fellow, and shook his head over the hardships of bachelorhood all the
way home.

In the autumn, new trials and experiences came to Meg. Sallie Moffat
renewed her friendship, was always running out for a dish of gossip at
the little house, or inviting "that poor dear" to come in and spend the
day at the big house. It was pleasant, for in dull weather Meg often
felt lonely; all were busy at home, John absent till night, and nothing
to do but sew, or read, or potter about. So it naturally fell out that
Meg got into the way of gadding and gossiping with her friend. Seeing
Sallie\textquotesingle s pretty things made her long for such, and pity
herself because she had not got them. Sallie was very kind, and often
offered her the coveted trifles; but Meg declined them, knowing that
John wouldn\textquotesingle t like it; and then this foolish little
woman went and did what John disliked infinitely worse.

She knew her husband\textquotesingle s income, and she loved to feel
that he trusted her, not only with his happiness, but what some men seem
to value more,---his money. She knew where it was, was free to take what
she liked, and all he asked was that she should keep account of every
penny, pay bills once a month, and remember that she was a poor
man\textquotesingle s wife. Till now, she had done well, been prudent
and exact, kept her little account-books neatly, and showed them to him
monthly without fear. But that autumn the serpent got into
Meg\textquotesingle s paradise, and tempted her, like many a modern Eve,
not with apples, but with dress. Meg didn\textquotesingle t like to be
pitied and made to feel poor; it irritated her, but she was ashamed to
confess it, and now and then she tried to console herself by buying
something pretty, so that Sallie needn\textquotesingle t think she had
to economize. She always felt wicked after it, for the pretty things
were seldom necessaries; but then they cost so little, it
wasn\textquotesingle t worth worrying about; so the trifles increased
unconsciously, and in the shopping excursions she was no longer a
passive looker-on.

But the trifles cost more than one would imagine; and when she cast up
her accounts at the end of the month, the sum total rather scared her.
John was busy that month, and left the bills to her; the next month he
was absent; but the third he had a grand quarterly settling up, and Meg
never forgot it. A few days before she had done a dreadful thing, and it
weighed upon her conscience. Sallie had been buying silks, and Meg
longed for a new one,---just a handsome light one for parties, her black
silk was so common, and thin things for evening wear were only proper
for girls. Aunt March usually gave the sisters a present of twenty-five
dollars apiece at New Year; that was only a month to wait, and here was
a lovely violet silk going at a bargain, and she had the money, if she
only dared to take it. John always said what was his was hers; but would
he think it right to spend not only the prospective five-and-twenty, but
another five-and-twenty out of the household fund? That was the
question. Sallie had urged her to do it, had offered to loan the money,
and with the best intentions in life, had tempted Meg beyond her
strength. In an evil moment the shopman held up the lovely, shimmering
folds, and said, "A bargain, I assure you, ma\textquotesingle am." She
answered, "I\textquotesingle ll take it;" and it was cut off and paid
for, and Sallie had exulted, and she had laughed as if it were a thing
of no consequence, and driven away, feeling as if she had stolen
something, and the police were after her.

\protect\phantomsection\label{6672479776654687619_37106-h-5.htm.xhtml_b132.png}{}
\pandocbounded{\includegraphics[keepaspectratio]{303483661336987339_b132.png}}

When she got home, she tried to assuage the pangs of remorse by
spreading forth the lovely silk; but it looked less silvery now,
didn\textquotesingle t become her, after all, and the words "fifty
dollars" seemed stamped like a pattern down each breadth. She put it
away; but it haunted her, not delightfully, as a new dress should, but
dreadfully, like the ghost of a folly that was not easily laid. When
John got out his books that night, Meg\textquotesingle s heart sank, and
for the first time in her married life, she was afraid of her husband.
The kind, brown eyes looked as if they could be stern; and though he was
unusually merry, she fancied he had found her out, but
didn\textquotesingle t mean to let her know it. The house-bills were all
paid, the books all in order. John had praised her, and was undoing the
old pocket-book which they called the "bank," when Meg, knowing that it
was quite empty, stopped his hand, saying nervously,---

"You haven\textquotesingle t seen my private expense book yet."

John never asked to see it; but she always insisted on his doing so, and
used to enjoy his masculine amazement at the queer things women wanted,
and made him guess what "piping" was, demand fiercely the meaning of a
"hug-me-tight," or wonder how a little thing composed of three rosebuds,
a bit of velvet, and a pair of strings, could possibly be a bonnet, and
cost five or six dollars. That night he looked as if he would like the
fun of quizzing her figures and pretending to be horrified at her
extravagance, as he often did, being particularly proud of his prudent
wife.

The little book was brought slowly out, and laid down before him. Meg
got behind his chair under pretence of smoothing the wrinkles out of his
tired forehead, and standing there, she said, with her panic increasing
with every word,---

"John, dear, I\textquotesingle m ashamed to show you my book, for
I\textquotesingle ve really been dreadfully extravagant lately. I go
about so much I must have things, you know, and Sallie advised my
getting it, so I did; and my New-Year\textquotesingle s money will
partly pay for it: but I was sorry after I\textquotesingle d done it,
for I knew you\textquotesingle d think it wrong in me."

John laughed, and drew her round beside him, saying good-humoredly,
"Don\textquotesingle t go and hide. I won\textquotesingle t beat you if
you \emph{have} got a pair of killing boots; I\textquotesingle m rather
proud of my wife\textquotesingle s feet, and don\textquotesingle t mind
if she does pay eight or nine dollars for her boots, if they are good
ones."

That had been one of her last "trifles," and John\textquotesingle s eye
had fallen on it as he spoke. "Oh, what \emph{will} he say when he comes
to that awful fifty dollars!" thought Meg, with a shiver.

"It\textquotesingle s worse than boots, it\textquotesingle s a silk
dress," she said, with the calmness of desperation, for she wanted the
worst over.

"Well, dear, what is the \textquotesingle dem\textquotesingle d
total,\textquotesingle{} as Mr. Mantalini says?"

That didn\textquotesingle t sound like John, and she knew he was looking
up at her with the straightforward look that she had always been ready
to meet and answer with one as frank till now. She turned the page and
her head at the same time, pointing to the sum which would have been bad
enough without the fifty, but which was appalling to her with that
added. For a minute the room was very still; then John said
slowly,---but she could feel it cost him an effort to express no
displeasure,---

"Well, I don\textquotesingle t know that fifty is much for a dress, with
all the furbelows and notions you have to have to finish it off these
days."

"It isn\textquotesingle t made or trimmed," sighed Meg faintly, for a
sudden recollection of the cost still to be incurred quite overwhelmed
her.

"Twenty-five yards of silk seems a good deal to cover one small woman,
but I\textquotesingle ve no doubt my wife will look as fine as Ned
Moffat\textquotesingle s when she gets it on," said John dryly.

"I know you are angry, John, but I can\textquotesingle t help it. I
don\textquotesingle t mean to waste your money, and I
didn\textquotesingle t think those little things would count up so. I
can\textquotesingle t resist them when I see Sallie buying all she
wants, and pitying me because I don\textquotesingle t. I try to be
contented, but it is hard, and I\textquotesingle m tired of being poor."

The last words were spoken so low she thought he did not hear them, but
he did, and they wounded him deeply, for he had denied himself many
pleasures for Meg\textquotesingle s sake. She could have bitten her
tongue out the minute she had said it, for John pushed the books away,
and got up, saying, with a little quiver in his voice, "I was afraid of
this; I do my best, Meg." If he had scolded her, or even shaken her, it
would not have broken her heart like those few words. She ran to him and
held him close, crying, with repentant tears, "O John, my dear, kind,
hard-working boy, I didn\textquotesingle t mean it! It was so wicked, so
untrue and ungrateful, how could I say it! Oh, how could I say it!"

He was very kind, forgave her readily, and did not utter one reproach;
but Meg knew that she had done and said a thing which would not be
forgotten soon, although he might never allude to it again. She had
promised to love him for better for worse; and then she, his wife, had
reproached him with his poverty, after spending his earnings recklessly.
It was dreadful; and the worst of it was John went on so quietly
afterward, just as if nothing had happened, except that he stayed in
town later, and worked at night when she had gone to cry herself to
sleep. A week of remorse nearly made Meg sick; and the discovery that
John had countermanded the order for his new great-coat reduced her to a
state of despair which was pathetic to behold. He had simply said, in
answer to her surprised inquiries as to the change, "I
can\textquotesingle t afford it, my dear."

Meg said no more, but a few minutes after he found her in the hall, with
her face buried in the old great-coat, crying as if her heart would
break.

They had a long talk that night, and Meg learned to love her husband
better for his poverty, because it seemed to have made a man of him,
given him the strength and courage to fight his own way, and taught him
a tender patience with which to bear and comfort the natural longings
and failures of those he loved.

Next day she put her pride in her pocket, went to Sallie, told the
truth, and asked her to buy the silk as a favor. The good-natured Mrs.
Moffat willingly did so, and had the delicacy not to make her a present
of it immediately afterward. Then Meg ordered home the great-coat, and,
when John arrived, she put it on, and asked him how he liked her new
silk gown. One can imagine what answer he made, how he received his
present, and what a blissful state of things ensued. John came home
early, Meg gadded no more; and that great-coat was put on in the morning
by a very happy husband, and taken off at night by a most devoted little
wife. So the year rolled round, and at midsummer there came to Meg a new
experience,---the deepest and tenderest of a woman\textquotesingle s
life.

Laurie came sneaking into the kitchen of the Dove-cote, one Saturday,
with an excited face, and was received with the clash of cymbals; for
Hannah clapped her hands with a saucepan in one and the cover in the
other.

"How\textquotesingle s the little mamma? Where is everybody? Why
didn\textquotesingle t you tell me before I came home?" began Laurie, in
a loud whisper.

"Happy as a queen, the dear! Every soul of \textquotesingle em is
upstairs a worshipin\textquotesingle; we didn\textquotesingle t want no
hurrycanes round. Now you go into the parlor, and I\textquotesingle ll
send \textquotesingle em down to you," with which somewhat involved
reply Hannah vanished, chuckling ecstatically.

Presently Jo appeared, proudly bearing a flannel bundle laid forth upon
a large pillow. Jo\textquotesingle s face was very sober, but her eyes
twinkled, and there was an odd sound in her voice of repressed emotion
of some sort.

"Shut your eyes and hold out your arms," she said invitingly.

Laurie backed precipitately into a corner, and put his hands behind him
with an imploring gesture: "No, thank you, I\textquotesingle d rather
not. I shall drop it or smash it, as sure as fate."

"Then you sha\textquotesingle n\textquotesingle t see your nevvy," said
Jo decidedly, turning as if to go.

"I will, I will! only you must be responsible for damages;" and, obeying
orders, Laurie heroically shut his eyes while something was put into his
arms. A peal of laughter from Jo, Amy, Mrs. March, Hannah, and John
caused him to open them the next minute, to find himself invested with
two babies instead of one.

\protect\phantomsection\label{6672479776654687619_37106-h-5.htm.xhtml_b133.png}{}
\pandocbounded{\includegraphics[keepaspectratio]{303483661336987339_b133.png}}

No wonder they laughed, for the expression of his face was droll enough
to convulse a Quaker, as he stood and stared wildly from the unconscious
innocents to the hilarious spectators, with such dismay that Jo sat down
on the floor and screamed.

"Twins, by Jupiter!" was all he said for a minute; then, turning to the
women with an appealing look that was comically piteous, he added, "Take
\textquotesingle em quick, somebody! I\textquotesingle m going to laugh,
and I shall drop \textquotesingle em."

John rescued his babies, and marched up and down, with one on each arm,
as if already initiated into the mysteries of baby-tending, while Laurie
laughed till the tears ran down his cheeks.

"It\textquotesingle s the best joke of the season, isn\textquotesingle t
it? I wouldn\textquotesingle t have you told, for I set my heart on
surprising you, and I flatter myself I\textquotesingle ve done it," said
Jo, when she got her breath.

"I never was more staggered in my life. Isn\textquotesingle t it fun?
Are they boys? What are you going to name them? Let\textquotesingle s
have another look. Hold me up, Jo; for upon my life it\textquotesingle s
one too many for me," returned Laurie, regarding the infants with the
air of a big, benevolent Newfoundland looking at a pair of infantile
kittens.

"Boy and girl. Aren\textquotesingle t they beauties?" said the proud
papa, beaming upon the little, red squirmers as if they were unfledged
angels.

"Most remarkable children I ever saw. Which is which?" and Laurie bent
like a well-sweep to examine the prodigies.

"Amy put a blue ribbon on the boy and a pink on the girl, French
fashion, so you can always tell. Besides, one has blue eyes and one
brown. Kiss them, Uncle Teddy," said wicked Jo.

"I\textquotesingle m afraid they mightn\textquotesingle t like it,"
began Laurie, with unusual timidity in such matters.

"Of course they will; they are used to it now. Do it this minute, sir!"
commanded Jo, fearing he might propose a proxy.

Laurie screwed up his face, and obeyed with a gingerly peck at each
little cheek that produced another laugh, and made the babies squeal.

"There, I knew they didn\textquotesingle t like it!
That\textquotesingle s the boy; see him kick; he hits out with his fists
like a good one. Now then, young Brooke, pitch into a man of your own
size, will you?" cried Laurie, delighted with a poke in the face from a
tiny fist, flapping aimlessly about.

"He\textquotesingle s to be named John Laurence, and the girl Margaret,
after mother and grandmother. We shall call her Daisy, so as not to have
two Megs, and I suppose the mannie will be Jack, unless we find a better
name," said Amy, with aunt-like interest.

"Name him Demijohn, and call him \textquotesingle Demi\textquotesingle{}
for short," said Laurie.

"Daisy and Demi,---just the thing! I \emph{knew} Teddy would do it,"
cried Jo, clapping her hands.

Teddy certainly had done it that time, for the babies were "Daisy" and
"Demi" to the end of the chapter.

\begin{center}\rule{0.5\linewidth}{0.5pt}\end{center}

\subsection{XXIX.
Calls.}\label{6672479776654687619_37106-h-5.htm.xhtml_pgepubid00031}

\protect\phantomsection\label{6672479776654687619_37106-h-5.htm.xhtml_XXIX}{}\hyperref[6672479776654687619_37106-h-0.htm.xhtml_contents2]{XXIX.}

CALLS.

\protect\phantomsection\label{6672479776654687619_37106-h-5.htm.xhtml_b134.png}{}
\pandocbounded{\includegraphics[keepaspectratio]{303483661336987339_b134.png}}

{"Come,} Jo, it\textquotesingle s time."

"For what?"

"You don\textquotesingle t mean to say you have forgotten that you
promised to make half a dozen calls with me to-day?"

"I\textquotesingle ve done a good many rash and foolish things in my
life, but I don\textquotesingle t think I ever was mad enough to say
I\textquotesingle d make six calls in one day, when a single one upsets
me for a week."

"Yes, you did; it was a bargain between us. I was to finish the crayon
of Beth for you, and you were to go properly with me, and return our
neighbors\textquotesingle{} visits."

"If it was fair---that was in the bond; and I stand to the letter of my
bond, Shylock. There is a pile of clouds in the east;
it\textquotesingle s \emph{not} fair, and I don\textquotesingle t go."

"Now, that\textquotesingle s shirking. It\textquotesingle s a lovely
day, no prospect of rain, and you pride yourself on keeping promises; so
be honorable; come and do your duty, and then be at peace for another
six months."

At that minute Jo was particularly absorbed in dressmaking; for she was
mantua-maker general to the family, and took especial credit to herself
because she could use a needle as well as a pen. It was very provoking
to be arrested in the act of a first trying-on, and ordered out to make
calls in her best array, on a warm July day. She hated calls of the
formal sort, and never made any till Amy compelled her with a bargain,
bribe, or promise. In the present instance, there was no escape; and
having clashed her scissors rebelliously, while protesting that she
smelt thunder, she gave in, put away her work, and taking up her hat and
gloves with an air of resignation, told Amy the victim was ready.

"Jo March, you are perverse enough to provoke a saint! You
don\textquotesingle t intend to make calls in that state, I hope," cried
Amy, surveying her with amazement.

"Why not? I\textquotesingle m neat and cool and comfortable; quite
proper for a dusty walk on a warm day. If people care more for my
clothes than they do for me, I don\textquotesingle t wish to see them.
You can dress for both, and be as elegant as you please: it pays for you
to be fine; it doesn\textquotesingle t for me, and furbelows only worry
me."

"Oh dear!" sighed Amy; "now she\textquotesingle s in a contrary fit, and
will drive me distracted before I can get her properly ready.
I\textquotesingle m sure it\textquotesingle s no pleasure to me to go
to-day, but it\textquotesingle s a debt we owe society, and
there\textquotesingle s no one to pay it but you and me.
I\textquotesingle ll do anything for you, Jo, if you\textquotesingle ll
only dress yourself nicely, and come and help me do the civil. You can
talk so well, look so aristocratic in your best things, and behave so
beautifully, if you try, that I\textquotesingle m proud of you.
I\textquotesingle m afraid to go alone; do come and take care of me."

"You\textquotesingle re an artful little puss to flatter and wheedle
your cross old sister in that way. The idea of my being aristocratic and
well-bred, and your being afraid to go anywhere alone! I
don\textquotesingle t know which is the most absurd. Well,
I\textquotesingle ll go if I must, and do my best. You shall be
commander of the expedition, and I\textquotesingle ll obey blindly; will
that satisfy you?" said Jo, with a sudden change from perversity to
lamb-like submission.

"You\textquotesingle re a perfect cherub! Now put on all your best
things, and I\textquotesingle ll tell you how to behave at each place,
so that you will make a good impression. I want people to like you, and
they would if you\textquotesingle d only try to be a little more
agreeable. Do your hair the pretty way, and put the pink rose in your
bonnet; it\textquotesingle s becoming, and you look too sober in your
plain suit. Take your light gloves and the embroidered handkerchief.
We\textquotesingle ll stop at Meg\textquotesingle s, and borrow her
white sunshade, and then you can have my dove-colored one."

While Amy dressed, she issued her orders, and Jo obeyed them; not
without entering her protest, however, for she sighed as she rustled
into her new organdie, frowned darkly at herself as she tied her bonnet
strings in an irreproachable bow, wrestled viciously with pins as she
put on her collar, wrinkled up her features generally as she shook out
the handkerchief, whose embroidery was as irritating to her nose as the
present mission was to her feelings; and when she had squeezed her hands
into tight gloves with three buttons and a tassel, as the last touch of
elegance, she turned to Amy with an imbecile expression of countenance,
saying meekly,---

"I\textquotesingle m perfectly miserable; but if you consider me
presentable, I die happy."

"You are highly satisfactory; turn slowly round, and let me get a
careful view." Jo revolved, and Amy gave a touch here and there, then
fell back, with her head on one side, observing graciously, "Yes,
you\textquotesingle ll do; your head is all I could ask, for that white
bonnet \emph{with} the rose is quite ravishing. Hold back your
shoulders, and carry your hands easily, no matter if your gloves do
pinch. There\textquotesingle s one thing you can do well, Jo, that is,
wear a shawl---I can\textquotesingle t; but it\textquotesingle s very
nice to see you, and I\textquotesingle m so glad Aunt March gave you
that lovely one; it\textquotesingle s simple, but handsome, and those
folds over the arm are really artistic. Is the point of my mantle in the
middle, and have I looped my dress evenly? I like to show my boots, for
my feet \emph{are} pretty, though my nose isn\textquotesingle t."

"You are a thing of beauty and a joy forever," said Jo, looking through
her hand with the air of a connoisseur at the blue feather against the
gold hair. "Am I to drag my best dress through the dust, or loop it up,
please, ma\textquotesingle am?"

"Hold it up when you walk, but drop it in the house; the sweeping style
suits you best, and you must learn to trail your skirts gracefully. You
haven\textquotesingle t half buttoned one cuff; do it at once.
You\textquotesingle ll never look finished if you are not careful about
the little details, for they make up the pleasing whole."

Jo sighed, and proceeded to burst the buttons off her glove, in doing up
her cuff; but at last both were ready, and sailed away, looking as
"pretty as picters," Hannah said, as she hung out of the upper window to
watch them.

"Now, Jo dear, the Chesters consider themselves very elegant people, so
I want you to put on your best deportment. Don\textquotesingle t make
any of your abrupt remarks, or do anything odd, will you? Just be calm,
cool, and quiet,---that\textquotesingle s safe and ladylike; and you can
easily do it for fifteen minutes," said Amy, as they approached the
first place, having borrowed the white parasol and been inspected by
Meg, with a baby on each arm.

"Let me see. \textquotesingle Calm, cool, and
quiet,\textquotesingle---yes, I think I can promise that.
I\textquotesingle ve played the part of a prim young lady on the stage,
and I\textquotesingle ll try it off. My powers are great, as you shall
see; so be easy in your mind, my child."

Amy looked relieved, but naughty Jo took her at her word; for, during
the first call, she sat with every limb gracefully composed, every fold
correctly draped, calm as a summer sea, cool as a snow-bank, and as
silent as a sphinx. In vain Mrs. Chester alluded to her "charming
novel," and the Misses Chester introduced parties, picnics, the opera,
and the fashions; each and all were answered by a smile, a bow, and a
demure "Yes" or "No," with the chill on. In vain Amy telegraphed the
word "Talk," tried to draw her out, and administered covert pokes with
her foot. Jo sat as if blandly unconscious of it all, with deportment
like Maud\textquotesingle s face, "icily regular, splendidly null."

"What a haughty, uninteresting creature that oldest Miss March is!" was
the unfortunately audible remark of one of the ladies, as the door
closed upon their guests. Jo laughed noiselessly all through the hall,
but Amy looked disgusted at the failure of her instructions, and very
naturally laid the blame upon Jo.

"How could you mistake me so? I merely meant you to be properly
dignified and composed, and you made yourself a perfect stock and stone.
Try to be sociable at the Lambs\textquotesingle, gossip as other girls
do, and be interested in dress and flirtations and whatever nonsense
comes up. They move in the best society, are valuable persons for us to
know, and I wouldn\textquotesingle t fail to make a good impression
there for anything."

"I\textquotesingle ll be agreeable; I\textquotesingle ll gossip and
giggle, and have horrors and raptures over any trifle you like. I rather
enjoy this, and now I\textquotesingle ll imitate what is called
\textquotesingle a charming girl;\textquotesingle{} I can do it, for I
have May Chester as a model, and I\textquotesingle ll improve upon her.
See if the Lambs don\textquotesingle t say, \textquotesingle What a
lively, nice creature that Jo March is!\textquotesingle"

Amy felt anxious, as well she might, for when Jo turned freakish there
was no knowing where she would stop. Amy\textquotesingle s face was a
study when she saw her sister skim into the next drawing-room, kiss all
the young ladies with effusion, beam graciously upon the young
gentlemen, and join in the chat with a spirit which amazed the beholder.
Amy was taken possession of by Mrs. Lamb, with whom she was a favorite,
and forced to hear a long account of Lucretia\textquotesingle s last
attack, while three delightful young gentlemen hovered near, waiting for
a pause when they might rush in and rescue her. So situated, she was
powerless to check Jo, who seemed possessed by a spirit of mischief, and
talked away as volubly as the old lady. A knot of heads gathered about
her, and Amy strained her ears to hear what was going on; for broken
sentences filled her with alarm, round eyes and uplifted hands tormented
her with curiosity, and frequent peals of laughter made her wild to
share the fun. One may imagine her suffering on overhearing fragments of
this sort of conversation:---

"She rides splendidly,---who taught her?"

"No one; she used to practise mounting, holding the reins, and sitting
straight on an old saddle in a tree. Now she rides anything, for she
doesn\textquotesingle t know what fear is, and the stable-man lets her
have horses cheap, because she trains them to carry ladies so well. She
has such a passion for it, I often tell her if everything else fails she
can be a horse-breaker, and get her living so."

At this awful speech Amy contained herself with difficulty, for the
impression was being given that she was rather a fast young lady, which
was her especial aversion. But what could she do? for the old lady was
in the middle of her story, and long before it was done Jo was off
again, making more droll revelations, and committing still more fearful
blunders.

"Yes, Amy was in despair that day, for all the good beasts were gone,
and of three left, one was lame, one blind, and the other so balky that
you had to put dirt in his mouth before he would start. Nice animal for
a pleasure party, wasn\textquotesingle t it?"

"Which did she choose?" asked one of the laughing gentlemen, who enjoyed
the subject.

"None of them; she heard of a young horse at the farmhouse over the
river, and, though a lady had never ridden him, she resolved to try,
because he was handsome and spirited. Her struggles were really
pathetic; there was no one to bring the horse to the saddle, so she took
the saddle to the horse. My dear creature, she actually rowed it over
the river, put it on her head, and marched up to the barn to the utter
amazement of the old man!"

\protect\phantomsection\label{6672479776654687619_37106-h-5.htm.xhtml_b135.png}{}
\pandocbounded{\includegraphics[keepaspectratio]{303483661336987339_b135.png}}

"Did she ride the horse?"

"Of course she did, and had a capital time. I expected to see her
brought home in fragments, but she managed him perfectly, and was the
life of the party."

"Well, I call that plucky!" and young Mr. Lamb turned an approving
glance upon Amy, wondering what his mother could be saying to make the
girl look so red and uncomfortable.

She was still redder and more uncomfortable a moment after, when a
sudden turn in the conversation introduced the subject of dress. One of
the young ladies asked Jo where she got the pretty drab hat she wore to
the picnic; and stupid Jo, instead of mentioning the place where it was
bought two years ago, must needs answer, with unnecessary frankness,
"Oh, Amy painted it; you can\textquotesingle t buy those soft shades, so
we paint ours any color we like. It\textquotesingle s a great comfort to
have an artistic sister."

"Isn\textquotesingle t that an original idea?" cried Miss Lamb, who
found Jo great fun.

"That\textquotesingle s nothing compared to some of her brilliant
performances. There\textquotesingle s nothing the child
can\textquotesingle t do. Why, she wanted a pair of blue boots for
Sallie\textquotesingle s party, so she just painted her soiled white
ones the loveliest shade of sky-blue you ever saw, and they looked
exactly like satin," added Jo, with an air of pride in her
sister\textquotesingle s accomplishments that exasperated Amy till she
felt that it would be a relief to throw her card-case at her.

"We read a story of yours the other day, and enjoyed it very much,"
observed the elder Miss Lamb, wishing to compliment the literary lady,
who did not look the character just then, it must be confessed.

Any mention of her "works" always had a bad effect upon Jo, who either
grew rigid and looked offended, or changed the subject with a
\emph{brusque} remark, as now. "Sorry you could find nothing better to
read. I write that rubbish because it sells, and ordinary people like
it. Are you going to New York this winter?"

As Miss Lamb had "enjoyed" the story, this speech was not exactly
grateful or complimentary. The minute it was made Jo saw her mistake;
but, fearing to make the matter worse, suddenly remembered that it was
for her to make the first move toward departure, and did so with an
abruptness that left three people with half-finished sentences in their
mouths.

"Amy, we \emph{must} go. \emph{Good}-by, dear; \emph{do} come and see
us; we are \emph{pining} for a visit. I don\textquotesingle t dare to
ask \emph{you}, Mr. Lamb; but if you \emph{should} come, I
don\textquotesingle t think I shall have the heart to send you away."

Jo said this with such a droll imitation of May
Chester\textquotesingle s gushing style that Amy got out of the room as
rapidly as possible, feeling a strong desire to laugh and cry at the
same time.

"Didn\textquotesingle t I do that well?" asked Jo, with a satisfied air,
as they walked away.

"Nothing could have been worse," was Amy\textquotesingle s crushing
reply. "What possessed you to tell those stories about my saddle, and
the hats and boots, and all the rest of it?"

"Why, it\textquotesingle s funny, and amuses people. They know we are
poor, so it\textquotesingle s no use pretending that we have grooms, buy
three or four hats a season, and have things as easy and fine as they
do."

"You needn\textquotesingle t go and tell them all our little shifts, and
expose our poverty in that perfectly unnecessary way. You
haven\textquotesingle t a bit of proper pride, and never will learn when
to hold your tongue and when to speak," said Amy despairingly.

Poor Jo looked abashed, and silently chafed the end of her nose with the
stiff handkerchief, as if performing a penance for her misdemeanors.

"How shall I behave here?" she asked, as they approached the third
mansion.

"Just as you please; I wash my hands of you," was Amy\textquotesingle s
short answer.

"Then I\textquotesingle ll enjoy myself. The boys are at home, and
we\textquotesingle ll have a comfortable time. Goodness knows I need a
little change, for elegance has a bad effect upon my constitution,"
returned Jo gruffly, being disturbed by her failures to suit.

An enthusiastic welcome from three big boys and several pretty children
speedily soothed her ruffled feelings; and, leaving Amy to entertain the
hostess and Mr. Tudor, who happened to be calling likewise, Jo devoted
herself to the young folks, and found the change refreshing. She
listened to college stories with deep interest, caressed pointers and
poodles without a murmur, agreed heartily that "Tom Brown was a brick,"
regardless of the improper form of praise; and when one lad proposed a
visit to his turtle-tank, she went with an alacrity which caused mamma
to smile upon her, as that motherly lady settled the cap which was left
in a ruinous condition by filial hugs, bear-like but affectionate, and
dearer to her than the most faultless \emph{coiffure} from the hands of
an inspired Frenchwoman.

Leaving her sister to her own devices, Amy proceeded to enjoy herself to
her heart\textquotesingle s content. Mr. Tudor\textquotesingle s uncle
had married an English lady who was third cousin to a living lord, and
Amy regarded the whole family with great respect; for, in spite of her
American birth and breeding, she possessed that reverence for titles
which haunts the best of us,---that unacknowledged loyalty to the early
faith in kings which set the most democratic nation under the sun in a
ferment at the coming of a royal yellow-haired laddie, some years ago,
and which still has something to do with the love the young country
bears the old, like that of a big son for an imperious little mother,
who held him while she could, and let him go with a farewell scolding
when he rebelled. But even the satisfaction of talking with a distant
connection of the British nobility did not render Amy forgetful of time;
and when the proper number of minutes had passed, she reluctantly tore
herself from this aristocratic society, and looked about for Jo,
fervently hoping that her incorrigible sister would not be found in any
position which should bring disgrace upon the name of March.

\protect\phantomsection\label{6672479776654687619_37106-h-5.htm.xhtml_b136.png}{}
\pandocbounded{\includegraphics[keepaspectratio]{303483661336987339_b136.png}}

It might have been worse, but Amy considered it bad; for Jo sat on the
grass, with an encampment of boys about her, and a dirty-footed dog
reposing on the skirt of her state and festival dress, as she related
one of Laurie\textquotesingle s pranks to her admiring audience. One
small child was poking turtles with Amy\textquotesingle s cherished
parasol, a second was eating gingerbread over Jo\textquotesingle s best
bonnet, and a third playing ball with her gloves. But all were enjoying
themselves; and when Jo collected her damaged property to go, her escort
accompanied her, begging her to come again, "it was such fun to hear
about Laurie\textquotesingle s larks."

"Capital boys, aren\textquotesingle t they? I feel quite young and brisk
again after that," said Jo, strolling along with her hands behind her,
partly from habit, partly to conceal the bespattered parasol.

"Why do you always avoid Mr. Tudor?" asked Amy, wisely refraining from
any comment upon Jo\textquotesingle s dilapidated appearance.

"Don\textquotesingle t like him; he puts on airs, snubs his sisters,
worries his father, and doesn\textquotesingle t speak respectfully of
his mother. Laurie says he is fast, and \emph{I} don\textquotesingle t
consider him a desirable acquaintance; so I let him alone."

"You might treat him civilly, at least. You gave him a cool nod; and
just now you bowed and smiled in the politest way to Tommy Chamberlain,
whose father keeps a grocery store. If you had just reversed the nod and
the bow, it would have been right," said Amy reprovingly.

"No, it wouldn\textquotesingle t," returned perverse Jo; "I neither
like, respect, nor admire Tudor, though his
grandfather\textquotesingle s uncle\textquotesingle s
nephew\textquotesingle s niece \emph{was} third cousin to a lord. Tommy
is poor and bashful and good and very clever; I think well of him, and
like to show that I do, for he \emph{is} a gentleman in spite of the
brown-paper parcels."

"It\textquotesingle s no use trying to argue with you," began Amy.

"Not the least, my dear," interrupted Jo; "so let us look amiable, and
drop a card here, as the Kings are evidently out, for which
I\textquotesingle m deeply grateful."

The family card-case having done its duty, the girls walked on, and Jo
uttered another thanksgiving on reaching the fifth house, and being told
that the young ladies were engaged.

"Now let us go home, and never mind Aunt March to-day. We can run down
there any time, and it\textquotesingle s really a pity to trail through
the dust in our best bibs and tuckers, when we are tired and cross."

"Speak for yourself, if you please. Aunt likes to have us pay her the
compliment of coming in style, and making a formal call;
it\textquotesingle s a little thing to do, but it gives her pleasure,
and I don\textquotesingle t believe it will hurt your things half so
much as letting dirty dogs and clumping boys spoil them. Stoop down, and
let me take the crumbs off of your bonnet."

"What a good girl you are, Amy!" said Jo, with a repentant glance from
her own damaged costume to that of her sister, which was fresh and
spotless still. "I wish it was as easy for me to do little things to
please people as it is for you. I think of them, but it takes too much
time to do them; so I wait for a chance to confer a great favor, and let
the small ones slip; but they tell best in the end, I fancy."

Amy smiled, and was mollified at once, saying with a maternal air,---

"Women should learn to be agreeable, particularly poor ones; for they
have no other way of repaying the kindnesses they receive. If
you\textquotesingle d remember that, and practise it,
you\textquotesingle d be better liked than I am, because there is more
of you."

"I\textquotesingle m a crotchety old thing, and always shall be, but
I\textquotesingle m willing to own that you are right; only
it\textquotesingle s easier for me to risk my life for a person than to
be pleasant to him when I don\textquotesingle t feel like it.
It\textquotesingle s a great misfortune to have such strong likes and
dislikes, isn\textquotesingle t it?"

"It\textquotesingle s a greater not to be able to hide them. I
don\textquotesingle t mind saying that I don\textquotesingle t approve
of Tudor any more than you do; but I\textquotesingle m not called upon
to tell him so; neither are you, and there is no use in making yourself
disagreeable because he is."

"But I think girls ought to show when they disapprove of young men; and
how can they do it except by their manners? Preaching does not do any
good, as I know to my sorrow, since I\textquotesingle ve had Teddy to
manage; but there are many little ways in which I can influence him
without a word, and I say we \emph{ought} to do it to others if we can."

"Teddy is a remarkable boy, and can\textquotesingle t be taken as a
sample of other boys," said Amy, in a tone of solemn conviction, which
would have convulsed the "remarkable boy," if he had heard it. "If we
were belles, or women of wealth and position, we might do something,
perhaps; but for us to frown at one set of young gentlemen because we
don\textquotesingle t approve of them, and smile upon another set
because we do, wouldn\textquotesingle t have a particle of effect, and
we should only be considered odd and puritanical."

"So we are to countenance things and people which we detest, merely
because we are not belles and millionaires, are we?
That\textquotesingle s a nice sort of morality."

"I can\textquotesingle t argue about it, I only know that
it\textquotesingle s the way of the world; and people who set themselves
against it only get laughed at for their pains. I don\textquotesingle t
like reformers, and I hope you will never try to be one."

"I do like them, and I shall be one if I can; for in spite of the
laughing, the world would never get on without them. We
can\textquotesingle t agree about that, for you belong to the old set,
and I to the new: you will get on the best, but I shall have the
liveliest time of it. I should rather enjoy the brickbats and hooting, I
think."

"Well, compose yourself now, and don\textquotesingle t worry aunt with
your new ideas."

"I\textquotesingle ll try not to, but I\textquotesingle m always
possessed to burst out with some particularly blunt speech or
revolutionary sentiment before her; it\textquotesingle s my doom, and I
can\textquotesingle t help it."

They found Aunt Carrol with the old lady, both absorbed in some very
interesting subject; but they dropped it as the girls came in, with a
conscious look which betrayed that they had been talking about their
nieces. Jo was not in a good humor, and the perverse fit returned; but
Amy, who had virtuously done her duty, kept her temper, and pleased
everybody, was in a most angelic frame of mind. This amiable spirit was
felt at once, and both the aunts "my deared" her affectionately, looking
what they afterwards said emphatically,---"That child improves every
day."

"Are you going to help about the fair, dear?" asked Mrs. Carrol, as Amy
sat down beside her with the confiding air elderly people like so well
in the young.

"Yes, aunt. Mrs. Chester asked me if I would, and I offered to tend a
table, as I have nothing but my time to give."

"I\textquotesingle m not," put in Jo decidedly. "I hate to be
patronized, and the Chesters think it\textquotesingle s a great favor to
allow us to help with their highly connected fair. I wonder you
consented, Amy: they only want you to work."

"I am willing to work: it\textquotesingle s for the freedmen as well as
the Chesters, and I think it very kind of them to let me share the labor
and the fun. Patronage does not trouble me when it is well meant."

"Quite right and proper. I like your grateful spirit, my dear;
it\textquotesingle s a pleasure to help people who appreciate our
efforts: some do not, and that is trying," observed Aunt March, looking
over her spectacles at Jo, who sat apart, rocking herself, with a
somewhat morose expression.

\protect\phantomsection\label{6672479776654687619_37106-h-5.htm.xhtml_b137.png}{}
\pandocbounded{\includegraphics[keepaspectratio]{303483661336987339_b137.png}}

If Jo had only known what a great happiness was wavering in the balance
for one of them, she would have turned dovelike in a minute; but,
unfortunately, we don\textquotesingle t have windows in our breasts, and
cannot see what goes on in the minds of our friends; better for us that
we cannot as a general thing, but now and then it would be such a
comfort, such a saving of time and temper. By her next speech, Jo
deprived herself of several years of pleasure, and received a timely
lesson in the art of holding her tongue.

"I don\textquotesingle t like favors; they oppress and make me feel like
a slave. I\textquotesingle d rather do everything for myself, and be
perfectly independent."

"Ahem!" coughed Aunt Carrol softly, with a look at Aunt March.

"I told you so," said Aunt March, with a decided nod to Aunt Carrol.

Mercifully unconscious of what she had done, Jo sat with her nose in the
air, and a revolutionary aspect which was anything but inviting.

"Do you speak French, dear?" asked Mrs. Carrol, laying her hand on
Amy\textquotesingle s.

"Pretty well, thanks to Aunt March, who lets Esther talk to me as often
as I like," replied Amy, with a grateful look, which caused the old lady
to smile affably.

"How are you about languages?" asked Mrs. Carrol of Jo.

"Don\textquotesingle t know a word; I\textquotesingle m very stupid
about studying anything; can\textquotesingle t bear French,
it\textquotesingle s such a slippery, silly sort of language," was the
\emph{brusque} reply.

Another look passed between the ladies, and Aunt March said to Amy, "You
are quite strong and well, now, dear, I believe? Eyes
don\textquotesingle t trouble you any more, do they?"

"Not at all, thank you, ma\textquotesingle am. I\textquotesingle m very
well, and mean to do great things next winter, so that I may be ready
for Rome, whenever that joyful time arrives."

"Good girl! You deserve to go, and I\textquotesingle m sure you will
some day," said Aunt March, with an approving pat on the head, as Amy
picked up her ball for her.

\ul{"Cross-patch, draw the latch,}

Sit by the fire and spin,"

squalled Polly, bending down from his perch on the back of her chair to
peep into Jo\textquotesingle s face, with such a comical air of
impertinent inquiry that it was impossible to help laughing.

"Most observing bird," said the old lady.

"Come and take a walk, my dear?" cried Polly, hopping toward the
china-closet, with a look suggestive of lump-sugar.

"Thank you, I will. Come, Amy;" and Jo brought the visit to an end,
feeling more strongly than ever that calls did have a bad effect upon
her constitution. She shook hands in a gentlemanly manner, but Amy
kissed both the aunts, and the girls departed, leaving behind them the
impression of shadow and sunshine; which impression caused Aunt March to
say, as they vanished,---

"You\textquotesingle d better do it, Mary; I\textquotesingle ll supply
the money," and Aunt Carrol to reply decidedly, "I certainly will, if
her father and mother consent."

\protect\phantomsection\label{6672479776654687619_37106-h-5.htm.xhtml_b138.png}{}
\pandocbounded{\includegraphics[keepaspectratio]{303483661336987339_b138.png}}

\begin{center}\rule{0.5\linewidth}{0.5pt}\end{center}

\subsection{XXX.
Consequences.}\label{6672479776654687619_37106-h-5.htm.xhtml_pgepubid00032}

\protect\phantomsection\label{6672479776654687619_37106-h-5.htm.xhtml_b139.png}{}
\pandocbounded{\includegraphics[keepaspectratio]{303483661336987339_b139.png}}

\protect\phantomsection\label{6672479776654687619_37106-h-5.htm.xhtml_XXX}{}\hyperref[6672479776654687619_37106-h-0.htm.xhtml_contents2]{XXX.}

CONSEQUENCES.

{Mrs. Chester\textquotesingle s} fair was so very elegant and select
that it was considered a great honor by the young ladies of the
neighborhood to be invited to take a table, and every one was much
interested in the matter. Amy was asked, but Jo was not, which was
fortunate for all parties, as her elbows were decidedly akimbo at this
period of her life, and it took a good many hard knocks to teach her how
to get on easily. The "haughty, uninteresting creature" was let severely
alone; but Amy\textquotesingle s talent and taste were duly complimented
by the offer of the art-table, and she exerted herself to prepare and
secure appropriate and valuable contributions to it.

Everything went on smoothly till the day before the fair opened; then
there occurred one of the little skirmishes which it is almost
impossible to avoid, when some five and twenty women, old and young,
with all their private piques and prejudices, try to work together.

May Chester was rather jealous of Amy because the latter was a greater
favorite than herself, and, just at this time, several trifling
circumstances occurred to increase the feeling. Amy\textquotesingle s
dainty pen-and-ink work entirely eclipsed May\textquotesingle s painted
vases,---that was one thorn; then the all-conquering Tudor had danced
four times with Amy, at a late party, and only once with May,---that was
thorn number two; but the chief grievance that rankled in her soul, and
gave her an excuse for her unfriendly conduct, was a rumor which some
obliging gossip had whispered to her, that the March girls had made fun
of her at the Lambs\textquotesingle. All the blame of this should have
fallen upon Jo, for her naughty imitation had been too lifelike to
escape detection, and the frolicsome Lambs had permitted the joke to
escape. No hint of this had reached the culprits, however, and
Amy\textquotesingle s dismay can be imagined, when, the very evening
before the fair, as she was putting the last touches to her pretty
table, Mrs. Chester, who, of course, resented the supposed ridicule of
her daughter, said, in a bland tone, but with a cold look,---

"I find, dear, that there is some feeling among the young ladies about
my giving this table to any one but my girls. As this is the most
prominent, and some say the most attractive table of all, and they are
the chief getters-up of the fair, it is thought best for them to take
this place. I\textquotesingle m sorry, but I know you are too sincerely
interested in the cause to mind a little personal disappointment, and
you shall have another table if you like."

Mrs. Chester had fancied beforehand that it would be easy to deliver
this little speech; but when the time came, she found it rather
difficult to utter it naturally, with Amy\textquotesingle s unsuspicious
eyes looking straight at her, full of surprise and trouble.

Amy felt that there was something behind this, but could not guess what,
and said quietly, feeling hurt, and showing that she did,---

"Perhaps you had rather I took no table at all?"

"Now, my dear, don\textquotesingle t have any ill feeling, I beg;
it\textquotesingle s merely a matter of expediency, you see; my girls
will naturally take the lead, and this table is considered their proper
place. \emph{I} think it very appropriate to you, and feel very grateful
for your efforts to make it so pretty; but we must give up our private
wishes, of course, and I will see that you have a good place elsewhere.
Wouldn\textquotesingle t you like the flower-table? The little girls
undertook it, but they are discouraged. You could make a charming thing
of it, and the flower-table is always attractive, you know."

"Especially to gentlemen," added May, with a look which enlightened Amy
as to one cause of her sudden fall from favor. She colored angrily, but
took no other notice of that girlish sarcasm, and answered, with
unexpected amiability,---

"It shall be as you please, Mrs. Chester. I\textquotesingle ll give up
my place here at once, and attend to the flowers, if you like."

"You can put your own things on your own table, if you prefer," began
May, feeling a little conscience-stricken, as she looked at the pretty
racks, the painted shells, and quaint illuminations Amy had so carefully
made and so gracefully arranged. She meant it kindly, but Amy mistook
her meaning, and said quickly,---

"Oh, certainly, if they are in your way;" and sweeping her contributions
into her apron, pell-mell, she walked off, feeling that herself and her
works of art had been insulted past forgiveness.

"Now she\textquotesingle s mad. Oh, dear, I wish I
hadn\textquotesingle t asked you to speak, mamma," said May, looking
disconsolately at the empty spaces on her table.

"Girls\textquotesingle{} quarrels are soon over," returned her mother,
feeling a trifle ashamed of her own part in this one, as well she might.

The little girls hailed Amy and her treasures with delight, which
cordial reception somewhat soothed her perturbed spirit, and she fell to
work, determined to succeed florally, if she could not artistically. But
everything seemed against her: it was late, and she was tired; every one
was too busy with their own affairs to help her; and the little girls
were only hindrances, for the dears fussed and chattered like so many
magpies, making a great deal of confusion in their artless efforts to
preserve the most perfect order. The evergreen arch
wouldn\textquotesingle t stay firm after she got it up, but wiggled and
threatened to tumble down on her head when the hanging baskets were
filled; her best tile got a splash of water, which left a sepia tear on
the Cupid\textquotesingle s cheek; she bruised her hands with hammering,
and got cold working in a draught, which last affliction filled her with
apprehensions for the morrow. Any girl-reader who has suffered like
afflictions will sympathize with poor Amy, and wish her well through
with her task.

There was great indignation at home when she told her story that
evening. Her mother said it was a shame, but told her she had done
right; Beth declared she wouldn\textquotesingle t go to the fair at all;
and Jo demanded why she didn\textquotesingle t take all her pretty
things and leave those mean people to get on without her.

"Because they are mean is no reason why I should be. I hate such things,
and though I think I\textquotesingle ve a right to be hurt, I
don\textquotesingle t intend to show it. They will feel that more than
angry speeches or huffy actions, won\textquotesingle t they, Marmee?"

"That\textquotesingle s the right spirit, my dear; a kiss for a blow is
always best, though it\textquotesingle s not very easy to give it
sometimes," said her mother, with the air of one who had learned the
difference between preaching and practising.

In spite of various very natural temptations to resent and retaliate,
Amy adhered to her resolution all the next day, bent on conquering her
enemy by kindness. She began well, thanks to a silent reminder that came
to her unexpectedly, but most opportunely. As she arranged her table
that morning, while the little girls were in an ante-room filling the
baskets, she took up her pet production,---a little book, the antique
cover of which her father had found among his treasures, and in which,
on leaves of vellum, she had beautifully illuminated different texts. As
she turned the pages, rich in dainty devices, with very pardonable
pride, her eye fell upon one verse that made her stop and think. Framed
in a brilliant scroll-work of scarlet, blue, and gold, with little
spirits of good-will helping one another up and down among the thorns
and flowers, were the words, "Thou shalt love thy neighbor as thyself."

"I ought, but I don\textquotesingle t," thought Amy, as her eye went
from the bright page to May\textquotesingle s discontented face behind
the big vases, that could not hide the vacancies her pretty work had
once filled. Amy stood a minute, turning the leaves in her hand, reading
on each some sweet rebuke for all heart-burnings and uncharitableness of
spirit. Many wise and true sermons are preached us every day by
unconscious ministers in street, school, office, or home; even a
fair-table may become a pulpit, if it can offer the good and helpful
words which are never out of season. Amy\textquotesingle s conscience
preached her a little sermon from that text, then and there; and she did
what many of us do not always do,---took the sermon to heart, and
straightway put it in practice.

A group of girls were standing about May\textquotesingle s table,
admiring the pretty things, and talking over the change of saleswomen.
They dropped their voices, but Amy knew they were speaking of her,
hearing one side of the story, and judging accordingly. It was not
pleasant, but a better spirit had come over her, and presently a chance
offered for proving it. She heard May say sorrowfully,---

"It\textquotesingle s too bad, for there is no time to make other
things, and I don\textquotesingle t want to fill up with odds and ends.
The table was just complete then: now it\textquotesingle s spoilt."

"I dare say she\textquotesingle d put them back if you asked her,"
suggested some one.

"How could I after all the fuss?" began May, but she did not finish, for
Amy\textquotesingle s voice came across the hall, saying pleasantly,---

"You may have them, and welcome, without asking, if you want them. I was
just thinking I\textquotesingle d offer to put them back, for they
belong to your table rather than mine. Here they are; please take them,
and forgive me if I was hasty in carrying them away last night."

As she spoke, Amy returned her contribution, with a nod and a smile, and
hurried away again, feeling that it was easier to do a friendly thing
than it was to stay and be thanked for it.

"Now, I call that lovely of her, don\textquotesingle t you?" cried one
girl.

May\textquotesingle s answer was inaudible; but another young lady,
whose temper was evidently a little soured by making lemonade, added,
with a disagreeable laugh, "Very lovely; for she knew she
wouldn\textquotesingle t sell them at her own table."

Now, that was hard; when we make little sacrifices we like to have them
appreciated, at least; and for a minute Amy was sorry she had done it,
feeling that virtue was not always its own reward. But it is,---as she
presently discovered; for her spirits began to rise, and her table to
blossom under her skilful hands; the girls were very kind, and that one
little act seemed to have cleared the atmosphere amazingly.

It was a very long day, and a hard one to Amy, as she sat behind her
table, often quite alone, for the little girls deserted very soon: few
cared to buy flowers in summer, and her bouquets began to droop long
before night.

The art-table \emph{was} the most attractive in the room; there was a
crowd about it all day long, and the tenders were constantly flying to
and fro with important faces and rattling money-boxes. Amy often looked
wistfully across, longing to be there, where she felt at home and happy,
instead of in a corner with nothing to do. It might seem no hardship to
some of us; but to a pretty, blithe young girl, it was not only tedious,
but very trying; and the thought of being found there in the evening by
her family, and Laurie and his friends, made it a real martyrdom.

She did not go home till night, and then she looked so pale and quiet
that they knew the day had been a hard one, though she made no
complaint, and did not even tell what she had done. Her mother gave her
an extra cordial cup of tea, Beth helped her dress, and made a charming
little wreath for her hair, while Jo astonished her family by getting
herself up with unusual care, and hinting darkly that the tables were
about to be turned.

"Don\textquotesingle t do anything rude, pray, Jo. I
won\textquotesingle t have any fuss made, so let it all pass, and behave
yourself," begged Amy, as she departed early, hoping to find a
reinforcement of flowers to refresh her poor little table.

"I merely intend to make myself entrancingly agreeable to every one I
know, and to keep them in your corner as long as possible. Teddy and his
boys will lend a hand, and we\textquotesingle ll have a good time yet,"
returned Jo, leaning over the gate to watch for Laurie. Presently the
familiar tramp was heard in the dusk, and she ran out to meet him.

"Is that my boy?"

"As sure as this is my girl!" and Laurie tucked her hand under his arm,
with the air of a man whose every wish was gratified.

"O Teddy, such doings!" and Jo told Amy\textquotesingle s wrongs with
sisterly zeal.

"A flock of our fellows are going to drive over by and by, and
I\textquotesingle ll be hanged if I don\textquotesingle t make them buy
every flower she\textquotesingle s got, and camp down before her table
afterward," said Laurie, espousing her cause with warmth.

"The flowers are not at all nice, Amy says, and the fresh ones may not
arrive in time. I don\textquotesingle t wish to be unjust or suspicious,
but I shouldn\textquotesingle t wonder if they never came at all. When
people do one mean thing they are very likely to do another," observed
Jo, in a disgusted tone.

"Didn\textquotesingle t Hayes give you the best out of our gardens? I
told him to."

"I didn\textquotesingle t know that; he forgot, I suppose; and, as your
grandpa was poorly, I didn\textquotesingle t like to worry him by
asking, though I did want some."

"Now, Jo, how could you think there was any need of asking! They are
just as much yours as mine. Don\textquotesingle t we always go halves in
everything?" began Laurie, in the tone that always made Jo turn thorny.

"Gracious, I hope not! half of some of your things
wouldn\textquotesingle t suit me at all. But we mustn\textquotesingle t
stand philandering here; I\textquotesingle ve got to help Amy, so you go
and make yourself splendid; and if you\textquotesingle ll be so very
kind as to let Hayes take a few nice flowers up to the Hall,
I\textquotesingle ll bless you forever."

"Couldn\textquotesingle t you do it now?" asked Laurie, so suggestively
that Jo shut the gate in his face with inhospitable haste, and called
through the bars, "Go away, Teddy; I\textquotesingle m busy."

Thanks to the conspirators, the tables \emph{were} turned that night;
for Hayes sent up a wilderness of flowers, with a lovely basket,
arranged in his best manner, for a centre-piece; then the March family
turned out \emph{en masse}, and Jo exerted herself to some purpose, for
people not only came, but stayed, laughing at her nonsense, admiring
Amy\textquotesingle s taste, and apparently enjoying themselves very
much. Laurie and his friends gallantly threw themselves into the breach,
bought up the bouquets, encamped before the table, and made that corner
the liveliest spot in the room. Amy was in her element now, and, out of
gratitude, if nothing more, was as sprightly and gracious as
possible,---coming to the conclusion, about that time, that virtue
\emph{was} its own reward, after all.

\protect\phantomsection\label{6672479776654687619_37106-h-5.htm.xhtml_b140.png}{}
\pandocbounded{\includegraphics[keepaspectratio]{303483661336987339_b140.png}}

Jo behaved herself with exemplary propriety; and when Amy was happily
surrounded by her guard of honor, Jo circulated about the hall, picking
up various bits of gossip, which enlightened her upon the subject of the
Chester change of base. She reproached herself for her share of the
ill-feeling, and resolved to exonerate Amy as soon as possible; she also
discovered what Amy had done about the things in the morning, and
considered her a model of magnanimity. As she passed the art-table, she
glanced over it for her sister\textquotesingle s things, but saw no
signs of them. "Tucked away out of sight, I dare say," thought Jo, who
could forgive her own wrongs, but hotly resented any insult offered to
her family.

"Good evening, Miss Jo. How does Amy get on?" asked May, with a
conciliatory air, for she wanted to show that she also could be
generous.

"She has sold everything she had that was worth selling, and now she is
enjoying herself. The flower-table is always attractive, you know,
\textquotesingle especially to gentlemen.\textquotesingle"

Jo \emph{couldn\textquotesingle t} resist giving that little slap, but
May took it so meekly she regretted it a minute after, and fell to
praising the great vases, which still remained unsold.

"Is Amy\textquotesingle s illumination anywhere about? I took a fancy to
buy that for father," said Jo, very anxious to learn the fate of her
sister\textquotesingle s work.

"Everything of Amy\textquotesingle s sold long ago; I took care that the
right people saw them, and they made a nice little sum of money for us,"
returned May, who had overcome sundry small temptations, as well as Amy,
that day.

Much gratified, Jo rushed back to tell the good news; and Amy looked
both touched and surprised by the report of May\textquotesingle s words
and manner.

"Now, gentlemen, I want you to go and do your duty by the other tables
as generously as you have by mine---especially the art-table," she said,
ordering out "Teddy\textquotesingle s Own," as the girls called the
college friends.

"\textquotesingle Charge, Chester, charge!\textquotesingle{} is the
motto for that table; but do your duty like men, and
you\textquotesingle ll get your money\textquotesingle s worth of
\emph{art} in every sense of the word," said the irrepressible Jo, as
the devoted phalanx prepared to take the field.

"To hear is to obey, but March is fairer far than May," said little
Parker, making a frantic effort to be both witty and tender, and getting
promptly quenched by Laurie, who said, "Very well, my son, for a small
boy!" and walked him off, with a paternal pat on the head.

"Buy the vases," whispered Amy to Laurie, as a final heaping of coals of
fire on her enemy\textquotesingle s head.

To May\textquotesingle s great delight, Mr. Laurence not only bought the
vases, but pervaded the hall with one under each arm. The other
gentlemen speculated with equal rashness in all sorts of frail trifles,
and wandered helplessly about afterward, burdened with wax flowers,
painted fans, filigree portfolios, and other useful and appropriate
purchases.

Aunt Carrol was there, heard the story, looked pleased, and said
something to Mrs. March in a corner, which made the latter lady beam
with satisfaction, and watch Amy with a face full of mingled pride and
anxiety, though she did not betray the cause of her pleasure till
several days later.

The fair was pronounced a success; and when May bade Amy good night, she
did not "gush" as usual, but gave her an affectionate kiss, and a look
which said, "Forgive and forget." That satisfied Amy; and when she got
home she found the vases paraded on the parlor chimney-piece, with a
great bouquet in each. "The reward of merit for a magnanimous March," as
Laurie announced with a flourish.

"You\textquotesingle ve a deal more principle and generosity and
nobleness of character than I ever gave you credit for, Amy.
You\textquotesingle ve behaved sweetly, and I respect you with all my
heart," said Jo warmly, as they brushed their hair together late that
night.

"Yes, we all do, and love her for being so ready to forgive. It must
have been dreadfully hard, after working so long, and setting your heart
on selling your own pretty things. I don\textquotesingle t believe I
could have done it as kindly as you did," added Beth from her pillow.

"Why, girls, you needn\textquotesingle t praise me so; I only did as
I\textquotesingle d be done by. You laugh at me when I say I want to be
a lady, but I mean a true gentlewoman in mind and manners, and I try to
do it as far as I know how. I can\textquotesingle t explain exactly, but
I want to be above the little meannesses and follies and faults that
spoil so many women. I\textquotesingle m far from it now, but I do my
best, and hope in time to be what mother is."

Amy spoke earnestly, and Jo said, with a cordial hug,---

"I understand now what you mean, and I\textquotesingle ll never laugh at
you again. You are getting on faster than you think, and
I\textquotesingle ll take lessons of you in true politeness, for
you\textquotesingle ve learned the secret, I believe. Try away, deary;
you\textquotesingle ll get your reward some day, and no one will be more
delighted than I shall."

A week later Amy did get her reward, and poor Jo found it hard to be
delighted. A letter came from Aunt Carrol, and Mrs.
March\textquotesingle s face was illuminated to such a degree, when she
read it, that Jo and Beth, who were with her, demanded what the glad
tidings were.

"Aunt Carrol is going abroad next month, and wants---"

"Me to go with her!" burst in Jo, flying out of her chair in an
uncontrollable rapture.

"No, dear, not you; it\textquotesingle s Amy."

"O mother! she\textquotesingle s too young; it\textquotesingle s my turn
first. I\textquotesingle ve wanted it so long---it would do me so much
good, and be so altogether splendid---I \emph{must} go."

"I\textquotesingle m afraid it\textquotesingle s impossible, Jo. Aunt
says Amy, decidedly, and it is not for us to dictate when she offers
such a favor."

"It\textquotesingle s always so. Amy has all the fun and I have all the
work. It isn\textquotesingle t fair, oh, it isn\textquotesingle t fair!"
cried Jo passionately.

"I\textquotesingle m afraid it is partly your own fault, dear. When Aunt
spoke to me the other day, she regretted your blunt manners and too
independent spirit; and here she writes, as if quoting something you had
said,---\textquotesingle I planned at first to ask Jo; but as "favors
burden her," and she "hates French," I think I won\textquotesingle t
venture to invite her. Amy is more docile, will make a good companion
for Flo, and receive gratefully any help the trip may give
her.\textquotesingle"

"Oh, my tongue, my abominable tongue! why can\textquotesingle t I learn
to keep it quiet?" groaned Jo, remembering words which had been her
undoing. When she had heard the explanation of the quoted phrases, Mrs.
March said sorrowfully,---

"I wish you could have gone, but there is no hope of it this time; so
try to bear it cheerfully, and don\textquotesingle t sadden
Amy\textquotesingle s pleasure by reproaches or regrets."

"I\textquotesingle ll try," said Jo, winking hard, as she knelt down to
pick up the basket she had joyfully upset. "I\textquotesingle ll take a
leaf out of her book, and try not only to seem glad, but to be so, and
not grudge her one minute of happiness; but it won\textquotesingle t be
easy, for it is a dreadful disappointment;" and poor Jo bedewed the
little fat pincushion she held with several very bitter tears.

"Jo, dear, I\textquotesingle m very selfish, but I
couldn\textquotesingle t spare you, and I\textquotesingle m glad you are
not going quite yet," whispered Beth, embracing her, basket and all,
with such a clinging touch and loving face, that Jo felt comforted in
spite of the sharp regret that made her want to box her own ears, and
humbly beg Aunt Carrol to burden her with this favor, and see how
gratefully she would bear it.

By the time Amy came in, Jo was able to take her part in the family
jubilation; not quite as heartily as usual, perhaps, but without
repinings at Amy\textquotesingle s good fortune. The young lady herself
received the news as tidings of great joy, went about in a solemn sort
of rapture, and began to sort her colors and pack her pencils that
evening, leaving such trifles as clothes, money, and passports to those
less absorbed in visions of art than herself.

"It isn\textquotesingle t a mere pleasure trip to me, girls," she said
impressively, as she scraped her best palette. "It will decide my
career; for if I have any genius, I shall find it out in Rome, and will
do something to prove it."

"Suppose you haven\textquotesingle t?" said Jo, sewing away, with red
eyes, at the new collars which were to be handed over to Amy.

"Then I shall come home and teach drawing for my living," replied the
aspirant for fame, with philosophic composure; but she made a wry face
at the prospect, and scratched away at her palette as if bent on
vigorous measures before she gave up her hopes.

"No, you won\textquotesingle t; you hate hard work, and
you\textquotesingle ll marry some rich man, and come home to sit in the
lap of luxury all your days," said Jo.

"Your predictions sometimes come to pass, but I don\textquotesingle t
believe that one will. I\textquotesingle m sure I wish it would, for if
I can\textquotesingle t be an artist myself, I should like to be able to
help those who are," said Amy, smiling, as if the part of Lady Bountiful
would suit her better than that of a poor drawing-teacher.

"Hum!" said Jo, with a sigh; "if you wish it you\textquotesingle ll have
it, for your wishes are always granted---mine never."

"Would you like to go?" asked Amy, thoughtfully patting her nose with
her knife.

"Rather!"

"Well, in a year or two I\textquotesingle ll send for you, and
we\textquotesingle ll dig in the Forum for relics, and carry out all the
plans we\textquotesingle ve made so many times."

"Thank you; I\textquotesingle ll remind you of your promise when that
joyful day comes, if it ever does," returned Jo, accepting the vague but
magnificent offer as gratefully as she could.

There was not much time for preparation, and the house was in a ferment
till Amy was off. Jo bore up very well till the last flutter of blue
ribbon vanished, when she retired to her refuge, the garret, and cried
till she couldn\textquotesingle t cry any more. Amy likewise bore up
stoutly till the steamer sailed; then, just as the gangway was about to
be withdrawn, it suddenly came over her that a whole ocean was soon to
roll between her and those who loved her best, and she clung to Laurie,
the last lingerer, saying with a sob,---

"Oh, take care of them for me; and if anything should happen---"

"I will, dear, I will; and if anything happens, I\textquotesingle ll
come and comfort you," whispered Laurie, little dreaming that he would
be called upon to keep his word.

So Amy sailed away to find the old world, which is always new and
beautiful to young eyes, while her father and friend watched her from
the shore, fervently hoping that none but gentle fortunes would befall
the happy-hearted girl, who waved her hand to them till they could see
nothing but the summer sunshine dazzling on the sea.

\protect\phantomsection\label{6672479776654687619_37106-h-5.htm.xhtml_b141.png}{}
\pandocbounded{\includegraphics[keepaspectratio]{303483661336987339_b141.png}}

\begin{center}\rule{0.5\linewidth}{0.5pt}\end{center}

\subsection{XXXI. Our Foreign
Correspondent.}\label{6672479776654687619_37106-h-5.htm.xhtml_pgepubid00033}

\protect\phantomsection\label{6672479776654687619_37106-h-5.htm.xhtml_b142.png}{}
\pandocbounded{\includegraphics[keepaspectratio]{303483661336987339_b142.png}}

\protect\phantomsection\label{6672479776654687619_37106-h-5.htm.xhtml_XXXI}{}\hyperref[6672479776654687619_37106-h-0.htm.xhtml_contents2]{XXXI.}

OUR FOREIGN CORRESPONDENT.

\begin{quote}
"{London.}

"Dearest People,---

"Here I really sit at a front window of the Bath Hotel, Piccadilly.
It\textquotesingle s not a fashionable place, but uncle stopped here
years ago, and won\textquotesingle t go anywhere else; however, we
don\textquotesingle t mean to stay long, so it\textquotesingle s no
great matter. Oh, I can\textquotesingle t begin to tell you how I enjoy
it all! I never can, so I\textquotesingle ll only give you bits out of
my note-book, for I\textquotesingle ve done nothing but sketch and
scribble since I started.

"I sent a line from Halifax, when I felt pretty miserable, but after
that I got on delightfully, seldom ill, on deck all day, with plenty of
pleasant people to amuse me. Every one was very kind to me, especially
the officers. Don\textquotesingle t laugh, Jo; gentlemen really are very
necessary aboard ship, to hold on to, or to wait upon one; and as they
have nothing to do, it\textquotesingle s a mercy to make them useful,
otherwise they would smoke themselves to death, I\textquotesingle m
afraid.
\end{quote}

\protect\phantomsection\label{6672479776654687619_37106-h-5.htm.xhtml_b143.png}{}
\pandocbounded{\includegraphics[keepaspectratio]{303483661336987339_b143.png}}\\
\protect\phantomsection\label{6672479776654687619_37106-h-5.htm.xhtml_ebm_caption5}{
"Every one was very kind, especially the officers."---Page 378.}

\begin{quote}
"Aunt and Flo were poorly all the way, and liked to be let alone, so
when I had done what I could for them, I went and enjoyed myself. Such
walks on deck, such sunsets, such splendid air and waves! It was almost
as exciting as riding a fast horse, when we went rushing on so grandly.
I wish Beth could have come, it would have done her so much good; \ul{as
for Jo, she would have gone up} and sat on the main-top jib, or whatever
the high thing is called, made friends with the engineers, and tooted on
the captain\textquotesingle s speaking-trumpet, she\textquotesingle d
have been in such a state of rapture.

"It was all heavenly, but I was glad to see the Irish coast, and found
it very lovely, so green and sunny, with brown cabins here and there,
ruins on some of the hills, and gentlemen\textquotesingle s
country-seats in the valleys, with deer feeding in the parks. It was
early in the morning, but I didn\textquotesingle t regret getting up to
see it, for the bay was full of little boats, the shore \emph{so}
picturesque, and a rosy sky overhead. I never shall forget it.

"At Queenstown one of my new acquaintances left us,---Mr. Lennox,---and
when I said something about the Lakes of Killarney, he sighed and sung,
with a look at me,---

\textquotesingle Oh, have you e\textquotesingle er heard of Kate
Kearney?

She lives on the banks of Killarney;

From the glance of her eye,

Shun danger and fly,

For fatal\textquotesingle s the glance of Kate
Kearney.\textquotesingle{}

Wasn\textquotesingle t that nonsensical?

"We only stopped at Liverpool a few hours. It\textquotesingle s a dirty,
noisy place, and I was glad to leave it. Uncle rushed out and bought a
pair of dog-skin gloves, some ugly, thick shoes, and an umbrella, and
got shaved \emph{à la} mutton-chop, the first thing. Then he flattered
himself that he looked like a true Briton; but the first time he had the
mud cleaned off his shoes, the little bootblack knew that an American
stood in them, and said, with a grin, \textquotesingle There yer har,
sir. I\textquotesingle ve give \textquotesingle em the latest Yankee
shine.\textquotesingle{} It amused uncle immensely. Oh, I \emph{must}
tell you what that absurd Lennox did! He got his friend Ward, who came
on with us, to order a bouquet for me, and the first thing I saw in my
room was a lovely one, with \textquotesingle Robert
Lennox\textquotesingle s compliments,\textquotesingle{} on the card.
Wasn\textquotesingle t that fun, girls? I like travelling.

"I never \emph{shall} get to London if I don\textquotesingle t hurry.
The trip was like riding through a long picture-gallery, full of lovely
landscapes. The farmhouses were my delight; with thatched roofs, ivy up
to the eaves, latticed windows, and stout women with rosy children at
the doors. The very cattle looked more tranquil than ours, as they stood
knee-deep in clover, and the hens had a contented cluck, as if they
never got nervous, like Yankee biddies. Such perfect color I never
saw,---the grass so green, sky so blue, grain so yellow, woods so
dark,---I was in a rapture all the way. So was Flo; and we kept bouncing
from one side to the other, trying to see everything while we were
whisking along at the rate of sixty miles an hour. Aunt was tired and
went to sleep, but uncle read his guide-book, and
wouldn\textquotesingle t be astonished at anything. This is the way we
went on: Amy, flying up,---\textquotesingle Oh, that must be Kenilworth,
that gray place among the trees!\textquotesingle{} Flo, darting to my
window,---\textquotesingle How sweet! We must go there some time,
won\textquotesingle t we, papa?\textquotesingle{} Uncle, calmly admiring
his boots,---\textquotesingle No, my dear, not unless you want beer;
that\textquotesingle s a brewery.\textquotesingle{}

"A pause,---then Flo cried out, \textquotesingle Bless me,
there\textquotesingle s a gallows and a man going up.\textquotesingle{}
\textquotesingle Where, where?\textquotesingle{} shrieks Amy, staring
out at two tall posts with a cross-beam and some dangling chains.
\textquotesingle A colliery,\textquotesingle{} remarks uncle, with a
twinkle of the eye. \textquotesingle Here\textquotesingle s a lovely
flock of \ul{lambs all lying down,\textquotesingle{} says Amy.}
\textquotesingle See, papa, aren\textquotesingle t they
pretty!\textquotesingle{} added Flo sentimentally.
\textquotesingle Geese, young ladies,\textquotesingle{} returns uncle,
in a tone that keeps us quiet till Flo settles down to enjoy
\textquotesingle The Flirtations of Capt. Cavendish,\textquotesingle{}
and I have the scenery all to myself.

"Of course it rained when we got to London, and there was nothing to be
seen but fog and umbrellas. We rested, unpacked, and shopped a little
between the showers. Aunt Mary got me some new things, for I came off in
such a hurry I wasn\textquotesingle t half ready. A white hat and blue
feather, a muslin dress to match, and the loveliest mantle you ever saw.
Shopping in Regent Street is perfectly splendid; things seem so
cheap---nice ribbons only sixpence a yard. I laid in a stock, but shall
get my gloves in Paris. Doesn\textquotesingle t that sound sort of
elegant and rich?

"Flo and I, for the fun of it, ordered a hansom cab, while aunt and
uncle were out, and went for a drive, though we learned afterward that
it wasn\textquotesingle t the thing for young ladies to ride in them
alone. It was so droll! for when we were shut in by the wooden apron,
the man drove so fast that Flo was frightened, and told me to stop him.
But he was up outside behind somewhere, and I couldn\textquotesingle t
get at him. He didn\textquotesingle t hear me call, nor see me flap my
parasol in front, and there we were, quite helpless, rattling away, and
whirling around corners at a break-neck pace. At last, in my despair, I
saw a little door in the roof, and on poking it open, a red eye
appeared, and a beery voice said,---

"\textquotesingle Now then, mum?\textquotesingle{}

"I gave my order as soberly as I could, and slamming down the door, with
an \textquotesingle Aye, aye, mum,\textquotesingle{} the man made his
horse walk, as if going to a funeral. I poked again, and said,
\textquotesingle A little faster;\textquotesingle{} then off he went,
helter-skelter, as before, and we resigned ourselves to our fate.

"To-day was fair and we went to Hyde Park, close by, for we are more
aristocratic than we look. The Duke of Devonshire lives near. I often
see his footmen lounging at the back gate; and the Duke of
Wellington\textquotesingle s house is not far off. Such sights as I saw,
my dear! It was as good as Punch, for there were fat dowagers rolling
about in their red and yellow coaches, with gorgeous Jeameses in silk
stockings and velvet coats, up behind, and powdered coachmen in front.
Smart maids, with the rosiest children I ever saw; handsome girls,
looking half asleep; dandies, in queer English hats and lavender kids,
lounging about, and tall soldiers, in short red jackets and muffin caps
stuck on one side, looking so funny I longed to sketch them.

"Rotten Row means \textquotesingle{}\emph{Route de
Roi},\textquotesingle{} or the king\textquotesingle s way; but now
it\textquotesingle s more like a riding-school than anything else. The
horses are splendid, and the men, especially the grooms, ride well; but
the women are stiff, and bounce, which isn\textquotesingle t according
to our rules. I longed to show them a tearing American gallop, for they
trotted solemnly up and down, in their scant habits and high hats,
looking like the women in a toy Noah\textquotesingle s Ark. Every one
rides,---old men, stout ladies, little children,---and the young folks
do a deal of flirting here; I saw a pair exchange rosebuds, for
it\textquotesingle s the thing to wear one in the button-hole, and I
thought it rather a nice little idea.

"In the {p.m.} to Westminster Abbey; but don\textquotesingle t expect me
to describe it, that\textquotesingle s impossible---so
I\textquotesingle ll only say it was sublime! This evening we are going
to see Fechter, which will be an appropriate end to the happiest day of
my life.

\begin{center}\rule{0.5\linewidth}{0.5pt}\end{center}

"{Midnight.}

"It\textquotesingle s very late, but I can\textquotesingle t let my
letter go in the morning without telling you what happened last evening.
Who do you think came in, as we were at tea? Laurie\textquotesingle s
English friends, Fred and Frank Vaughn! I was \emph{so} surprised, for I
shouldn\textquotesingle t have known them but for the cards. Both are
tall fellows, with whiskers; Fred handsome in the English style, and
Frank much better, for he only limps slightly, and uses no crutches.
They had heard from Laurie where we were to be, and came to ask us to
their house; but uncle won\textquotesingle t go, so we shall return the
call, and see them as we can. They went to the theatre with us, and we
did have \emph{such} a good time, for Frank devoted himself to Flo, and
Fred and I talked over past, present, and future fun as if we had known
each other all our days. Tell Beth Frank asked for her, and was sorry to
hear of her ill health. Fred laughed when I spoke of Jo, and sent his
\textquotesingle respectful compliments to the big
hat.\textquotesingle{} Neither of them had forgotten Camp Laurence, or
the fun we had there. What ages ago it seems, doesn\textquotesingle t
it?

"Aunt is tapping on the wall for the third time, so I \emph{must} stop.
I really feel like a dissipated London fine lady, writing here so late,
with my room full of pretty things, and my head a jumble of parks,
theatres, new gowns, and gallant creatures who say
\textquotesingle Ah!\textquotesingle{} and twirl their blond mustaches
with the true English lordliness. I long to see you all, and in spite of
my nonsense am, as ever, your loving

Amy."
\end{quote}

\begin{center}\rule{0.5\linewidth}{0.5pt}\end{center}

\begin{quote}
"Paris"

"Dear Girls,---

"In my last I told you about our London visit,---how kind the Vaughns
were, and what pleasant parties they made for us. I enjoyed the trips to
Hampton Court and the Kensington Museum more than anything else,---for
at Hampton I saw Raphael\textquotesingle s cartoons, and, at the Museum,
rooms full of pictures by Turner, Lawrence, Reynolds, Hogarth, and the
other great creatures. The day in Richmond Park was charming, for we had
a regular English picnic, and I had more splendid oaks and groups of
deer than I could copy; also heard a nightingale, and saw larks go up.
We \textquotesingle did\textquotesingle{} London to our
hearts\textquotesingle{} content, thanks to Fred and Frank, and were
sorry to go away; for, though English people are slow to take you in,
when they once make up their minds to do it they cannot be outdone in
hospitality, \emph{I} think. The Vaughns hope to meet us in Rome next
winter, and I shall be dreadfully disappointed if they
don\textquotesingle t, for Grace and I are great friends, and the boys
very nice fellows,---especially Fred.

"Well, we were hardly settled here, when he turned up again, saying he
had come for a holiday, and was going to Switzerland. Aunt looked sober
at first, but he was so cool about it she couldn\textquotesingle t say a
word; and now we get on nicely, and are very glad he came, for he speaks
French like a native, and I don\textquotesingle t know what we should do
without him. Uncle doesn\textquotesingle t know ten words, and insists
on talking English very loud, as if that would make people understand
him. Aunt\textquotesingle s pronunciation is old-fashioned, and Flo and
I, though we flattered ourselves that we knew a good deal, find we
don\textquotesingle t, and are very grateful to have Fred do the
\textquotesingle{}\emph{parley vooing},\textquotesingle{} as uncle calls
it.

"Such delightful times as we are having! sight-seeing from morning till
night, stopping for nice lunches in the gay \emph{cafés}, and meeting
with all sorts of droll adventures. Rainy days I spend in the Louvre,
revelling in pictures. Jo would turn up her naughty nose at some of the
finest, because she has no soul for art; but \emph{I} have, and
I\textquotesingle m cultivating eye and taste as fast as I can. She
would like the relics of great people better, for I\textquotesingle ve
seen her Napoleon\textquotesingle s cocked hat and gray coat, his
baby\textquotesingle s cradle and his old toothbrush; also Marie
Antoinette\textquotesingle s little shoe, the ring of Saint Denis,
Charlemagne\textquotesingle s sword, and many other interesting things.
I\textquotesingle ll talk for hours about them when I come, but
haven\textquotesingle t time to write.

"The Palais Royale is a heavenly place,---so full of \emph{bijouterie}
and lovely things that I\textquotesingle m nearly distracted because I
can\textquotesingle t buy them. Fred wanted to get me some, but of
course I didn\textquotesingle t allow it. Then the Bois and the Champs
Elysées are \emph{très magnifique}. I\textquotesingle ve seen the
imperial family several times,---the emperor an ugly, hard-looking man,
the empress pale and pretty, but dressed in bad taste, \emph{I}
thought,---purple dress, green hat, and yellow gloves. Little Nap. is a
handsome boy, who sits chatting to his tutor, and kisses his hand to the
people as he passes in his four-horse barouche, with postilions in red
satin jackets, and a mounted guard before and behind.
\end{quote}

\protect\phantomsection\label{6672479776654687619_37106-h-5.htm.xhtml_b144.png}{}
\pandocbounded{\includegraphics[keepaspectratio]{303483661336987339_b144.png}}

\begin{quote}
"We often walk in the Tuileries Gardens, for they are lovely, though the
antique Luxembourg Gardens suit me better. Père la Chaise is very
curious, for many of the tombs are like small rooms, and, looking in,
one sees a table, with images or pictures of the dead, and chairs for
the mourners to sit in when they come to lament. That is so Frenchy.

"Our rooms are on the Rue de Rivoli, and, sitting in the balcony, we
look up and down the long, brilliant street. It is so pleasant that we
spend our evenings talking there, when too tired with our
day\textquotesingle s work to go out. Fred is very entertaining, and is
altogether the most agreeable young man I ever knew,---except Laurie,
whose manners are more charming. I wish Fred was dark, for I
don\textquotesingle t fancy light men; however, the Vaughns are very
rich, and come of an excellent family, so I won\textquotesingle t find
fault with their yellow hair, as my own is yellower.

"Next week we are off to Germany and Switzerland; and, as we shall
travel fast, I shall only be able to give you hasty letters. I keep my
diary, and try to \textquotesingle remember correctly and describe
clearly all that I see and admire,\textquotesingle{} as father advised.
It is good practice for me, and, with my sketch-book, will give you a
better idea of my tour than these scribbles.

"Adieu; I embrace you tenderly.

Votre Amie."
\end{quote}

\begin{center}\rule{0.5\linewidth}{0.5pt}\end{center}

\begin{quote}
"Heidelberg.

"My dear Mamma,---

"Having a quiet hour before we leave for Berne, I\textquotesingle ll try
to tell you what has happened, for some of it is very important, as you
will see.

"The sail up the Rhine was perfect, and I just sat and enjoyed it with
all my might. Get father\textquotesingle s old guide-books, and read
about it; I haven\textquotesingle t words beautiful enough to describe
it. At Coblentz we had a lovely time, for some students from Bonn, with
whom Fred got acquainted on the boat, gave us a serenade. It was a
moonlight night, and, about one o\textquotesingle clock, Flo and I were
waked by the most delicious music under our windows. We flew up, and hid
behind the curtains; but sly peeps showed us Fred and the students
singing away down below. It was the most romantic thing I ever
saw,---the river, the bridge of boats, the great fortress opposite,
moonlight everywhere, and music fit to melt a heart of stone.

"When they were done we threw down some flowers, and saw them scramble
for them, kiss their hands to the invisible ladies, and go laughing
away,---to smoke and drink beer, I suppose. Next morning Fred showed me
one of the crumpled flowers in his vest-pocket, and looked very
sentimental. I laughed at him, and said I didn\textquotesingle t throw
it, but Flo, which seemed to disgust him, for he tossed it out of the
window, and turned sensible again. I\textquotesingle m afraid
I\textquotesingle m going to have trouble with that boy, it begins to
look like it.

"The baths at Nassau were very gay, so was Baden-Baden, where Fred lost
some money, and I scolded him. He needs some one to look after him when
Frank is not with him. Kate said once she hoped he\textquotesingle d
marry soon, and I quite agree with her that it would be well for him.
Frankfort was delightful; I saw Goethe\textquotesingle s house,
Schiller\textquotesingle s statue, and Dannecker\textquotesingle s
famous \textquotesingle Ariadne.\textquotesingle{} It was very lovely,
but I should have enjoyed it more if I had known the story better. I
didn\textquotesingle t like to ask, as every one knew it, or pretended
they did. I wish Jo would tell me all about it; I ought to have read
more, for I find I don\textquotesingle t know anything, and it mortifies
me.

"Now comes the serious part,---for it happened here, and Fred is just
gone. He has been so kind and jolly that we all got quite fond of him; I
never thought of anything but a travelling friendship, till the serenade
night. Since then I\textquotesingle ve begun to feel that the moonlight
walks, balcony talks, and daily adventures were something more to him
than fun. I haven\textquotesingle t flirted, mother, truly, but
remembered what you said to me, and have done my very best. I
can\textquotesingle t help it if people like me; I don\textquotesingle t
try to make them, and it worries me if I don\textquotesingle t care for
them, though Jo says I haven\textquotesingle t got any heart. Now I know
mother will shake her head, and the girls say, \textquotesingle Oh, the
mercenary little wretch!\textquotesingle{} but I\textquotesingle ve made
up my mind, and, if Fred asks me, I shall accept him, though
I\textquotesingle m not madly in love. I like him, and we get on
comfortably together. He is handsome, young, clever enough, and very
rich,---ever so much richer than the Laurences. I don\textquotesingle t
think his family would object, and I should be very happy, for they are
all kind, well-bred, generous people, and they like me. Fred, as the
eldest twin, will have the estate, I suppose, and such a splendid one as
it is! A city house in a fashionable street, not so showy as our big
houses, but twice as comfortable, and full of solid luxury, such as
English people believe in. I like it, for it\textquotesingle s genuine.
I\textquotesingle ve seen the plate, the family jewels, the old
servants, and pictures of the country place, with its park, great house,
lovely grounds, and fine horses. Oh, it would be all I should ask! and
I\textquotesingle d rather have it than any title such as girls snap up
so readily, and find nothing behind. I may be mercenary, but I hate
poverty, and don\textquotesingle t mean to bear it a minute longer than
I can help. One of us \emph{must} marry well; Meg
didn\textquotesingle t, Jo won\textquotesingle t, Beth
can\textquotesingle t yet, so I shall, and make everything cosey all
round. I wouldn\textquotesingle t marry a man I hated or despised. You
may be sure of that; and, though Fred is not my model hero, he does very
well, and, in time, I should get fond enough of him if he was very fond
of me, and let me do just as I liked. So I\textquotesingle ve been
turning the matter over in my mind the last week, for it was impossible
to help seeing that Fred liked me. He said nothing, but little things
showed it; he never goes with Flo, always gets on my side of the
carriage, table, or promenade, looks sentimental when we are alone, and
frowns at any one else who ventures to speak to me. Yesterday, at
dinner, when an Austrian officer stared at us, and then said something
to his friend,---a rakish-looking baron,---about
\textquotesingle{}\emph{ein wonderschönes Blöndchen},\textquotesingle{}
Fred looked as fierce as a lion, and cut his meat so savagely, it nearly
flew off his plate. He isn\textquotesingle t one of the cool, stiff
Englishmen, but is rather peppery, for he has Scotch blood in him, as
one might guess from his bonnie blue eyes.

\protect\phantomsection\label{6672479776654687619_37106-h-5.htm.xhtml_b145.png}{}
\pandocbounded{\includegraphics[keepaspectratio]{303483661336987339_b145.png}}

"Well, last evening we went up to the castle about sunset,---at least
all of us but Fred, who was to meet us there, after going to the Post
Restante for letters. We had a charming time poking about the ruins, the
vaults where the monster tun is, and the beautiful gardens made by the
elector, long ago, for his English wife. I liked the great terrace best,
for the view was divine; so, while the rest went to see the rooms
inside, I sat there trying to sketch the gray stone
lion\textquotesingle s head on the wall, with scarlet woodbine sprays
hanging round it. I felt as if I\textquotesingle d got into a romance,
sitting there, watching the Neckar rolling through the valley, listening
to the music of the Austrian band below, and waiting for my lover, like
a real story-book girl. I had a feeling that something was going to
happen, and I was ready for it. I didn\textquotesingle t feel blushy or
quakey, but quite cool, and only a little excited.
\end{quote}

\begin{quote}
"By and by I heard Fred\textquotesingle s voice, and then he came
hurrying through the great arch to find me. He looked so troubled that I
forgot all about myself, and asked what the matter was. He said
he\textquotesingle d just got a letter begging him to come home, for
Frank was very ill; so he was going at once, in the night train, and
only had time to say good-by. I was very sorry for him, and disappointed
for myself, but only for a minute, because he said, as he shook
hands,---and said it in a way that I could not
mistake,---\textquotesingle I shall soon come back; you
won\textquotesingle t forget me, Amy?\textquotesingle{}

"I didn\textquotesingle t promise, but I looked at him, and he seemed
satisfied, and there was no time for anything but messages and
good-byes, for he was off in an hour, and we all miss him very much. I
know he wanted to speak, but I think, from something he once hinted,
that he had promised his father not to do anything of the sort yet
awhile, for he is a rash boy, and the old gentleman dreads a foreign
daughter-in-law. We shall soon meet in Rome; and then, if I
don\textquotesingle t change my mind, I\textquotesingle ll say
\textquotesingle Yes, thank you,\textquotesingle{} when he says
\textquotesingle Will you, please?\textquotesingle{}

"Of course this is all \emph{very private}, but I wished you to know
what was going on. Don\textquotesingle t be anxious about me; remember I
am your \textquotesingle prudent Amy,\textquotesingle{} and be sure I
will do nothing rashly. Send me as much advice as you like;
I\textquotesingle ll use it if I can. I wish I could see you for a good
talk, Marmee. Love and trust me.

"Ever your

Amy."
\end{quote}

\begin{center}\rule{0.5\linewidth}{0.5pt}\end{center}

\subsection{XXXII. Tender
Troubles.}\label{6672479776654687619_37106-h-5.htm.xhtml_pgepubid00034}

\protect\phantomsection\label{6672479776654687619_37106-h-5.htm.xhtml_XXXII}{}\hyperref[6672479776654687619_37106-h-0.htm.xhtml_contents2]{XXXII.}

TENDER TROUBLES.

"{Jo}, I\textquotesingle m anxious about Beth."

"Why, mother, she has seemed unusually well since the babies came."

"It\textquotesingle s not her health that troubles me now;
it\textquotesingle s her spirits. I\textquotesingle m sure there is
something on her mind, and I want you to discover what it is."

"What makes you think so, mother?"

"She sits alone a good deal, and doesn\textquotesingle t talk to her
father as much as she used. I found her crying over the babies the other
day. When she sings, the songs are always sad ones, and now and then I
see a look in her face that I don\textquotesingle t understand. This
isn\textquotesingle t like Beth, and it worries me."

"Have you asked her about it?"

"I have tried once or twice; but she either evaded my questions, or
looked so distressed that I stopped. I never force my
children\textquotesingle s confidence, and I seldom have to wait for it
long."

Mrs. March glanced at Jo as she spoke, but the face opposite seemed
quite unconscious of any secret disquietude but Beth\textquotesingle s;
and, after sewing thoughtfully for a minute, Jo said,---

"I think she is growing up, and so begins to dream dreams, and have
hopes and fears and fidgets, without knowing why, or being able to
explain them. Why, mother, Beth\textquotesingle s eighteen, but we
don\textquotesingle t realize it, and treat her like a child, forgetting
she\textquotesingle s a woman."

"So she is. Dear heart, how fast you do grow up," returned her mother,
with a sigh and a smile.

"Can\textquotesingle t be helped, Marmee, so you must resign yourself to
all sorts of worries, and let your birds hop out of the nest, one by
one. I promise never to hop very far, if that is any comfort to you."

"It is a great comfort, Jo; I always feel strong when you are at home,
now Meg is gone. Beth is too feeble and Amy too young to depend upon;
but when the tug comes, you are always ready."

"Why, you know I don\textquotesingle t mind hard jobs much, and there
must always be one scrub in a family. Amy is splendid in fine works, and
I\textquotesingle m not; but I feel in my element when all the carpets
are to be taken up, or half the family fall sick at once. Amy is
distinguishing herself abroad; but if anything is amiss at home,
I\textquotesingle m your man."

"I leave Beth to your hands, then, for she will open her tender little
heart to her Jo sooner than to any one else. Be very kind, and
don\textquotesingle t let her think any one watches or talks about her.
If she only would get quite strong and cheerful again, I
shouldn\textquotesingle t have a wish in the world."

"Happy woman! I\textquotesingle ve got heaps."

"My dear, what are they?"

"I\textquotesingle ll settle Bethy\textquotesingle s troubles, and then
I\textquotesingle ll tell you mine. They are not very wearing, so
they\textquotesingle ll keep;" and Jo stitched away, with a wise nod
which set her mother\textquotesingle s heart at rest about her, for the
present at least.

While apparently absorbed in her own affairs, Jo watched Beth; and,
after many conflicting conjectures, finally settled upon one which
seemed to explain the change in her. A slight incident gave Jo the clue
to the mystery, she thought, and lively fancy, loving heart did the
rest. She was affecting to write busily one Saturday afternoon, when she
and Beth were alone together; yet as she scribbled, she kept her eye on
her sister, who seemed unusually quiet. Sitting at the window,
Beth\textquotesingle s work often dropped into her lap, and she leaned
her head upon her hand, in a dejected attitude, while her eyes rested on
the dull, autumnal landscape. Suddenly some one passed below, whistling
like an operatic blackbird, and a voice called out,---

\protect\phantomsection\label{6672479776654687619_37106-h-5.htm.xhtml_b146.png}{}
\pandocbounded{\includegraphics[keepaspectratio]{303483661336987339_b146.png}}

"All serene! Coming in to-night."

Beth started, leaned forward, smiled and nodded, watched the passer-by
till his quick tramp died away, then said softly, as if to herself,---

"How strong and well and happy that dear boy looks."

"Hum!" said Jo, still intent upon her sister\textquotesingle s face; for
the bright color faded as quickly as it came, the smile vanished, and
presently a tear lay shining on the window-ledge. Beth whisked it off,
and glanced apprehensively at Jo; but she was scratching away at a
tremendous rate, apparently engrossed in "Olympia\textquotesingle s
Oath." The instant Beth turned, Jo began her watch again, saw
Beth\textquotesingle s hand go quietly to her eyes more than once, and,
in her half-averted face, read a tender sorrow that made her own eyes
fill. Fearing to betray herself, she slipped away, murmuring something
about needing more paper.

"Mercy on me, Beth loves Laurie!" she said, sitting down in her own
room, pale with the shock of the discovery which she believed she had
just made. "I never dreamt of such a thing. What \emph{will} mother say?
I wonder if he---" there Jo stopped, and turned scarlet with a sudden
thought. "If he shouldn\textquotesingle t love back again, how dreadful
it would be. He must; I\textquotesingle ll make him!" and she shook her
head threateningly at the picture of the mischievous-looking boy
laughing at her from the wall. "Oh dear, we \emph{are} growing up with a
vengeance. Here\textquotesingle s Meg married and a mamma, Amy
flourishing away at Paris, and Beth in love. I\textquotesingle m the
only one that has sense enough to keep out of mischief." Jo thought
intently for a minute, with her eyes fixed on the picture; then she
smoothed out her wrinkled forehead, and said, with a decided nod at the
face opposite, "No, thank you, sir; you\textquotesingle re very
charming, but you\textquotesingle ve no more stability than a
weathercock; so you needn\textquotesingle t write touching notes, and
smile in that insinuating way, for it won\textquotesingle t do a bit of
good, and I won\textquotesingle t have it."

Then she sighed, and fell into a reverie, from which she did not wake
till the early twilight sent her down to take new observations, which
only confirmed her suspicion. Though Laurie flirted with Amy and joked
with Jo, his manner to Beth had always been peculiarly kind and gentle,
but so was everybody\textquotesingle s; therefore, no one thought of
imagining that he cared more for her than for the others. Indeed, a
general impression had prevailed in the family, of late, that "our boy"
was getting fonder than ever of Jo, who, however,
wouldn\textquotesingle t hear a word upon the subject, and scolded
violently if any one dared to suggest it. If they had known the various
tender passages of the past year, or rather attempts at tender passages
which had been nipped in the bud, they would have had the immense
satisfaction of saying, "I told you so." But Jo hated "philandering,"
and wouldn\textquotesingle t allow it, always having a joke or a smile
ready at the least sign of impending danger.

When Laurie first went to college, he fell in love about once a month;
but these small flames were as brief as ardent, did no damage, and much
amused Jo, who took great interest in the alternations of hope, despair,
and resignation, which were confided to her in their weekly conferences.
But there came a time when Laurie ceased to worship at many shrines,
hinted darkly at one all-absorbing passion, and indulged occasionally in
Byronic fits of gloom. Then he avoided the tender subject altogether,
wrote philosophical notes to Jo, turned studious, and gave out that he
was going to "dig," intending to graduate in a blaze of glory. This
suited the young lady better than twilight confidences, tender pressures
of the hand, and eloquent glances of the eye; for with Jo, brain
developed earlier than heart, and she preferred imaginary heroes to real
ones, because, when tired of them, the former could be shut up in the
tin-kitchen till called for, and the latter were less manageable.

Things were in this state when the grand discovery was made, and Jo
watched Laurie that night as she had never done before. If she had not
got the new idea into her head, she would have seen nothing unusual in
the fact that Beth was very quiet, and Laurie very kind to her. But
having given the rein to her lively fancy, it galloped away with her at
a great pace; and common sense, being rather weakened by a long course
of romance writing, did not come to the rescue. As usual, Beth lay on
the sofa, and Laurie sat in a low chair close by, amusing her with all
sorts of gossip; for she depended on her weekly "spin," and he never
disappointed her. But that evening, Jo fancied that
Beth\textquotesingle s eyes rested on the lively, dark face beside her
with peculiar pleasure, and that she listened with intense interest to
an account of some exciting cricket-match, though the phrases, "caught
off a tice," "stumped off his ground," and "the leg hit for three," were
as intelligible to her as Sanscrit. She also fancied, having set her
heart upon seeing it, that she saw a certain increase of gentleness in
Laurie\textquotesingle s manner, that he dropped his voice now and then,
laughed less than usual, was a little absent-minded, and settled the
afghan over Beth\textquotesingle s feet with an assiduity that was
really almost tender.

"Who knows? stranger things have happened," thought Jo, as she fussed
about the room. "She will make quite an angel of him, and he will make
life delightfully easy and pleasant for the dear, if they only love each
other. I don\textquotesingle t see how he can help it; and I do believe
he would if the rest of us were out of the way."

As every one \emph{was} out of the way but herself, Jo began to feel
that she ought to dispose of herself with all speed. But where should
she go? and burning to lay herself upon the shrine of sisterly devotion,
she sat down to settle that point.

Now, the old sofa was a regular patriarch of a sofa,---long, broad,
well-cushioned, and low; a trifle shabby, as well it might be, for the
girls had slept and sprawled on it as babies, fished over the back, rode
on the arms, and had menageries under it as children, and rested tired
heads, dreamed dreams, and listened to tender talk on it as young women.
They all loved it, for it was a family refuge, and one corner had always
been Jo\textquotesingle s favorite lounging-place. Among the many
pillows that adorned the venerable couch was one, hard, round, covered
with prickly horsehair, and furnished with a knobby button at each end;
this repulsive pillow was her especial property, being used as a weapon
of defence, a barricade, or a stern preventive of too much slumber.

Laurie knew this pillow well, and had cause to regard it with deep
aversion, having been unmercifully pummelled with it in former days,
when romping was allowed, and now frequently debarred by it from taking
the seat he most coveted, next to Jo in the sofa corner. If "the
sausage" as they called it, stood on end, it was a sign that he might
approach and repose; but if it lay flat across the sofa, woe to the man,
woman, or child who dared disturb it! That evening Jo forgot to
barricade her corner, and had not been in her seat five minutes, before
a massive form appeared beside her, and, with both arms spread over the
sofa-back, both long legs stretched out before him, Laurie exclaimed,
with a sigh of satisfaction,---

"Now, \emph{this} is filling at the price."

\protect\phantomsection\label{6672479776654687619_37106-h-5.htm.xhtml_b147.png}{}
\pandocbounded{\includegraphics[keepaspectratio]{303483661336987339_b147.png}}

"No slang," snapped Jo, slamming down the pillow. But it was too late,
there was no room for it; and, coasting on to the floor, it disappeared
in a most mysterious manner.

"Come, Jo, don\textquotesingle t be thorny. After studying himself to a
skeleton all the week, a fellow deserves petting, and ought to get it."

"Beth will pet you; I\textquotesingle m busy."

"No, she\textquotesingle s not to be bothered with me; but you like that
sort of thing, unless you\textquotesingle ve suddenly lost your taste
for it. Have you? Do you hate your boy, and want to fire pillows at
him?"

Anything more wheedlesome than that touching appeal was seldom heard,
but Jo quenched "her boy" by turning on him with the stern query,---

"How many bouquets have you sent Miss Randal this week?"

"Not one, upon my word. She\textquotesingle s engaged. Now then."

"I\textquotesingle m glad of it; that\textquotesingle s one of your
foolish extravagances,---sending flowers and things to girls for whom
you don\textquotesingle t care two pins," continued Jo reprovingly.

"Sensible girls, for whom I do care whole papers of pins,
won\textquotesingle t let me send them \textquotesingle flowers and
things,\textquotesingle{} so what can I do? My feelings must have a
\emph{went}."

"Mother doesn\textquotesingle t approve of flirting, even in fun; and
you do flirt desperately, Teddy."

"I\textquotesingle d give anything if I could answer,
\textquotesingle So do you.\textquotesingle{} As I
can\textquotesingle t, I\textquotesingle ll merely say that I
don\textquotesingle t see any harm in that pleasant little game, if all
parties understand that it\textquotesingle s only play."

"Well, it does look pleasant, but I can\textquotesingle t learn how
it\textquotesingle s done. I\textquotesingle ve tried, because one feels
awkward in company, not to do as everybody else is doing; but I
don\textquotesingle t seem to get on," said Jo, forgetting to play
Mentor.

"Take lessons of Amy; she has a regular talent for it."

"Yes, she does it very prettily, and never seems to go too far. I
suppose it\textquotesingle s natural to some people to please without
trying, and others to always say and do the wrong thing in the wrong
place."

"I\textquotesingle m glad you can\textquotesingle t flirt;
it\textquotesingle s really refreshing to see a sensible,
straightforward girl, who can be jolly and kind without making a fool of
herself. Between ourselves, Jo, some of the girls I know really do go on
at such a rate I\textquotesingle m ashamed of them. They
don\textquotesingle t mean any harm, I\textquotesingle m sure; but if
they knew how we fellows talked about them afterward,
they\textquotesingle d mend their ways, I fancy."

"They do the same; and, as their tongues are the sharpest, you fellows
get the worst of it, for you are as silly as they, every bit. If you
behaved properly, they would; but, knowing you like their nonsense, they
keep it up, and then you blame them."

"Much you know about it, ma\textquotesingle am," said Laurie, in a
superior tone. "We don\textquotesingle t like romps and flirts, though
we may act as if we did sometimes. The pretty, modest girls are never
talked about, except respectfully, among gentlemen. Bless your innocent
soul! If you could be in my place for a month you\textquotesingle d see
things that would astonish you a trifle. Upon my word, when I see one of
those harum-scarum girls, I always want to say with our friend Cock
Robin,---

"\textquotesingle Out upon you, fie upon you,

Bold-faced jig!\textquotesingle"

It was impossible to help laughing at the funny conflict between
Laurie\textquotesingle s chivalrous reluctance to speak ill of
womankind, and his very natural dislike of the unfeminine folly of which
fashionable society showed him many samples. Jo knew that "young
Laurence" was regarded as a most eligible \emph{parti} by worldly
mammas, was much smiled upon by their daughters, and flattered enough by
ladies of all ages to make a coxcomb of him; so she watched him rather
jealously, fearing he would be spoilt, and rejoiced more than she
confessed to find that he still believed in modest girls. Returning
suddenly to her admonitory tone, she said, dropping her voice, "If you
\emph{must} have a \textquotesingle went,\textquotesingle{} Teddy, go
and devote yourself to one of the \textquotesingle pretty, modest
girls\textquotesingle{} whom you do respect, and not waste your time
with the silly ones."

"You really advise it?" and Laurie looked at her with an odd mixture of
anxiety and merriment in his face.

"Yes, I do; but you\textquotesingle d better wait till you are through
college, on the whole, and be fitting yourself for the place meantime.
You\textquotesingle re not half good enough for---well, whoever the
modest girl may be," and Jo looked a little queer likewise, for a name
had almost escaped her.

"That I\textquotesingle m not!" acquiesced Laurie, with an expression of
humility quite new to him, as he dropped his eyes, and absently wound
Jo\textquotesingle s apron-tassel round his finger.

"Mercy on us, this will never do," thought Jo; adding aloud, "Go and
sing to me. I\textquotesingle m dying for some music, and always like
yours."

"I\textquotesingle d rather stay here, thank you."

"Well, you can\textquotesingle t; there isn\textquotesingle t room. Go
and make yourself useful, since you are too big to be ornamental. I
thought you hated to be tied to a woman\textquotesingle s apron-string?"
retorted Jo, quoting certain rebellious words of his own.

"Ah, that depends on who wears the apron!" and Laurie gave an audacious
tweak at the tassel.

"Are you going?" demanded Jo, diving for the pillow.

He fled at once, and the minute it was well "Up with the bonnets of
bonnie Dundee," she slipped away, to return no more till the young
gentleman had departed in high dudgeon.

\protect\phantomsection\label{6672479776654687619_37106-h-5.htm.xhtml_b148.png}{}
\pandocbounded{\includegraphics[keepaspectratio]{303483661336987339_b148.png}}

Jo lay long awake that night, and was just dropping off when the sound
of a stifled sob made her fly to Beth\textquotesingle s bedside, with
the anxious inquiry, "What is it, dear?"

"I thought you were asleep," sobbed Beth.

"Is it the old pain, my precious?"

"No; it\textquotesingle s a new one; but I can bear it," and Beth tried
to check her tears.

"Tell me all about it, and let me cure it as I often did the other."

"You can\textquotesingle t; there is no cure." There
Beth\textquotesingle s voice gave way, and, clinging to her sister, she
cried so despairingly that Jo was frightened.

"Where is it? Shall I call mother?"

Beth did not answer the first question; but in the dark one hand went
involuntarily to her heart, as if the pain were there; with the other
she held Jo fast, whispering eagerly, "No, no, don\textquotesingle t
call her, don\textquotesingle t tell her. I shall be better soon. Lie
down here and \textquotesingle poor\textquotesingle{} my head.
I\textquotesingle ll be quiet, and go to sleep; indeed I will."

Jo obeyed; but as her hand went softly to and fro across
Beth\textquotesingle s hot forehead and wet eyelids, her heart was very
full, and she longed to speak. But young as she was, Jo had learned that
hearts, like flowers, cannot be rudely handled, but must open naturally;
so, though she believed she knew the cause of Beth\textquotesingle s new
pain, she only said, in her tenderest tone, "Does anything trouble you,
deary?"

"Yes, Jo," after a long pause.

"Wouldn\textquotesingle t it comfort you to tell me what it is?"

"Not now, not yet."

"Then I won\textquotesingle t ask; but remember, Bethy, that mother and
Jo are always glad to hear and help you, if they can."

"I know it. I\textquotesingle ll tell you by and by."

"Is the pain better now?"

"Oh, yes, much better; you are so comfortable, Jo!"

"Go to sleep, dear; I\textquotesingle ll stay with you."

So cheek to cheek they fell asleep, and on the morrow Beth seemed quite
herself again; for at eighteen, neither heads nor hearts ache long, and
a loving word can medicine most ills.

But Jo had made up her mind, and, after pondering over a project for
some days, she confided it to her mother.

"You asked me the other day what my wishes were. I\textquotesingle ll
tell you one of them, Marmee," she began, as they sat alone together. "I
want to go away somewhere this winter for a change."

"Why, Jo?" and her mother looked up quickly, as if the words suggested a
double meaning.

With her eyes on her work, Jo answered soberly, "I want something new; I
feel restless, and anxious to be seeing, doing, and learning more than I
am. I brood too much over my own small affairs, and need stirring up,
so, as I can be spared this winter, I\textquotesingle d like to hop a
little way, and try my wings."

"Where will you hop?"

"To New York. I had a bright idea yesterday, and this is it. You know
Mrs. Kirke wrote to you for some respectable young person to teach her
children and sew. It\textquotesingle s rather hard to find just the
thing, but I think I should suit if I tried."

"My dear, go out to service in that great boarding-house!" and Mrs.
March looked surprised, but not displeased.

"It\textquotesingle s not exactly going out to service; for Mrs. Kirke
is your friend,---the kindest soul that ever lived,---and would make
things pleasant for me, I know. Her family is separate from the rest,
and no one knows me there. Don\textquotesingle t care if they do;
it\textquotesingle s honest work, and I\textquotesingle m not ashamed of
it."

"Nor I; but your writing?"

"All the better for the change. I shall see and hear new things, get new
ideas, and, even if I haven\textquotesingle t much time there, I shall
bring home quantities of material for my rubbish."

"I have no doubt of it; but are these your only reasons for this sudden
fancy?"

"No, mother."

"May I know the others?"

Jo looked up and Jo looked down, then said slowly, with sudden color in
her cheeks, "It may be vain and wrong to say it,
but---I\textquotesingle m afraid---Laurie is getting too fond of me."

"Then you don\textquotesingle t care for him in the way it is evident he
begins to care for you?" and Mrs. March looked anxious as she put the
question.

"Mercy, no! I love the dear boy, as I always have, and am immensely
proud of him; but as for anything more, it\textquotesingle s out of the
question."

"I\textquotesingle m glad of that, Jo."

"Why, please?"

"Because, dear, I don\textquotesingle t think you suited to one another.
As friends you are very happy, and your frequent quarrels soon blow
over; but I fear you would both rebel if you were mated for life. You
are too much alike and too fond of freedom, not to mention hot tempers
and strong wills, to get on happily together, in a relation which needs
infinite patience and forbearance, as well as love."

"That\textquotesingle s just the feeling I had, though I
couldn\textquotesingle t express it. I\textquotesingle m glad you think
he is only beginning to care for me. It would trouble me sadly to make
him unhappy; for I couldn\textquotesingle t fall in love with the dear
old fellow merely out of gratitude, could I?"

"You are sure of his feeling for you?"

The color deepened in Jo\textquotesingle s cheeks, as she answered, with
the look of mingled pleasure, pride, and pain which young girls wear
when speaking of first lovers,---

"I\textquotesingle m afraid it is so, mother; he hasn\textquotesingle t
said anything, but he looks a great deal. I think I had better go away
before it comes to anything."

"I agree with you, and if it can be managed you shall go."

Jo looked relieved, and, after a pause, said, smiling, "How Mrs. Moffat
would wonder at your want of management, if she knew; and how she will
rejoice that Annie still may hope."

"Ah, Jo, mothers may differ in their management, but the hope is the
same in all,---the desire to see their children happy. Meg is so, and I
am content with her success. You I leave to enjoy your liberty till you
tire of it; for only then will you find that there is something sweeter.
Amy is my chief care now, but her good sense will help her. For Beth, I
indulge no hopes except that she may be well. By the way, she seems
brighter this last day or two. Have you spoken to her?"

"Yes; she owned she had a trouble, and promised to tell me by and by. I
said no more, for I think I know it;" and Jo told her little story.

Mrs. March shook her head, and did not take so romantic a view of the
case, but looked grave, and repeated her opinion that, for
Laurie\textquotesingle s sake, Jo should go away for a time.

"Let us say nothing about it to him till the plan is settled; then
I\textquotesingle ll run away before he can collect his wits and be
tragical. Beth must think I\textquotesingle m going to please myself, as
I am, for I can\textquotesingle t talk about Laurie to her; but she can
pet and comfort him after I\textquotesingle m gone, and so cure him of
this romantic notion. He\textquotesingle s been through so many little
trials of the sort, he\textquotesingle s used to it, and will soon get
over his love-lornity."

Jo spoke hopefully, but could not rid herself of the foreboding fear
that this "little trial" would be harder than the others, and that
Laurie would not get over his "love-lornity" as easily as heretofore.

The plan was talked over in a family council, and agreed upon; for Mrs.
Kirke gladly accepted Jo, and promised to make a pleasant home for her.
The teaching would render her independent; and such leisure as she got
might be made profitable by writing, while the new scenes and society
would be both useful and agreeable. Jo liked the prospect and was eager
to be gone, for the home-nest was growing too narrow for her restless
nature and adventurous spirit. When all was settled, with fear and
trembling she told Laurie; but to her surprise he took it very quietly.
He had been graver than usual of late, but very pleasant; and, when
jokingly accused of turning over a new leaf, he answered soberly, "So I
am; and I mean this one shall stay turned."

Jo was very much relieved that one of his virtuous fits should come on
just then, and made her preparations with a lightened heart,---for Beth
seemed more cheerful,---and hoped she was doing the best for all.

"One thing I leave to your especial care," she said, the night before
she left.

"You mean your papers?" asked Beth.

"No, my boy. Be very good to him, won\textquotesingle t you?"

"Of course I will; but I can\textquotesingle t fill your place, and
he\textquotesingle ll miss you sadly."

"It won\textquotesingle t hurt him; so remember, I leave him in your
charge, to plague, pet, and keep in order."

"I\textquotesingle ll do my best, for your sake," promised Beth,
wondering why Jo looked at her so queerly.

When Laurie said "Good-by," he whispered significantly, "It
won\textquotesingle t do a bit of good, Jo. My eye is on you; so mind
what you do, or I\textquotesingle ll come and bring you home."

\begin{center}\rule{0.5\linewidth}{0.5pt}\end{center}

\subsection{XXXIII. Jo\textquotesingle s
Journal.}\label{6672479776654687619_37106-h-5.htm.xhtml_pgepubid00035}

\protect\phantomsection\label{6672479776654687619_37106-h-5.htm.xhtml_b149.png}{}
\pandocbounded{\includegraphics[keepaspectratio]{303483661336987339_b149.png}}

\protect\phantomsection\label{6672479776654687619_37106-h-5.htm.xhtml_XXXIII}{}\hyperref[6672479776654687619_37106-h-0.htm.xhtml_contents2]{XXXIII.}

JO\textquotesingle S JOURNAL.

\begin{quote}
"{New York}, November.

"Dear Marmee and Beth,---

"I\textquotesingle m going to write you a regular volume, for
I\textquotesingle ve got heaps to tell, though I\textquotesingle m not a
fine young lady travelling on the continent. When I lost sight of
father\textquotesingle s dear old face, I felt a trifle blue, and might
have shed a briny drop or two, if an Irish lady with four small
children, all crying more or less, hadn\textquotesingle t diverted my
mind; for I amused myself by dropping gingerbread nuts over the seat
every time they opened their mouths to roar.

"Soon the sun came out, and taking it as a good omen, I cleared up
likewise, and enjoyed my journey with all my heart.

"Mrs. Kirke welcomed me so kindly I felt at home at once, even in that
big house full of strangers. She gave me a funny little sky-parlor---all
she had; but there is a stove in it, and a nice table in a sunny window,
so I can sit here and write whenever I like. A fine view and a
church-tower opposite atone for the many stairs, and I took a fancy to
my den on the spot. The nursery, where I am to teach and sew, is a
pleasant room next Mrs. Kirke\textquotesingle s private parlor, and the
two little girls are pretty children,---rather spoilt, I fancy, but they
took to me after telling them \textquotesingle The Seven Bad
Pigs;\textquotesingle{} and I\textquotesingle ve no doubt I shall make a
model governess.

"I am to have my meals with the children, if I prefer it to the great
table, and for the present I do, for I \emph{am} bashful, though no one
will believe it.

"\textquotesingle Now, my dear, make yourself at home,\textquotesingle{}
said Mrs. K. in her motherly way; \textquotesingle I\textquotesingle m
on the drive from morning to night, as you may suppose with such a
family; but a great anxiety will be off my mind if I know the children
are safe with you. My rooms are always open to you, and your own shall
be as comfortable as I can make it. There are some pleasant people in
the house if you feel sociable, and your evenings are always free. Come
to me if anything goes wrong, and be as happy as you can.
There\textquotesingle s the tea-bell; I must run and change my
cap;\textquotesingle{} and off she bustled, leaving me to settle myself
in my new nest.

"As I went downstairs, soon after, I saw something I liked. The flights
are very long in this tall house, and as I stood waiting at the head of
the third one for a little servant girl to lumber up, I saw a gentleman
come along behind her, take the heavy hod of coal out of her hand, carry
it all the way up, put it down at a door near by, and walk away, saying,
with a kind nod and a foreign accent,---

"\textquotesingle It goes better so. The little back is too young to haf
such \ul{heaviness.\textquotesingle{}}

"Wasn\textquotesingle t it good of him? I like such things, for, as
father says, trifles show character. When I mentioned it to Mrs. K.,
that evening, she laughed, and said,---

"\textquotesingle That must have been Professor Bhaer;
he\textquotesingle s always doing things of that sort.\textquotesingle{}

"Mrs. K. told me he was from Berlin; very learned and good, but poor as
a church-mouse, and gives lessons to support himself and two little
orphan nephews whom he is educating here, according to the wishes of his
sister, who married an American. Not a very romantic story, but it
interested me; and I was glad to hear that Mrs. K. lends him her parlor
for some of his scholars. There is a glass door between it and the
nursery, and I mean to peep at him, and then I\textquotesingle ll tell
you how he looks. He\textquotesingle s almost forty, so
it\textquotesingle s no harm, Marmee.

"After tea and a go-to-bed romp with the little girls, I attacked the
big work-basket, and had a quiet evening chatting with my new friend. I
shall keep a journal-letter, and send it once a week; so good-night, and
more to-morrow."

\begin{center}\rule{0.5\linewidth}{0.5pt}\end{center}

"\emph{Tuesday Eve.}

"Had a lively time in my seminary, this morning, for the children acted
like Sancho; and at one time I really thought I should shake them all
round. Some good angel inspired me to try gymnastics, and I kept it up
till they were glad to sit down and keep still. After luncheon, the girl
took them out for a walk, and I went to my needle-work, like little
Mabel, \textquotesingle with a willing mind.\textquotesingle{} I was
thanking my stars that I\textquotesingle d learned to make nice
button-holes, when the parlor-door opened and shut, and some one began
to hum,---

\textquotesingle Kennst du das land,\textquotesingle{}

like a big bumble-bee. It was dreadfully improper, I know, but I
couldn\textquotesingle t resist the temptation; and lifting one end of
the curtain before the glass door, I peeped in. Professor Bhaer was
there; and while he arranged his books, I took a good look at him. A
regular German,---rather stout, with brown hair tumbled all over his
head, a bushy beard, good nose, the kindest eyes I ever saw, and a
splendid big voice that does one\textquotesingle s ears good, after our
sharp or slipshod American gabble. His clothes were rusty, his hands
were large, and he hadn\textquotesingle t a really handsome feature in
his face, except his beautiful teeth; yet I liked him, for he had a fine
head; his linen was very nice, and he looked like a gentleman, though
two buttons were off his coat, and there was a patch on one shoe. He
looked sober in spite of his humming, till he went to the window to turn
the hyacinth bulbs toward the sun, and stroke the cat, who received him
like an old friend. Then he smiled; and when a tap came at the door,
called out in a loud, brisk tone,---

"\textquotesingle Herein!\textquotesingle{}

"I was just going to run, when I caught sight of a morsel of a child
carrying a big book, and stopped to see what was going on.

"\textquotesingle Me wants my Bhaer,\textquotesingle{} said the mite,
slamming down her book, and running to meet him.
\end{quote}

\protect\phantomsection\label{6672479776654687619_37106-h-5.htm.xhtml_b150.png}{}
\pandocbounded{\includegraphics[keepaspectratio]{303483661336987339_b150.png}}

\begin{quote}
"\textquotesingle Thou shalt haf thy Bhaer; come, then, and take a goot
hug from him, my Tina,\textquotesingle{} said the Professor, catching
her up, with a laugh, and holding her so high over his head that she had
to stoop her little face to kiss him.

"\textquotesingle Now me mus tuddy my lessin,\textquotesingle{} went on
the funny little thing; so he put her up at the table, opened the great
dictionary she had brought, and gave her a paper and pencil, and she
scribbled away, turning a leaf now and then, and passing her little fat
finger down the page, as if finding a word, so soberly that I nearly
betrayed myself by a laugh, while Mr. Bhaer stood stroking her pretty
hair, with a fatherly look, that made me think she must be his own,
though she looked more French than German.

"Another knock and the appearance of two young ladies sent me back to my
work, and there I virtuously remained through all the noise and gabbling
that went on next door. One of the girls kept laughing affectedly, and
saying \textquotesingle Now Professor,\textquotesingle{} in a coquettish
tone, and the other pronounced her German with an accent that must have
made it hard for him to keep sober.

"Both seemed to try his patience sorely; for more than once I heard him
say emphatically, \textquotesingle No, no, it is \emph{not} so; you haf
not attend to what I say;\textquotesingle{} and once there was a loud
rap, as if he struck the table with his book, followed by the despairing
exclamation, \textquotesingle Prut! it all goes bad this
day.\textquotesingle{}

"Poor man, I pitied him; and when the girls were gone, took just one
more peep, to see if he survived it. He seemed to have thrown himself
back in his chair, tired out, and sat there with his eyes shut till the
clock struck two, when he jumped up, put his books in his pocket, as if
ready for another lesson, and, taking little Tina, who had fallen asleep
on the sofa, in his arms, he carried her quietly away. I fancy he has a
hard life of it.

"Mrs. Kirke asked me if I wouldn\textquotesingle t go down to the five
o\textquotesingle clock dinner; and, feeling a little bit homesick, I
thought I would, just to see what sort of people are under the same roof
with me. So I made myself respectable, and tried to slip in behind Mrs.
Kirke; but as she is short, and I\textquotesingle m tall, my efforts at
concealment were rather a failure. She gave me a seat by her, and after
my face cooled off, I plucked up courage, and looked about me. The long
table was full, and every one intent on getting their dinner,---the
gentlemen especially, who seemed to be eating on time, for they
\emph{bolted} in every sense of the word, vanishing as soon as they were
done. There was the usual assortment of young men absorbed in
themselves; young couples absorbed in each other; married ladies in
their babies, and old gentlemen in politics. I don\textquotesingle t
think I shall care to have much to do with any of them, except one
sweet-faced maiden lady, who looks as if she had something in her.

"Cast away at the very bottom of the table was the Professor, shouting
answers to the questions of a very inquisitive, deaf old gentleman on
one side, and talking philosophy with a Frenchman on the other. If Amy
had been here, she\textquotesingle d have turned her back on him
forever, because, sad to relate, he had a great appetite, and shovelled
in his dinner in a manner which would have horrified
\textquotesingle her ladyship.\textquotesingle{} I
didn\textquotesingle t mind, for I like \textquotesingle to see folks
eat with a relish,\textquotesingle{} as Hannah says, and the poor man
must have needed a deal of food after teaching idiots all day.

"As I went upstairs after dinner, two of the young men were settling
their hats before the hall-mirror, and I heard one say low to the other,
\textquotesingle Who\textquotesingle s the new party?\textquotesingle{}

"\textquotesingle Governess, or something of that
sort.\textquotesingle{}

"\textquotesingle What the deuce is she at our table
for?\textquotesingle{}

"\textquotesingle Friend of the old
lady\textquotesingle s.\textquotesingle{}

"\textquotesingle Handsome head, but no style.\textquotesingle{}

"\textquotesingle Not a bit of it. Give us a light and come
on.\textquotesingle{}

"I felt angry at first, and then I didn\textquotesingle t care, for a
governess is as good as a clerk, and I\textquotesingle ve got sense, if
I haven\textquotesingle t style, which is more than some people have,
judging from the remarks of the elegant beings who clattered away,
smoking like bad chimneys. I hate ordinary people!"

\begin{center}\rule{0.5\linewidth}{0.5pt}\end{center}

"\emph{Thursday.}

"Yesterday was a quiet day, spent in teaching, sewing, and writing in my
little room, which is very cosey, with a light and fire. I picked up a
few bits of news, and was introduced to the Professor. It seems that
Tina is the child of the Frenchwoman who does the fine ironing in the
laundry here. The little thing has lost her heart to Mr. Bhaer, and
follows him about the house like a dog whenever he is at home, which
delights him, as he is very fond of children, though a
\textquotesingle bacheldore.\textquotesingle{} Kitty and Minnie Kirke
likewise regard him with affection, and tell all sorts of stories about
the plays he invents, the presents he brings, and the splendid tales he
tells. The young men quiz him, it seems, call him Old Fritz, Lager Beer,
Ursa Major, and make all manner of jokes on his name. But he enjoys it
like a boy, Mrs. K. says, and takes it so good-naturedly that they all
like him, in spite of his foreign ways.

"The maiden lady is a Miss Norton,---rich, cultivated, and kind. She
spoke to me at dinner to-day (for I went to table again,
it\textquotesingle s such fun to watch people), and asked me to come and
see her at her room. She has fine books and pictures, knows interesting
persons, and seems friendly; so I shall make myself agreeable, for I
\emph{do} want to get into good society, only it isn\textquotesingle t
the same sort that Amy likes.

"I was in our parlor last evening, when Mr. Bhaer came in with some
newspapers for Mrs. Kirke. She wasn\textquotesingle t there, but Minnie,
who is a little old woman, introduced me very prettily:
\textquotesingle This is mamma\textquotesingle s friend, Miss
March.\textquotesingle{}

"\textquotesingle Yes; and she\textquotesingle s jolly and we like her
lots,\textquotesingle{} added Kitty, who is an \emph{enfant terrible}.

"We both bowed, and then we laughed, for the prim introduction and the
blunt addition were rather a comical contrast.

"\textquotesingle Ah, yes, I hear these naughty ones go to vex you, Mees
Marsch. If so again, call at me and I come,\textquotesingle{} he said,
with a threatening frown that delighted the little wretches.

\protect\phantomsection\label{6672479776654687619_37106-h-5.htm.xhtml_b151.png}{}
\pandocbounded{\includegraphics[keepaspectratio]{303483661336987339_b151.png}}

"I promised I would, and he departed; but it seems as if I was doomed to
see a good deal of him, for to-day, as I passed his door on my way out,
by accident I knocked against it with my umbrella. It flew open, and
there he stood in his dressing gown, with a big blue sock on one hand,
and a darning-needle in the other; he didn\textquotesingle t seem at all
ashamed of it, for when I explained and hurried on, he waved his hand,
sock and all, saying in his loud, cheerful way,---

"\textquotesingle You haf a fine day to make your walk. \emph{Bon
voyage, mademoiselle.}\textquotesingle{}

"I laughed all the way downstairs; but it was a little pathetic, also,
to think of the poor man having to mend his own clothes. The German
gentlemen embroider, I know; but darning hose is another thing, and not
so pretty."

\begin{center}\rule{0.5\linewidth}{0.5pt}\end{center}

"\emph{Saturday.}

"Nothing has happened to write about, except a call on Miss Norton, who
has a room full of lovely things, and who was very charming, for she
showed me all her treasures, and asked me if I would sometimes go with
her to lectures and concerts, as her escort,---if I enjoyed them. She
put it as a favor, but I\textquotesingle m sure Mrs. Kirke has told her
about us, and she does it out of kindness to me. I\textquotesingle m as
proud as Lucifer, but such favors from such people don\textquotesingle t
burden me, and I accepted gratefully.

"When I got back to the nursery there was such an uproar in the parlor
that I looked in; and there was Mr. Bhaer down on his hands and knees,
with Tina on his back, Kitty leading him with a jump-rope, and Minnie
feeding two small boys with seed-cakes, as they roared and ramped in
cages built of chairs.

"\textquotesingle We are playing \emph{nargerie},\textquotesingle{}
explained Kitty.

"\textquotesingle Dis is mine effalunt!\textquotesingle{} added Tina,
holding on by the Professor\textquotesingle s hair.

\protect\phantomsection\label{6672479776654687619_37106-h-5.htm.xhtml_b152.png}{}
\pandocbounded{\includegraphics[keepaspectratio]{303483661336987339_b152.png}}

"\textquotesingle Mamma always allows us to do what we like Saturday
afternoon, when Franz and Emil come, doesn\textquotesingle t she, Mr.
Bhaer?\textquotesingle{} said Minnie.

"The \textquotesingle effalunt\textquotesingle{} sat up, looking as much
in earnest as any of them, and said soberly to me,---

"\textquotesingle I gif you my wort it is so. If we make too large a
noise you shall say "Hush!" to us, and we go more
softly.\textquotesingle{}

"I promised to do so, but left the door open, and enjoyed the fun as
much as they did,---for a more glorious frolic I never witnessed. They
played tag and soldiers, danced and sung, and when it began to grow dark
they all piled on to the sofa about the Professor, while he told
charming fairy stories of the storks on the chimney-tops, and the little
\textquotesingle kobolds,\textquotesingle{} who ride the snow-flakes as
they fall. I wish Americans were as simple and natural as Germans,
don\textquotesingle t you?

"I\textquotesingle m so fond of writing, I should go spinning on forever
if motives of economy didn\textquotesingle t stop me, for though
I\textquotesingle ve used thin paper and written fine, I tremble to
think of the stamps this long letter will need. Pray forward
Amy\textquotesingle s as soon as you can spare them. My small news will
sound very flat after her splendors, but you will like them, I know. Is
Teddy studying so hard that he can\textquotesingle t find time to write
to his friends? Take good care of him for me, Beth, and tell me all
about the babies, and give heaps of love to every one.

{"From your faithful}

Jo.

"P.~S. On reading over my letter it strikes me as rather Bhaery; but I
am always interested in odd people, and I really had nothing else to
write about. Bless you!"
\end{quote}

\begin{center}\rule{0.5\linewidth}{0.5pt}\end{center}

\begin{quote}
"December.

"My Precious Betsey,---

"As this is to be a scribble-scrabble letter, I direct it to you, for it
may amuse you, and give you some idea of my goings on; for, though
quiet, they are rather amusing, for which, oh, be joyful! After what Amy
would call Herculaneum efforts, in the way of mental and moral
agriculture, my young ideas begin to shoot and my little twigs to bend
as I could wish. They are not so interesting to me as Tina and the boys,
but I do my duty by them, and they are fond of me. Franz and Emil are
jolly little lads, quite after my own heart; for the mixture of German
and American spirit in them produces a constant state of effervescence.
Saturday afternoons are riotous times, whether spent in the house or
out; for on pleasant days they all go to walk, like a seminary, with the
Professor and myself to keep order; and then such fun!

"We are very good friends now, and I\textquotesingle ve begun to take
lessons. I really couldn\textquotesingle t help it, and it all came
about in such a droll way that I must tell you. To begin at the
beginning, Mrs. Kirke called to me, one day, as I passed Mr.
Bhaer\textquotesingle s room, where she was rummaging.

"\textquotesingle Did you ever see such a den, my dear? Just come and
help me put these books to rights, for I\textquotesingle ve turned
everything upside down, trying to discover what he has done with the six
new handkerchiefs I gave him not long ago.\textquotesingle{}

"I went in, and while we worked I looked about me, for it was
\textquotesingle a den,\textquotesingle{} to be sure. Books and papers
everywhere; a broken meerschaum, and an old flute over the mantel-piece
as if done with; a ragged bird, without any tail, chirped on one
window-seat, and a box of white mice adorned the other; half-finished
boats and bits of string lay among the manuscripts; dirty little boots
stood drying before the fire; and traces of the dearly beloved boys, for
whom he makes a slave of himself, were to be seen all over the room.
After a grand rummage three of the missing articles were found,---one
over the bird-cage, one covered with ink, and a third burnt brown,
having been used as a holder.

"\textquotesingle Such a man!\textquotesingle{} laughed good-natured
Mrs. K., as she put the relics in the rag-bag. \textquotesingle I
suppose the others are torn up to rig ships, bandage cut fingers, or
make kite-tails. It\textquotesingle s dreadful, but I
can\textquotesingle t scold him: he\textquotesingle s so absent-minded
and good-natured, he lets those boys ride over him rough-shod. I agreed
to do his washing and mending, but he forgets to give out his things and
I forget to look them over, so he comes to a sad pass
sometimes.\textquotesingle{}

"\textquotesingle Let me mend them,\textquotesingle{} said I.
\textquotesingle I don\textquotesingle t mind it, and he
needn\textquotesingle t know. I\textquotesingle d like
to,---he\textquotesingle s so kind to me about bringing my letters and
lending books.\textquotesingle{}

"So I have got his things in order, and knit heels into two pairs of the
socks,---for they were boggled out of shape with his queer darns.
Nothing was said, and I hoped he wouldn\textquotesingle t find it out,
but one day last week he caught me at it. Hearing the lessons he gives
to others has interested and amused me so much that I took a fancy to
learn; for Tina runs in and out, leaving the door open, and I can hear.
I had been sitting near this door, finishing off the last sock, and
trying to understand what he said to a new scholar, who is as stupid as
I am. The girl had gone, and I thought he had also, it was so still, and
I was busily gabbling over a verb, and rocking to and fro in a most
absurd way, when a little crow made me look up, and there was Mr. Bhaer
looking and laughing quietly, while he made signs to Tina not to betray
him.

"\textquotesingle So!\textquotesingle{} he said, as I stopped and stared
like a goose, \textquotesingle you peep at me, I peep at you, and that
is not bad; but see, I am not pleasanting when I say, haf you a wish for
German?\textquotesingle{}

"\textquotesingle Yes; but you are too busy. I am too stupid to
learn,\textquotesingle{} I blundered out, as red as a peony.

"\textquotesingle Prut! we will make the time, and we fail not to find
the sense. At efening I shall gif a little lesson with much gladness;
for, look you, Mees Marsch, I haf this debt to pay,\textquotesingle{}
and he pointed to my work. \textquotesingle Yes, they say to one
another, these so kind ladies, "he is a stupid old fellow; he will see
not what we do; he will never opserve that his sock-heels go not in
holes any more, he will think his buttons grow out new when they fall,
and believe that strings make theirselves." Ah! but I haf an eye, and I
see much. I haf a heart, and I feel the thanks for this. Come, a little
lesson then and now, or no more good fairy works for me and
mine.\textquotesingle{}

"Of course I couldn\textquotesingle t say anything after that, and as it
really is a splendid opportunity, I made the bargain, and we began. I
took four lessons, and then I stuck fast in a grammatical bog. The
Professor was very patient with me, but it must have been torment to
him, and now and then he\textquotesingle d look at me with such an
expression of mild despair that it was a toss-up with me whether to
laugh or cry. I tried both ways; and when it came to a sniff of utter
mortification and woe, he just threw the grammar on to the floor, and
marched out of the room. I felt myself disgraced and deserted forever,
but didn\textquotesingle t blame him a particle, and was scrambling my
papers together, meaning to rush upstairs and shake myself hard, when in
he came, as brisk and beaming as if I\textquotesingle d covered myself
with glory.

"\textquotesingle Now we shall try a new way. You and I will read these
pleasant little Märchen together, and dig no more in that dry book, that
goes in the corner for making us trouble.\textquotesingle{}

"He spoke so kindly, and opened Hans Andersen\textquotesingle s fairy
tales so invitingly before me, that I was more ashamed than ever, and
went at my lesson in a neck-or-nothing style that seemed to amuse him
immensely. I forgot my bashfulness, and pegged away (no other word will
express it) with all my might, tumbling over long words, pronouncing
according to the inspiration of the minute, and doing my very best. When
I finished reading my first page, and stopped for breath, he clapped his
hands and cried out, in his hearty way, \textquotesingle Das ist gute!
Now we go well! My turn. I do him in German; gif me your
ear.\textquotesingle{} And away he went, rumbling out the words with his
strong voice, and a relish which was good to see as well as hear.
Fortunately the story was the \textquotesingle Constant Tin
Soldier,\textquotesingle{} which is droll, you know, so I could
laugh,---and I did,---though I didn\textquotesingle t understand half he
read, for I couldn\textquotesingle t help it, he was so earnest, I so
excited, and the whole thing so comical.

"After that we got on better, and now I read my lessons pretty well; for
this way of studying suits me, and I can see that the grammar gets
tucked into the tales and poetry as one gives pills in jelly. I like it
very much, and he doesn\textquotesingle t seem tired of it yet,---which
is very good of him, isn\textquotesingle t it? I mean to give him
something on Christmas, for I dare not offer money. Tell me something
nice, Marmee.

"I\textquotesingle m glad Laurie seems so happy and busy, that he has
given up smoking, and lets his hair grow. You see Beth manages him
better than I did. I\textquotesingle m not jealous, dear; do your best,
only don\textquotesingle t make a saint of him. I\textquotesingle m
afraid I couldn\textquotesingle t like him without a spice of human
naughtiness. Read him bits of my letters. I haven\textquotesingle t time
to write much, and that will do just as well. Thank Heaven Beth
continues so comfortable."

\begin{center}\rule{0.5\linewidth}{0.5pt}\end{center}

"{January.}

"A Happy New Year to you all, my dearest family, which of course
includes Mr. L. and a young man by the name of Teddy. I
can\textquotesingle t tell you how much I enjoyed your Christmas bundle,
for I didn\textquotesingle t get it till night, and had given up hoping.
Your letter came in the morning, but you said nothing about a parcel,
meaning it for a surprise; so I was disappointed, for
I\textquotesingle d had a \textquotesingle kind of a
feeling\textquotesingle{} that you wouldn\textquotesingle t forget me. I
felt a little low in my mind, as I sat up in my room, after tea; and
when the big, muddy, battered-looking bundle was brought to me, I just
hugged it, and pranced. It was so \emph{homey} and refreshing, that I
sat down on the floor and read and looked and ate and laughed and cried,
in my usual absurd way. The things were just what I wanted, and all the
better for being made instead of bought. Beth\textquotesingle s new
\textquotesingle ink-bib\textquotesingle{} was capital; and
Hannah\textquotesingle s box of hard gingerbread will be a treasure.
I\textquotesingle ll be sure and wear the nice flannels you sent,
Marmee, and read carefully the books father has marked. Thank you all,
heaps and heaps!
\end{quote}

\protect\phantomsection\label{6672479776654687619_37106-h-6.htm.xhtml}{}

\protect\phantomsection\label{6672479776654687619_37106-h-6.htm.xhtml_b153.png}{}
\pandocbounded{\includegraphics[keepaspectratio]{303483661336987339_b153.png}}

\begin{quote}
"Speaking of books reminds me that I\textquotesingle m getting rich in
that line for, on New Year\textquotesingle s Day, Mr. Bhaer gave me a
fine Shakespeare. It is one he values much, and I\textquotesingle ve
often admired it, set up in the place of honor, with his German Bible,
Plato, Homer, and Milton; so you may imagine how I felt when he brought
it down, without its cover, and showed me my name in it,
\textquotesingle from my friend Friedrich Bhaer.\textquotesingle{}

"\textquotesingle You say often you wish a library: here I gif you one;
for between these lids (he meant covers) is many books in one. Read him
well, and he will help you much; for the study of character in this book
will help you to read it in the world and paint it with your
pen.\textquotesingle{}

"I thanked him as well as I could, and talk now about
\textquotesingle my library,\textquotesingle{} as if I had a hundred
books. I never knew how much there was in Shakespeare before; but then I
never had a Bhaer to explain it to me. Now \emph{don\textquotesingle t}
laugh at his horrid name; it isn\textquotesingle t pronounced either
Bear or Beer, as people \emph{will} say it, but something between the
two, as only Germans can give it. I\textquotesingle m glad you both like
what I tell you about him, and hope you will know him some day. Mother
would admire his warm heart, father his wise head. I admire both, and
feel rich in my new \textquotesingle friend Friedrich
Bhaer.\textquotesingle{}

"Not having much money, or knowing what he\textquotesingle d like, I got
several little things, and put them about the room, where he would find
them unexpectedly. They were useful, pretty, or funny,---a new standish
on his table, a little vase for his flower,---he always has one, or a
bit of green in a glass, to keep him fresh, he says,---and a holder for
his blower, so that he needn\textquotesingle t burn up what Amy calls
\textquotesingle mouchoirs.\textquotesingle{} I made it like those Beth
invented,---a big butterfly with a fat body, and black and yellow wings,
worsted feelers, and bead eyes. It took his fancy immensely, and he put
it on his mantel-piece as an article of \emph{vertu}; so it was rather a
failure after all. Poor as he is, he didn\textquotesingle t forget a
servant or a child in the house; and not a soul here, from the French
laundry-woman to Miss Norton, forgot him. I was so glad of that.

"They got up a masquerade, and had a gay time New Year\textquotesingle s
Eve. I didn\textquotesingle t mean to go down, having no dress; but at
the last minute, Mrs. Kirke remembered some old brocades, and Miss
Norton lent me lace and feathers; so I dressed up as Mrs. Malaprop, and
sailed in with a mask on. No one knew me, for I disguised my voice, and
no one dreamed of the silent, haughty Miss March (for they think I am
very stiff and cool, most of them; and so I am to whipper-snappers)
could dance and dress, and burst out into a \textquotesingle nice
derangement of epitaphs, like an allegory on the banks of the
Nile.\textquotesingle{} I enjoyed it very much; and when we unmasked, it
was fun to see them stare at me. I heard one of the young men tell
another that he knew I\textquotesingle d been an actress; in fact, he
thought he remembered seeing me at one of the minor theatres. Meg will
relish that joke. Mr. Bhaer was Nick Bottom, and Tina was Titania,---a
perfect little fairy in his arms. To see them dance was
\textquotesingle quite a landscape,\textquotesingle{} to use a Teddyism.

"I had a very happy New Year, after all; and when I thought it over in
my room, I felt as if I was getting on a little in spite of my many
failures; for I\textquotesingle m cheerful all the time now, work with a
will, and take more interest in other people than I used to, which is
satisfactory. Bless you all! Ever your loving

Jo."
\end{quote}

\protect\phantomsection\label{6672479776654687619_37106-h-6.htm.xhtml_b154.png}{}
\pandocbounded{\includegraphics[keepaspectratio]{303483661336987339_b154.png}}

\begin{center}\rule{0.5\linewidth}{0.5pt}\end{center}

\subsection{XXXIV. A
Friend.}\label{6672479776654687619_37106-h-6.htm.xhtml_pgepubid00036}

\protect\phantomsection\label{6672479776654687619_37106-h-6.htm.xhtml_b155.png}{}
\pandocbounded{\includegraphics[keepaspectratio]{303483661336987339_b155.png}}

\protect\phantomsection\label{6672479776654687619_37106-h-6.htm.xhtml_XXXIV}{}\hyperref[6672479776654687619_37106-h-0.htm.xhtml_contents2]{XXXIV.}

A FRIEND.

{Though} very happy in the social atmosphere about her, and very busy
with the daily work that earned her bread, and made it sweeter for the
effort, Jo still found time for literary labors. The purpose which now
took possession of her was a natural one to a poor and ambitious girl;
but the means she took to gain her end were not the best. She saw that
money conferred power: money and power, therefore, she resolved to have;
not to be used for herself alone, but for those whom she loved more than
self.

The dream of filling home with comforts, giving Beth everything she
wanted, from strawberries in winter to an organ in her bedroom; going
abroad herself, and always having \emph{more} than enough, so that she
might indulge in the luxury of charity, had been for years
Jo\textquotesingle s most cherished castle in the air.

The prize-story experience had seemed to open a way which might, after
long travelling and much up-hill work lead to this delightful
\emph{château en Espagne}. But the novel disaster quenched her courage
for a time, for public opinion is a giant which has frightened
stouter-hearted Jacks on bigger bean-stalks than hers. Like that
immortal hero, she reposed awhile after the first attempt, which
resulted in a tumble, and the least lovely of the
giant\textquotesingle s treasures, if I remember rightly. But the "up
again and take another" spirit was as strong in Jo as in Jack; so she
scrambled up, on the shady side this time, and got more booty, but
nearly left behind her what was far more precious than the money-bags.

She took to writing sensation stories; for in those dark ages, even
all-perfect America read rubbish. She told no one, but concocted a
"thrilling tale," and boldly carried it herself to Mr. Dashwood, editor
of the "Weekly Volcano." She had never read "Sartor Resartus," but she
had a womanly instinct that clothes possess an influence more powerful
over many than the worth of character or the magic of manners. So she
dressed herself in her best, and, trying to persuade herself that she
was neither excited nor nervous, bravely climbed two pairs of dark and
dirty stairs to find herself in a disorderly room, a cloud of
cigar-smoke, and the presence of three gentlemen, sitting with their
heels rather higher than their hats, which articles of dress none of
them took the trouble to remove on her appearance. Somewhat daunted by
this reception, Jo hesitated on the threshold, murmuring in much
embarrassment,---

"Excuse me, I was looking for the \textquotesingle Weekly
Volcano\textquotesingle{} office; I wished to see Mr. Dashwood."

Down went the highest pair of heels, up rose the smokiest gentleman,
and, carefully cherishing his cigar between his fingers, he advanced,
with a nod, and a countenance expressive of nothing but sleep. Feeling
that she must get through the matter somehow, Jo produced her
manuscript, and, blushing redder and redder with each sentence,
blundered out fragments of the little speech carefully prepared for the
occasion.

"A friend of mine desired me to offer---a story---just as an
experiment---would like your opinion---be glad to write more if this
suits."

While she blushed and blundered, Mr. Dashwood had taken the manuscript,
and was turning over the leaves with a pair of rather dirty fingers, and
casting critical glances up and down the neat pages.

"Not a first attempt, I take it?" observing that the pages were
numbered, covered only on one side, and not tied up with a
ribbon,---sure sign of a novice.

"No, sir; she has had some experience, and got a prize for a tale in the
\textquotesingle Blarneystone Banner.\textquotesingle"

"Oh, did she?" and Mr. Dashwood gave Jo a quick look, which seemed to
take note of everything she had on, from the bow in her bonnet to the
buttons on her boots. "Well, you can leave it, if you like.
We\textquotesingle ve more of this sort of thing on hand than we know
what to do with at present; but I\textquotesingle ll run my eye over it,
and give you an answer next week."

Now, Jo did \emph{not} like to leave it, for Mr. Dashwood
didn\textquotesingle t suit her at all; but, under the circumstances,
there was nothing for her to do but bow and walk away, looking
particularly tall and dignified, as she was apt to do when nettled or
abashed. Just then she was both; for it was perfectly evident, from the
knowing glances exchanged among the gentlemen, that her little fiction
of "my friend" was considered a good joke; and a laugh, produced by some
inaudible remark of the editor, as he closed the door, completed her
discomfiture. Half resolving never to return, she went home, and worked
off her irritation by stitching pinafores vigorously; and in an hour or
two was cool enough to laugh over the scene, and long for next week.

When she went again, Mr. Dashwood was alone, whereat she rejoiced; Mr.
Dashwood was much wider awake than before, which was agreeable; and Mr.
Dashwood was not too deeply absorbed in a cigar to remember his manners:
so the second interview was much more comfortable than the first.

"We\textquotesingle ll take this" (editors never say I), "if you
don\textquotesingle t object to a few alterations. It\textquotesingle s
too long, but omitting the passages I\textquotesingle ve marked will
make it just the right length," he said, in a business-like tone.

Jo hardly knew her own MS. again, so crumpled and underscored were its
pages and paragraphs; but, feeling as a tender parent might on being
asked to cut off her baby\textquotesingle s legs in order that it might
fit into a new cradle, she looked at the marked passages, and was
surprised to find that all the moral reflections---which she had
carefully put in as ballast for much romance---had been stricken out.

"But, sir, I thought every story should have some sort of a moral, so I
took care to have a few of my sinners repent."

Mr. Dashwood\textquotesingle s editorial gravity relaxed into a smile,
for Jo had forgotten her "friend," and spoken as only an author could.

"People want to be amused, not preached at, you know. Morals
don\textquotesingle t sell nowadays;" which was not quite a correct
statement, by the way.

"You think it would do with these alterations, then?"

"Yes; it\textquotesingle s a new plot, and pretty well worked
up---language good, and so on," was Mr. Dashwood\textquotesingle s
affable reply.

"What do you---that is, what compensation---" began Jo, not exactly
knowing how to express herself.

"Oh, yes, well, we give from twenty-five to thirty for things of this
sort. Pay when it comes out," returned Mr. Dashwood, as if that point
had escaped him; such trifles often do escape the editorial mind, it is
said.

"Very well; you can have it," said Jo, handing back the story, with a
satisfied air; for, after the dollar-a-column work, even twenty-five
seemed good pay.

"Shall I tell my friend you will take another if she has one better than
this?" asked Jo, unconscious of her little slip of the tongue, and
emboldened by her success.

"Well, we\textquotesingle ll look at it; can\textquotesingle t promise
to take it. Tell her to make it short and spicy, and never mind the
moral. What name would your friend like to put to it?" in a careless
tone.

"None at all, if you please; she doesn\textquotesingle t wish her name
to appear, and has no \emph{nom de plume}," said Jo, blushing in spite
of herself.

"Just as she likes, of course. The tale will be out next week; will you
call for the money, or shall I send it?" asked Mr. Dashwood, who felt a
natural desire to know who his new contributor might be.

"I\textquotesingle ll call. Good morning, sir."

As she departed, Mr. Dashwood put up his feet, with the graceful remark,
"Poor and proud, as usual, but she\textquotesingle ll do."

Following Mr. Dashwood\textquotesingle s directions, and making Mrs.
Northbury her model, Jo rashly took a plunge into the frothy sea of
sensational literature; but, thanks to the life-preserver thrown her by
a friend, she came up again, not much the worse for her ducking.

Like most young scribblers, she went abroad for her characters and
scenery; and banditti, counts, gypsies, nuns, and duchesses appeared
upon her stage, and played their parts with as much accuracy and spirit
as could be expected. Her readers were not particular about such trifles
as grammar, punctuation, and probability, and Mr. Dashwood graciously
permitted her to fill his columns at the lowest prices, not thinking it
necessary to tell her that the real cause of his hospitality was the
fact that one of his hacks, on being offered higher wages, had basely
left him in the lurch.

She soon became interested in her work, for her emaciated purse grew
stout, and the little hoard she was making to take Beth to the mountains
next summer grew slowly but surely as the weeks passed. One thing
disturbed her satisfaction, and that was that she did not tell them at
home. She had a feeling that father and mother would not approve, and
preferred to have her own way first, and beg pardon afterward. It was
easy to keep her secret, for no name appeared with her stories; Mr.
Dashwood had, of course, found it out very soon, but promised to be
dumb; and, for a wonder, kept his word.

She thought it would do her no harm, for she sincerely meant to write
nothing of which she should be ashamed, and quieted all pricks of
conscience by anticipations of the happy minute when she should show her
earnings and laugh over her well-kept secret.

But Mr. Dashwood rejected any but thrilling tales; and, as thrills could
not be produced except by harrowing up the souls of the readers, history
and romance, land and sea, science and art, police records and lunatic
asylums, had to be ransacked for the purpose. Jo soon found that her
innocent experience had given her but few glimpses of the tragic world
which underlies society; so, regarding it in a business light, she set
about supplying her deficiencies with characteristic energy. Eager to
find material for stories, and bent on making them original in plot, if
not masterly in execution, she searched newspapers for accidents,
incidents, and crimes; she excited the suspicions of public librarians
by asking for works on poisons; she studied faces in the street, and
characters, good, bad, and indifferent, all about her; she delved in the
dust of ancient times for facts or fictions so old that they were as
good as new, and introduced herself to folly, sin, and misery, as well
as her limited opportunities allowed. She thought she was prospering
finely; but, unconsciously, she was beginning to desecrate some of the
womanliest attributes of a woman\textquotesingle s character. She was
living in bad society; and, imaginary though it was, its influence
affected her, for she was feeding heart and fancy on dangerous and
unsubstantial food, and was fast brushing the innocent bloom from her
nature by a premature acquaintance with the darker side of life, which
comes soon enough to all of us.

She was beginning to feel rather than see this, for much describing of
other people\textquotesingle s passions and feelings set her to studying
and speculating about her own,---a morbid amusement, in which healthy
young minds do not voluntarily indulge. Wrong-doing always brings its
own punishment; and, when Jo most needed hers, she got it.

I don\textquotesingle t know whether the study of Shakespeare helped her
to read character, or the natural instinct of a woman for what was
honest, brave, and strong; but while endowing her imaginary heroes with
every perfection under the sun, Jo was discovering a live hero, who
interested her in spite of many human imperfections. Mr. Bhaer, in one
of their conversations, had advised her to study simple, true, and
lovely characters, wherever she found them, as good training for a
writer. Jo took him at his word, for she coolly turned round and studied
him,---a proceeding which would have much surprised him, had he known
it, for the worthy Professor was very humble in his own conceit.

Why everybody liked him was what puzzled Jo, at first. He was neither
rich nor great, young nor handsome; in no respect what is called
fascinating, imposing, or brilliant; and yet he was as attractive as a
genial fire, and people seemed to gather about him as naturally as about
a warm hearth. He was poor, yet always appeared to be giving something
away; a stranger, yet every one was his friend; no longer young, but as
happy-hearted as a boy; plain and peculiar, yet his face looked
beautiful to many, and his oddities were freely forgiven for his sake.
Jo often watched him, trying to discover the charm, and, at last,
decided that it was benevolence which worked the miracle. If he had any
sorrow, "it sat with its head under its wing," and he turned only his
sunny side to the world. There were lines upon his forehead, but Time
seemed to have touched him gently, remembering how kind he was to
others. The pleasant curves about his mouth were the memorials of many
friendly words and cheery laughs; his eyes were never cold or hard, and
his big hand had a warm, strong grasp that was more expressive than
words.

His very clothes seemed to partake of the hospitable nature of the
wearer. They looked as if they were at ease, and liked to make him
comfortable; his capacious waistcoat was suggestive of a large heart
underneath; his rusty coat had a social air, and the baggy pockets
plainly proved that little hands often went in empty and came out full;
his very boots were benevolent, and his collars never stiff and raspy
like other people\textquotesingle s.

"That\textquotesingle s it!" said Jo to herself, when she at length
discovered that genuine good-will towards one\textquotesingle s
fellow-men could beautify and dignify even a stout German teacher, who
shovelled in his dinner, darned his own socks, and was burdened with the
name of Bhaer.

Jo valued goodness highly, but she also possessed a most feminine
respect for intellect, and a little discovery which she made about the
Professor added much to her regard for him. He never spoke of himself,
and no one ever knew that in his native city he had been a man much
honored and esteemed for learning and integrity, till a countryman came
to see him, and, in a conversation with Miss Norton, divulged the
pleasing fact. From her Jo learned it, and liked it all the better
because Mr. Bhaer had never told it. She felt proud to know that he was
an honored Professor in Berlin, though only a poor language-master in
America; and his homely, hard-working life was much beautified by the
spice of romance which this discovery gave it.

Another and a better gift than intellect was shown her in a most
unexpected manner. Miss Norton had the \emph{entrée} into literary
society, which Jo would have had no chance of seeing but for her. The
solitary woman felt an interest in the ambitious girl, and kindly
conferred many favors of this sort both on Jo and the Professor. She
took them with her, one night, to a select symposium, held in honor of
several celebrities.

\protect\phantomsection\label{6672479776654687619_37106-h-6.htm.xhtml_b156.png}{}
\pandocbounded{\includegraphics[keepaspectratio]{303483661336987339_b156.png}}

Jo went prepared to bow down and adore the mighty ones whom she had
worshipped with youthful enthusiasm afar off. But her reverence for
genius received a severe shock that night, and it took her some time to
recover from the discovery that the great creatures were only men and
women after all. Imagine her dismay, on stealing a glance of timid
admiration at the poet whose lines suggested an ethereal being fed on
"spirit, fire, and dew," to behold him devouring his supper with an
ardor which flushed his intellectual countenance. Turning as from a
fallen idol, she made other discoveries which rapidly dispelled her
romantic illusions. The great novelist vibrated between two decanters
with the regularity of a pendulum; the famous divine flirted openly with
one of the Madame de Staëls of the age, who looked daggers at another
Corinne, who was amiably satirizing her, after out-manœuvring her in
efforts to absorb the profound philosopher, who imbibed tea Johnsonianly
and appeared to slumber, the loquacity of the lady rendering speech
impossible. The scientific celebrities, forgetting their mollusks and
glacial periods, gossiped about art, while devoting themselves to
oysters and ices with characteristic energy; the young musician, who was
charming the city like a second Orpheus, talked horses; and the specimen
of the British nobility present happened to be the most ordinary man of
the party.

Before the evening was half over, Jo felt so completely
\emph{désillusionée}, that she sat down in a corner to recover herself.
Mr. Bhaer soon joined her, looking rather out of his element, and
presently several of the philosophers, each mounted on his hobby, came
ambling up to hold an intellectual tournament in the recess. The
conversation was miles beyond Jo\textquotesingle s comprehension, but
she enjoyed it, though Kant and Hegel were unknown gods, the Subjective
and Objective unintelligible terms; and the only thing "evolved from her
inner consciousness," was a bad headache after it was all over. It
dawned upon her gradually that the world was being picked to pieces, and
put together on new, and, according to the talkers, on infinitely better
principles than before; that religion was in a fair way to be reasoned
into nothingness, and intellect was to be the only God. Jo knew nothing
about philosophy or metaphysics of any sort, but a curious excitement,
half pleasurable, half painful, came over her, as she listened with a
sense of being turned adrift into time and space, like a young balloon
out on a holiday.

She looked round to see how the Professor liked it, and found him
looking at her with the grimmest expression she had ever seen him wear.
He shook his head, and beckoned her to come away; but she was
fascinated, just then, by the freedom of Speculative Philosophy, and
kept her seat, trying to find out what the wise gentlemen intended to
rely upon after they had annihilated all the old beliefs.

Now, Mr. Bhaer was a diffident man, and slow to offer his own opinions,
not because they were unsettled, but too sincere and earnest to be
lightly spoken. As he glanced from Jo to several other young people,
attracted by the brilliancy of the philosophic pyrotechnics, he knit his
brows, and longed to speak, fearing that some inflammable young soul
would be led astray by the rockets, to find, when the display was over,
that they had only an empty stick or a scorched hand.

He bore it as long as he could; but when he was appealed to for an
opinion, he blazed up with honest indignation, and defended religion
with all the eloquence of truth,---an eloquence which made his broken
English musical, and his plain face beautiful. He had a hard fight, for
the wise men argued well; but he didn\textquotesingle t know when he was
beaten, and stood to his colors like a man. Somehow, as he talked, the
world got right again to Jo; the old beliefs, that had lasted so long,
seemed better than the new; God was not a blind force, and immortality
was not a pretty fable, but a blessed fact. She felt as if she had solid
ground under her feet again; and when Mr. Bhaer paused, out-talked, but
not one whit convinced, Jo wanted to clap her hands and thank him.

She did neither; but she remembered this scene, and gave the Professor
her heartiest respect, for she knew it cost him an effort to speak out
then and there, because his conscience would not let him be silent. She
began to see that character is a better possession than money, rank,
intellect, or beauty; and to feel that if greatness is what a wise man
has defined it to be, "truth, reverence, and good-will," then her friend
Friedrich Bhaer was not only good, but great.

This belief strengthened daily. She valued his esteem, she coveted his
respect, she wanted to be worthy of his friendship; and, just when the
wish was sincerest, she came near losing everything. It all grew out of
a cocked hat; for one evening the Professor came in to give Jo her
lesson, with a paper soldier-cap on his head, which Tina had put there,
and he had forgotten to take off.

"It\textquotesingle s evident he doesn\textquotesingle t look in his
glass before coming down," thought Jo, with a smile, as he said "Goot
efening," and sat soberly down, quite unconscious of the ludicrous
contrast between his subject and his head-gear, for he was going to read
her the "Death of Wallenstein."

\protect\phantomsection\label{6672479776654687619_37106-h-6.htm.xhtml_b157.png}{}
\pandocbounded{\includegraphics[keepaspectratio]{303483661336987339_b157.png}}

She said nothing at first, for she liked to hear him laugh out his big,
hearty laugh, when anything funny happened, so she left him to discover
it for himself, and presently forgot all about it; for to hear a German
read Schiller is rather an absorbing occupation. After the reading came
the lesson, which was a lively one, for Jo was in a gay mood that night,
and the cocked-hat kept her eyes dancing with merriment. The Professor
didn\textquotesingle t know what to make of her, and stopped at last, to
ask, with an air of mild surprise that was irresistible,---

"Mees Marsch, for what do you laugh in your master\textquotesingle s
face? Haf you no respect for me, that you go on so bad?"

"How can I be respectful, sir, when you forget to take your hat off?"
said Jo.

Lifting his hand to his head, the absent-minded Professor gravely felt
and removed the little cocked-hat, looked at it a minute, and then threw
back his head, and laughed like a merry bass-viol.

"Ah! I see him now; it is that imp Tina who makes me a fool with my cap.
Well, it is nothing; but see you, if this lesson goes not well, you too
shall wear him."

But the lesson did not go at all for a few minutes, because Mr. Bhaer
caught sight of a picture on the hat, and, unfolding it, said, with an
air of great disgust,---

"I wish these papers did not come in the house; they are not for
children to see, nor young people to read. It is not well, and I haf no
patience with those who make this harm."

Jo glanced at the sheet, and saw a pleasing illustration composed of a
lunatic, a corpse, a villain, and a viper. She did not like it; but the
impulse that made her turn it over was not one of displeasure, but fear,
because, for a minute, she fancied the paper was the "Volcano." It was
not, however, and her panic subsided as she remembered that, even if it
had been, and one of her own tales in it, there would have been no name
to betray her. She had betrayed herself, however, by a look and a blush;
for, though an absent man, the Professor saw a good deal more than
people fancied. He knew that Jo wrote, and had met her down among the
newspaper offices more than once; but as she never spoke of it, he asked
no questions, in spite of a strong desire to see her work. Now it
occurred to him that she was doing what she was ashamed to own, and it
troubled him. He did not say to himself, "It is none of my business;
I\textquotesingle ve no right to say anything," as many people would
have done; he only remembered that she was young and poor, a girl far
away from mother\textquotesingle s love and father\textquotesingle s
care; and he was moved to help her with an impulse as quick and natural
as that which would prompt him to put out his hand to save a baby from a
puddle. All this flashed through his mind in a minute, but not a trace
of it appeared in his face; and by the time the paper was turned, and
Jo\textquotesingle s needle threaded, he was ready to say quite
naturally, but very gravely,---

"Yes, you are right to put it from you. I do not like to think that good
young girls should see such things. They are made pleasant to some, but
I would more rather give my boys gunpowder to play with than this bad
trash."

"All may not be bad, only silly, you know; and if there is a demand for
it, I don\textquotesingle t see any harm in supplying it. Many very
respectable people make an honest living out of what are called
sensation stories," said Jo, scratching gathers so energetically that a
row of little slits followed her pin.

"There is a demand for whiskey, but I think you and I do not care to
sell it. If the respectable people knew what harm they did, they would
not feel that the living \emph{was} honest. They haf no right to put
poison in the sugar-plum, and let the small ones eat it. No; they should
think a little, and sweep mud in the street before they do this thing."

Mr. Bhaer spoke warmly, and walked to the fire, crumpling the paper in
his hands. Jo sat still, looking as if the fire had come to her; for her
cheeks burned long after the cocked hat had turned to smoke, and gone
harmlessly up the chimney.

"I should like much to send all the rest after him," muttered the
Professor, coming back with a relieved air.

Jo thought what a blaze her pile of papers upstairs would make, and her
hard-earned money lay rather heavily on her conscience at that minute.
Then she thought consolingly to herself, "Mine are not like that; they
are only silly, never bad, so I won\textquotesingle t be worried;" and
taking up her book, she said, with a studious face,---

"Shall we go on, sir? I\textquotesingle ll be very good and proper now."

"I shall hope so," was all he said, but he meant more than she imagined;
and the grave, kind look he gave her made her feel as if the words
"Weekly Volcano" were printed in large type on her forehead.

As soon as she went to her room, she got out her papers, and carefully
re-read every one of her stories. Being a little short-sighted, Mr.
Bhaer sometimes used eye-glasses, and Jo had tried them once, smiling to
see how they magnified the fine print of her book; now she seemed to
have got on the Professor\textquotesingle s mental or moral spectacles
also; for the faults of these poor stories glared at her dreadfully, and
filled her with dismay.

"They \emph{are} trash, and will soon be worse than trash if I go on;
for each is more sensational than the last. I\textquotesingle ve gone
blindly on, hurting myself and other people, for the sake of money; I
know it\textquotesingle s so, for I can\textquotesingle t read this
stuff in sober earnest without being horribly ashamed of it; and what
\emph{should} I do if they were seen at home, or Mr. Bhaer got hold of
them?"

Jo turned hot at the bare idea, and stuffed the whole bundle into her
stove, nearly setting the chimney afire with the blaze.

\protect\phantomsection\label{6672479776654687619_37106-h-6.htm.xhtml_b158.png}{}
\pandocbounded{\includegraphics[keepaspectratio]{303483661336987339_b158.png}}

"Yes, that\textquotesingle s the best place for such inflammable
nonsense; I\textquotesingle d better burn the house down, I suppose,
than let other people blow themselves up with my gunpowder," she
thought, as she watched the "Demon of the Jura" whisk away, a little
black cinder with fiery eyes.

But when nothing remained of all her three months\textquotesingle{} work
except a heap of ashes, and the money in her lap, Jo looked sober, as
she sat on the floor, wondering what she ought to do about her wages.

"I think I haven\textquotesingle t done much harm \emph{yet}, and may
keep this to pay for my time," she said, after a long meditation, adding
impatiently, "I almost wish I hadn\textquotesingle t any conscience,
it\textquotesingle s so inconvenient. If I didn\textquotesingle t care
about doing right, and didn\textquotesingle t feel uncomfortable when
doing wrong, I should get on capitally. I can\textquotesingle t help
wishing sometimes, that father and mother hadn\textquotesingle t been so
particular about such things."

Ah, Jo, instead of wishing that, thank God that "father and mother
\emph{were} particular," and pity from your heart those who have no such
guardians to hedge them round with principles which may seem like
prison-walls to impatient youth, but which will prove sure foundations
to build character upon in womanhood.

Jo wrote no more sensational stories, deciding that the money did not
pay for her share of the sensation; but, going to the other extreme, as
is the way with people of her stamp, she took a course of Mrs. Sherwood,
Miss Edgeworth, and Hannah More; and then produced a tale which might
have been more properly called an essay or a sermon, so intensely moral
was it. She had her doubts about it from the beginning; for her lively
fancy and girlish romance felt as ill at ease in the new style as she
would have done masquerading in the stiff and cumbrous costume of the
last century. She sent this didactic gem to several markets, but it
found no purchaser; and she was inclined to agree with Mr. Dashwood,
that morals didn\textquotesingle t sell.

Then she tried a child\textquotesingle s story, which she could easily
have disposed of if she had not been mercenary enough to demand filthy
lucre for it. The only person who offered enough to make it worth her
while to try juvenile literature was a worthy gentleman who felt it his
mission to convert all the world to his particular belief. But much as
she liked to write for children, Jo could not consent to depict all her
naughty boys as being eaten by bears or tossed by mad bulls, because
they did not go to a particular Sabbath-school, nor all the good
infants, who did go, as rewarded by every kind of bliss, from gilded
gingerbread to escorts of angels, when they departed this life with
psalms or sermons on their lisping tongues. So nothing came of these
trials; and Jo corked up her inkstand, and said, in a fit of very
wholesome humility,---

"I don\textquotesingle t know anything; I\textquotesingle ll wait till I
do before I try again, and, meantime, \textquotesingle sweep mud in the
street,\textquotesingle{} if I can\textquotesingle t do better;
that\textquotesingle s honest, at least;" which decision proved that her
second tumble down the bean-stalk had done her some good.

While these internal revolutions were going on, her external life had
been as busy and uneventful as usual; and if she sometimes looked
serious or a little sad no one observed it but Professor Bhaer. He did
it so quietly that Jo never knew he was watching to see if she would
accept and profit by his reproof; but she stood the test, and he was
satisfied; for, though no words passed between them, he knew that she
had given up writing. Not only did he guess it by the fact that the
second finger of her right hand was no longer inky, but she spent her
evenings downstairs now, was met no more among newspaper offices, and
studied with a dogged patience, which assured him that she was bent on
occupying her mind with something useful, if not pleasant.

He helped her in many ways, proving himself a true friend, and Jo was
happy; for, while her pen lay idle, she was learning other lessons
beside German, and laying a foundation for the sensation story of her
own life.

It was a pleasant winter and a long one, for she did not leave Mrs.
Kirke till June. Every one seemed sorry when the time came; the children
were inconsolable, and Mr. Bhaer\textquotesingle s hair stuck straight
up all over his head, for he always rumpled it wildly when disturbed in
mind.

"Going home? Ah, you are happy that you haf a home to go in," he said,
when she told him, and sat silently pulling his beard, in the corner,
while she held a little levee on that last evening.

She was going early, so she bade them all good-by over night; and when
his turn came, she said warmly,---

"Now, sir, you won\textquotesingle t forget to come and see us, if you
ever travel our way, will you? I\textquotesingle ll never forgive you if
you do, for I want them all to know my friend."

"Do you? Shall I come?" he asked, looking down at her with an eager
expression which she did not see.

"Yes, come next month; Laurie graduates then, and you\textquotesingle d
enjoy Commencement as something new."

"That is your best friend, of whom you speak?" he said, in an altered
tone.

"Yes, my boy Teddy; I\textquotesingle m very proud of him, and should
like you to see him."

Jo looked up then, quite unconscious of anything but her own pleasure in
the prospect of showing them to one another. Something in Mr.
Bhaer\textquotesingle s face suddenly recalled the fact that she might
find Laurie more than a "best friend," and, simply because she
particularly wished not to look as if anything was the matter, she
involuntarily began to blush; and the more she tried not to, the redder
she grew. If it had not been for Tina on her knee, she
didn\textquotesingle t know what would have become of her. Fortunately,
the child was moved to hug her; so she managed to hide her face an
instant, hoping the Professor did not see it. But he did, and his own
changed again from that momentary anxiety to its usual expression, as he
said cordially,---

"I fear I shall not make the time for that, but I wish the friend much
success, and you all happiness. Gott bless you!" and with that, he shook
hands warmly, shouldered Tina, and went away.

But after the boys were abed, he sat long before his fire, with the
tired look on his face, and the "\emph{heimweh}," or homesickness, lying
heavy at his heart. Once, when he remembered Jo, as she sat with the
little child in her lap and that new softness in her face, he leaned his
head on his hands a minute, and then roamed about the room, as if in
search of something that he could not find.

"It is not for me; I must not hope it now," he said to himself, with a
sigh that was almost a groan; then, as if reproaching himself for the
longing that he could not repress, he went and kissed the two towzled
heads upon the pillow, took down his seldom-used meerschaum, and opened
his Plato.

He did his best, and did it manfully; but I don\textquotesingle t think
he found that a pair of rampant boys, a pipe, or even the divine Plato,
were very satisfactory substitutes for wife and child and home.

Early as it was, he was at the station, next morning, to see Jo off;
and, thanks to him, she began her solitary journey with the pleasant
memory of a familiar face smiling its farewell, a bunch of violets to
keep her company, and, best of all, the happy thought,---

"Well, the winter\textquotesingle s gone, and I\textquotesingle ve
written no books, earned no fortune; but I\textquotesingle ve made a
friend worth having, and I\textquotesingle ll try to keep him all my
life."

\begin{center}\rule{0.5\linewidth}{0.5pt}\end{center}

\subsection{XXXV.
Heartache.}\label{6672479776654687619_37106-h-6.htm.xhtml_pgepubid00037}

\protect\phantomsection\label{6672479776654687619_37106-h-6.htm.xhtml_b159.png}{}
\pandocbounded{\includegraphics[keepaspectratio]{303483661336987339_b159.png}}

\protect\phantomsection\label{6672479776654687619_37106-h-6.htm.xhtml_XXXV}{}\hyperref[6672479776654687619_37106-h-0.htm.xhtml_contents2b]{XXXV.}

HEARTACHE.

{Whatever} his motive might have been, Laurie studied to some purpose
that year, for he graduated with honor, and gave the Latin oration with
the grace of a Phillips and the eloquence of a Demosthenes, so his
friends said. They were all there, his grandfather,---oh, so
proud!---Mr. and Mrs. March, John and Meg, Jo and Beth, and all exulted
over him with the sincere admiration which boys make light of at the
time, but fail to win from the world by any after-triumphs.

"I\textquotesingle ve got to stay for this confounded supper, but I
shall be home early to-morrow; you\textquotesingle ll come and meet me
as usual, girls?" Laurie said, as he put the sisters into the carriage
after the joys of the day were over. He said "girls," but he meant Jo,
for she was the only one who kept up the old custom; she had not the
heart to refuse her splendid, successful boy anything, and answered
warmly,---

"I\textquotesingle ll come, Teddy, rain or shine, and march before you,
playing \textquotesingle{}\emph{Hail the conquering hero
comes},\textquotesingle{} on a jews-harp."

Laurie thanked her with a look that made her think, in a sudden panic,
"Oh, deary me! I know he\textquotesingle ll say something, and then what
shall I do?"

Evening meditation and morning work somewhat allayed her fears, and
having decided that she wouldn\textquotesingle t be vain enough to think
people were going to propose when she had given them every reason to
know what her answer would be, she set forth at the appointed time,
hoping Teddy wouldn\textquotesingle t do anything to make her hurt his
poor little feelings. A call at Meg\textquotesingle s, and a refreshing
sniff and sip at the Daisy and Demijohn, still further fortified her for
the \emph{tête-à-tête}, but when she saw a stalwart figure looming in
the distance, she had a strong desire to turn about and run away.

"Where\textquotesingle s the jews-harp, Jo?" cried Laurie, as soon as he
was within speaking distance.

"I forgot it;" and Jo took heart again, for that salutation could not be
called lover-like.

She always used to take his arm on these occasions; now she did not, and
he made no complaint, which was a bad sign, but talked on rapidly about
all sorts of far-away subjects, till they turned from the road into the
little path that led homeward through the grove. Then he walked more
slowly, suddenly lost his fine flow of language, and, now and then, a
dreadful pause occurred. To rescue the conversation from one of the
wells of silence into which it kept falling, Jo said hastily,---

"Now you must have a good long holiday!"

"I intend to."

Something in his resolute tone made Jo look up quickly to find him
looking down at her with an expression that assured her the dreaded
moment had come, and made her put out her hand with an imploring,---

"No, Teddy, please don\textquotesingle t!"

"I will, and you \emph{must} hear me. It\textquotesingle s no use, Jo;
we\textquotesingle ve got to have it out, and the sooner the better for
both of us," he answered, getting flushed and excited all at once.

"Say what you like, then; I\textquotesingle ll listen," said Jo, with a
desperate sort of patience.

Laurie was a young lover, but he was in earnest, and meant to "have it
out," if he died in the attempt; so he plunged into the subject with
characteristic impetuosity, saying in a voice that \emph{would} get
choky now and then, in spite of manful efforts to keep it steady,---

"I\textquotesingle ve loved you ever since I\textquotesingle ve known
you, Jo; couldn\textquotesingle t help it, you\textquotesingle ve been
so good to me. I\textquotesingle ve tried to show it, but you
wouldn\textquotesingle t let me; now I\textquotesingle m going to make
you hear, and give me an answer, for I \emph{can\textquotesingle t} go
on so any longer."

"I wanted to save you this; I thought you\textquotesingle d
understand---" began Jo, finding it a great deal harder than she
expected.

"I know you did; but girls are so queer you never know what they mean.
They say No when they mean Yes, and drive a man out of his wits just for
the fun of it," returned Laurie, entrenching himself behind an
undeniable fact.

"\emph{I} don\textquotesingle t. I never wanted to make you care for me
so, and I went away to keep you from it if I could."

"I thought so; it was like you, but it was no use. I only loved you all
the more, and I worked hard to please you, and I gave up billiards and
everything you didn\textquotesingle t like, and waited and never
complained, for I hoped you\textquotesingle d love me, though
I\textquotesingle m not half good enough---" here there was a choke that
couldn\textquotesingle t be controlled, so he decapitated buttercups
while he cleared his "confounded throat."

"Yes, you are; you\textquotesingle re a great deal too good for me, and
I\textquotesingle m so grateful to you, and so proud and fond of you, I
don\textquotesingle t see why I can\textquotesingle t love you as you
want me to. I\textquotesingle ve tried, but I can\textquotesingle t
change the feeling, and it would be a lie to say I do when I
don\textquotesingle t."

"Really, truly, Jo?"

He stopped short, and caught both her hands as he put his question with
a look that she did not soon forget.

"Really, truly, dear."

They were in the grove now, close by the stile; and when the last words
fell reluctantly from Jo\textquotesingle s lips, Laurie dropped her
hands and turned as if to go on, but for once in his life that fence was
too much for him; so he just laid his head down on the mossy post, and
stood so still that Jo was frightened.

\protect\phantomsection\label{6672479776654687619_37106-h-6.htm.xhtml_b160.png}{}
\pandocbounded{\includegraphics[keepaspectratio]{303483661336987339_b160.png}}

"O Teddy, I\textquotesingle m so sorry, so desperately sorry, I could
kill myself if it would do any good! I wish you wouldn\textquotesingle t
take it so hard. I can\textquotesingle t help it; you know
it\textquotesingle s impossible for people to make themselves love other
people if they don\textquotesingle t," cried Jo inelegantly but
remorsefully, as she softly patted his shoulder, remembering the time
when he had comforted her so long ago.

"They do sometimes," said a muffled voice from the post.

"I don\textquotesingle t believe it\textquotesingle s the right sort of
love, and I\textquotesingle d rather not try it," was the decided
answer.

There was a long pause, while a blackbird sung blithely on the willow by
the river, and the tall grass rustled in the wind. Presently Jo said
very soberly, as she sat down on the step of the stile,---

"Laurie, I want to tell you something."

He started as if he had been shot, threw up his head, and cried out, in
a fierce tone---

"\emph{Don\textquotesingle t} tell me that, Jo; I can\textquotesingle t
bear it now!"

"Tell what?" she asked, wondering at his violence.

"That you love that old man."

"What old man?" demanded Jo, thinking he must mean his grandfather.

"That devilish Professor you were always writing about. If you say you
love him, I know I shall do something desperate;" and he looked as if he
would keep his word, as he clenched his hands, with a wrathful spark in
his eyes.

Jo wanted to laugh, but restrained herself, and said warmly, for she,
too, was getting excited with all this,---

"Don\textquotesingle t swear, Teddy! He isn\textquotesingle t old, nor
anything bad, but good and kind, and the best friend
I\textquotesingle ve got, next to you. Pray, don\textquotesingle t fly
into a passion; I want to be kind, but I know I shall get angry if you
abuse my Professor. I haven\textquotesingle t the least idea of loving
him or anybody else."

"But you will after a while, and then what will become of me?"

"You\textquotesingle ll love some one else too, like a sensible boy, and
forget all this trouble."

"I \emph{can\textquotesingle t} love any one else; and
I\textquotesingle ll never forget you, Jo, never! never!" with a stamp
to emphasize his passionate words.

"What \emph{shall} I do with him?" sighed Jo, finding that emotions were
more unmanageable than she expected. "You haven\textquotesingle t heard
what I wanted to tell you. Sit down and listen; for indeed I want to do
right and make you happy," she said, hoping to soothe him with a little
reason, which proved that she knew nothing about love.

Seeing a ray of hope in that last speech, Laurie threw himself down on
the grass at her feet, leaned his arm on the lower step of the stile,
and looked up at her with an expectant face. Now that arrangement was
not conducive to calm speech or clear thought on Jo\textquotesingle s
part; for how \emph{could} she say hard things to her boy while he
watched her with eyes full of love and longing, and lashes still wet
with the bitter drop or two her hardness of heart had wrung from him?
She gently turned his head away, saying, as she stroked the wavy hair
which had been allowed to grow for her sake,---how touching that was, to
be sure!---

"I agree with mother that you and I are not suited to each other,
because our quick tempers and strong wills would probably make us very
miserable, if we were so foolish as to---" Jo paused a little over the
last word, but Laurie uttered it with a rapturous expression,---

"Marry,---no, we shouldn\textquotesingle t! If you loved me, Jo, I
should be a perfect saint, for you could make me anything you like."

"No, I can\textquotesingle t. I\textquotesingle ve tried it and failed,
and I won\textquotesingle t risk our happiness by such a serious
experiment. We don\textquotesingle t agree and we never shall; so
we\textquotesingle ll be good friends all our lives, but we
won\textquotesingle t go and do anything rash."

"Yes, we will if we get the chance," muttered Laurie rebelliously.

"Now do be reasonable, and take a sensible view of the case," implored
Jo, almost at her wit\textquotesingle s end.

"I won\textquotesingle t be reasonable; I don\textquotesingle t want to
take what you call \textquotesingle a sensible view;\textquotesingle{}
it won\textquotesingle t help me, and it only makes you harder. I
don\textquotesingle t believe you\textquotesingle ve got any heart."

"I wish I hadn\textquotesingle t!"

There was a little quiver in Jo\textquotesingle s voice, and, thinking
it a good omen, Laurie turned round, bringing all his persuasive powers
to bear as he said, in the wheedlesome tone that had never been so
dangerously wheedlesome before,---

"Don\textquotesingle t disappoint us, dear! Every one expects it.
Grandpa has set his heart upon it, your people like it, and I
can\textquotesingle t get on without you. Say you will, and
let\textquotesingle s be happy. Do, do!"

Not until months afterward did Jo understand how she had the strength of
mind to hold fast to the resolution she had made when she decided that
she did not love her boy, and never could. It was very hard to do, but
she did it, knowing that delay was both useless and cruel.

"I can\textquotesingle t say \textquotesingle Yes\textquotesingle{}
truly, so I won\textquotesingle t say it at all. You\textquotesingle ll
see that I\textquotesingle m right, by and by, and thank me for
it"---she began solemnly.

"I\textquotesingle ll be hanged if I do!" and Laurie bounced up off the
grass, burning with indignation at the bare idea.

"Yes, you will!" persisted Jo; "you\textquotesingle ll get over this
after a while, and find some lovely, accomplished girl, who will adore
you, and make a fine mistress for your fine house. I
shouldn\textquotesingle t. I\textquotesingle m homely and awkward and
odd and old, and you\textquotesingle d be ashamed of me, and we should
quarrel,---we can\textquotesingle t help it even now, you see,---and I
shouldn\textquotesingle t like elegant society and you would, and
you\textquotesingle d hate my scribbling, and I couldn\textquotesingle t
get on without it, and we should be unhappy, and wish we
hadn\textquotesingle t done it, and everything would be horrid!"

"Anything more?" asked Laurie, finding it hard to listen patiently to
this prophetic burst.

"Nothing more, except that I don\textquotesingle t believe I shall ever
marry. I\textquotesingle m happy as I am, and love my liberty too well
to be in any hurry to give it up for any mortal man."

"I know better!" broke in Laurie. "You think so now; but
there\textquotesingle ll come a time when you \emph{will} care for
somebody, and you\textquotesingle ll love him tremendously, and live and
die for him. I know you will, it\textquotesingle s your way, and I shall
have to stand by and see it;" and the despairing lover cast his hat upon
the ground with a gesture that would have seemed comical, if his face
had not been so tragical.

"Yes, I \emph{will} live and die for him, if he ever comes and makes me
love him in spite of myself, and you must do the best you can!" cried
Jo, losing patience with poor Teddy. "I\textquotesingle ve done my best,
but you \emph{won\textquotesingle t} be reasonable, and
it\textquotesingle s selfish of you to keep teasing for what I
can\textquotesingle t give. I shall always be fond of you, very fond
indeed, as a friend, but I\textquotesingle ll never marry you; and the
sooner you believe it, the better for both of us,---so now!"

That speech was like fire to gunpowder. Laurie looked at her a minute as
if he did not quite know what to do with himself, then turned sharply
away, saying, in a desperate sort of tone,---

"You\textquotesingle ll be sorry some day, Jo."

"Oh, where are you going?" she cried, for his face frightened her.

"To the devil!" was the consoling answer.

For a minute Jo\textquotesingle s heart stood still, as he swung himself
down the bank, toward the river; but it takes much folly, sin, or misery
to send a young man to a violent death, and Laurie was not one of the
weak sort who are conquered by a single failure. He had no thought of a
melodramatic plunge, but some blind instinct led him to fling hat and
coat into his boat, and row away with all his might, making better time
up the river than he had done in many a race. Jo drew a long breath and
unclasped her hands as she watched the poor fellow trying to outstrip
the trouble which he carried in his heart.

"That will do him good, and he\textquotesingle ll come home in such a
tender, penitent state of mind, that I
sha\textquotesingle n\textquotesingle t dare to see him," she said;
adding, as she went slowly home, feeling as if she had murdered some
innocent thing, and buried it under the leaves,---

"Now I must go and prepare Mr. Laurence to be very kind to my poor boy.
I wish he\textquotesingle d love Beth; perhaps he may, in time, but I
begin to think I was mistaken about her. Oh dear! how can girls like to
have lovers and refuse them. I think it\textquotesingle s dreadful."

Being sure that no one could do it so well as herself, she went straight
to Mr. Laurence, told the hard story bravely through, and then broke
down, crying so dismally over her own insensibility that the kind old
gentleman, though sorely disappointed, did not utter a reproach. He
found it difficult to understand how any girl could help loving Laurie,
and hoped she would change her mind, but he knew even better than Jo
that love cannot be forced, so he shook his head sadly, and resolved to
carry his boy out of harm\textquotesingle s way; for Young
Impetuosity\textquotesingle s parting words to Jo disturbed him more
than he would confess.

When Laurie came home, dead tired, but quite composed, his grandfather
met him as if he knew nothing, and kept up the delusion very
successfully for an hour or two. But when they sat together in the
twilight, the time they used to enjoy so much, it was hard work for the
old man to ramble on as usual, and harder still for the young one to
listen to praises of the last year\textquotesingle s success, which to
him now seemed love\textquotesingle s labor lost. He bore it as long as
he could, then went to his piano, and began to play. The windows were
open; and Jo, walking in the garden with Beth, for once understood music
better than her sister, for he played the "Sonata Pathétique," and
played it as he never did before.

"That\textquotesingle s very fine, I dare say, but it\textquotesingle s
sad enough to make one cry; give us something gayer, lad," said Mr.
Laurence, whose kind old heart was full of sympathy, which he longed to
show, but knew not how.

Laurie dashed into a livelier strain, played stormily for several
minutes, and would have got through bravely, if, in a momentary lull,
Mrs. March\textquotesingle s voice had not been heard calling,---

"Jo, dear, come in; I want you."

Just what Laurie longed to say, with a different meaning! As he
listened, he lost his place; the music ended with a broken chord, and
the musician sat silent in the dark.

"I can\textquotesingle t stand this," muttered the old gentleman. Up he
got, groped his way to the piano, laid a kind hand on either of the
broad shoulders, and said, as gently as a woman,---

"I know, my boy, I know."

No answer for an instant; then Laurie asked sharply,---

"Who told you?"

"Jo herself."

"Then there\textquotesingle s an end of it!" and he shook off his
grandfather\textquotesingle s hands with an impatient motion; for,
though grateful for the sympathy, his man\textquotesingle s pride could
not bear a man\textquotesingle s pity.

"Not quite; I want to say one thing, and then there shall be an end of
it," returned Mr. Laurence, with unusual mildness. "You
won\textquotesingle t care to stay at home just now, perhaps?"

"I don\textquotesingle t intend to run away from a girl. Jo
can\textquotesingle t prevent my seeing her, and I shall stay and do it
as long as I like," interrupted Laurie, in a defiant tone.

"Not if you are the gentleman I think you. I\textquotesingle m
disappointed, but the girl can\textquotesingle t help it; and the only
thing left for you to do is to go away for a time. Where will you go?"

"Anywhere. I don\textquotesingle t care what becomes of me;" and Laurie
got up, with a reckless laugh, that grated on his
grandfather\textquotesingle s ear.

"Take it like a man, and don\textquotesingle t do anything rash, for
God\textquotesingle s sake. Why not go abroad, as you planned, and
forget it?"

"I can\textquotesingle t."

"But you\textquotesingle ve been wild to go, and I promised you should
when you got through college."

"Ah, but I didn\textquotesingle t mean to go alone!" and Laurie walked
fast through the room, with an expression which it was well his
grandfather did not see.

"I don\textquotesingle t ask you to go alone; there\textquotesingle s
some one ready and glad to go with you, anywhere in the world."

"Who, sir?" stopping to listen.

"Myself."

Laurie came back as quickly as he went, and put out his hand, saying
huskily,---

"I\textquotesingle m a selfish brute; but---you know---grandfather---"

"Lord help me, yes, I do know, for I\textquotesingle ve been through it
all before, once in my own young days, and then with your father. Now,
my dear boy, just sit quietly down, and hear my plan.
It\textquotesingle s all settled, and can be carried out at once," said
Mr. Laurence, keeping hold of the young man, as if fearful that he would
break away, as his father had done before him.

"Well, sir, what is it?" and Laurie sat down, without a sign of interest
in face or voice.

"There is business in London that needs looking after; I meant you
should attend to it; but I can do it better myself, and things here will
get on very well with Brooke to manage them. My partners do almost
everything; I\textquotesingle m merely holding on till you take my
place, and can be off at any time."

"But you hate travelling, sir; I can\textquotesingle t ask it of you at
your age," began Laurie, who was grateful for the sacrifice, but much
preferred to go alone, if he went at all.

The old gentleman knew that perfectly well, and particularly desired to
prevent it; for the mood in which he found his grandson assured him that
it would not be wise to leave him to his own devices. So, stifling a
natural regret at the thought of the home comforts he would leave behind
him, he said stoutly,---

"Bless your soul, I\textquotesingle m not superannuated yet. I quite
enjoy the idea; it will do me good, and my old bones
won\textquotesingle t suffer, for travelling nowadays is almost as easy
as sitting in a chair."

A restless movement from Laurie suggested that \emph{his} chair was not
easy, or that he did not like the plan, and made the old man add
hastily,---

"I don\textquotesingle t mean to be a marplot or a burden; I go because
I think you\textquotesingle d feel happier than if I was left behind. I
don\textquotesingle t intend to gad about with you, but leave you free
to go where you like, while I amuse myself in my own way.
I\textquotesingle ve friends in London and Paris, and should like to
visit them; meantime you can go to Italy, Germany, Switzerland, where
you will, and enjoy pictures, music, scenery, and adventures to your
heart\textquotesingle s content."

Now, Laurie felt just then that his heart was entirely broken, and the
world a howling wilderness; but at the sound of certain words which the
old gentleman artfully introduced into his closing sentence, the broken
heart gave an unexpected leap, and a green oasis or two suddenly
appeared in the howling wilderness. He sighed, and then said, in a
spiritless tone,---

"Just as you like, sir; it doesn\textquotesingle t matter where I go or
what I do."

"It does to me, remember that, my lad; I give you entire liberty, but I
trust you to make an honest use of it. Promise me that, Laurie."

"Anything you like, sir."

"Good," thought the old gentleman. "You don\textquotesingle t care now,
but there\textquotesingle ll come a time when that promise will keep you
out of mischief, or I\textquotesingle m much mistaken."

Being an energetic individual, Mr. Laurence struck while the iron was
hot; and before the blighted being recovered spirit enough to rebel,
they were off. During the time necessary for preparation, Laurie bore
himself as young gentlemen usually do in such cases. He was moody,
irritable, and pensive by turns; lost his appetite, neglected his dress,
and devoted much time to playing tempestuously on his piano; avoided Jo,
but consoled himself by staring at her from his window, with a tragical
face that haunted her dreams by night, and oppressed her with a heavy
sense of guilt by day. Unlike some sufferers, he never spoke of his
unrequited passion, and would allow no one, not even Mrs. March, to
attempt consolation or offer sympathy. On some accounts, this was a
relief to his friends; but the weeks before his departure were very
uncomfortable, and every one rejoiced that the "poor, dear fellow was
going away to forget his trouble, and come home happy." Of course, he
smiled darkly at their delusion, but passed it by, with the sad
superiority of one who knew that his fidelity, like his love, was
unalterable.

When the parting came he affected high spirits, to conceal certain
inconvenient emotions which seemed inclined to assert themselves. This
gayety did not impose upon anybody, but they tried to look as if it did,
for his sake, and he got on very well till Mrs. March kissed him, with a
whisper full of motherly solicitude; then, feeling that he was going
very fast, he hastily embraced them all round, not forgetting the
afflicted Hannah, and ran downstairs as if for his life. Jo followed a
minute after to wave her hand to him if he looked round. He did look
round, came back, put his arms about her, as she stood on the step above
him, and looked up at her with a face that made his short appeal both
eloquent and pathetic.

"O Jo, can\textquotesingle t you?"

\protect\phantomsection\label{6672479776654687619_37106-h-6.htm.xhtml_b161.png}{}
\pandocbounded{\includegraphics[keepaspectratio]{303483661336987339_b161.png}}

"Teddy, dear, I wish I could!"

That was all, except a little pause; then Laurie straightened himself
up, said "It\textquotesingle s all right, never mind," and went away
without another word. Ah, but it wasn\textquotesingle t all right, and
Jo \emph{did} mind; for while the curly head lay on her arm a minute
after her hard answer, she felt as if she had stabbed her dearest
friend; and when he left her without a look behind him, she knew that
the boy Laurie never would come again.

\protect\phantomsection\label{6672479776654687619_37106-h-6.htm.xhtml_b162.png}{}
\pandocbounded{\includegraphics[keepaspectratio]{303483661336987339_b162.png}}

\begin{center}\rule{0.5\linewidth}{0.5pt}\end{center}

\subsection{XXXVI. Beth\textquotesingle s
Secret.}\label{6672479776654687619_37106-h-6.htm.xhtml_pgepubid00038}

\protect\phantomsection\label{6672479776654687619_37106-h-6.htm.xhtml_XXXVI}{}\hyperref[6672479776654687619_37106-h-0.htm.xhtml_contents2b]{XXXVI.}

BETH\textquotesingle S SECRET.

{When} Jo came home that spring, she had been struck with the change in
Beth. No one spoke of it or seemed aware of it, for it had come too
gradually to startle those who saw her daily; but to eyes sharpened by
absence, it was very plain; and a heavy weight fell on
Jo\textquotesingle s heart as she saw her sister\textquotesingle s face.
It was no paler and but little thinner than in the autumn; yet there was
a strange, transparent look about it, as if the mortal was being slowly
refined away, and the immortal shining through the frail flesh with an
indescribably pathetic beauty. Jo saw and felt it, but said nothing at
the time, and soon the first impression lost much of its power; for Beth
seemed happy, no one appeared to doubt that she was better; and,
presently, in other cares, Jo for a time forgot her fear.

But when Laurie was gone, and peace prevailed again, the vague anxiety
returned and haunted her. She had confessed her sins and been forgiven;
but when she showed her savings and proposed the mountain trip, Beth had
thanked her heartily, but begged not to go so far away from home.
Another little visit to the seashore would suit her better, and, as
grandma could not be prevailed upon to leave the babies, Jo took Beth
down to the quiet place, where she could live much in the open air, and
let the fresh sea-breezes blow a little color into her pale cheeks.

It was not a fashionable place, but, even among the pleasant people
there, the girls made few friends, preferring to live for one another.
Beth was too shy to enjoy society, and Jo too wrapped up in her to care
for any one else; so they were all in all to each other, and came and
went, quite unconscious of the interest they excited in those about
them, who watched with sympathetic eyes the strong sister and the feeble
one, always together, as if they felt instinctively that a long
separation was not far away.

They did feel it, yet neither spoke of it; for often between ourselves
and those nearest and dearest to us there exists a reserve which it is
very hard to overcome. Jo felt as if a veil had fallen between her heart
and Beth\textquotesingle s; but when she put out her hand to lift it up,
there seemed something sacred in the silence, and she waited for Beth to
speak. She wondered, and was thankful also, that her parents did not
seem to see what she saw; and, during the quiet weeks, when the shadow
grew so plain to her, she said nothing of it to those at home, believing
that it would tell itself when Beth came back no better. She wondered
still more if her sister really guessed the hard truth, and what
thoughts were passing through her mind during the long hours when she
lay on the warm rocks, with her head in Jo\textquotesingle s lap, while
the winds blew healthfully over her, and the sea made music at her feet.

\protect\phantomsection\label{6672479776654687619_37106-h-6.htm.xhtml_b163.png}{}
\pandocbounded{\includegraphics[keepaspectratio]{303483661336987339_b163.png}}

One day Beth told her. Jo thought she was asleep, she lay so still; and,
putting down her book, sat looking at her with wistful eyes, trying to
see signs of hope in the faint color on Beth\textquotesingle s cheeks.
But she could not find enough to satisfy her, for the cheeks were very
thin, and the hands seemed too feeble to hold even the rosy little
shells they had been gathering. It came to her then more bitterly than
ever that Beth was slowly drifting away from her, and her arms
instinctively tightened their hold upon the dearest treasure she
possessed. For a minute her eyes were too dim for seeing, and, when they
cleared, Beth was looking up at her so tenderly that there was hardly
any need for her to say,---

"Jo, dear, I\textquotesingle m glad you know it. I\textquotesingle ve
tried to tell you, but I couldn\textquotesingle t."

There was no answer except her sister\textquotesingle s cheek against
her own, not even tears; for when most deeply moved, Jo did not cry. She
was the weaker, then, and Beth tried to comfort and sustain her, with
her arms about her, and the soothing words she whispered in her ear.

"I\textquotesingle ve known it for a good while, dear, and, now
I\textquotesingle m used to it, it isn\textquotesingle t hard to think
of or to bear. Try to see it so, and don\textquotesingle t be troubled
about me, because it\textquotesingle s best; indeed it is."

"Is this what made you so unhappy in the autumn, Beth? You did not feel
it then, and keep it to yourself so long, did you?" asked Jo, refusing
to see or say that it \emph{was} best, but glad to know that Laurie had
no part in Beth\textquotesingle s trouble.

"Yes, I gave up hoping then, but I didn\textquotesingle t like to own
it. I tried to think it was a sick fancy, and would not let it trouble
any one. But when I saw you all so well and strong, and full of happy
plans, it was hard to feel that I could never be like you, and then I
was miserable, Jo."

"O Beth, and you didn\textquotesingle t tell me, didn\textquotesingle t
let me comfort and help you! How could you shut me out, and bear it all
alone?"

Jo\textquotesingle s voice was full of tender reproach, and her heart
ached to think of the solitary struggle that must have gone on while
Beth learned to say good-by to health, love, and life, and take up her
cross so cheerfully.

"Perhaps it was wrong, but I tried to do right; I wasn\textquotesingle t
sure, no one said anything, and I hoped I was mistaken. It would have
been selfish to frighten you all when Marmee was so anxious about Meg,
and Amy away, and you so happy with Laurie,---at least, I thought so
then."

"And I thought that you loved him, Beth, and I went away because I
couldn\textquotesingle t," cried Jo, glad to say all the truth.

Beth looked so amazed at the idea that Jo smiled in spite of her pain,
and added softly,---

"Then you didn\textquotesingle t, deary? I was afraid it was so, and
imagined your poor little heart full of love-lornity all that while."

"Why, Jo, how could I, when he was so fond of you?" asked Beth, as
innocently as a child. "I do love him dearly; he is so good to me, how
can I help it? But he never could be anything to me but my brother. I
hope he truly will be, sometime."

"Not through me," said Jo decidedly. "Amy is left for him, and they
would suit excellently; but I have no heart for such things, now. I
don\textquotesingle t care what becomes of anybody but you, Beth. You
\emph{must} get well."

"I want to, oh, so much! I try, but every day I lose a little, and feel
more sure that I shall never gain it back. It\textquotesingle s like the
tide, Jo, when it turns, it goes slowly, but it can\textquotesingle t be
stopped."

"It \emph{shall} be stopped, your tide must not turn so soon, nineteen
is too young. Beth, \ul{I can\textquotesingle t let you go.}
I\textquotesingle ll work and pray and fight against it.
I\textquotesingle ll keep you in spite of everything; there must be
ways, it can\textquotesingle t be too late. God won\textquotesingle t be
so cruel as to take you from me," cried poor Jo rebelliously, for her
spirit was far less piously submissive than Beth\textquotesingle s.

Simple, sincere people seldom speak much of their piety; it shows itself
in acts, rather than in words, and has more influence than homilies or
protestations. Beth could not reason upon or explain the faith that gave
her courage and patience to give up life, and cheerfully wait for death.
Like a confiding child, she asked no questions, but left everything to
God and nature, Father and mother of us all, feeling sure that they, and
they only, could teach and strengthen heart and spirit for this life and
the life to come. She did not rebuke Jo with saintly speeches, only
loved her better for her passionate affection, and clung more closely to
the dear human love, from which our Father never means us to be weaned,
but through which He draws us closer to Himself. She could not say,
"I\textquotesingle m glad to go," for life was very sweet to her; she
could only sob out, "I try to be willing," while she held fast to Jo, as
the first bitter wave of this great sorrow broke over them together.

By and by Beth said, with recovered serenity,---

"You\textquotesingle ll tell them this when we go home?"

"I think they will see it without words," sighed Jo; for now it seemed
to her that Beth changed every day.

"Perhaps not; I\textquotesingle ve heard that the people who love best
are often blindest to such things. If they don\textquotesingle t see it,
you will tell them for me. I don\textquotesingle t want any secrets, and
it\textquotesingle s kinder to prepare them. Meg has John and the babies
to comfort her, but you must stand by father and mother,
won\textquotesingle t you, Jo?"

"If I can; but, Beth, I don\textquotesingle t give up yet;
I\textquotesingle m going to believe that it \emph{is} a sick fancy, and
not let you think it\textquotesingle s true," said Jo, trying to speak
cheerfully.

Beth lay a minute thinking, and then said in her quiet way,---

"I don\textquotesingle t know how to express myself, and
shouldn\textquotesingle t try, to any one but you, because I
can\textquotesingle t speak out, except to my Jo. I only mean to say
that I have a feeling that it never was intended I should live long.
I\textquotesingle m not like the rest of you; I never made any plans
about what I\textquotesingle d do when I grew up; I never thought of
being married, as you all did. I couldn\textquotesingle t seem to
imagine myself anything but stupid little Beth, trotting about at home,
of no use anywhere but there. I never wanted to go away, and the hard
part now is the leaving you all. I\textquotesingle m not afraid, but it
seems as if I should be homesick for you even in heaven."

Jo could not speak; and for several minutes there was no sound but the
sigh of the wind and the lapping of the tide. A white-winged gull flew
by, with the flash of sunshine on its silvery breast; Beth watched it
till it vanished, and her eyes were full of sadness. A little
gray-coated sand-bird came tripping over the beach, "peeping" softly to
itself, as if enjoying the sun and sea; it came quite close to Beth,
looked at her with a friendly eye, and sat upon a warm stone, dressing
its wet feathers, quite at home. Beth smiled, and felt comforted, for
the tiny thing seemed to offer its small friendship, and remind her that
a pleasant world was still to be enjoyed.

"Dear little bird! See, Jo, how tame it is. I like peeps better than the
gulls: they are not so wild and handsome, but they seem happy, confiding
little things. I used to call them my birds, last summer; and mother
said they reminded her of me,---busy, quaker-colored creatures, always
near the shore, and always chirping that contented little song of
theirs. You are the gull, Jo, strong and wild, fond of the storm and the
wind, flying far out to sea, and happy all alone. Meg is the
turtle-dove, and Amy is like the lark she writes about, trying to get up
among the clouds, but always dropping down into its nest again. Dear
little girl! she\textquotesingle s so ambitious, but her heart is good
and tender; and no matter how high she flies, she never will forget
home. I hope I shall see her again, but she seems \emph{so} far away."

"She is coming in the spring, and I mean that you shall be all ready to
see and enjoy her. I\textquotesingle m going to have you well and rosy
by that time," began Jo, feeling that of all the changes in Beth, the
talking change was the greatest, for it seemed to cost no effort now,
and she thought aloud in a way quite unlike bashful Beth.

"Jo, dear, don\textquotesingle t hope any more; it won\textquotesingle t
do any good, I\textquotesingle m sure of that. We won\textquotesingle t
be miserable, but enjoy being together while we wait.
We\textquotesingle ll have happy times, for I don\textquotesingle t
suffer much, and I think the tide will go out easily, if you help me."

Jo leaned down to kiss the tranquil face; and with that silent kiss, she
dedicated herself soul and body to Beth.

She was right: there was no need of any words when they got home, for
father and mother saw plainly, now, what they had prayed to be saved
from seeing. Tired with her short journey, Beth went at once to bed,
saying how glad she was to be at home; and when Jo went down, she found
that she would be spared the hard task of telling Beth\textquotesingle s
secret. Her father stood leaning his head on the mantel-piece, and did
not turn as she came in; but her mother stretched out her arms as if for
help, and Jo went to comfort her without a word.

\protect\phantomsection\label{6672479776654687619_37106-h-6.htm.xhtml_b164.png}{}
\pandocbounded{\includegraphics[keepaspectratio]{303483661336987339_b164.png}}

\begin{center}\rule{0.5\linewidth}{0.5pt}\end{center}

\subsection{XXXVII. New
Impressions.}\label{6672479776654687619_37106-h-6.htm.xhtml_pgepubid00039}

\protect\phantomsection\label{6672479776654687619_37106-h-6.htm.xhtml_b165.png}{}
\pandocbounded{\includegraphics[keepaspectratio]{303483661336987339_b165.png}}

\protect\phantomsection\label{6672479776654687619_37106-h-6.htm.xhtml_XXXVII}{}\hyperref[6672479776654687619_37106-h-0.htm.xhtml_contents2b]{XXXVII.}

NEW IMPRESSIONS.

{At} three o\textquotesingle clock in the afternoon, all the fashionable
world at Nice may be seen on the Promenade des Anglais,---a charming
place; for the wide walk, bordered with palms, flowers, and tropical
shrubs, is bounded on one side by the sea, on the other by the grand
drive, lined with hotels and villas, while beyond lie orange-orchards
and the hills. Many nations are represented, many languages spoken, many
costumes worn; and, on a sunny day, the spectacle is as gay and
brilliant as a carnival. Haughty English, lively French, sober Germans,
handsome Spaniards, ugly Russians, meek Jews, free-and-easy Americans,
all drive, sit, or saunter here, chatting over the news, and criticising
the latest celebrity who has arrived,---Ristori or Dickens, Victor
Emmanuel or the Queen of the Sandwich Islands. The equipages are as
varied as the company, and attract as much attention, especially the low
basket-barouches in which ladies drive themselves, with a pair of
dashing ponies, gay nets to keep their voluminous flounces from
overflowing the diminutive vehicles, and little grooms on the perch
behind.

Along this walk, on Christmas Day, a tall young man walked slowly, with
his hands behind him, and a somewhat absent expression of countenance.
He looked like an Italian, was dressed like an Englishman, and had the
independent air of an American,---a combination which caused sundry
pairs of feminine eyes to look approvingly after him, and sundry dandies
in black velvet suits, with rose-colored neckties, buff gloves, and
orange-flowers in their button-holes, to shrug their shoulders, and then
envy him his inches. There were plenty of pretty faces to admire, but
the young man took little notice of them, except to glance, now and
then, at some blonde girl, or lady in blue. Presently he strolled out of
the promenade, and stood a moment at the crossing, as if undecided
whether to go and listen to the band in the Jardin Publique, or to
wander along the beach toward Castle Hill. The quick trot of
ponies\textquotesingle{} feet made him look up, as one of the little
carriages, containing a single lady, came rapidly down the street. The
lady was young, blonde, and dressed in blue. He stared a minute, then
his whole face woke up, and, waving his hat like a boy, he hurried
forward to meet her.

"O Laurie, is it really you? I thought you\textquotesingle d never
come!" cried Amy, dropping the reins, and holding out both hands, to the
great scandalization of a French mamma, who hastened her
daughter\textquotesingle s steps, lest she should be demoralized by
beholding the free manners of these "mad English."

"I was detained by the way, but I promised to spend Christmas with you,
and here I am."

"How is your grandfather? When did you come? Where are you staying?"

"Very well---last night---at the Chauvain. I called at your hotel, but
you were all out."

"I have so much to say, I don\textquotesingle t know where to begin! Get
in, and we can talk at our ease; I was going for a drive, and longing
for company. Flo\textquotesingle s saving up for to-night."

"What happens then, a ball?"

"A Christmas party at our hotel. There are many Americans there, and
they give it in honor of the day. You\textquotesingle ll go with us, of
course? Aunt will be charmed."

"Thank you. Where now?" asked Laurie, leaning back and folding his arms,
a proceeding which suited Amy, who preferred to drive; for her
parasol-whip and blue reins over the white ponies\textquotesingle{}
backs, afforded her infinite satisfaction.

"I\textquotesingle m going to the banker\textquotesingle s first, for
letters, and then to Castle Hill; the view is so lovely, and I like to
feed the peacocks. Have you ever been there?"

"Often, years ago; but I don\textquotesingle t mind having a look at
it."

"Now tell me all about yourself. The last I heard of you, your
grandfather wrote that he expected you from Berlin."

"Yes, I spent a month there, and then joined him in Paris, where he has
settled for the winter. He has friends there, and finds plenty to amuse
him; so I go and come, and we get on capitally."

"That\textquotesingle s a sociable arrangement," said Amy, missing
something in Laurie\textquotesingle s manner, though she
couldn\textquotesingle t tell what.

"Why, you see he hates to travel, and I hate to keep still; so we each
suit ourselves, and there is no trouble. I am often with him, and he
enjoys my adventures, while I like to feel that some one is glad to see
me when I get back from my wanderings. Dirty old hole,
isn\textquotesingle t it?" he added, with a look of disgust, as they
drove along the boulevard to the Place Napoleon, in the old city.

"The dirt is picturesque, so I don\textquotesingle t mind. The river and
the hills are delicious, and these glimpses of the narrow cross-streets
are my delight. Now we shall have to wait for that procession to pass;
it\textquotesingle s going to the Church of St. John."

While Laurie listlessly watched the procession of priests under their
canopies, white-veiled nuns bearing lighted tapers, and some brotherhood
in blue, chanting as they walked, Amy watched him, and felt a new sort
of shyness steal over her; for he was changed, and she could not find
the merry-faced boy she left in the moody-looking man beside her. He was
handsomer than ever, and greatly improved, she thought; but now that the
flush of pleasure at meeting her was over, he looked tired and
spiritless,---not sick, nor exactly unhappy, but older and graver than a
year or two of prosperous life should have made him. She
couldn\textquotesingle t understand it, and did not venture to ask
questions; so she shook her head, and touched up her ponies, as the
procession wound away across the arches of the Paglioni bridge, and
vanished in the church.

"\emph{Que pensez vous}?" she said, airing her French, which had
improved in quantity, if not in quality, since she came abroad.

"That mademoiselle has made good use of her time, and the result is
charming," replied Laurie, bowing, with his hand on his heart, and an
admiring look.

She blushed with pleasure, but somehow the compliment did not satisfy
her like the blunt praises he used to give her at home, when he
promenaded round her on festival occasions, and told her she was
"altogether jolly," with a hearty smile and an approving pat on the
head. She didn\textquotesingle t like the new tone; for, though not
\emph{blasé}, it sounded indifferent in spite of the look.

"If that\textquotesingle s the way he\textquotesingle s going to grow
up, I wish he\textquotesingle d stay a boy," she thought, with a curious
sense of disappointment and discomfort, trying meantime to seem quite
easy and gay.

At Avigdor\textquotesingle s she found the precious home-letters, and,
giving the reins to Laurie, read them luxuriously as they wound up the
shady road between green hedges, where tea-roses bloomed as freshly as
in June.

"Beth is very poorly, mother says. I often think I ought to go home, but
they all say \textquotesingle stay;\textquotesingle{} so I do, for I
shall never have another chance like this," said Amy, looking sober over
one page.

"I think you are right, there; you could do nothing at home, and it is a
great comfort to them to know that you are well and happy, and enjoying
so much, my dear."

He drew a little nearer, and looked more like his old self, as he said
that; and the fear that sometimes weighed on Amy\textquotesingle s heart
was lightened, for the look, the act, the brotherly "my dear," seemed to
assure her that if any trouble did come, she would not be alone in a
strange land. Presently she laughed, and showed him a small sketch of Jo
in her scribbling-suit, with the bow rampantly erect upon her cap, and
issuing from her mouth the words, "Genius burns!"

Laurie smiled, took it, put it in his vest-pocket, "to keep it from
blowing away," and listened with interest to the lively letter Amy read
him.

"This will be a regularly merry Christmas to me, with presents in the
morning, you and letters in the afternoon, and a party at night," said
Amy, as they alighted among the ruins of the old fort, and a flock of
splendid peacocks came trooping about them, tamely waiting to be fed.
While Amy stood laughing on the bank above him as she scattered crumbs
to the brilliant birds, Laurie looked at her as she had looked at him,
with a natural curiosity to see what changes time and absence had
wrought. He found nothing to perplex or disappoint, much to admire and
approve; for, overlooking a few little affectations of speech and
manner, she was as sprightly and graceful as ever, with the addition of
that indescribable something in dress and bearing which we call
elegance. Always mature for her age, she had gained a certain
\emph{aplomb} in both carriage and conversation, which made her seem
more of a woman of the world than she was; but her old petulance now and
then showed itself, her strong will still held its own, and her native
frankness was unspoiled by foreign polish.

Laurie did not read all this while he watched her feed the peacocks, but
he saw enough to satisfy and interest him, and carried away a pretty
little picture of a bright-faced girl standing in the sunshine, which
brought out the soft hue of her dress, the fresh color of her cheeks,
the golden gloss of her hair, and made her a prominent figure in the
pleasant scene.

As they came up on to the stone plateau that crowns the hill, Amy waved
her hand as if welcoming him to her favorite haunt, and said, pointing
here and there,---

"Do you remember the Cathedral and the Corso, the fishermen dragging
their nets in the bay, and the lovely road to Villa Franca,
Schubert\textquotesingle s Tower, just below, and, best of all, that
speck far out to sea which they say is Corsica?"

"I remember; it\textquotesingle s not much changed," he answered,
without enthusiasm.

"What Jo would give for a sight of that famous speck!" said Amy, feeling
in good spirits, and anxious to see him so also.

"Yes," was all he said, but he turned and strained his eyes to see the
island which a greater usurper than even Napoleon now made interesting
in his sight.

"Take a good look at it for her sake, and then come and tell me what you
have been doing with yourself all this while," said Amy, seating
herself, ready for a good talk.

But she did not get it; for, though he joined her, and answered all her
questions freely, she could only learn that he had roved about the
continent and been to Greece. So, after idling away an hour, they drove
home again; and, having paid his respects to Mrs. Carrol, Laurie left
them, promising to return in the evening.

It must be recorded of Amy that she deliberately "prinked" that night.
Time and absence had done its work on both the young people; she had
seen her old friend in a new light, not as "our boy," but as a handsome
and agreeable man, and she was conscious of a very natural desire to
find favor in his sight. Amy knew her good points, and made the most of
them, with the taste and skill which is a fortune to a poor and pretty
woman.

Tarlatan and tulle were cheap at Nice, so she enveloped herself in them
on such occasions, and, following the sensible English fashion of simple
dress for young girls, got up charming little toilettes with fresh
flowers, a few trinkets, and all manner of dainty devices, which were
both inexpensive and effective. It must be confessed that the artist
sometimes got possession of the woman, and indulged in antique
\emph{coiffures}, statuesque attitudes, and classic draperies. But, dear
heart, we all have our little weaknesses, and find it easy to pardon
such in the young, who satisfy our eyes with their comeliness, and keep
our hearts merry with their artless vanities.

"I do want him to think I look well, and tell them so at home," said Amy
to herself, as she put on Flo\textquotesingle s old white silk
ball-dress, and covered it with a cloud of fresh illusion, out of which
her white shoulders and golden head emerged with a most artistic effect.
Her hair she had the sense to let alone, after gathering up the thick
waves and curls into a Hebe-like knot at the back of her head.

"It\textquotesingle s not the fashion, but it\textquotesingle s
becoming, and I can\textquotesingle t afford to make a fright of
myself," she used to say, when advised to frizzle, puff, or braid, as
the latest style commanded.

Having no ornaments fine enough for this important occasion, Amy looped
her fleecy skirts with rosy clusters of azalea, and framed the white
shoulders in delicate green vines. Remembering the painted boots, she
surveyed her white satin slippers with girlish satisfaction, and
\emph{chasséed} down the room, admiring her aristocratic feet all by
herself.

"My new fan just matches my flowers, my gloves fit to a charm, and the
real lace on aunt\textquotesingle s \emph{mouchoir} gives an air to my
whole dress. If I only had a classical nose and mouth I should be
perfectly happy," she said, surveying herself with a critical eye, and a
candle in each hand.

In spite of this affliction, she looked unusually gay and graceful as
she glided away; she seldom ran,---it did not suit her style, she
thought, for, being tall, the stately and Junoesque was more appropriate
than the sportive or piquante. She walked up and down the long saloon
while waiting for Laurie, and once arranged herself under the
chandelier, which had a good effect upon her hair; then she thought
better of it, and went away to the other end of the room, as if ashamed
of the girlish desire to have the first view a propitious one. It so
happened that she could not have done a better thing, for Laurie came in
so quietly she did not hear him; and, as she stood at the distant
window, with her head half turned, and one hand gathering up her dress,
the slender, white figure against the red curtains was as effective as a
well-placed statue.

"Good evening, Diana!" said Laurie, with the look of satisfaction she
liked to see in his eyes when they rested on her.

"Good evening, Apollo!" she answered, smiling back at him, for he, too,
looked unusually \emph{debonnaire}, and the thought of entering the
ball-room on the arm of such a personable man caused Amy to pity the
four plain Misses Davis from the bottom of her heart.

"Here are your flowers; I arranged them myself, remembering that you
didn\textquotesingle t like what Hannah calls a
\textquotesingle sot-bookay,\textquotesingle" said Laurie, handing her a
delicate nosegay, in a holder that she had long coveted as she daily
passed it in Cardiglia\textquotesingle s window.

\protect\phantomsection\label{6672479776654687619_37106-h-6.htm.xhtml_b166.png}{}
\pandocbounded{\includegraphics[keepaspectratio]{303483661336987339_b166.png}}

"How kind you are!" she exclaimed gratefully. "If I\textquotesingle d
known you were coming I\textquotesingle d have had something ready for
you to-day, though not as pretty as this, I\textquotesingle m afraid."

"Thank you; it isn\textquotesingle t what it should be, but you have
improved it," he added, as she snapped the silver bracelet on her wrist.

"Please don\textquotesingle t."

"I thought you liked that sort of thing?"

"Not from you; it doesn\textquotesingle t sound natural, and I like your
old bluntness better."

"I\textquotesingle m glad of it," he answered, with a look of relief;
then buttoned her gloves for her, and asked if his tie was straight,
just as he used to do when they went to parties together, at home.

The company assembled in the long \emph{salle à manger}, that evening,
was such as one sees nowhere but on the Continent. The hospitable
Americans had invited every acquaintance they had in Nice, and, having
no prejudice against titles, secured a few to add lustre to their
Christmas ball.

A Russian prince condescended to sit in a corner for an hour, and talk
with a massive lady, dressed like Hamlet\textquotesingle s mother, in
black velvet, with a pearl bridle under her chin. A Polish count, aged
eighteen, devoted himself to the ladies, who pronounced him "a
fascinating dear," and a German Serene Something, having come for the
supper alone, roamed vaguely about, seeking what he might devour. Baron
Rothschild\textquotesingle s private secretary, a large-nosed Jew, in
tight boots, affably beamed upon the world, as if his
master\textquotesingle s name crowned him with a golden halo; a stout
Frenchman, who knew the Emperor, came to indulge his mania for dancing,
and Lady de Jones, a British matron, adorned the scene with her little
family of eight. Of course, there were many light-footed, shrill-voiced
American girls, handsome, lifeless-looking English ditto, and a few
plain but piquante French demoiselles; likewise the usual set of
travelling young gentlemen, who disported themselves gayly, while mammas
of all nations lined the walls, and smiled upon them benignly when they
danced with their daughters.

Any young girl can imagine Amy\textquotesingle s state of mind when she
"took the stage" that night, leaning on Laurie\textquotesingle s arm.
She knew she looked well, she loved to dance, she felt that her foot was
on her native heath in a ball-room, and enjoyed the delightful sense of
power which comes when young girls first discover the new and lovely
kingdom they are born to rule by virtue of beauty, youth, and womanhood.
She did pity the Davis girls, who were awkward, plain, and destitute of
escort, except a grim papa and three grimmer maiden aunts, and she bowed
to them in her friendliest manner as she passed; which was good of her,
as it permitted them to see her dress, and burn with curiosity to know
who her distinguished-looking friend might be. With the first burst of
the band, Amy\textquotesingle s color rose, her eyes began to sparkle,
and her feet to tap the floor impatiently; for she danced well, and
wanted Laurie to know it: therefore the shock she received can better be
imagined than described, when he said, in a perfectly tranquil tone,---

"Do you care to dance?"

"One usually does at a ball."

Her amazed look and quick answer caused Laurie to repair his error as
fast as possible.

"I meant the first dance. May I have the honor?"

"I can give you one if I put off the Count. He dances divinely; but he
will excuse me, as you are an old friend," said Amy, hoping that the
name would have a good effect, and show Laurie that she was not to be
trifled with.

"Nice little boy, but rather a short Pole to support

"\textquotesingle A daughter of the gods,

Divinely tall, and most divinely fair,\textquotesingle"

was all the satisfaction she got, however.

The set in which they found themselves was composed of English, and Amy
was compelled to walk decorously through a cotillon, feeling all the
while as if she could dance the Tarantula with a relish. Laurie resigned
her to the "nice little boy," and went to do his duty to Flo, without
securing Amy for the joys to come, which reprehensible want of
forethought was properly punished, for she immediately engaged herself
till supper, meaning to relent if he then gave any signs of penitence.
She showed him her ball-book with demure satisfaction when he strolled,
instead of rushing, up to claim her for the next, a glorious
polka-redowa; but his polite regrets didn\textquotesingle t impose upon
her, and when she gallopaded away with the Count, she saw Laurie sit
down by her aunt with an actual expression of relief.

That was unpardonable; and Amy took no more notice of him for a long
while, except a word now and then, when she came to her chaperon,
between the dances, for a necessary pin or a moment\textquotesingle s
rest. Her anger had a good effect, however, for she hid it under a
smiling face, and seemed unusually blithe and brilliant.
Laurie\textquotesingle s eyes followed her with pleasure, for she
neither romped nor sauntered, but danced with spirit and grace, making
the delightsome pastime what it should be. He very naturally fell to
studying her from this new point of view; and, before the evening was
half over, had decided that "little Amy was going to make a very
charming woman."

It was a lively scene, for soon the spirit of the social season took
possession of every one, and Christmas merriment made all faces shine,
hearts happy, and heels light. The musicians fiddled, tooted, and banged
as if they enjoyed it; everybody danced who could, and those who
couldn\textquotesingle t admired their neighbors with uncommon warmth.
The air was dark with Davises, and many Joneses gambolled like a flock
of young giraffes. The golden secretary darted through the room like a
meteor, with a dashing Frenchwoman, who carpeted the floor with her pink
satin train. The Serene Teuton found the supper-table, and was happy,
eating steadily through the bill of fare, and dismayed the
\emph{garçons} by the ravages he committed. But the
Emperor\textquotesingle s friend covered himself with glory, for he
danced everything, whether he knew it or not, and introduced impromptu
pirouettes when the figures bewildered him. The boyish abandon of that
stout man was charming to behold; for, though he "carried weight," he
danced like an india-rubber ball. He ran, he flew, he pranced; his face
glowed, his bald head shone; his coat-tails waved wildly, his pumps
actually twinkled in the air, and when the music stopped, he wiped the
drops from his brow, and beamed upon his fellow-men like a French
Pickwick without glasses.

Amy and her Pole distinguished themselves by equal enthusiasm, but more
graceful agility; and Laurie found himself involuntarily keeping time to
the rhythmic rise and fall of the white slippers as they flew by as
indefatigably as if winged. When little Vladimir finally relinquished
her, with assurances that he was "desolated to leave so early," she was
ready to rest, and see how her recreant knight had borne his punishment.

It had been successful; for, at three-and-twenty, blighted affections
find a balm in friendly society, and young nerves will thrill, young
blood dance, and healthy young spirits rise, when subjected to the
enchantment of beauty, light, music, and motion. Laurie had a waked-up
look as he rose to give her his seat; and when he hurried away to bring
her some supper, she said to herself, with a satisfied smile,---

"Ah, I thought that would do him good!"

"You look like Balzac\textquotesingle s \textquotesingle Femme peinte
par elle-même,\textquotesingle" he said, as he fanned her with one hand,
and held her coffee-cup in the other.

"My rouge won\textquotesingle t come off;" and Amy rubbed her brilliant
cheek, and showed him her white glove with a sober simplicity that made
him laugh outright.

"What do you call this stuff?" he asked, touching a fold of her dress
that had blown over his knee.

"Illusion."

"Good name for it; it\textquotesingle s very pretty---new thing,
isn\textquotesingle t it?"

"It\textquotesingle s as old as the hills; you have seen it on dozens of
girls, and you never found out that it was pretty till
now---\emph{stupide}!"

"I never saw it on you before, which accounts for the mistake, you see."

"None of that, it is forbidden; I\textquotesingle d rather take coffee
than compliments just now. No, don\textquotesingle t lounge, it makes me
nervous."

Laurie sat bolt upright, and meekly took her empty plate, feeling an odd
sort of pleasure in having "little Amy" order him about; for she had
lost her shyness now, and felt an irresistible desire to trample on him,
as girls have a delightful way of doing when lords of creation show any
signs of subjection.

"Where did you learn all this sort of thing?" he asked, with a quizzical
look.

"As \textquotesingle this sort of thing\textquotesingle{} is rather a
vague expression, would you kindly explain?" returned Amy, knowing
perfectly well what he meant, but wickedly leaving him to describe what
is indescribable.

"Well---the general air, the style, the self-possession,
the---the---illusion---you know," laughed Laurie, breaking down, and
helping himself out of his quandary with the new word.

Amy was gratified, but, of course, didn\textquotesingle t show it, and
demurely answered, "Foreign life polishes one in spite of
one\textquotesingle s self; I study as well as play; and as for
this"---with a little gesture toward her dress---"why, tulle is cheap,
posies to be had for nothing, and I am used to making the most of my
poor little things."

Amy rather regretted that last sentence, fearing it
wasn\textquotesingle t in good taste; but Laurie liked her the better
for it, and found himself both admiring and respecting the brave
patience that made the most of opportunity, and the cheerful spirit that
covered poverty with flowers. Amy did not know why he looked at her so
kindly, nor why he filled up her book with his own name, and devoted
himself to her for the rest of the evening, in the most delightful
manner; but the impulse that wrought this agreeable change was the
result of one of the new impressions which both of them were
unconsciously giving and receiving.

\begin{center}\rule{0.5\linewidth}{0.5pt}\end{center}

\subsection{XXXVIII. On the
Shelf.}\label{6672479776654687619_37106-h-6.htm.xhtml_pgepubid00040}

\protect\phantomsection\label{6672479776654687619_37106-h-6.htm.xhtml_b167.png}{}
\pandocbounded{\includegraphics[keepaspectratio]{303483661336987339_b167.png}}

\protect\phantomsection\label{6672479776654687619_37106-h-6.htm.xhtml_XXXVIII}{}\hyperref[6672479776654687619_37106-h-0.htm.xhtml_contents2b]{XXXVIII.}

ON THE SHELF.

{In} France the young girls have a dull time of it till they are
married, when "\emph{Vive la liberté}" becomes their motto. In America,
as every one knows, girls early sign the declaration of independence,
and enjoy their freedom with republican zest; but the young matrons
usually abdicate with the first heir to the throne, and go into a
seclusion almost as close as a French nunnery, though by no means as
quiet. Whether they like it or not, they are virtually put upon the
shelf as soon as the wedding excitement is over, and most of them might
exclaim, as did a very pretty woman the other day, "I\textquotesingle m
as handsome as ever, but no one takes any notice of me because
I\textquotesingle m married."

Not being a belle or even a fashionable lady, Meg did not experience
this affliction till her babies were a year old, for in her little world
primitive customs prevailed, and she found herself more admired and
beloved than ever.

As she was a womanly little woman, the maternal instinct was very
strong, and she was entirely absorbed in her children, to the utter
exclusion of everything and everybody else. Day and night she brooded
over them with tireless devotion and anxiety, leaving John to the tender
mercies of the help, for an Irish lady now presided over the kitchen
department. Being a domestic man, John decidedly missed the wifely
attentions he had been accustomed to receive; but, as he adored his
babies, he cheerfully relinquished his comfort for a time, supposing,
with masculine ignorance, that peace would soon be restored. But three
months passed, and there was no return of repose; Meg looked worn and
nervous, the babies absorbed every minute of her time, the house was
neglected, and Kitty, the cook, who took life "aisy," kept him on short
commons. When he went out in the morning he was bewildered by small
commissions for the captive mamma; if he came gayly in at night, eager
to embrace his family, he was quenched by a "Hush! they are just asleep
after worrying all day." If he proposed a little amusement at home, "No,
it would disturb the babies." If he hinted at a lecture or concert, he
was answered with a reproachful look, and a decided "Leave my children
for pleasure, never!" His sleep was broken by infant wails and visions
of a phantom figure pacing noiselessly to and fro in the watches of the
night; his meals were interrupted by the frequent flight of the
presiding genius, who deserted him, half-helped, if a muffled chirp
sounded from the nest above; and when he read his paper of an evening,
Demi\textquotesingle s colic got into the shipping-list, and
Daisy\textquotesingle s fall affected the price of stocks, for Mrs.
Brooke was only interested in domestic news.

The poor man was very uncomfortable, for the children had bereft him of
his wife; home was merely a nursery, and the perpetual "hushing" made
him feel like a brutal intruder whenever he entered the sacred precincts
of Babyland. He bore it very patiently for six months, and, when no
signs of amendment appeared, he did what other paternal exiles
do,---tried to get a little comfort elsewhere. Scott had married and
gone to housekeeping not far off, and John fell into the way of running
over for an hour or two of an evening, when his own parlor was empty,
and his own wife singing lullabies that seemed to have no end. Mrs.
Scott was a lively, pretty girl, with nothing to do but be agreeable,
and she performed her mission most successfully. The parlor was always
bright and attractive, the chess-board ready, the piano in tune, plenty
of gay gossip, and a nice little supper set forth in tempting style.

John would have preferred his own fireside if it had not been so lonely;
but as it was, he gratefully took the next best thing, and enjoyed his
neighbor\textquotesingle s society.

Meg rather approved of the new arrangement at first, and found it a
relief to know that John was having a good time instead of dozing in the
parlor, or tramping about the house and waking the children. But by and
by, when the teething worry was over, and the idols went to sleep at
proper hours, leaving mamma time to rest, she began to miss John, and
find her work-basket dull company, when he was not sitting opposite in
his old dressing-gown, comfortably scorching his slippers on the fender.
She would not ask him to stay at home, but felt injured because he did
not know that she wanted him without being told, entirely forgetting the
many evenings he had waited for her in vain. She was nervous and worn
out with watching and worry, and in that unreasonable frame of mind
which the best of mothers occasionally experience when domestic cares
oppress them. Want of exercise robs them of cheerfulness, and too much
devotion to that idol of American women, the teapot, makes them feel as
if they were all nerve and no muscle.

"Yes," she would say, looking in the glass, "I\textquotesingle m getting
old and ugly; John doesn\textquotesingle t find me interesting any
longer, so he leaves his faded wife and goes to see his pretty neighbor,
who has no incumbrances. Well, the babies love me; they
don\textquotesingle t care if I am thin and pale, and
haven\textquotesingle t time to crimp my hair; they are my comfort, and
some day John will see what I\textquotesingle ve gladly sacrificed for
them, won\textquotesingle t he, my precious?"

To which pathetic appeal Daisy would answer with a coo, or Demi with a
crow, and Meg would put by her lamentations for a maternal revel, which
soothed her solitude for the time being. But the pain increased as
politics absorbed John, who was always running over to discuss
interesting points with Scott, quite unconscious that Meg missed him.
Not a word did she say, however, till her mother found her in tears one
day, and insisted on knowing what the matter was, for
Meg\textquotesingle s drooping spirits had not escaped her observation.

"I wouldn\textquotesingle t tell any one except you, mother; but I
really do need advice, for, if John goes on so much longer I might as
well be widowed," replied Mrs. Brooke, drying her tears on
Daisy\textquotesingle s bib, with an injured air.

"Goes on how, my dear?" asked her mother anxiously.

"He\textquotesingle s away all day, and at night, when I want to see
him, he is continually going over to the Scotts\textquotesingle. It
isn\textquotesingle t fair that I should have the hardest work, and
never any amusement. Men are very selfish, even the best of them."

"So are women; don\textquotesingle t blame John till you see where you
are wrong yourself."

"But it can\textquotesingle t be right for him to neglect me."

"Don\textquotesingle t you neglect him?"

"Why, mother, I thought you\textquotesingle d take my part!"

"So I do, as far as sympathizing goes; but I think the fault is yours,
Meg."

"I don\textquotesingle t see how."

"Let me show you. Did John ever neglect you, as you call it, while you
made it a point to give him your society of an evening, his only leisure
time?"

"No; but I can\textquotesingle t do it now, with two babies to tend."

"I think you could, dear; and I think you ought. May I speak quite
freely, and will you remember that it\textquotesingle s mother who
blames as well as mother who sympathizes?"

"Indeed I will! Speak to me as if I were little Meg again. I often feel
as if I needed teaching more than ever since these babies look to me for
everything."

Meg drew her low chair beside her mother\textquotesingle s, and, with a
little interruption in either lap, the two women rocked and talked
lovingly together, feeling that the tie of motherhood made them more one
than ever.

"You have only made the mistake that most young wives make,---forgotten
your duty to your husband in your love for your children. A very natural
and forgivable mistake, Meg, but one that had better be remedied before
you take to different ways; for children should draw you nearer than
ever, not separate you, as if they were all yours, and John had nothing
to do but support them. I\textquotesingle ve seen it for some weeks, but
have not spoken, feeling sure it would come right in time."

"I\textquotesingle m afraid it won\textquotesingle t. If I ask him to
stay, he\textquotesingle ll think I\textquotesingle m jealous; and I
wouldn\textquotesingle t insult him by such an idea. He
doesn\textquotesingle t see that I want him, and I don\textquotesingle t
know how to tell him without words."

"Make it so pleasant he won\textquotesingle t want to go away. My dear,
he\textquotesingle s longing for his little home; but it
isn\textquotesingle t home without you, and you are always in the
nursery."

"Oughtn\textquotesingle t I to be there?"

"Not all the time; too much confinement makes you nervous, and then you
are unfitted for everything. Besides, you owe something to John as well
as to the babies; don\textquotesingle t neglect husband for children,
don\textquotesingle t shut him out of the nursery, but teach him how to
help in it. His place is there as well as yours, and the children need
him; let him feel that he has his part to do, and he will do it gladly
and faithfully, and it will be better for you all."

"You really think so, mother?"

"I know it, Meg, for I\textquotesingle ve tried it; and I seldom give
advice unless I\textquotesingle ve proved its practicability. When you
and Jo were little, I went on just as you are, feeling as if I
didn\textquotesingle t do my duty unless I devoted myself wholly to you.
Poor father took to his books, after I had refused all offers of help,
and left me to try my experiment alone. I struggled along as well as I
could, but Jo was too much for me. I nearly spoilt her by indulgence.
You were poorly, and I worried about you till I fell sick myself. Then
father came to the rescue, quietly managed everything, and made himself
so helpful that I saw my mistake, and never have been able to get on
without him since. That is the secret of our home happiness: he does not
let business wean him from the little cares and duties that affect us
all, and I try not to let domestic worries destroy my interest in his
pursuits. Each do our part alone in many things, but at home we work
together, always."

"It is so, mother; and my great wish is to be to my husband and children
what you have been to yours. Show me how; I\textquotesingle ll do
anything you say."

"You always were my docile daughter. Well, dear, if I were you,
I\textquotesingle d let John have more to do with the management of
Demi, for the boy needs training, and it\textquotesingle s none too soon
to begin. Then I\textquotesingle d do what I have often proposed, let
Hannah come and help you; she is a capital nurse, and you may trust the
precious babies to her while you do more housework. You need the
exercise, Hannah would enjoy the rest, and John would find his wife
again. Go out more; keep cheerful as well as busy, for you are the
sunshine-maker of the family, and if you get dismal there is no fair
weather. Then I\textquotesingle d try to take an interest in whatever
John likes,---talk with him, let him read to you, exchange ideas, and
help each other in that way. Don\textquotesingle t shut yourself up in a
bandbox because you are a woman, but understand what is going on, and
educate yourself to take your part in the world\textquotesingle s work,
for it all affects you and yours."

"John is so sensible, I\textquotesingle m afraid he will think
I\textquotesingle m stupid if I ask questions about politics and
things."

"I don\textquotesingle t believe he would; love covers a multitude of
sins, and of whom could you ask more freely than of him? Try it, and see
if he doesn\textquotesingle t find your society far more agreeable than
Mrs. Scott\textquotesingle s suppers."

"I will. Poor John! I\textquotesingle m afraid I \emph{have} neglected
him sadly, but I thought I was right, and he never said anything."

"He tried not to be selfish, but he \emph{has} felt rather forlorn, I
fancy. This is just the time, Meg, when young married people are apt to
grow apart, and the very time when they ought to be most together; for
the first tenderness soon wears off, unless care is taken to preserve
it; and no time is so beautiful and precious to parents as the first
years of the little lives given them to train. Don\textquotesingle t let
John be a stranger to the babies, for they will do more to keep him safe
and happy in this world of trial and temptation than anything else, and
through them you will learn to know and love one another as you should.
Now, dear, good-by; think over mother\textquotesingle s preachment, act
upon it if it seems good, and God bless you all!"

Meg did think it over, found it good, and acted upon it, though the
first attempt was not made exactly as she planned to have it. Of course
the children tyrannized over her, and ruled the house as so on as they
found out that kicking and squalling brought them whatever they wanted.
Mamma was an abject slave to their caprices, but papa was not so easily
subjugated, and occasionally afflicted his tender spouse by an attempt
at paternal discipline with his obstreperous son. For Demi inherited a
trifle of his sire\textquotesingle s firmness of character,---we
won\textquotesingle t call it obstinacy,---and when he made up his
little mind to have or to do anything, all the king\textquotesingle s
horses and all the king\textquotesingle s men could not change that
pertinacious little mind. Mamma thought the dear too young to be taught
to conquer his prejudices, but papa believed that it never was too soon
to learn obedience; so Master Demi early discovered that when he
undertook to "wrastle" with "parpar," he always got the worst of it;
yet, like the Englishman, Baby respected the man who conquered him, and
loved the father whose grave "No, no," was more impressive than all
mamma\textquotesingle s love-pats.

A few days after the talk with her mother, Meg resolved to try a social
evening with John; so she ordered a nice supper, set the parlor in
order, dressed herself prettily, and put the children to bed early, that
nothing should interfere with her experiment. But, unfortunately,
Demi\textquotesingle s most unconquerable prejudice was against going to
bed, and that night he decided to go on a rampage; so poor Meg sung and
rocked, told stories and tried every sleep-provoking wile she could
devise, but all in vain, the big eyes wouldn\textquotesingle t shut; and
long after Daisy had gone to byelow, like the chubby little bunch of
good-nature she was, naughty Demi lay staring at the light, with the
most discouragingly wide-awake expression of countenance.

"Will Demi lie still like a good boy, while mamma runs down and gives
poor papa his tea?" asked Meg, as the hall-door softly closed, and the
well-known step went tiptoeing into the dining-room.

"Me has tea!" said Demi, preparing to join in the revel.

"No; but I\textquotesingle ll save you some little cakies for breakfast,
if you\textquotesingle ll go bye-by like Daisy. Will you, lovey?"

"Iss!" and Demi shut his eyes tight, as if to catch sleep and hurry the
desired day.

Taking advantage of the propitious moment, Meg slipped away, and ran
down to greet her husband with a smiling face, and the little blue bow
in her hair which was his especial admiration. He saw it at once, and
said, with pleased surprise,---

"Why, little mother, how gay we are to-night. Do you expect company?"

"Only you, dear."

"Is it a birthday, anniversary, or anything?"

"No; I\textquotesingle m tired of being a dowdy, so I dressed up as a
change. You always make yourself nice for table, no matter how tired you
are; so why shouldn\textquotesingle t I when I have the time?"

"I do it out of respect to you, my dear," said old-fashioned John.

"Ditto, ditto, Mr. Brooke," laughed Meg, looking young and pretty again,
as she nodded to him over the teapot.

"Well, it\textquotesingle s altogether delightful, and like old times.
This tastes right. I drink your health, dear." And John sipped his tea
with an air of reposeful rapture, which was of very short duration,
however; for, as he put down his cup, the door-handle rattled
mysteriously, and a little voice was heard, saying impatiently,---

"Opy doy; me\textquotesingle s tummin!"

"It\textquotesingle s that naughty boy. I told him to go to sleep alone,
and here he is, downstairs, getting his death a-cold pattering over that
canvas," said Meg, answering the call.

\protect\phantomsection\label{6672479776654687619_37106-h-6.htm.xhtml_b168.png}{}
\pandocbounded{\includegraphics[keepaspectratio]{303483661336987339_b168.png}}

"Mornin\textquotesingle{} now," announced Demi, in a joyful tone, as he
entered, with his long night-gown gracefully festooned over his arm, and
every curl bobbing gayly as he pranced about the table, eying the
"cakies" with loving glances.

"No, it isn\textquotesingle t morning yet. You must go to bed, and not
trouble poor mamma; then you can have the little cake with sugar on it."

"Me loves parpar," said the artful one, preparing to climb the paternal
knee, and revel in forbidden joys. But John shook his head, and said to
Meg,---

"If you told him to stay up there, and go to sleep alone, make him do
it, or he will never learn to mind you."

"Yes, of course. Come, Demi;" and Meg led her son away, feeling a strong
desire to spank the little marplot who hopped beside her, laboring under
the delusion that the bribe was to be administered as soon as they
reached the nursery.

Nor was he disappointed; for that short-sighted woman actually gave him
a lump of sugar, tucked him into his bed, and forbade any more
promenades till morning.

"Iss!" said Demi the perjured, blissfully sucking his sugar, and
regarding his first attempt as eminently successful.

Meg returned to her place, and supper was progressing pleasantly, when
the little ghost walked again, and exposed the maternal delinquencies by
boldly demanding,---

"More sudar, marmar."

"Now this won\textquotesingle t do," said John, hardening his heart
against the engaging little sinner. "We shall never know any peace till
that child learns to go to bed properly. You have made a slave of
yourself long enough; give him one lesson, and then there will be an end
of it. Put him in his bed and leave him, Meg."

"He won\textquotesingle t stay there; he never does, unless I sit by
him."

"I\textquotesingle ll manage him. Demi, go upstairs, and get into your
bed, as mamma bids you."

"S\textquotesingle ant!" replied the young rebel, helping himself to the
coveted "cakie," and beginning to eat the same with calm audacity.

"You must never say that to papa; I shall carry you if you
don\textquotesingle t go yourself."

"Go \textquotesingle way; me don\textquotesingle t love parpar;" and
Demi retired to his mother\textquotesingle s skirts for protection.

But even that refuge proved unavailing, for he was delivered over to the
enemy, with a "Be gentle with him, John," which struck the culprit with
dismay; for when mamma deserted him, then the judgment-day was at hand.
Bereft of his cake, defrauded of his frolic, and borne away by a strong
hand to that detested bed, poor Demi could not restrain his wrath, but
openly defied papa, and kicked and screamed lustily all the way
upstairs. The minute he was put into bed on one side, he rolled out on
the other, and made for the door, only to be ignominiously caught up by
the tail of his little toga, and put back again, which lively
performance was kept up till the young man\textquotesingle s strength
gave out, when he devoted himself to roaring at the top of his voice.
This vocal exercise usually conquered Meg; but John sat as unmoved as
the post which is popularly believed to be deaf. No coaxing, no sugar,
no lullaby, no story; even the light was put out, and only the red glow
of the fire enlivened the "big dark" which Demi regarded with curiosity
rather than fear. This new order of things disgusted him, and he howled
dismally for "marmar," as his angry passions subsided, and recollections
of his tender bondwoman returned to the captive autocrat. The plaintive
wail which succeeded the passionate roar went to Meg\textquotesingle s
heart, and she ran up to say beseechingly,---

"Let me stay with him; he\textquotesingle ll be good, now, John."

"No, my dear, I\textquotesingle ve told him he must go to sleep, as you
bid him; and he must, if I stay here all night."

"But he\textquotesingle ll cry himself sick," pleaded Meg, reproaching
herself for deserting her boy.

"No, he won\textquotesingle t, he\textquotesingle s so tired he will
soon drop off, and then the matter is settled; for he will understand
that he has got to mind. Don\textquotesingle t interfere;
I\textquotesingle ll manage him."

"He\textquotesingle s my child, and I can\textquotesingle t have his
spirit broken by harshness."

"He\textquotesingle s my child, and I won\textquotesingle t have his
temper spoilt by indulgence. Go down, my dear, and leave the boy to me."

When John spoke in that masterful tone, Meg always obeyed, and never
regretted her docility.

"Please let me kiss him once, John?"

"Certainly. Demi, say \textquotesingle good-night\textquotesingle{} to
mamma, and let her go and rest, for she is very tired with taking care
of you all day."

Meg always insisted upon it that the kiss won the victory; for after it
was given, Demi sobbed more quietly, and lay quite still at the bottom
of the bed, whither he had wriggled in his anguish of mind.

"Poor little man, he\textquotesingle s worn out with sleep and crying.
I\textquotesingle ll cover him up, and then go and set
Meg\textquotesingle s heart at rest," thought John, creeping to the
bedside, hoping to find his rebellious heir asleep.

But he wasn\textquotesingle t; for the moment his father peeped at him,
Demi\textquotesingle s eyes opened, his little chin began to quiver, and
he put up his arms, saying, with a penitent hiccough,
"Me\textquotesingle s dood, now."

Sitting on the stairs, outside, Meg wondered at the long silence which
followed the uproar; and, after imagining all sorts of impossible
accidents, she slipped into the room, to set her fears at rest. Demi lay
fast asleep; not in his usual spread-eagle attitude, but in a subdued
bunch, cuddled close in the circle of his father\textquotesingle s arm
and holding his father\textquotesingle s finger, as if he felt that
justice was tempered with mercy, and had gone to sleep a sadder and a
wiser baby. So held, John had waited with womanly patience till the
little hand relaxed its hold; and, while waiting, had fallen asleep,
more tired by that tussle with his son than with his whole
day\textquotesingle s work.

As Meg stood watching the two faces on the pillow, she smiled to
herself, and then slipped away again, saying, in a satisfied tone,---

"I never need fear that John will be too harsh with my babies: he
\emph{does} know how to manage them, and will be a great help, for Demi
\emph{is} getting too much for me."

When John came down at last, expecting to find a pensive or reproachful
wife, he was agreeably surprised to find Meg placidly trimming a bonnet,
and to be greeted with the request to read something about the election,
if he was not too tired. John saw in a minute that a revolution of some
kind was going on, but wisely asked no questions, knowing that Meg was
such a transparent little person, she couldn\textquotesingle t keep a
secret to save her life, and therefore the clew would soon appear. He
read a long debate with the most amiable readiness, and then explained
it in his most lucid manner, while Meg tried to look deeply interested,
to ask intelligent questions, and keep her thoughts from wandering from
the state of the nation to the state of her bonnet. In her secret soul,
however, she decided that politics were as bad as mathematics, and that
the mission of politicians seemed to be calling each other names; but
she kept these feminine ideas to herself, and when John paused, shook
her head, and said with what she thought diplomatic ambiguity,---

"Well, I really don\textquotesingle t see what we are coming to."

John laughed, and watched her for a minute, as she poised a pretty
little preparation of lace and flowers on her hand, and regarded it with
the genuine interest which his harangue had failed to waken.

"She is trying to like politics for my sake, so I\textquotesingle ll try
and like millinery for hers, that\textquotesingle s only fair," thought
John the Just, adding aloud,---

"That\textquotesingle s very pretty; is it what you call a
breakfast-cap?"

\protect\phantomsection\label{6672479776654687619_37106-h-6.htm.xhtml_b169.png}{}
\pandocbounded{\includegraphics[keepaspectratio]{303483661336987339_b169.png}}

"My dear man, it\textquotesingle s a bonnet! My very best
go-to-concert-and-theatre bonnet."

"I beg your pardon; it was so small, I naturally mistook it for one of
the fly-away things you sometimes wear. How do you keep it on?"

"These bits of lace are fastened under the chin with a rosebud, so;" and
Meg illustrated by putting on the bonnet, and regarding him with an air
of calm satisfaction that was irresistible.

"It\textquotesingle s a love of a bonnet, but I prefer the face inside,
for it looks young and happy again," and John kissed the smiling face,
to the great detriment of the rosebud under the chin.

"I\textquotesingle m glad you like it, for I want you to take me to one
of the new concerts some night; I really need some music to put me in
tune. Will you, please?"

"Of course I will, with all my heart, or anywhere else you like. You
have been shut up so long, it will do you no end of good, and I shall
enjoy it, of all things. What put it into your head, little mother?"

"Well, I had a talk with Marmee the other day, and told her how nervous
and cross and out of sorts I felt, and she said I needed change and less
care; so Hannah is to help me with the children, and I\textquotesingle m
to see to things about the house more, and now and then have a little
fun, just to keep me from getting to be a fidgety, broken-down old woman
before my time. It\textquotesingle s only an experiment, John, and I
want to try it for your sake as much as for mine, because
I\textquotesingle ve neglected you shamefully lately, and
I\textquotesingle m going to make home what it used to be, if I can. You
don\textquotesingle t object, I hope?"

Never mind what John said, or what a very narrow escape the little
bonnet had from utter ruin; all that we have any business to know, is
that John did \emph{not} appear to object, judging from the changes
which gradually took place in the house and its inmates. It was not all
Paradise by any means, but every one was better for the division of
labor system; the children throve under the paternal rule, for accurate,
steadfast John brought order and obedience into Babydom, while Meg
recovered her spirits and composed her nerves by plenty of wholesome
exercise, a little pleasure, and much confidential conversation with her
sensible husband. Home grew home-like again, and John had no wish to
leave it, unless he took Meg with him. The Scotts came to the
Brookes\textquotesingle{} now, and every one found the little house a
cheerful place, full of happiness, content, and family love. Even gay
Sallie Moffatt liked to go there. "It is always so quiet and pleasant
here; it does me good, Meg," she used to say, looking about her with
wistful eyes, as if trying to discover the charm, that she might use it
in her great house, full of splendid loneliness; for there were no
riotous, sunny-faced babies there, and Ned lived in a world of his own,
where there was no place for her.

This household happiness did not come all at once, but John and Meg had
found the key to it, and each year of married life taught them how to
use it, unlocking the treasuries of real home-love and mutual
helpfulness, which the poorest may possess, and the richest cannot buy.
This is the sort of shelf on which young wives and mothers may consent
to be laid, safe from the restless fret and fever of the world, finding
loyal lovers in the little sons and daughters who cling to them,
undaunted by sorrow, poverty, or age; walking side by side, through fair
and stormy weather, with a faithful friend, who is, in the true sense of
the good old Saxon word, the "house-band," and learning, as Meg learned,
that a woman\textquotesingle s happiest kingdom is home, her highest
honor the art of ruling it, not as a queen, but a wise wife and mother.

\protect\phantomsection\label{6672479776654687619_37106-h-6.htm.xhtml_b170.png}{}
\pandocbounded{\includegraphics[keepaspectratio]{303483661336987339_b170.png}}

\begin{center}\rule{0.5\linewidth}{0.5pt}\end{center}

\subsection{XXXIX. Lazy
Laurence.}\label{6672479776654687619_37106-h-6.htm.xhtml_pgepubid00041}

\protect\phantomsection\label{6672479776654687619_37106-h-6.htm.xhtml_b171.png}{}
\pandocbounded{\includegraphics[keepaspectratio]{303483661336987339_b171.png}}

\pandocbounded{\includegraphics[keepaspectratio]{303483661336987339_b171b.png}}

\protect\phantomsection\label{6672479776654687619_37106-h-6.htm.xhtml_XXXIX}{}\hyperref[6672479776654687619_37106-h-0.htm.xhtml_contents2b]{XXXIX.}

LAZY LAURENCE.

{Laurie} went to Nice intending to stay a week, and remained a month. He
was tired of wandering about alone, and Amy\textquotesingle s familiar
presence seemed to give a home-like charm to the foreign scenes in which
she bore a part. He rather missed the "petting" he used to receive, and
enjoyed a taste of it again; for no attentions, however flattering, from
strangers, were half so pleasant as the sisterly adoration of the girls
at home. Amy never would pet him like the others, but she was very glad
to see him now, and quite clung to him, feeling that he was the
representative of the dear family for whom she longed more than she
would confess. They naturally took comfort in each
other\textquotesingle s society, and were much together, riding,
walking, dancing, or dawdling, for, at Nice, no one can be very
industrious during the gay season. But, while apparently amusing
themselves in the most careless fashion, they were half-consciously
making discoveries and forming opinions about each other. Amy rose daily
in the estimation of her friend, but he sunk in hers, and each felt the
truth before a word was spoken. Amy tried to please, and succeeded, for
she was grateful for the many pleasures he gave her, and repaid him with
the little services to which womanly women know how to lend an
indescribable charm. Laurie made no effort of any kind, but just let
himself drift along as comfortably as possible, trying to forget, and
feeling that all women owed him a kind word because one had been cold to
him. It cost him no effort to be generous, and he would have given Amy
all the trinkets in Nice if she would have taken them; but, at the same
time, he felt that he could not change the opinion she was forming of
him, and he rather dreaded the keen blue eyes that seemed to watch him
with such half-sorrowful, half-scornful surprise.

"All the rest have gone to Monaco for the day; I preferred to stay at
home and write letters. They are done now, and I am going to Valrosa to
sketch; will you come?" said Amy, as she joined Laurie one lovely day
when he lounged in as usual, about noon.

"Well, yes; but isn\textquotesingle t it rather warm for such a long
walk?" he answered slowly, for the shaded \emph{salon} looked inviting,
after the glare without.

"I\textquotesingle m going to have the little carriage, and Baptiste can
drive, so you\textquotesingle ll have nothing to do but hold your
umbrella and keep your gloves nice," returned Amy, with a sarcastic
glance at the immaculate kids, which were a weak point with Laurie.

"Then I\textquotesingle ll go with pleasure;" and he put out his hand
for her sketch-book. But she tucked it under her arm with a sharp---

"Don\textquotesingle t trouble yourself; it\textquotesingle s no
exertion to me, but \emph{you} don\textquotesingle t look equal to it."

Laurie lifted his eyebrows, and followed at a leisurely pace as she ran
downstairs; but when they got into the carriage he took the reins
himself, and left little Baptiste nothing to do but fold his arms and
fall asleep on his perch.

The two never quarrelled,---Amy was too well-bred, and just now Laurie
was too lazy; so, in a minute he peeped under her hat-brim with an
inquiring air; she answered with a smile, and they went on together in
the most amicable manner.

It was a lovely drive, along winding roads rich in the picturesque
scenes that delight beauty-loving eyes. Here an ancient monastery,
whence the solemn chanting of the monks came down to them. There a
bare-legged shepherd, in wooden shoes, pointed hat, and rough jacket
over one shoulder, sat piping on a stone, while his goats skipped among
the rocks or lay at his feet. Meek, mouse-colored donkeys, laden with
panniers of freshly-cut grass, passed by, with a pretty girl in a
\emph{capaline} sitting between the green piles, or an old woman
spinning with a distaff as she went. Brown, soft-eyed children ran out
from the quaint stone hovels to offer nosegays, or bunches of oranges
still on the bough. Gnarled olive-trees covered the hills with their
dusky foliage, fruit hung golden in the orchard, and great scarlet
anemones fringed the roadside; while beyond green slopes and craggy
heights, the Maritime Alps rose sharp and white against the blue Italian
sky.

Valrosa well deserved its name, for, in that climate of perpetual
summer, roses blossomed everywhere. They overhung the archway, thrust
themselves between the bars of the great gate with a sweet welcome to
passers-by, and lined the avenue, winding through lemon-trees and
feathery palms up to the villa on the hill. Every shadowy nook, where
seats invited one to stop and rest, was a mass of bloom; every cool
grotto had its marble nymph smiling from a veil of flowers, and every
fountain reflected crimson, white, or pale pink roses, leaning down to
smile at their own beauty. Roses covered the walls of the house, draped
the cornices, climbed the pillars, and ran riot over the balustrade of
the wide terrace, whence one looked down on the sunny Mediterranean, and
the white-walled city on its shore.

"This is a regular honeymoon Paradise, isn\textquotesingle t it? Did you
ever see such roses?" asked Amy, pausing on the terrace to enjoy the
view, and a luxurious whiff of perfume that came wandering by.

"No, nor felt such thorns," returned Laurie, with his thumb in his
mouth, after a vain attempt to capture a solitary scarlet flower that
grew just beyond his reach.

"Try lower down, and pick those that have no thorns," said Amy,
gathering three of the tiny cream-colored ones that starred the wall
behind her. She put them in his button-hole, as a peace-offering, and he
stood a minute looking down at them with a curious expression, for in
the Italian part of his nature there was a touch of superstition, and he
was just then in that state of half-sweet, half-bitter melancholy, when
imaginative young men find significance in trifles, and food for romance
everywhere. He had thought of Jo in reaching after the thorny red rose,
for vivid flowers became her, and she had often worn ones like that from
the greenhouse at home. The pale roses Amy gave him were the sort that
the Italians lay in dead hands, never in bridal wreaths, and, for a
moment, he wondered if the omen was for Jo or for himself; but the next
instant his American common-sense got the better of sentimentality, and
he laughed a heartier laugh than Amy had heard since he came.

"It\textquotesingle s good advice; you\textquotesingle d better take it
and save your fingers," she said, thinking her speech amused him.

"Thank you, I will," he answered in jest, and a few months later he did
it in earnest.

"Laurie, when are you going to your grandfather?" she asked presently,
as she settled herself on a rustic seat.

"Very soon."

"You have said that a dozen times within the last three weeks."

"I dare say; short answers save trouble."

"He expects you, and you really ought to go."

"Hospitable creature! I know it."

"Then why don\textquotesingle t you do it?"

"Natural depravity, I suppose."

"Natural indolence, you mean. It\textquotesingle s really dreadful!" and
Amy looked severe.

"Not so bad as it seems, for I should only plague him if I went, so I
might as well stay, and plague you a little longer, you can bear it
better; in fact, I think it agrees with you excellently;" and Laurie
composed himself for a lounge on the broad ledge of the balustrade.

Amy shook her head, and opened her sketch-book with an air of
resignation; but she had made up her mind to lecture "that boy," and in
a minute she began again.

"What are you doing just now?"

"Watching lizards."

"No, no; I mean what do you intend and wish to do?"

"Smoke a cigarette, if you\textquotesingle ll allow me."

"How provoking you are! I don\textquotesingle t approve of cigars, and I
will only allow it on condition that you let me put you into my sketch;
I need a figure."

"With all the pleasure in life. How will you have me,---full-length or
three-quarters, on my head or my heels? I should respectfully suggest a
recumbent posture, then put yourself in also, and call it
\textquotesingle{}\emph{Dolce far niente.}\textquotesingle"

"Stay as you are, and go to sleep if you like. \emph{I} intend to work
hard," said Amy, in her most energetic tone.

"What delightful enthusiasm!" and he leaned against a tall urn with an
air of entire satisfaction.

"What would Jo say if she saw you now?" asked Amy impatiently, hoping to
stir him up by the mention of her still more energetic
sister\textquotesingle s name.

"As usual, \textquotesingle Go away, Teddy, I\textquotesingle m
busy!\textquotesingle" He laughed as he spoke, but the laugh was not
natural, and a shade passed over his face, for the utterance of the
familiar name touched the wound that was not healed yet. Both tone and
shadow struck Amy, for she had seen and heard them before, and now she
looked up in time to catch a new expression on Laurie\textquotesingle s
face,---a hard, bitter look, full of pain, dissatisfaction, and regret.
It was gone before she could study it, and the listless expression back
again. She watched him for a moment with artistic pleasure, thinking how
like an Italian he looked, as he lay basking in the sun with uncovered
head, and eyes full of southern dreaminess; for he seemed to have
forgotten her, and fallen into a reverie.

"You look like the effigy of a young knight asleep on his tomb," she
said, carefully tracing the well-cut profile defined against the dark
stone.

"Wish I was!"

"That\textquotesingle s a foolish wish, unless you have spoilt your
life. You are so changed, I sometimes think---" there Amy stopped, with
a half-timid, half-wistful look, more significant than her unfinished
speech.

Laurie saw and understood the affectionate anxiety which she hesitated
to express, and looking straight into her eyes, said, just as he used to
say it to her mother,---

"It\textquotesingle s all right, ma\textquotesingle am."

That satisfied her and set at rest the doubts that had begun to worry
her lately. It also touched her, and she showed that it did, by the
cordial tone in which she said,---

"I\textquotesingle m glad of that! I didn\textquotesingle t think
you\textquotesingle d been a very bad boy, but I fancied you might have
wasted money at that wicked Baden-Baden, lost your heart to some
charming Frenchwoman with a husband, or got into some of the scrapes
that young men seem to consider a necessary part of a foreign tour.
Don\textquotesingle t stay out there in the sun; come and lie on the
grass here, and \textquotesingle let us be friendly,\textquotesingle{}
as Jo used to say when we got in the sofa-corner and told secrets."

\protect\phantomsection\label{6672479776654687619_37106-h-7.htm.xhtml}{}

\protect\phantomsection\label{6672479776654687619_37106-h-7.htm.xhtml_b172.png}{}
\pandocbounded{\includegraphics[keepaspectratio]{303483661336987339_b172.png}}

Laurie obediently threw himself down on the turf, and began to amuse
himself by sticking daisies into the ribbons of Amy\textquotesingle s
hat, that lay there.

"I\textquotesingle m all ready for the secrets;" and he glanced up with
a decided expression of interest in his eyes.

"I\textquotesingle ve none to tell; you may begin."

"Haven\textquotesingle t one to bless myself with. I thought perhaps
you\textquotesingle d had some news from home."

"You have heard all that has come lately. Don\textquotesingle t you hear
often? I fancied Jo would send you volumes."

"She\textquotesingle s very busy; I\textquotesingle m roving about so,
it\textquotesingle s impossible to be regular, you know. When do you
begin your great work of art, Raphaella?" he asked, changing the subject
abruptly after another pause, in which he had been wondering if Amy knew
his secret, and wanted to talk about it.

"Never," she answered, with a despondent but decided air. "Rome took all
the vanity out of me; for after seeing the wonders there, I felt too
insignificant to live, and gave up all my foolish hopes in despair."

"Why should you, with so much energy and talent?"

"That\textquotesingle s just why,---because talent isn\textquotesingle t
genius, and no amount of energy can make it so. I want to be great, or
nothing. I won\textquotesingle t be a common-place dauber, so I
don\textquotesingle t intend to try any more."

"And what are you going to do with yourself now, if I may ask?"

"Polish up my other talents, and be an ornament to society, if I get the
chance."

It was a characteristic speech, and sounded daring; but audacity becomes
young people, and Amy\textquotesingle s ambition had a good foundation.
Laurie smiled, but he liked the spirit with which she took up a new
purpose when a long-cherished one died, and spent no time lamenting.

"Good! and here is where Fred Vaughn comes in, I fancy."

Amy preserved a discreet silence, but there was a conscious look in her
downcast face, that made Laurie sit up and say gravely,---

"Now I\textquotesingle m going to play brother, and ask questions. May
I?"

"I don\textquotesingle t promise to answer."

"Your face will, if your tongue won\textquotesingle t. You
aren\textquotesingle t woman of the world enough yet to hide your
feelings, my dear. I heard rumors about Fred and you last year, and
it\textquotesingle s my private opinion that, if he had not been called
home so suddenly and detained so long, something would have come of
it---hey?"

"That\textquotesingle s not for me to say," was Amy\textquotesingle s
prim reply; but her lips would smile, and there was a traitorous sparkle
of the eye, which betrayed that she knew her power and enjoyed the
knowledge.

"You are not engaged, I hope?" and Laurie looked very elder-brotherly
and grave all of a sudden.

"No."

"But you will be, if he comes back and goes properly down upon his
knees, won\textquotesingle t you?"

"Very likely."

"Then you are fond of old Fred?"

"I could be, if I tried."

"But you don\textquotesingle t intend to try till the proper moment?
Bless my soul, what unearthly prudence! He\textquotesingle s a good
fellow, Amy, but not the man I fancied you\textquotesingle d like."

"He is rich, a gentleman, and has delightful manners," began Amy, trying
to be quite cool and dignified, but feeling a little ashamed of herself,
in spite of the sincerity of her intentions.

"I understand; queens of society can\textquotesingle t get on without
money, so you mean to make a good match, and start in that way? Quite
right and proper, as the world goes, but it sounds odd from the lips of
one of your mother\textquotesingle s girls."

"True, nevertheless."

A short speech, but the quiet decision with which it was uttered
contrasted curiously with the young speaker. Laurie felt this
instinctively, and laid himself down again, with a sense of
disappointment which he could not explain. His look and silence, as well
as a certain inward self-disapproval, ruffled Amy, and made her resolve
to deliver her lecture without delay.

"I wish you\textquotesingle d do me the favor to rouse yourself a
little," she said sharply.

"Do it for me, there\textquotesingle s a dear girl."

"I could, if I tried;" and she looked as if she would like doing it in
the most summary style.

"Try, then; I give you leave," returned Laurie, who enjoyed having some
one to tease, after his long abstinence from his favorite pastime.

"You\textquotesingle d be angry in five minutes."

"I\textquotesingle m never angry with you. It takes two flints to make a
fire: you are as cool and soft as snow."

"You don\textquotesingle t know what I can do; snow produces a glow and
a tingle, if applied rightly. Your indifference is half affectation, and
a good stirring up would prove it."

"Stir away; it won\textquotesingle t hurt me and it may amuse you, as
the big man said when his little wife beat him. Regard me in the light
of a husband or a carpet, and beat till you are tired, if that sort of
exercise agrees with you."

Being decidedly nettled herself, and longing to see him shake off the
apathy that so altered him, Amy sharpened both tongue and pencil, and
began:---

"Flo and I have got a new name for you; it\textquotesingle s
\textquotesingle Lazy Laurence.\textquotesingle{} How do you like it?"

She thought it would annoy him; but he only folded his arms under his
head, with an imperturbable "That\textquotesingle s not bad. Thank you,
ladies."

"Do you want to know what I honestly think of you?"

"Pining to be told."

"Well, I despise you."

If she had even said "I hate you," in a petulant or coquettish tone, he
would have laughed, and rather liked it; but the grave, almost sad,
accent of her voice made him open his eyes, and ask quickly,---

"Why, if you please?"

"Because, with every chance for being good, useful, and happy, you are
faulty, lazy, and miserable."

"Strong language, mademoiselle."

"If you like it, I\textquotesingle ll go on."

"Pray, do; it\textquotesingle s quite interesting."

"I thought you\textquotesingle d find it so; selfish people always like
to talk about themselves."

"Am \emph{I} selfish?" The question slipped out involuntarily and in a
tone of surprise, for the one virtue on which he prided himself was
generosity.

"Yes, very selfish," continued Amy, in a calm, cool voice, twice as
effective, just then, as an angry one. "I\textquotesingle ll show you
how, for I\textquotesingle ve studied you while we have been frolicking,
and I\textquotesingle m not at all satisfied with you. Here you have
been abroad nearly six months, and done nothing but waste time and money
and disappoint your friends."

"Isn\textquotesingle t a fellow to have any pleasure after a four-years
grind?"

"You don\textquotesingle t look as if you\textquotesingle d had much; at
any rate, you are none the better for it, as far as I can see. I said,
when we first met, that you had improved. Now I take it all back, for I
don\textquotesingle t think you half so nice as when I left you at home.
You have grown abominably lazy; you like gossip, and waste time on
frivolous things; you are contented to be petted and admired by silly
people, instead of being loved and respected by wise ones. With money,
talent, position, health, and beauty,---ah, you like that, Old Vanity!
but it\textquotesingle s the truth, so I can\textquotesingle t help
saying it,---with all these splendid things to use and enjoy, you can
find nothing to do but dawdle; and, instead of being the man you might
and ought to be, you are only---" There she stopped, with a look that
had both pain and pity in it.

"Saint Laurence on a gridiron," added Laurie, blandly finishing the
sentence. But the lecture began to take effect, for there was a
wide-awake sparkle in his eyes now, and a half-angry, half-injured
expression replaced the former indifference.

"I supposed you\textquotesingle d take it so. You men tell us we are
angels, and say we can make you what we will; but the instant we
honestly try to do you good, you laugh at us, and won\textquotesingle t
listen, which proves how much your flattery is worth." Amy spoke
bitterly, and turned her back on the exasperating martyr at her feet.

In a minute a hand came down over the page, so that she could not draw,
and Laurie\textquotesingle s voice said, with a droll imitation of a
penitent child,---

"I will be good, oh, I will be good!"

But Amy did not laugh, for she was in earnest; and, tapping on the
outspread hand with her pencil, said soberly,---

"Aren\textquotesingle t you ashamed of a hand like that?
It\textquotesingle s as soft and white as a woman\textquotesingle s, and
looks as if it never did anything but wear Jouvin\textquotesingle s best
gloves, and pick flowers for ladies. You are not a dandy, thank Heaven!
so I\textquotesingle m glad to see there are no diamonds or big
seal-rings on it, only the little old one Jo gave you so long ago. Dear
soul, I wish she was here to help me!"

"So do I!"

The hand vanished as suddenly as it came, and there was energy enough in
the echo of her wish to suit even Amy. She glanced down at him with a
new thought in her mind; but he was lying with his hat half over his
face, as if for shade, and his mustache hid his mouth. She only saw his
chest rise and fall, with a long breath that might have been a sigh, and
the hand that wore the ring nestled down into the grass, as if to hide
something too precious or too tender to be spoken of. All in a minute
various hints and trifles assumed shape and significance in
Amy\textquotesingle s mind, and told her what her sister never had
confided to her. She remembered that Laurie never spoke voluntarily of
Jo; she recalled the shadow on his face just now, the change in his
character, and the wearing of the little old ring, which was no ornament
to a handsome hand. Girls are quick to read such signs and feel their
eloquence. Amy had fancied that perhaps a love trouble was at the bottom
of the alteration, and now she was sure of it. Her keen eyes filled,
and, when she spoke again, it was in a voice that could be beautifully
soft and kind when she chose to make it so.

"I know I have no right to talk so to you, Laurie; and if you
weren\textquotesingle t the sweetest-tempered fellow in the world,
you\textquotesingle d be very angry with me. But we are all so fond and
proud of you, I couldn\textquotesingle t bear to think they should be
disappointed in you at home as I have been, though, perhaps, they would
understand the change better than I do."

"I think they would," came from under the hat, in a grim tone, quite as
touching as a broken one.

"They ought to have told me, and not let me go blundering and scolding,
when I should have been more kind and patient than ever. I never did
like that Miss Randal, and now I hate her!" said artful Amy, wishing to
be sure of her facts this time.

"Hang Miss Randal!" and Laurie knocked the hat off his face with a look
that left no doubt of his sentiments toward that young lady.

"I beg pardon; I thought---" and there she paused diplomatically.

"No, you didn\textquotesingle t; you knew perfectly well I never cared
for any one but Jo." Laurie said that in his old, impetuous tone, and
turned his face away as he spoke.

"I did think so; but as they never said anything about it, and you came
away, I supposed I was mistaken. And Jo wouldn\textquotesingle t be kind
to you? Why, I was sure she loved you dearly."

"She \emph{was} kind, but not in the right way; and it\textquotesingle s
lucky for her she didn\textquotesingle t love me, if I\textquotesingle m
the good-for-nothing fellow you think me. It\textquotesingle s her
fault, though, and you may tell her so."

The hard, bitter look came back again as he said that, and it troubled
Amy, for she did not know what balm to apply.

"I was wrong, I didn\textquotesingle t know. I\textquotesingle m very
sorry I was so cross, but I can\textquotesingle t help wishing
you\textquotesingle d bear it better, Teddy, dear."

"Don\textquotesingle t, that\textquotesingle s her name for me!" and
Laurie put up his hand with a quick gesture to stop the words spoken in
Jo\textquotesingle s half-kind, half-reproachful tone. "Wait till
you\textquotesingle ve tried it yourself," he added, in a low voice, as
he pulled up the grass by the handful.

"I\textquotesingle d take it manfully, and be respected if I
couldn\textquotesingle t be loved," said Amy, with the decision of one
who knew nothing about it.

Now, Laurie flattered himself that he \emph{had} borne it remarkably
well, making no moan, asking no sympathy, and taking his trouble away to
live it down alone. Amy\textquotesingle s lecture put the matter in a
new light, and for the first time it did look weak and selfish to lose
heart at the first failure, and shut himself up in moody indifference.
He felt as if suddenly shaken out of a pensive dream, and found it
impossible to go to sleep again. Presently he sat up, and asked
slowly,---

"Do you think Jo would despise me as you do?"

"Yes, if she saw you now. She hates lazy people. Why
don\textquotesingle t you do something splendid, and \emph{make} her
love you?"

"I did my best, but it was no use."

"Graduating well, you mean? That was no more than you ought to have
done, for your grandfather\textquotesingle s sake. It would have been
shameful to fail after spending so much time and money, when every one
knew you \emph{could} do well."

"I did fail, say what you will, for Jo wouldn\textquotesingle t love
me," began Laurie, leaning his head on his hand in a despondent
attitude.

"No, you didn\textquotesingle t, and you\textquotesingle ll say so in
the end, for it did you good, and proved that you could do something if
you tried. If you\textquotesingle d only set about another task of some
sort, you\textquotesingle d soon be your hearty, happy self again, and
forget your trouble."

"That\textquotesingle s impossible."

"Try it and see. You needn\textquotesingle t shrug your shoulders, and
think, \textquotesingle Much she knows about such
things.\textquotesingle{} I don\textquotesingle t pretend to be wise,
but I \emph{am} observing, and I see a great deal more than
you\textquotesingle d imagine. I\textquotesingle m interested in other
people\textquotesingle s experiences and inconsistencies; and, though I
can\textquotesingle t explain, I remember and use them for my own
benefit. Love Jo all your days, if you choose, but don\textquotesingle t
let it spoil you, for it\textquotesingle s wicked to throw away so many
good gifts because you can\textquotesingle t have the one you want.
There, I won\textquotesingle t lecture any more, for I know
you\textquotesingle ll wake up and be a man in spite of that hardhearted
girl."

Neither spoke for several minutes. Laurie sat turning the little ring on
his finger, and Amy put the last touches to the hasty sketch she had
been working at while she talked. Presently she put it on his knee,
merely saying,---

"How do you like that?"

He looked and then he smiled, as he could not well help doing, for it
was capitally done,---the long, lazy figure on the grass, with listless
face, half-shut eyes, and one hand holding a cigar, from which came the
little wreath of smoke that encircled the dreamer\textquotesingle s
head.

"How well you draw!" he said, with genuine surprise and pleasure at her
skill, adding, with a half-laugh,---

"Yes, that\textquotesingle s me."

"As you are: this is as you were;" and Amy laid another sketch beside
the one he held.

It was not nearly so well done, but there was a life and spirit in it
which atoned for many faults, and it recalled the past so vividly that a
sudden change swept over the young man\textquotesingle s face as he
looked. Only a rough sketch of Laurie taming a horse; hat and coat were
off, and every line of the active figure, resolute face, and commanding
attitude, was full of energy and meaning. The handsome brute, just
subdued, stood arching his neck under the tightly drawn rein, with one
foot impatiently pawing the ground, and ears pricked up as if listening
for the voice that had mastered him. In the ruffled mane, the
rider\textquotesingle s breezy hair and erect attitude, there was a
suggestion of suddenly arrested motion, of strength, courage, and
youthful buoyancy, that contrasted sharply with the supine grace of the
"\emph{Dolce far niente}" sketch. Laurie said nothing; but, as his eye
went from one to the other, Amy saw him flush up and fold his lips
together as if he read and accepted the little lesson she had given him.
That satisfied her; and, without waiting for him to speak, she said, in
her sprightly way,---

\protect\phantomsection\label{6672479776654687619_37106-h-7.htm.xhtml_b173.png}{}
\pandocbounded{\includegraphics[keepaspectratio]{303483661336987339_b173.png}}

"Don\textquotesingle t you remember the day you played Rarey with Puck,
and we all looked on? Meg and Beth were frightened, but Jo clapped and
pranced, and I sat on the fence and drew you. I found that sketch in my
portfolio the other day, touched it up, and kept it to show you."

"Much obliged. You\textquotesingle ve improved immensely since then, and
I congratulate you. May I venture to suggest in \textquotesingle a
honeymoon Paradise\textquotesingle{} that five o\textquotesingle clock
is the dinner-hour at your hotel?"

Laurie rose as he spoke, returned the pictures with a smile and a bow,
and looked at his watch, as if to remind her that even moral lectures
should have an end. He tried to resume his former easy, indifferent air,
but it \emph{was} an affectation now, for the rousing had been more
efficacious than he would confess. Amy felt the shade of coldness in his
manner, and said to herself,---

"Now I\textquotesingle ve offended him. Well, if it does him good,
I\textquotesingle m glad; if it makes him hate me, I\textquotesingle m
sorry; but it\textquotesingle s true, and I can\textquotesingle t take
back a word of it."

They laughed and chatted all the way home; and little Baptiste, up
behind, thought that monsieur and mademoiselle were in charming spirits.
But both felt ill at ease; the friendly frankness was disturbed, the
sunshine had a shadow over it, and despite their apparent gayety, there
was a secret discontent in the heart of each.

"Shall we see you this evening, \emph{mon frère}?" asked Amy as they
parted at her aunt\textquotesingle s door.

"Unfortunately I have an engagement. \emph{Au revoir, mademoiselle},"
and Laurie bent as if to kiss her hand, in the foreign fashion, which
became him better than many men. Something in his face made Amy say
quickly and warmly,---

"No; be yourself with me, Laurie, and part in the good old way.
I\textquotesingle d rather have a hearty English hand-shake than all the
sentimental salutations in France."

"Good-by, dear," and with these words, uttered in the tone she liked,
Laurie left her, after a hand-shake almost painful in its heartiness.

Next morning, instead of the usual call, Amy received a note which made
her smile at the beginning and sigh at the end:---

\begin{quote}
"My dear Mentor,---

"Please make my adieux to your aunt, and exult within yourself, for
\textquotesingle Lazy Laurence\textquotesingle{} has gone to his
grandpa, like the best of boys. A pleasant winter to you, and may the
gods grant you a blissful honeymoon at Valrosa! I think Fred would be
benefited by a rouser. Tell him so, with my congratulations.

{"Yours gratefully,}

Telemachus."
\end{quote}

"Good boy! I\textquotesingle m glad he\textquotesingle s gone," said
Amy, with an approving smile; the next minute her face fell as she
glanced about the empty room, adding, with an involuntary sigh,---

"Yes, I \emph{am} glad, but how I shall miss him!"

\begin{center}\rule{0.5\linewidth}{0.5pt}\end{center}

\subsection{XL. The Valley of the
Shadow.}\label{6672479776654687619_37106-h-7.htm.xhtml_pgepubid00042}

\protect\phantomsection\label{6672479776654687619_37106-h-7.htm.xhtml_b174.png}{}
\pandocbounded{\includegraphics[keepaspectratio]{303483661336987339_b174.png}}

\protect\phantomsection\label{6672479776654687619_37106-h-7.htm.xhtml_XL}{}\hyperref[6672479776654687619_37106-h-0.htm.xhtml_contents2b]{XL.}

THE VALLEY OF THE SHADOW.

{When} the first bitterness was over, the family accepted the
inevitable, and tried to bear it cheerfully, helping one another by the
increased affection which comes to bind households tenderly together in
times of trouble. They put away their grief, and each did his or her
part toward making that last year a happy one.

The pleasantest room in the house was set apart for Beth, and in it was
gathered everything that she most loved,---flowers, pictures, her piano,
the little work-table, and the beloved pussies. Father\textquotesingle s
best books found their way there, mother\textquotesingle s easy-chair,
Jo\textquotesingle s desk, Amy\textquotesingle s finest sketches; and
every day Meg brought her babies on a loving pilgrimage, to make
sunshine for Aunty Beth. John quietly set apart a little sum, that he
might enjoy the pleasure of keeping the invalid supplied with the fruit
she loved and longed for; old Hannah never wearied of concocting dainty
dishes to tempt a capricious appetite, dropping tears as she worked; and
from across the sea came little gifts and cheerful letters, seeming to
bring breaths of warmth and fragrance from lands that know no winter.

Here, cherished like a household saint in its shrine, sat Beth, tranquil
and busy as ever; for nothing could change the sweet, unselfish nature,
and even while preparing to leave life, she tried to make it happier for
those who should remain behind. The feeble fingers were never idle, and
one of her pleasures was to make little things for the school-children
daily passing to and fro,---to drop a pair of mittens from her window
for a pair of purple hands, a needle-book for some small mother of many
dolls, pen-wipers for young penmen toiling through forests of pot-hooks,
scrap-books for picture-loving eyes, and all manner of pleasant devices,
till the reluctant climbers up the ladder of learning found their way
strewn with flowers, as it were, and came to regard the gentle giver as
a sort of fairy godmother, who sat above there, and showered down gifts
miraculously suited to their tastes and needs. If Beth had wanted any
reward, she found it in the bright little faces always turned up to her
window, with nods and smiles, and the droll little letters which came to
her, full of blots and gratitude.

The first few months were very happy ones, and Beth often used to look
round, and say "How beautiful this is!" as they all sat together in her
sunny room, the babies kicking and crowing on the floor, mother and
sisters working near, and father reading, in his pleasant voice, from
the wise old books which seemed rich in good and comfortable words, as
applicable now as when written centuries ago; a little chapel, where a
paternal priest taught his flock the hard lessons all must learn, trying
to show them that hope can comfort love, and faith make resignation
possible. Simple sermons, that went straight to the souls of those who
listened; for the father\textquotesingle s heart was in the
minister\textquotesingle s religion, and the frequent falter in the
voice gave a double eloquence to the words he spoke or read.

It was well for all that this peaceful time was given them as
preparation for the sad hours to come; for, by and by, Beth said the
needle was "so heavy," and put it down forever; talking wearied her,
faces troubled her, pain claimed her for its own, and her tranquil
spirit was sorrowfully perturbed by the ills that vexed her feeble
flesh. Ah me! such heavy days, such long, long nights, such aching
hearts and imploring prayers, when those who loved her best were forced
to see the thin hands stretched out to them beseechingly, to hear the
bitter cry, "Help me, help me!" and to feel that there was no help. A
sad eclipse of the serene soul, a sharp struggle of the young life with
death; but both were mercifully brief, and then, the natural rebellion
over, the old peace returned more beautiful than ever. With the wreck of
her frail body, Beth\textquotesingle s soul grew strong; and, though she
said little, those about her felt that she was ready, saw that the first
pilgrim called was likewise the fittest, and waited with her on the
shore, trying to see the Shining Ones coming to receive her when she
crossed the river.

Jo never left her for an hour since Beth had said, "I feel stronger when
you are here." She slept on a couch in the room, waking often to renew
the fire, to feed, lift, or wait upon the patient creature who seldom
asked for anything, and "tried not to be a trouble." All day she haunted
the room, jealous of any other nurse, and prouder of being chosen then
than of any honor her life ever brought her. Precious and helpful hours
to Jo, for now her heart received the teaching that it needed; lessons
in patience were so sweetly taught her that she could not fail to learn
them; charity for all, the lovely spirit that can forgive and truly
forget unkindness, the loyalty to duty that makes the hardest easy, and
the sincere faith that fears nothing, but trusts undoubtingly.

Often, when she woke, Jo found Beth reading in her well-worn little
book, heard her singing softly, to beguile the sleepless night, or saw
her lean her face upon her hands, while slow tears dropped through the
transparent fingers; and Jo would lie watching her, with thoughts too
deep for tears, feeling that Beth, in her simple, unselfish way, was
trying to wean herself from the dear old life, and fit herself for the
life to come, by sacred words of comfort, quiet prayers, and the music
she loved so well.

Seeing this did more for Jo than the wisest sermons, the saintliest
hymns, the most fervent prayers that any voice could utter; for, with
eyes made clear by many tears, and a heart softened by the tenderest
sorrow, she recognized the beauty of her sister\textquotesingle s
life,---uneventful, unambitious, yet full of the genuine virtues which
"smell sweet, and blossom in the dust," the self-forgetfulness that
makes the humblest on earth remembered soonest in heaven, the true
success which is possible to all.

One night, when Beth looked among the books upon her table, to find
something to make her forget the mortal weariness that was almost as
hard to bear as pain, as she turned the leaves of her old favorite
Pilgrim\textquotesingle s Progress, she found a little paper, scribbled
over in Jo\textquotesingle s hand. The name caught her eye, and the
blurred look of the lines made her sure that tears had fallen on it.

"Poor Jo! she\textquotesingle s fast asleep, so I won\textquotesingle t
wake her to ask leave; she shows me all her things, and I
don\textquotesingle t think she\textquotesingle ll mind if I look at
this," thought Beth, with a glance at her sister, who lay on the rug,
with the tongs beside her, ready to wake up the minute the log fell
apart.

\hfill\break

"MY BETH.

"Sitting patient in the shadow

Till the blessed light shall come,

A serene and saintly presence

Sanctifies our troubled home.

Earthly joys and hopes and sorrows

Break like ripples on the strand

Of the deep and solemn river

Where her willing feet now stand.

\hfill\break

"O my sister, passing from me,

Out of human care and strife,

Leave me, as a gift, those virtues

Which have beautified your life.

Dear, bequeath me that great patience

Which has power to sustain

A cheerful, uncomplaining spirit

In its prison-house of pain.

\hfill\break

"Give me, for I need it sorely,

Of that courage, wise and sweet,

Which has made the path of duty

Green beneath your willing feet.

Give me that unselfish nature,

That with charity divine

Can pardon wrong for love\textquotesingle s dear sake---

Meek heart, forgive me mine!

\hfill\break

"Thus our parting daily loseth

Something of its bitter pain,

And while learning this hard lesson,

My great loss becomes my gain.

For the touch of grief will render

My wild nature more serene,

Give to life new aspirations,

A new trust in the unseen.

\hfill\break

"Henceforth, safe across the river,

I shall see forevermore

A beloved, household spirit

Waiting for me on the shore.

Hope and faith, born of my sorrow,

Guardian angels shall become,

And the sister gone before me

By their hands shall lead me home."

Blurred and blotted, faulty and feeble, as the lines were, they brought
a look of inexpressible comfort to Beth\textquotesingle s face, for her
one regret had been that she had done so little; and this seemed to
assure her that her life had not been useless, that her death would not
bring the despair she feared. As she sat with the paper folded between
her hands, the charred log fell asunder. Jo started up, revived the
blaze, and crept to the bedside, hoping Beth slept.

"Not asleep, but so happy, dear. See, I found this and read it; I knew
you wouldn\textquotesingle t care. Have I been all that to you, Jo?" she
asked, with wistful, humble earnestness.

"O Beth, so much, so much!" and Jo\textquotesingle s head went down upon
the pillow, beside her sister\textquotesingle s.

"Then I don\textquotesingle t feel as if I\textquotesingle d wasted my
life. I\textquotesingle m not so good as you make me, but I \emph{have}
tried to do right; and now, when it\textquotesingle s too late to begin
even to do better, it\textquotesingle s such a comfort to know that some
one loves me so much, and feels as if I\textquotesingle d helped them."

"More than any one in the world, Beth. I used to think I
couldn\textquotesingle t let you go; but I\textquotesingle m learning to
feel that I don\textquotesingle t lose you; that you\textquotesingle ll
be more to me than ever, and death can\textquotesingle t part us, though
it seems to."

"I know it cannot, and I don\textquotesingle t fear it any longer, for
I\textquotesingle m sure I shall be your Beth still, to love and help
you more than ever. You must take my place, Jo, and be everything to
father and mother when I\textquotesingle m gone. They will turn to you,
don\textquotesingle t fail them; and if it\textquotesingle s hard to
work alone, remember that I don\textquotesingle t forget you, and that
you\textquotesingle ll be happier in doing that than writing splendid
books or seeing all the world; for love is the only thing that we can
carry with us when we go, and it makes the end so easy."

"I\textquotesingle ll try, Beth;" and then and there Jo renounced her
old ambition, pledged herself to a new and better one, acknowledging the
poverty of other desires, and feeling the blessed solace of a belief in
the immortality of love.

So the spring days came and went, the sky grew clearer, the earth
greener, the flowers were up fair and early, and the birds came back in
time to say good-by to Beth, who, like a tired but trustful child, clung
to the hands that had led her all her life, as father and mother guided
her tenderly through the Valley of the Shadow, and gave her up to God.

Seldom, except in books, do the dying utter memorable words, see
visions, or depart with beatified countenances; and those who have sped
many parting souls know that to most the end comes as naturally and
simply as sleep. As Beth had hoped, the "tide went out easily;" and in
the dark hour before the dawn, on the bosom where she had drawn her
first breath, she quietly drew her last, with no farewell but one loving
look, one little sigh.

With tears and prayers and tender hands, mother and sisters made her
ready for the long sleep that pain would never mar again, seeing with
grateful eyes the beautiful serenity that soon replaced the pathetic
patience that had wrung their hearts so long, and feeling, with reverent
joy, that to their darling death was a benignant angel, not a phantom
full of dread.

When morning came, for the first time in many months the fire was out,
Jo\textquotesingle s place was empty, and the room was very still. But a
bird sang blithely on a budding bough, close by, the snow-drops
blossomed freshly at the window, and the spring sunshine streamed in
like a benediction over the placid face upon the pillow,---a face so
full of painless peace that those who loved it best smiled through their
tears, and thanked God that Beth was well at last.

\protect\phantomsection\label{6672479776654687619_37106-h-7.htm.xhtml_b175.png}{}
\pandocbounded{\includegraphics[keepaspectratio]{303483661336987339_b175.png}}

\begin{center}\rule{0.5\linewidth}{0.5pt}\end{center}

\subsection{XLI. Learning to
Forget.}\label{6672479776654687619_37106-h-7.htm.xhtml_pgepubid00043}

\protect\phantomsection\label{6672479776654687619_37106-h-7.htm.xhtml_b176.png}{}
\pandocbounded{\includegraphics[keepaspectratio]{303483661336987339_b176.png}}

\protect\phantomsection\label{6672479776654687619_37106-h-7.htm.xhtml_XLI}{}\hyperref[6672479776654687619_37106-h-0.htm.xhtml_contents2b]{XLI.}

LEARNING TO FORGET.

{Amy\textquotesingle s} lecture did Laurie good, though, of course, he
did not own it till long afterward; men seldom do, for when women are
the advisers, the lords of creation don\textquotesingle t take the
advice till they have persuaded themselves that it is just what they
intended to do; then they act upon it, and, if it succeeds, they give
the weaker vessel half the credit of it; if it fails, they generously
give her the whole. Laurie went back to his grandfather, and was so
dutifully devoted for several weeks that the old gentleman declared the
climate of Nice had improved him wonderfully, and he had better try it
again. There was nothing the young gentleman would have liked better,
but elephants could not have dragged him back after the scolding he had
received; pride forbid, and whenever the longing grew very strong, he
fortified his resolution by repeating the words that had made the
deepest impression, "I despise you;" "Go and do something splendid that
will \emph{make} her love you."

Laurie turned the matter over in his mind so often that he soon brought
himself to confess that he \emph{had} been selfish and lazy; but then
when a man has a great sorrow, he should be indulged in all sorts of
vagaries till he has lived it down. He felt that his blighted affections
were quite dead now; and, though he should never cease to be a faithful
mourner, there was no occasion to wear his weeds ostentatiously. Jo
\emph{wouldn\textquotesingle t} love him, but he might \emph{make} her
respect and admire him by doing something which should prove that a
girl\textquotesingle s "No" had not spoilt his life. He had always meant
to do something, and Amy\textquotesingle s advice was quite unnecessary.
He had only been waiting till the aforesaid blighted affections were
decently interred; that being done, he felt that he was ready to "hide
his stricken heart, and still toil on."

As Goethe, when he had a joy or a grief, put it into a song, so Laurie
resolved to embalm his love-sorrow in music, and compose a Requiem which
should harrow up Jo\textquotesingle s soul and melt the heart of every
hearer. Therefore the next time the old gentleman found him getting
restless and moody, and ordered him off, he went to Vienna, where he had
musical friends, and fell to work with the firm determination to
distinguish himself. But, whether the sorrow was too vast to be embodied
in music, or music too ethereal to uplift a mortal woe, he soon
discovered that the Requiem was beyond him, just at present. It was
evident that his mind was not in working order yet, and his ideas needed
clarifying; for often in the middle of a plaintive strain, he would find
himself humming a dancing tune that vividly recalled the Christmas ball
at Nice, especially the stout Frenchman, and put an effectual stop to
tragic composition for the time being.

Then he tried an Opera, for nothing seemed impossible in the beginning;
but here, again, unforeseen difficulties beset him. He wanted Jo for his
heroine, and called upon his memory to supply him with tender
recollections and romantic visions of his love. But memory turned
traitor; and, as if possessed by the perverse spirit of the girl, would
only recall Jo\textquotesingle s oddities, faults, and freaks, would
only show her in the most unsentimental aspects,---beating mats with her
head tied up in a bandanna, barricading herself with the sofa-pillow, or
throwing cold water over his passion \emph{à la} Gummidge,---and an
irresistible laugh spoilt the pensive picture he was endeavoring to
paint. Jo wouldn\textquotesingle t be put into the Opera at any price,
and he had to give her up with a "Bless that girl, what a torment she
is!" and a clutch at his hair, as became a distracted composer.

When he looked about him for another and a less intractable damsel to
immortalize in melody, memory produced one with the most obliging
readiness. This phantom wore many faces, but it always had golden hair,
was enveloped in a diaphanous cloud, and floated airily before his
mind\textquotesingle s eye in a pleasing chaos of roses, peacocks, white
ponies, and blue ribbons. He did not give the complacent wraith any
name, but he took her for his heroine, and grew quite fond of her, as
well he might; for he gifted her with every gift and grace under the
sun, and escorted her, unscathed, through trials which would have
annihilated any mortal woman.

Thanks to this inspiration, he got on swimmingly for a time, but
gradually the work lost its charm, and he forgot to compose, while he
sat musing, pen in hand, or roamed about the gay city to get new ideas
and refresh his mind, which seemed to be in a somewhat unsettled state
that winter. He did not do much, but he thought a great deal and was
conscious of a change of some sort going on in spite of himself.
"It\textquotesingle s genius simmering, perhaps. I\textquotesingle ll
let it simmer, and see what comes of it," he said, with a secret
suspicion, all the while, that it wasn\textquotesingle t genius, but
something far more common. Whatever it was, it simmered to some purpose,
for he grew more and more discontented with his desultory life, began to
long for some real and earnest work to go at, soul and body, and finally
came to the wise conclusion that every one who loved music was not a
composer. Returning from one of Mozart\textquotesingle s grand operas,
splendidly performed at the Royal Theatre, he looked over his own,
played a few of the best parts, sat staring up at the busts of
Mendelssohn, Beethoven, and Bach, who stared benignly back again; then
suddenly he tore up his music-sheets, one by one, and, as the last
fluttered out of his hand, he said soberly to himself,---

"She is right! Talent isn\textquotesingle t genius, and you
can\textquotesingle t make it so. That music has taken the vanity out of
me as Rome took it out of her, and I won\textquotesingle t be a humbug
any longer. Now what shall I do?"

That seemed a hard question to answer, and Laurie began to wish he had
to work for his daily bread. Now, if ever, occurred an eligible
opportunity for "going to the devil," as he once forcibly expressed it,
for he had plenty of money and nothing to do, and Satan is proverbially
fond of providing employment for full and idle hands. The poor fellow
had temptations enough from without and from within, but he withstood
them pretty well; for, much as he valued liberty, he valued good faith
and confidence more, so his promise to his grandfather, and his desire
to be able to look honestly into the eyes of the women who loved him,
and say "All\textquotesingle s well," kept him safe and steady.

Very likely some Mrs. Grundy will observe, "I don\textquotesingle t
believe it; boys will be boys, young men must sow their wild oats, and
women must not expect miracles." I dare say \emph{you}
don\textquotesingle t, Mrs. Grundy, but it\textquotesingle s true
nevertheless. Women work a good many miracles, and I have a persuasion
that they may perform even that of raising the standard of manhood by
refusing to echo such sayings. Let the boys be boys, the longer the
better, and let the young men sow their wild oats if they must; but
mothers, sisters, and friends may help to make the crop a small one, and
keep many tares from spoiling the harvest, by believing, and showing
that they believe, in the possibility of loyalty to the virtues which
make men manliest in good women\textquotesingle s eyes. If it \emph{is}
a feminine delusion, leave us to enjoy it while we may, for without it
half the beauty and the romance of life is lost, and sorrowful
forebodings would embitter all our hopes of the brave, tender-hearted
little lads, who still love their mothers better than themselves, and
are not ashamed to own it.

Laurie thought that the task of forgetting his love for Jo would absorb
all his powers for years; but, to his great surprise, he discovered it
grew easier every day. He refused to believe it at first, got angry with
himself, and couldn\textquotesingle t understand it; but these hearts of
ours are curious and contrary things, and time and nature work their
will in spite of us. Laurie\textquotesingle s heart
\emph{wouldn\textquotesingle t} ache; the wound persisted in healing
with a rapidity that astonished him, and, instead of trying to forget,
he found himself trying to remember. He had not foreseen this turn of
affairs, and was not prepared for it. He was disgusted with himself,
surprised at his own fickleness, and full of a queer mixture of
disappointment and relief that he could recover from such a tremendous
blow so soon. He carefully stirred up the embers of his lost love, but
they refused to burst into a blaze: there was only a comfortable glow
that warmed and did him good without putting him into a fever, and he
was reluctantly obliged to confess that the boyish passion was slowly
subsiding into a more tranquil sentiment, very tender, a little sad and
resentful still, but that was sure to pass away in time, leaving a
brotherly affection which would last unbroken to the end.

As the word "brotherly" passed through his mind in one of these
reveries, he smiled, and glanced up at the picture of Mozart that was
before him:---

"Well, he was a great man; and when he couldn\textquotesingle t have one
sister he took the other, and was happy."

Laurie did not utter the words, but he thought them; and the next
instant kissed the little old ring, saying to himself,---

"No, I won\textquotesingle t! I haven\textquotesingle t forgotten, I
never can. I\textquotesingle ll try again, and if that fails, why,
then---"

Leaving his sentence unfinished, he seized pen and paper and wrote to
Jo, telling her that he could not settle to anything while there was the
least hope of her changing her mind. Couldn\textquotesingle t she,
wouldn\textquotesingle t she, and let him come home and be happy? While
waiting for an answer he did nothing, but he did it energetically, for
he was in a fever of impatience. It came at last, and settled his mind
effectually on one point, for Jo decidedly couldn\textquotesingle t and
wouldn\textquotesingle t. She was wrapped up in Beth, and never wished
to hear the word "love" again. Then she begged him to be happy with
somebody else, but always to keep a little corner of his heart for his
loving sister Jo. In a postscript she desired him not to tell Amy that
Beth was worse; she was coming home in the spring, and there was no need
of saddening the remainder of her stay. That would be time enough,
please God, but Laurie must write to her often, and not let her feel
lonely, homesick, or anxious.

"So I will, at once. Poor little girl; it will be a sad going home for
her, I\textquotesingle m afraid;" and Laurie opened his desk, as if
writing to Amy had been the proper conclusion of the sentence left
unfinished some weeks before.

But he did not write the letter that day; for, as he rummaged out his
best paper, he came across something which changed his purpose. Tumbling
about in one part of the desk, among bills, passports, and business
documents of various kinds, were several of Jo\textquotesingle s
letters, and in another compartment were three notes from Amy, carefully
tied up with one of her blue ribbons, and sweetly suggestive of the
little dead roses put away inside. With a half-repentant, half-amused
expression, Laurie gathered up all Jo\textquotesingle s letters,
smoothed, folded, and put them neatly into a small drawer of the desk,
stood a minute turning the ring thoughtfully on his finger, then slowly
drew it off, laid it with the letters, locked the drawer, and went out
to hear High Mass at Saint Stefan\textquotesingle s, feeling as if there
had been a funeral; and, though not overwhelmed with affliction, this
seemed a more proper way to spend the rest of the day than in writing
letters to charming young ladies.

\protect\phantomsection\label{6672479776654687619_37106-h-7.htm.xhtml_b177.png}{}
\pandocbounded{\includegraphics[keepaspectratio]{303483661336987339_b177.png}}

The letter went very soon, however, and was promptly answered, for Amy
\emph{was} homesick, and confessed it in the most delightfully confiding
manner. The correspondence flourished famously, and letters flew to and
fro, with unfailing regularity, all through the early spring. Laurie
sold his busts, made allumettes of his opera, and went back to Paris,
hoping somebody would arrive before long. He wanted desperately to go to
Nice, but would not till he was asked; and Amy would not ask him, for
just then she was having little experiences of her own, which made her
rather wish to avoid the quizzical eyes of "our boy."

Fred Vaughn had returned, and put the question to which she had once
decided to answer "Yes, thank you;" but now she said, "No, thank you,"
kindly but steadily; for, when the time came, her courage failed her,
and she found that something more than money and position was needed to
satisfy the new longing that filled her heart so full of tender hopes
and fears. The words, "Fred is a good fellow, but not at all the man I
fancied you would ever like," and Laurie\textquotesingle s face when he
uttered them, kept returning to her as pertinaciously as her own did
when she said in look, if not in words, "I shall marry for money." It
troubled her to remember that now, she wished she could take it back, it
sounded so unwomanly. She didn\textquotesingle t want Laurie to think
her a heartless, worldly creature; she didn\textquotesingle t care to be
a queen of society now half so much as she did to be a lovable woman;
she was so glad he didn\textquotesingle t hate her for the dreadful
things she said, but took them so beautifully, and was kinder than ever.
His letters were such a comfort, for the home letters were very
irregular, and were not half so satisfactory as his when they did come.
It was not only a pleasure, but a duty to answer them, for the poor
fellow was forlorn, and needed petting, since Jo persisted in being
stony-hearted. She ought to have made an effort, and tried to love him;
it couldn\textquotesingle t be very hard, many people would be proud and
glad to have such a dear boy care for them; but Jo never would act like
other girls, so there was nothing to do but be very kind, and treat him
like a brother.

If all brothers were treated as well as Laurie was at this period, they
would be a much happier race of beings than they are. Amy never lectured
now; she asked his opinion on all subjects; she was interested in
everything he did, made charming little presents for him, and sent him
two letters a week, full of lively gossip, sisterly confidences, and
captivating sketches of the lovely scenes about her. As few brothers are
complimented by having their letters carried about in their
sisters\textquotesingle{} pockets, read and reread diligently, cried
over when short, kissed when long, and treasured carefully, we will not
hint that Amy did any of these fond and foolish things. But she
certainly did grow a little pale and pensive that spring, lost much of
her relish for society, and went out sketching alone a good deal. She
never had much to show when she came home, but was studying nature, I
dare say, while she sat for hours, with her hands folded, on the terrace
at Valrosa, or absently sketched any fancy that occurred to her,---a
stalwart knight carved on a tomb, a young man asleep in the grass, with
his hat over his eyes, or a curly-haired girl in gorgeous array,
promenading down a ball-room on the arm of a tall gentleman, both faces
being left a blur according to the last fashion in art, which was safe,
but not altogether satisfactory.

Her aunt thought that she regretted her answer to Fred; and, finding
denials useless and explanations impossible, Amy left her to think what
she liked, taking care that Laurie should know that Fred had gone to
Egypt. That was all, but he understood it, and looked relieved, as he
said to himself, with a venerable air,---

"I was sure she would think better of it. Poor old fellow!
I\textquotesingle ve been through it all, and I can sympathize."

With that he heaved a great sigh, and then, as if he had discharged his
duty to the past, put his feet up on the sofa, and enjoyed
Amy\textquotesingle s letter luxuriously.

While these changes were going on abroad, trouble had come at home; but
the letter telling that Beth was failing never reached Amy, and when the
next found her, the grass was green above her sister. The sad news met
her at Vevay, for the heat had driven them from Nice in May, and they
had travelled slowly to Switzerland, by way of Genoa and the Italian
lakes. She bore it very well, and quietly submitted to the family decree
that she should not shorten her visit, for, since it was too late to say
good-by to Beth, she had better stay, and let absence soften her sorrow.
But her heart was very heavy; she longed to be at home, and every day
looked wistfully across the lake, waiting for Laurie to come and comfort
her.

He did come very soon; for the same mail brought letters to them both,
but he was in Germany, and it took some days to reach him. The moment he
read it, he packed his knapsack, bade adieu to his fellow-pedestrians,
and was off to keep his promise, with a heart full of joy and sorrow,
hope and suspense.

He knew Vevay well; and as soon as the boat touched the little quay, he
hurried along the shore to La Tour, where the Carrols were living
\emph{en pension}. The \emph{garçon} was in despair that the whole
family had gone to take a promenade on the lake; but no, the blond
mademoiselle might be in the chateau garden. If monsieur would give
himself the pain of sitting down, a flash of time should present her.
But monsieur could not wait even "a flash of time," and, in the middle
of the speech, departed to find mademoiselle himself.

A pleasant old garden on the borders of the lovely lake, with chestnuts
rustling overhead, ivy climbing everywhere, and the black shadow of the
tower falling far across the sunny water. At one corner of the wide, low
wall was a seat, and here Amy often came to read or work, or console
herself with the beauty all about her. She was sitting here that day,
leaning her head on her hand, with a homesick heart and heavy eyes,
thinking of Beth, and wondering why Laurie did not come. She did not
hear him cross the court-yard beyond, nor see him pause in the archway
that led from the subterranean path into the garden. He stood a minute,
looking at her with new eyes, seeing what no one had ever seen
before,---the tender side of Amy\textquotesingle s character. Everything
about her mutely suggested love and sorrow,---the blotted letters in her
lap, the black ribbon that tied up her hair, the womanly pain and
patience in her face; even the little ebony cross at her throat seemed
pathetic to Laurie, for he had given it to her, and she wore it as her
only ornament. If he had any doubts about the reception she would give
him, they were set at rest the minute she looked up and saw him; for,
dropping everything, she ran to him, exclaiming, in a tone of
unmistakable love and longing,---

"O Laurie, Laurie, I knew you\textquotesingle d come to me!"

\protect\phantomsection\label{6672479776654687619_37106-h-7.htm.xhtml_b178.png}{}
\pandocbounded{\includegraphics[keepaspectratio]{303483661336987339_b178.png}}

I think everything was said and settled then; for, as they stood
together quite silent for a moment, with the dark head bent down
protectingly over the light one, Amy felt that no one could comfort and
sustain her so well as Laurie, and Laurie decided that Amy was the only
woman in the world who could fill Jo\textquotesingle s place, and make
him happy. He did not tell her so; but she was not disappointed, for
both felt the truth, were satisfied, and gladly left the rest to
silence.

In a minute Amy went back to her place; and, while she dried her tears,
Laurie gathered up the scattered papers, finding in the sight of sundry
well-worn letters and suggestive sketches good omens for the future. As
he sat down beside her, Amy felt shy again, and turned rosy red at the
recollection of her impulsive greeting.

"I couldn\textquotesingle t help it; I felt so lonely and sad, and was
so very glad to see you. It was such a surprise to look up and find you,
just as I was beginning to fear you wouldn\textquotesingle t come," she
said, trying in vain to speak quite naturally.

"I came the minute I heard. I wish I could say something to comfort you
for the loss of dear little Beth; but I can only feel, and---" He could
not get any further, for he, too, turned bashful all of a sudden, and
did not quite know what to say. He longed to lay Amy\textquotesingle s
head down on his shoulder, and tell her to have a good cry, but he did
not dare; so took her hand instead, and gave it a sympathetic squeeze
that was better than words.

"You needn\textquotesingle t say anything; this comforts me," \ul{she
said softly.} "Beth is well and happy, and I mustn\textquotesingle t
wish her back; but I dread the going home, much as I long to see them
all. We won\textquotesingle t talk about it now, for it makes me cry,
and I want to enjoy you while you stay. You needn\textquotesingle t go
right back, need you?"

"Not if you want me, dear."

"I do, so much. Aunt and Flo are very kind; but you seem like one of the
family, and it would be so comfortable to have you for a little while."

Amy spoke and looked so like a homesick child, whose heart was full,
that Laurie forgot his bashfulness all at once, and gave her just what
she wanted,---the petting she was used to and the cheerful conversation
she needed.

"Poor little soul, you look as if you\textquotesingle d grieved yourself
half-sick! I\textquotesingle m going to take care of you, so
don\textquotesingle t cry any more, but come and walk about with me; the
wind is too chilly for you to sit still," he said, in the
half-caressing, half-commanding way that Amy liked, as he tied on her
hat, drew her arm through his, and began to pace up and down the sunny
walk, under the new-leaved chestnuts. He felt more at ease upon his
legs; and Amy found it very pleasant to have a strong arm to lean upon,
a familiar face to smile at her, and a kind voice to talk delightfully
for her alone.

The quaint old garden had sheltered many pairs of lovers, and seemed
expressly made for them, so sunny and secluded was it, with nothing but
the tower to overlook them, and the wide lake to carry away the echo of
their words, as it rippled by below. For an hour this new pair walked
and talked, or rested on the wall, enjoying the sweet influences which
gave such a charm to time and place; and when an unromantic dinner-bell
warned them away, Amy felt as if she left her burden of loneliness and
sorrow behind her in the chateau garden.

The moment Mrs. Carrol saw the girl\textquotesingle s altered face, she
was illuminated with a new idea, and exclaimed to herself, "Now I
understand it all,---the child has been pining for young Laurence. Bless
my heart, I never thought of such a thing!"

With praiseworthy discretion, the good lady said nothing, and betrayed
no sign of enlightenment; but cordially urged Laurie to stay, and begged
Amy to enjoy his society, for it would do her more good than so much
solitude. Amy was a model of docility; and, as her aunt was a good deal
occupied with Flo, she was left to entertain her friend, and did it with
more than her usual success.

At Nice, Laurie had lounged and Amy had scolded; at Vevay, Laurie was
never idle, but always walking, riding, boating, or studying, in the
most energetic manner, while Amy admired everything he did, and followed
his example as far and as fast as she could. He said the change was
owing to the climate, and she did not contradict him, being glad of a
like excuse for her own recovered health and spirits.

The invigorating air did them both good, and much exercise worked
wholesome changes in minds as well as bodies. They seemed to get clearer
views of life and duty up there among the everlasting hills; the fresh
winds blew away desponding doubts, delusive fancies, and moody mists;
the warm spring sunshine brought out all sorts of aspiring ideas, tender
hopes, and happy thoughts; the lake seemed to wash away the troubles of
the past, and the grand old mountains to look benignly down upon them,
saying, "Little children, love one another."

In spite of the new sorrow, it was a very happy time, so happy that
Laurie could not bear to disturb it by a word. It took him a little
while to recover from his surprise at the rapid cure of his first, and,
as he had firmly believed, his last and only love. He consoled himself
for the seeming disloyalty by the thought that Jo\textquotesingle s
sister was almost the same as Jo\textquotesingle s self, and the
conviction that it would have been impossible to love any other woman
but Amy so soon and so well. His first wooing had been of the
tempestuous order, and he looked back upon it as if through a long vista
of years, with a feeling of compassion blended with regret. He was not
ashamed of it, but put it away as one of the bitter-sweet experiences of
his life, for which he could be grateful when the pain was over. His
second wooing he resolved should be as calm and simple as possible;
there was no need of having a scene, hardly any need of telling Amy that
he loved her; she knew it without words, and had given him his answer
long ago. It all came about so naturally that no one could complain, and
he knew that everybody would be pleased, even Jo. But when our first
little passion has been crushed, we are apt to be wary and slow in
making a second trial; so Laurie let the days pass, enjoying every hour,
and leaving to chance the utterance of the word that would put an end to
the first and sweetest part of his new romance.

He had rather imagined that the \emph{dénouement} would take place in
the chateau garden by moonlight, and in the most graceful and decorous
manner; but it turned out exactly the reverse, for the matter was
settled on the lake, at noonday, in a few blunt words. They had been
floating about all the morning, from gloomy St. Gingolf to sunny
Montreux, with the Alps of Savoy on one side, Mont St. Bernard and the
Dent du Midi on the other, pretty Vevay in the valley, and Lausanne upon
the hill beyond, a cloudless blue sky overhead, and the bluer lake
below, dotted with the picturesque boats that look like white-winged
gulls.

They had been talking of Bonnivard, as they glided past Chillon, and of
Rousseau, as they looked up at Clarens, where he wrote his "Héloise."
Neither had read it, but they knew it was a love-story, and each
privately wondered if it was half as interesting as their own. Amy had
been dabbling her hand in the water during the little pause that fell
between them, and, when she looked up, Laurie was leaning on his oars,
with an expression in his eyes that made her say hastily, merely for the
sake of saying something,---

"You must be tired; rest a little, and let me row; it will do me good;
for, since you came, I have been altogether lazy and luxurious."

"I\textquotesingle m not tired; but you may take an oar, if you like.
There\textquotesingle s room enough, though I have to sit nearly in the
middle, else the boat won\textquotesingle t trim," returned Laurie, as
if he rather liked the arrangement.

Feeling that she had not mended matters much, Amy took the offered third
of a seat, shook her hair over her face, and accepted an oar. She rowed
as well as she did many other things; and, though she used both hands,
and Laurie but one, the oars kept time, and the boat went smoothly
through the water.

\protect\phantomsection\label{6672479776654687619_37106-h-7.htm.xhtml_b179.png}{}
\pandocbounded{\includegraphics[keepaspectratio]{303483661336987339_b179.png}}

"How well we pull together, don\textquotesingle t we?" said Amy, who
objected to silence just then.

"So well that I wish we might always pull in the same boat. Will you,
Amy?" very tenderly.

"Yes, Laurie," very low.

Then they both stopped rowing, and unconsciously added a pretty little
\emph{tableau} of human love and happiness to the dissolving views
reflected in the lake.

\begin{center}\rule{0.5\linewidth}{0.5pt}\end{center}

\subsection{XLII. All
Alone.}\label{6672479776654687619_37106-h-7.htm.xhtml_pgepubid00044}

\protect\phantomsection\label{6672479776654687619_37106-h-7.htm.xhtml_XLII}{}\hyperref[6672479776654687619_37106-h-0.htm.xhtml_contents2b]{XLII.}

ALL ALONE.

{It} was easy to promise self-abnegation when self was wrapped up in
another, and heart and soul were purified by a sweet example; but when
the helpful voice was silent, the daily lesson over, the beloved
presence gone, and nothing remained but loneliness and grief, then Jo
found her promise very hard to keep. How could she "comfort father and
mother," when her own heart ached with a ceaseless longing for her
sister; how could she "make the house cheerful," when all its light and
warmth and beauty seemed to have deserted it when Beth left the old home
for the new; and where in all the world could she "find some useful,
happy work to do," that would take the place of the loving service which
had been its own reward? She tried in a blind, hopeless way to do her
duty, secretly rebelling against it all the while, for it seemed unjust
that her few joys should be lessened, her burdens made heavier, and life
get harder and harder as she toiled along. Some people seemed to get all
sunshine, and some all shadow; it was not fair, for she tried more than
Amy to be good, but never got any reward, only disappointment, trouble,
and hard work.

Poor Jo, these were dark days to her, for something like despair came
over her when she thought of spending all her life in that quiet house,
devoted to humdrum cares, a few small pleasures, and the duty that never
seemed to grow any easier. "I can\textquotesingle t do it. I
wasn\textquotesingle t meant for a life like this, and I know I shall
break away and do something desperate if somebody don\textquotesingle t
come and help me," she said to herself, when her first efforts failed,
and she fell into the moody, miserable state of mind which often comes
when strong wills have to yield to the inevitable.

But some one did come and help her, though Jo did not recognize her good
angels at once, because they wore familiar shapes, and used the simple
spells best fitted to poor humanity. Often she started up at night,
thinking Beth called her; and when the sight of the little empty bed
made her cry with the bitter cry of an unsubmissive sorrow, "O Beth,
come back! come back!" she did not stretch out her yearning arms in
vain; for, as quick to hear her sobbing as she had been to hear her
sister\textquotesingle s faintest whisper, her mother came to comfort
her, not with words only, but the patient tenderness that soothes by a
touch, tears that were mute reminders of a greater grief than
Jo\textquotesingle s, and broken whispers, more eloquent than prayers,
because hopeful resignation went hand-in-hand with natural sorrow.
Sacred moments, when heart talked to heart in the silence of the night,
turning affliction to a blessing, which chastened grief and strengthened
love. Feeling this, Jo\textquotesingle s burden seemed easier to bear,
duty grew sweeter, and life looked more endurable, seen from the safe
shelter of her mother\textquotesingle s arms.

When aching heart was a little comforted, troubled mind likewise found
help; for one day she went to the study, and, leaning over the good gray
head lifted to welcome her with a tranquil smile, she said, very
humbly,---

"Father, talk to me as you did to Beth. I need it more than she did, for
I\textquotesingle m all wrong."

"My dear, nothing can comfort me like this," he answered, with a falter
in his voice, and both arms round her, as if he, too, needed help, and
did not fear to ask it.

\protect\phantomsection\label{6672479776654687619_37106-h-7.htm.xhtml_b180.png}{}
\pandocbounded{\includegraphics[keepaspectratio]{303483661336987339_b180.png}}

Then, sitting in Beth\textquotesingle s little chair close beside him,
Jo told her troubles,---the resentful sorrow for her loss, the fruitless
efforts that discouraged her, the want of faith that made life look so
dark, and all the sad bewilderment which we call despair. She gave him
entire confidence, he gave her the help she needed, and both found
consolation in the act; for the time had come when they could talk
together not only as father and daughter, but as man and woman, able and
glad to serve each other with mutual sympathy as well as mutual love.
Happy, thoughtful times there in the old study which Jo called "the
church of one member," and from which she came with fresh courage,
recovered cheerfulness, and a more submissive spirit; for the parents
who had taught one child to meet death without fear, were trying now to
teach another to accept life without despondency or distrust, and to use
its beautiful opportunities with gratitude and power.

Other helps had Jo,---humble, wholesome duties and delights that would
not be denied their part in serving her, and which she slowly learned to
see and value. Brooms and dishcloths never could be as distasteful as
they once had been, for Beth had presided over both; and something of
her housewifely spirit seemed to linger round the little mop and the old
brush, that was never thrown away. As she used them, Jo found herself
humming the songs Beth used to hum, imitating Beth\textquotesingle s
orderly ways, and giving the little touches here and there that kept
everything fresh and cosey, which was the first step toward making home
happy, though she didn\textquotesingle t know it, till Hannah said with
an approving squeeze of the hand,---

"You thoughtful creter, you\textquotesingle re determined we
sha\textquotesingle n\textquotesingle t miss that dear lamb ef you can
help it. We don\textquotesingle t say much, but we see it, and the Lord
will bless you for\textquotesingle t, see ef He don\textquotesingle t."

As they sat sewing together, Jo discovered how much improved her sister
Meg was; how well she could talk, how much she knew about good, womanly
impulses, thoughts, and feelings, how happy she was in husband and
children, and how much they were all doing for each other.

"Marriage is an excellent thing, after all. I wonder if I should blossom
out half as well as you have, if I tried it?" said Jo, as she
constructed a kite for Demi, in the topsy-turvy nursery.

"It\textquotesingle s just what you need to bring out the tender,
womanly half of your nature, Jo. You are like a chestnut-burr, prickly
outside, but silky-soft within, and a sweet kernel, if one can only get
at it. Love will make you show your heart some day, and then the rough
burr will fall off."

"Frost opens chestnut-burrs, ma\textquotesingle am, and it takes a good
shake to bring them down. Boys go nutting, and I don\textquotesingle t
care to be bagged by them," returned Jo, pasting away at the kite which
no wind that blows would ever carry up, for Daisy had tied herself on as
a bob.

Meg laughed, for she was glad to see a glimmer of Jo\textquotesingle s
old spirit, but she felt it her duty to enforce her opinion by every
argument in her power; and the sisterly chats were not wasted,
especially as two of Meg\textquotesingle s most effective arguments were
the babies, whom Jo loved tenderly. Grief is the best opener for some
hearts, and Jo\textquotesingle s was nearly ready for the bag: a little
more sunshine to ripen the nut, then, not a boy\textquotesingle s
impatient shake, but a man\textquotesingle s hand reached up to pick it
gently from the burr, and find the kernel sound and sweet. If she had
suspected this, she would have shut up tight, and been more prickly than
ever; fortunately she wasn\textquotesingle t thinking about herself, so,
when the time came, down she dropped.

Now, if she had been the heroine of a moral story-book, she ought at
this period of her life to have become quite saintly, renounced the
world, and gone about doing good in a mortified bonnet, with tracts in
her pocket. But, you see, Jo wasn\textquotesingle t a heroine; she was
only a struggling human girl, like hundreds of others, and she just
acted out her nature, being sad, cross, listless, or energetic, as the
mood suggested. It\textquotesingle s highly virtuous to say
we\textquotesingle ll be good, but we can\textquotesingle t do it all at
once, and it takes a long pull, a strong pull, and a pull all together,
before some of us even get our feet set in the right way. Jo had got so
far, she was learning to do her duty, and to feel unhappy if she did
not; but to do it cheerfully---ah, that was another thing! She had often
said she wanted to do something splendid, no matter how hard; and now
she had her wish, for what could be more beautiful than to devote her
life to father and mother, trying to make home as happy to them as they
had to her? And, if difficulties were necessary to increase the splendor
of the effort, what could be harder for a restless, ambitious girl than
to give up her own hopes, plans, and desires, and cheerfully live for
others?

Providence had taken her at her word; here was the task, not what she
had expected, but better, because self had no part in it: now, could she
do it? She decided that she would try; and, in her first attempt, she
found the helps I have suggested. Still another was given her, and she
took it, not as a reward, but as a comfort, as Christian took the
refreshment afforded by the little arbor where he rested, as he climbed
the hill called Difficulty.

"Why don\textquotesingle t you write? That always used to make you
happy," said her mother, once, when the desponding fit overshadowed Jo.

"I\textquotesingle ve no heart to write, and if I had, nobody cares for
my things."

"We do; write something for us, and never mind the rest of the world.
Try it, dear; I\textquotesingle m sure it would do you good, and please
us very much."

"Don\textquotesingle t believe I can;" but Jo got out her desk, and
began to overhaul her half-finished manuscripts.

An hour afterward her mother peeped in, and there she was, scratching
away, with her black pinafore on, and an absorbed expression, which
caused Mrs. March to smile, and slip away, well pleased with the success
of her suggestion. Jo never knew how it happened, but something got into
that story that went straight to the hearts of those who read it; for,
when her family had laughed and cried over it, her father sent it, much
against her will, to one of the popular magazines, and, to her utter
surprise, it was not only paid for, but others requested. Letters from
several persons, whose praise was honor, followed the appearance of the
little story, newspapers copied it, and strangers as well as friends
admired it. For a small thing it was a great success; and Jo was more
astonished than when her novel was commended and condemned all at once.

"I don\textquotesingle t understand it. What \emph{can} there be in a
simple little story like that, to make people praise it so?" she said,
quite bewildered.

"There is truth in it, Jo, that\textquotesingle s the secret; humor and
pathos make it alive, and you have found your style at last. You wrote
with no thought of fame or money, and put your heart into it, my
daughter; you have had the bitter, now comes the sweet. Do your best,
and grow as happy as we are in your success."

"If there \emph{is} anything good or true in what I write, it
isn\textquotesingle t mine; I owe it all to you and mother and to Beth,"
said Jo, more touched by her father\textquotesingle s words than by any
amount of praise from the world.

So, taught by love and sorrow, Jo wrote her little stories, and sent
them away to make friends for themselves and her, finding it a very
charitable world to such humble wanderers; for they were kindly
welcomed, and sent home comfortable tokens to their mother, like dutiful
children whom good fortune overtakes.

When Amy and Laurie wrote of their engagement, Mrs. March feared that Jo
would find it difficult to rejoice over it, but her fears were soon set
at rest; for, though Jo looked grave at first, she took it very quietly,
and was full of hopes and plans for "the children" before she read the
letter twice. It was a sort of written duet, wherein each glorified the
other in lover-like fashion, very pleasant to read and satisfactory to
think of, for no one had any objection to make.

"You like it, mother?" said Jo, as they laid down the closely written
sheets, and looked at one another.

"Yes, I hoped it would be so, ever since Amy wrote that she had refused
Fred. I felt sure then that something better than what you call the
\textquotesingle mercenary spirit\textquotesingle{} had come over her,
and a hint here and there in her letters made me suspect that love and
Laurie would win the day."

"How sharp you are, Marmee, and how silent! You never said a word to
me."

"Mothers have need of sharp eyes and discreet tongues when they have
girls to manage. I was half afraid to put the idea into your head, lest
you should write and congratulate them before the thing was settled."

"I\textquotesingle m not the scatter-brain I was; you may trust me,
I\textquotesingle m sober and sensible enough for any
one\textquotesingle s \emph{confidante} now."

"So you are, dear, and I should have made you mine, only I fancied it
might pain you to learn that your Teddy loved any one else."

"Now, mother, did you really think I could be so silly and selfish,
after I\textquotesingle d refused his love, when it was freshest, if not
best?"

"I knew you were sincere then, Jo, but lately I have thought that if he
came back, and asked again, you might, perhaps, feel like giving another
answer. Forgive me, dear, I can\textquotesingle t help seeing that you
are very lonely, and sometimes there is a hungry look in your eyes that
goes to my heart; so I fancied that your boy might fill the empty place
if he tried now."

"No, mother, it is better as it is, and I\textquotesingle m glad Amy has
learned to love him. But you are right in one thing: I \emph{am} lonely,
and perhaps if Teddy had tried again, I might have said
\textquotesingle Yes,\textquotesingle{} not because I love him any more,
but because I care more to be loved than when he went away."

"I\textquotesingle m glad of that, Jo, for it shows that you are getting
on. There are plenty to love you, so try to be satisfied with father and
mother, sisters and brothers, friends and babies, till the best lover of
all comes to give you your reward."

"Mothers are the \emph{best} lovers in the world; but I
don\textquotesingle t mind whispering to Marmee that I\textquotesingle d
like to try all kinds. It\textquotesingle s very curious, but the more I
try to satisfy myself with all sorts of natural affections, the more I
seem to want. I\textquotesingle d no idea hearts could take in so many;
mine is so elastic, it never seems full now, and I used to be quite
contented with my family. I don\textquotesingle t understand it."

"I do;" and Mrs. March smiled her wise smile, as Jo turned back the
leaves to read what Amy said of Laurie.

"It is so beautiful to be loved as Laurie loves me; he
isn\textquotesingle t sentimental, doesn\textquotesingle t say much
about it, but I see and feel it in all he says and does, and it makes me
so happy and so humble that I don\textquotesingle t seem to be the same
girl I was. I never knew how good and generous and tender he was till
now, for he lets me read his heart, and I find it full of noble impulses
and hopes and purposes, and am so proud to know it\textquotesingle s
mine. He says he feels as if he \textquotesingle could make a prosperous
voyage now with me aboard as mate, and lots of love for
ballast.\textquotesingle{} I pray he may, and try to be all he believes
me, for I love my gallant captain with all my heart and soul and might,
and never will desert him, while God lets us be together. O mother, I
never knew how much like heaven this world could be, when two people
love and live for one another!"

"And that\textquotesingle s our cool, reserved, and worldly Amy! Truly,
love does work miracles. How very, very happy they must be!" And Jo laid
the rustling sheets together with a careful hand, as one might shut the
covers of a lovely romance, which holds the reader fast till the end
comes, and he finds himself alone in the work-a-day world again.

By and by Jo roamed away upstairs, for it was rainy, and she could not
walk. A restless spirit possessed her, and the old feeling came again,
not bitter as it once was, but a sorrowfully patient wonder why one
sister should have all she asked, the other nothing. It was not true;
she knew that, and tried to put it away, but the natural craving for
affection was strong, and Amy\textquotesingle s happiness woke the
hungry longing for some one to "love with heart and soul, and cling to
while God let them be together."

Up in the garret, where Jo\textquotesingle s unquiet wanderings ended,
stood four little wooden chests in a row, each marked with its
owner\textquotesingle s name, and each filled with relics of the
childhood and girlhood ended now for all. Jo glanced into them, and when
she came to her own, leaned her chin on the edge, and stared absently at
the chaotic collection, till a bundle of old exercise-books caught her
eye. She drew them out, turned them over, and re-lived that pleasant
winter at kind Mrs. Kirke\textquotesingle s. She had smiled at first,
then she looked thoughtful, next sad, and when she came to a little
message written in the Professor\textquotesingle s hand, her lips began
to tremble, the books slid out of her lap, and she sat looking at the
friendly words, as if they took a new meaning, and touched a tender spot
in her heart.

"Wait for me, my friend. I may be a little late, but I shall surely
come."

"Oh, if he only would! So kind, so good, so patient with me always; my
dear old Fritz, I didn\textquotesingle t value him half enough when I
had him, but now how I should love to see him, for every one seems going
away from me, and I\textquotesingle m all alone."

And holding the little paper fast, as if it were a promise yet to be
fulfilled, Jo laid her head down on a comfortable rag-bag, and cried, as
if in opposition to the rain pattering on the roof.

\protect\phantomsection\label{6672479776654687619_37106-h-7.htm.xhtml_b181.png}{}
\pandocbounded{\includegraphics[keepaspectratio]{303483661336987339_b181.png}}

Was it all self-pity, loneliness, or low spirits? or was it the waking
up of a sentiment which had bided its time as patiently as its inspirer?
Who shall say?

\begin{center}\rule{0.5\linewidth}{0.5pt}\end{center}

\subsection{XLIII.
Surprises}\label{6672479776654687619_37106-h-7.htm.xhtml_pgepubid00045}

\protect\phantomsection\label{6672479776654687619_37106-h-7.htm.xhtml_b182.png}{}
\pandocbounded{\includegraphics[keepaspectratio]{303483661336987339_b182.png}}

\protect\phantomsection\label{6672479776654687619_37106-h-7.htm.xhtml_XLIII}{}\hyperref[6672479776654687619_37106-h-0.htm.xhtml_contents2b]{XLIII.}

SURPRISES.

{Jo} was alone in the twilight, lying on the old sofa, looking at the
fire, and thinking. It was her favorite way of spending the hour of
dusk; no one disturbed her, and she used to lie there on
Beth\textquotesingle s little red pillow, planning stories, dreaming
dreams, or thinking tender thoughts of the sister who never seemed far
away. Her face looked tired, grave, and rather sad; for to-morrow was
her birthday, and she was thinking how fast the years went by, how old
she was getting, and how little she seemed to have accomplished. Almost
twenty-five, and nothing to show for it. Jo was mistaken in that; there
was a good deal to show, and by and by she saw, and was grateful for it.

"An old maid, that\textquotesingle s what I\textquotesingle m to be. A
literary spinster, with a pen for a spouse, a family of stories for
children, and twenty years hence a morsel of fame, perhaps; when, like
poor Johnson, I\textquotesingle m old, and can\textquotesingle t enjoy
it, solitary, and can\textquotesingle t share it, independent, and
don\textquotesingle t need it. Well, I needn\textquotesingle t be a sour
saint nor a selfish sinner; and, I dare say, old maids are very
comfortable when they get used to it; but---" and there Jo sighed, as if
the prospect was not inviting.

It seldom is, at first, and thirty seems the end of all things to
five-and-twenty; but it\textquotesingle s not so bad as it looks, and
one can get on quite happily if one has something in
one\textquotesingle s self to fall back upon. At twenty-five, girls
begin to talk about being old maids, but secretly resolve that they
never will be; at thirty they say nothing about it, but quietly accept
the fact, and, if sensible, console themselves by remembering that they
have twenty more useful, happy years, in which they may be learning to
grow old gracefully. Don\textquotesingle t laugh at the spinsters, dear
girls, for often very tender, tragical romances are hidden away in the
hearts that beat so quietly under the sober gowns, and many silent
sacrifices of youth, health, ambition, love itself, make the faded faces
beautiful in God\textquotesingle s sight. Even the sad, sour sisters
should be kindly dealt with, because they have missed the sweetest part
of life, if for no other reason; and, looking at them with compassion,
not contempt, girls in their bloom should remember that they too may
miss the blossom time; that rosy cheeks don\textquotesingle t last
forever, that silver threads will come in the bonnie brown hair, and
that, by and by, kindness and respect will be as sweet as love and
admiration now.

Gentlemen, which means boys, be courteous to the old maids, no matter
how poor and plain and prim, for the only chivalry worth having is that
which is the readiest to pay deference to the old, protect the feeble,
and serve womankind, regardless of rank, age, or color. Just recollect
the good aunts who have not only lectured and fussed, but nursed and
petted, too often without thanks; the scrapes they have helped you out
of, the "tips" they have given you from their small store, the stitches
the patient old fingers have set for you, the steps the willing old feet
have taken, and gratefully pay the dear old ladies the little attentions
that women love to receive as long as they live. The bright-eyed girls
are quick to see such traits, and will like you all the better for them;
and if death, almost the only power that can part mother and son, should
rob you of yours, you will be sure to find a tender welcome and maternal
cherishing from some Aunt Priscilla, who has kept the warmest corner of
her lonely old heart for "the best nevvy in the world."

Jo must have fallen asleep (as I dare say my reader has during this
little homily), for suddenly Laurie\textquotesingle s ghost seemed to
stand before her,---a substantial, lifelike ghost,---leaning over her,
with the very look he used to wear when he felt a good deal and
didn\textquotesingle t like to show it. But, like Jenny in \ul{the
ballad,---}

"She could not think it he,"

and lay staring up at him in startled silence, till he stooped and
kissed her. Then she knew him, and flew up, crying joyfully,---

"O my Teddy! O my Teddy!"

"Dear Jo, you are glad to see me, then?"

"Glad! My blessed boy, words can\textquotesingle t express my gladness.
Where\textquotesingle s Amy?"

"Your mother has got her down at Meg\textquotesingle s. We stopped there
by the way, and there was no getting my wife out of their clutches."

"Your what?" cried Jo, for Laurie uttered those two words with an
unconscious pride and satisfaction which betrayed him.

"Oh, the dickens! now I\textquotesingle ve done it;" and he looked so
guilty that Jo was down upon him like a flash.

"You\textquotesingle ve gone and got married!"

"Yes, please, but I never will again;" and he went down upon his knees,
with a penitent clasping of hands, and a face full of mischief, mirth,
and triumph.

"Actually married?"

"Very much so, thank you."

"Mercy on us! What dreadful thing will you do next?" and Jo fell into
her seat, with a gasp.

"A characteristic, but not exactly complimentary, congratulation,"
returned Laurie, still in an abject attitude, but beaming with
satisfaction.

"What can you expect, when you take one\textquotesingle s breath away,
creeping in like a burglar, and letting cats out of bags like that? Get
up, you ridiculous boy, and tell me all about it."

"Not a word, unless you let me come in my old place, and promise not to
barricade."

Jo laughed at that as she had not done for many a long day, and patted
the sofa invitingly, as she said, in a cordial tone,---

"The old pillow is up garret, and we don\textquotesingle t need it now;
so, come and \textquotesingle fess, Teddy."

"How good it sounds to hear you say
\textquotesingle Teddy\textquotesingle! No one ever calls me that but
you;" and Laurie sat down, with an air of great content.

"What does Amy call you?"

"My lord."

"That\textquotesingle s like her. Well, you look it;" and
Jo\textquotesingle s eyes plainly betrayed that she found her boy
comelier than ever.

The pillow was gone, but there \emph{was} a barricade, nevertheless,---a
natural one, raised by time, absence, and change of heart. Both felt it,
and for a minute looked at one another as if that invisible barrier cast
a little shadow over them. It was gone directly, however, for Laurie
said, with a vain attempt at dignity,---

"Don\textquotesingle t I look like a married man and the head of a
family?"

"Not a bit, and you never will. You\textquotesingle ve grown bigger and
bonnier, but you are the same scapegrace as ever."

"Now, really, Jo, you ought to treat me with more respect," began
Laurie, who enjoyed it all immensely.

"How can I, when the mere idea of you, married and settled, is so
irresistibly funny that I can\textquotesingle t keep sober!" answered
Jo, smiling all over her face, so infectiously that they had another
laugh, and then settled down for a good talk, quite in the pleasant old
fashion.

"It\textquotesingle s no use your going out in the cold to get Amy, for
they are all coming up presently. I couldn\textquotesingle t wait; I
wanted to be the one to tell you the grand surprise, and have
\textquotesingle first skim,\textquotesingle{} as we used to say when we
squabbled about the cream."

"Of course you did, and spoilt your story by beginning at the wrong end.
Now, start right, and tell me how it all happened; I\textquotesingle m
pining to know."

"Well, I did it to please Amy," began Laurie, with a twinkle that made
Jo exclaim,---

"Fib number one; Amy did it to please you. Go on, and tell the truth, if
you can, sir."

"Now she\textquotesingle s beginning to marm it; isn\textquotesingle t
it jolly to hear her?" said Laurie to the fire, and the fire glowed and
sparkled as if it quite agreed. "It\textquotesingle s all the same, you
know, she and I being one. We planned to come home with the Carrols, a
month or more ago, but they suddenly changed their minds, and decided to
pass another winter in Paris. But grandpa wanted to come home; he went
to please me, and I couldn\textquotesingle t let him go alone, neither
could I leave Amy; and Mrs. Carrol had got English notions about
chaperons and such nonsense, and wouldn\textquotesingle t let Amy come
with us. So I just settled the difficulty by saying,
\textquotesingle Let\textquotesingle s be married, and then we can do as
we like.\textquotesingle"

"Of course you did; you always have things to suit you."

"Not always;" and something in Laurie\textquotesingle s voice made Jo
say hastily,---

"How did you ever get aunt to agree?"

"It was hard work; but, between us, we talked her over, for we had heaps
of good reasons on our side. There wasn\textquotesingle t time to write
and ask leave, but you all liked it, had consented to it by and by, and
it was only \textquotesingle taking Time by the
fetlock,\textquotesingle{} as my wife says."

"Aren\textquotesingle t we proud of those two words, and
don\textquotesingle t we like to say them?" interrupted Jo, addressing
the fire in her turn, and watching with delight the happy light it
seemed to kindle in the eyes that had been so tragically gloomy when she
saw them last.

"A trifle, perhaps; she\textquotesingle s such a captivating little
woman I can\textquotesingle t help being proud of her. Well, then, uncle
and aunt were there to play propriety; we were so absorbed in one
another we were of no mortal use apart, and that charming arrangement
would make everything easy all round; so we did it."

"When, where, how?" asked Jo, in a fever of feminine interest and
curiosity, for she could not realize it a particle.

"Six weeks ago, at the American consul\textquotesingle s, in Paris; a
very quiet wedding, of course, for even in our happiness we
didn\textquotesingle t forget dear little Beth."

Jo put her hand in his as he said that, and Laurie gently smoothed the
little red pillow, which he remembered well.

"Why didn\textquotesingle t you let us know afterward?" asked Jo, in a
quieter tone, when they had sat quite still a minute.

"We wanted to surprise you; we thought we were coming directly home, at
first; but the dear old gentleman, as soon as we were married, found he
couldn\textquotesingle t be ready under a month, at least, and sent us
off to spend our honeymoon wherever we liked. Amy had once called
Valrosa a regular honeymoon home, so we went there, and were as happy as
people are but once in their lives. My faith! wasn\textquotesingle t it
love among the roses!"

Laurie seemed to forget Jo for a minute, and Jo was glad of it; for the
fact that he told her these things so freely and naturally assured her
that he had quite forgiven and forgotten. She tried to draw away her
hand; but, as if he guessed the thought that prompted the
half-involuntary impulse, Laurie held it fast, and said, with a manly
gravity she had never seen in him before,---

"Jo, dear, I want to say one thing, and then we\textquotesingle ll put
it by forever. As I told you in my letter, when I wrote that Amy had
been so kind to me, I never shall stop loving you; but the love is
altered, and I have learned to see that it is better as it is. Amy and
you change places in my heart, that\textquotesingle s all. I think it
was meant to be so, and would have come about naturally, if I had
waited, as you tried to make me; but I never could be patient, and so I
got a heartache. I was a boy then, headstrong and violent; and it took a
hard lesson to show me my mistake. For it \emph{was} one, Jo, as you
said, and I found it out, after making a fool of myself. Upon my word, I
was so tumbled up in my mind, at one time, that I didn\textquotesingle t
know which I loved best, you or Amy, and tried to love both alike; but I
couldn\textquotesingle t, and when I saw her in Switzerland, everything
seemed to clear up all at once. You both got into your right places, and
I felt sure that it was well off with the old love before it was on with
the new; that I could honestly share my heart between sister Jo and wife
Amy, and love them both dearly. Will you believe it, and go back to the
happy old times when we first knew one another?"

"I\textquotesingle ll believe it, with all my heart; but, Teddy, we
never can be boy and girl again: the happy old times
can\textquotesingle t come back, and we mustn\textquotesingle t expect
it. We are man and woman now, with sober work to do, for playtime is
over, and we must give up frolicking. I\textquotesingle m sure you feel
this; I see the change in you, and you\textquotesingle ll find it in me.
I shall miss my boy, but I shall love the man as much, and admire him
more, because he means to be what I hoped he would. We
can\textquotesingle t be little playmates any longer, but we will be
brother and sister, to love and help one another all our lives,
won\textquotesingle t we, Laurie?"

He did not say a word, but took the hand she offered him, and laid his
face down on it for a minute, feeling that out of the grave of a boyish
passion, there had risen a beautiful, strong friendship to bless them
both. Presently Jo said cheerfully, for she didn\textquotesingle t want
the coming home to be a sad one,---

"I can\textquotesingle t make it true that you children are really
married, and going to set up housekeeping. Why, it seems only yesterday
that I was buttoning Amy\textquotesingle s pinafore, and pulling your
hair when you teased. Mercy me, how time does fly!"

"As one of the children is older than yourself, you
needn\textquotesingle t talk so like a grandma. I flatter myself
I\textquotesingle m a \textquotesingle gentleman
growed,\textquotesingle{} as Peggotty said of David; and when you see
Amy, you\textquotesingle ll find her rather a precocious infant," said
Laurie, looking amused at her maternal air.

"You may be a little older in years, but I\textquotesingle m ever so
much older in feeling, Teddy. Women always are; and this last year has
been such a hard one that I feel forty."

"Poor Jo! we left you to bear it alone, while we went pleasuring. You
\emph{are} older; here\textquotesingle s a line, and
there\textquotesingle s another; unless you smile, your eyes look sad,
and when I touched the cushion, just now, I found a tear on it.
You\textquotesingle ve had a great deal to bear, and had to bear it all
alone. What a selfish beast I\textquotesingle ve been!" and Laurie
pulled his own hair, with a remorseful look.

But Jo only turned over the traitorous pillow, and answered, in a tone
which she tried to make quite cheerful,---

"No, I had father and mother to help me, the dear babies to comfort me,
and the thought that you and Amy were safe and happy, to make the
troubles here easier to bear. I \emph{am} lonely, sometimes, but I dare
say it\textquotesingle s good for me, and---"

"You never shall be again," broke in Laurie, putting his arm about her,
as if to fence out every human ill. "Amy and I can\textquotesingle t get
on without you, so you must come and teach \textquotesingle the
children\textquotesingle{} to keep house, and go halves in everything,
just as we used to do, and let us pet you, and all be blissfully happy
and friendly together."

"If I shouldn\textquotesingle t be in the way, it would be very
pleasant. I begin to feel quite young already; for, somehow, all my
troubles seemed to fly away when you came. You always were a comfort,
Teddy;" and Jo leaned her head on his shoulder, just as she did years
ago, when Beth lay ill, and Laurie told her to hold on to him.

He looked down at her, wondering if she remembered the time, but Jo was
smiling to herself, as if, in truth, her troubles \emph{had} all
vanished at his coming.

"You are the same Jo still, dropping tears about one minute, and
laughing the next. You look a little wicked now; what is it, grandma?"

"I was wondering how you and Amy get on together."

"Like angels!"

"Yes, of course, at first; but which rules?"

"I don\textquotesingle t mind telling you that she does, now; at least I
let her think so,---it pleases her, you know. By and by we shall take
turns, for marriage, they say, halves one\textquotesingle s rights and
doubles one\textquotesingle s duties."

"You\textquotesingle ll go on as you begin, and Amy will rule you all
the days of your life."

"Well, she does it so imperceptibly that I don\textquotesingle t think I
shall mind much. She is the sort of woman who knows how to rule well; in
fact, I rather like it, for she winds one round her finger as softly and
prettily as a skein of silk, and makes you feel as if she was doing you
a favor all the while."

"That ever I should live to see you a henpecked husband and enjoying
it!" cried Jo, with uplifted hands.

It was good to see Laurie square his shoulders, and smile with masculine
scorn at that insinuation, as he replied, with his "high and mighty"
air,---

"Amy is too well-bred for that, and I am not the sort of man to submit
to it. My wife and I respect ourselves and one another too much ever to
tyrannize or quarrel."

Jo liked that, and thought the new dignity very becoming, but the boy
seemed changing very fast into the man, and regret mingled with her
pleasure.

"I am sure of that; Amy and you never did quarrel as we used to. She is
the sun and I the wind, in the fable, and the sun managed the man best,
you remember."

"She can blow him up as well as shine on him," laughed Laurie. "Such a
lecture as I got at Nice! I give you my word it was a deal worse than
any of your scoldings,---a regular rouser. I\textquotesingle ll tell you
all about it sometime,---\emph{she} never will, because, after telling
me that she despised and was ashamed of me, she lost her heart to the
despicable party and married the good-for-nothing."

"What baseness! Well, if she abuses you, come to me, and
I\textquotesingle ll defend you."

"I look as if I needed it, don\textquotesingle t I?" said Laurie,
getting up and striking an attitude which suddenly changed from the
imposing to the rapturous, as Amy\textquotesingle s voice was heard
calling,---

"Where is she? Where\textquotesingle s my dear old Jo?"

In trooped the whole family, and every one was hugged and kissed all
over again, and, after several vain attempts, the three wanderers were
set down to be looked at and exulted over. Mr. Laurence, hale and hearty
as ever, was quite as much improved as the others by his foreign tour,
for the crustiness seemed to be nearly gone, and the old-fashioned
courtliness had received a polish which made it kindlier than ever. It
was good to see him beam at "my children," as he called the young pair;
it was better still to see Amy pay him the daughterly duty and affection
which completely won his old heart; and best of all, to watch Laurie
revolve about the two, as if never tired of enjoying the pretty picture
they made.

The minute she put her eyes upon Amy, Meg became conscious that her own
dress hadn\textquotesingle t a Parisian air, that young Mrs. Moffat
would be entirely eclipsed by young Mrs. Laurence, and that "her
ladyship" was altogether a most elegant and graceful woman. Jo thought,
as she watched the pair, "How well they look together! I was right, and
Laurie has found the beautiful, accomplished girl who will become his
home better than clumsy old Jo, and be a pride, not a torment to him."
Mrs. March and her husband smiled and nodded at each other with happy
faces, for they saw that their youngest had done well, not only in
worldly things, but the better wealth of love, confidence, and
happiness.

For Amy\textquotesingle s face was full of the soft brightness which
betokens a peaceful heart, her voice had a new tenderness in it, and the
cool, prim carriage was changed to a gentle dignity, both womanly and
winning. No little affectations marred it, and the cordial sweetness of
her manner was more charming than the new beauty or the old grace, for
it stamped her at once with the unmistakable sign of the true
gentlewoman she had hoped to become.

"Love has done much for our little girl," said her mother softly.

"She has had a good example before her all her life, my dear," Mr. March
whispered back, with a loving look at the worn face and gray head beside
him.

Daisy found it impossible to keep her eyes off her "pitty aunty," but
attached herself like a lap-dog to the wonderful châtelaine full of
delightful charms. Demi paused to consider the new relationship before
he compromised himself by the rash acceptance of a bribe, which took the
tempting form of a family of wooden bears from Berne. A flank movement
produced an unconditional surrender, however, for Laurie knew where to
have him.

"Young man, when I first had the honor of making your acquaintance you
hit me in the face: now I demand the satisfaction of a gentleman;" and
with that the tall uncle proceeded to toss and tousle the small nephew
in a way that damaged his philosophical dignity as much as it delighted
his boyish soul.

\protect\phantomsection\label{6672479776654687619_37106-h-7.htm.xhtml_b183.png}{}
\pandocbounded{\includegraphics[keepaspectratio]{303483661336987339_b183.png}}

"Blest if she ain\textquotesingle t in silk from head to foot?
Ain\textquotesingle t it a relishin\textquotesingle{} sight to see her
settin\textquotesingle{} there as fine as a fiddle, and hear folks
calling little Amy, Mis. Laurence?" muttered old Hannah, who could not
resist frequent "peeks" through the slide as she set the table in a most
decidedly promiscuous manner.

Mercy on us, how they did talk! first one, then the other, then all
burst out together, trying to tell the history of three years in half an
hour. It was fortunate that tea was at hand, to produce a lull and
provide refreshment, for they would have been hoarse and faint if they
had gone on much longer. Such a happy procession as filed away into the
little dining-room! Mr. March proudly escorted "Mrs. Laurence;" Mrs.
March as proudly leaned on the arm of "my son;" the old gentleman took
Jo, with a whispered "You must be my girl now," and a glance at the
empty corner by the fire, that made Jo whisper back, with trembling
lips, "I\textquotesingle ll try to fill her place, sir."

The twins pranced behind, feeling that the millennium was at hand, for
every one was so busy with the new-comers that they were left to revel
at their own sweet will, and you may be sure they made the most of the
opportunity. Didn\textquotesingle t they steal sips of tea, stuff
gingerbread \emph{ab libitum}, get a hot biscuit apiece, and, as a
crowning trespass, didn\textquotesingle t they each whisk a captivating
little tart into their tiny pockets, there to stick and crumble
treacherously, teaching them that both human nature and pastry are
frail? Burdened with the guilty consciousness of the sequestered tarts,
and fearing that Dodo\textquotesingle s sharp eyes would pierce the thin
disguise of cambric and merino which hid their booty, the little sinners
attached themselves to "Dranpa," who hadn\textquotesingle t his
spectacles on. Amy, who was handed about like refreshments, returned to
the parlor on Father Laurence\textquotesingle s arm; the others paired
off as before, and this arrangement left Jo companionless. She did not
mind it at the minute, for she lingered to answer
Hannah\textquotesingle s eager inquiry,---

"Will Miss Amy ride in her coop (\emph{coupé}), and use all them lovely
silver dishes that\textquotesingle s stored away over yander?"

"Shouldn\textquotesingle t wonder if she drove six white horses, ate off
gold plate, and wore diamonds and point-lace every day. Teddy thinks
nothing too good for her," returned Jo with infinite satisfaction.

"No more there is! Will you have hash or fish-balls for breakfast?"
asked Hannah, who wisely mingled poetry and prose.

"I don\textquotesingle t care;" and Jo shut the door, feeling that food
was an uncongenial topic just then. She stood a minute looking at the
party vanishing above, and, as Demi\textquotesingle s short plaid legs
toiled up the last stair, a sudden sense of loneliness came over her so
strongly that she looked about her with dim eyes, as if to find
something to lean upon, for even Teddy had deserted her. If she had
known what birthday gift was coming every minute nearer and nearer, she
would not have said to herself, "I\textquotesingle ll weep a little weep
when I go to bed; it won\textquotesingle t do to be dismal now." Then
she drew her hand over her eyes,---for one of her boyish habits was
never to know where her handkerchief was,---and had just managed to call
up a smile when there came a knock at the porch-door.

She opened it with hospitable haste, and started as if another ghost had
come to surprise her; for there stood a tall, bearded gentleman, beaming
on her from the darkness like a midnight sun.

"O Mr. Bhaer, I \emph{am} so glad to see you!" cried Jo, with a clutch,
as if she feared the night would swallow him up before she could get him
in.

\protect\phantomsection\label{6672479776654687619_37106-h-7.htm.xhtml_b184.png}{}
\pandocbounded{\includegraphics[keepaspectratio]{303483661336987339_b184.png}}

"And I to see Miss Marsch,---but no, you haf a party---" and the
Professor paused as the sound of voices and the tap of dancing feet came
down to them.

"No, we haven\textquotesingle t, only the family. My sister and friends
have just come home, and we are all very happy. Come in, and make one of
us."

Though a very social man, I think Mr. Bhaer would have gone decorously
away, and come again another day; but how could he, when Jo shut the
door behind him, and bereft him of his hat? Perhaps her face had
something to do with it, for she forgot to hide her joy at seeing him,
and showed it with a frankness that proved irresistible to the solitary
man, whose welcome far exceeded his boldest hopes.

"If I shall not be Monsieur de Trop, I will so gladly see them all. You
haf been ill, my friend?"

He put the question abruptly, for, as Jo hung up his coat, the light
fell on her face, and he saw a change in it.

"Not ill, but tired and sorrowful. We have had trouble since I saw you
last."

"Ah, yes, I know. My heart was sore for you when I heard that;" and he
shook hands again, with such a sympathetic face that Jo felt as if no
comfort could equal the look of the kind eyes, the grasp of the big,
warm hand.

"Father, mother, this is my friend, Professor Bhaer," she said, with a
face and tone of such irrepressible pride and pleasure that she might as
well have blown a trumpet and opened the door with a flourish.

If the stranger had had any doubts about his reception, they were set at
rest in a minute by the cordial welcome he received. Every one greeted
him kindly, for Jo\textquotesingle s sake at first, but very soon they
liked him for his own. They could not help it, for he carried the
talisman that opens all hearts, and these simple people warmed to him at
once, feeling even the more friendly because he was poor; for poverty
enriches those who live above it, and is a sure passport to truly
hospitable spirits. Mr. Bhaer sat looking about him with the air of a
traveller who knocks at a strange door, and, when it opens, finds
himself at home. The children went to him like bees to a honey-pot; and,
establishing themselves on each knee, proceeded to captivate him by
rifling his pockets, pulling his beard, and investigating his watch,
with juvenile audacity. The women telegraphed their approval to one
another, and Mr. March, feeling that he had got a kindred spirit, opened
his choicest stores for his guest\textquotesingle s benefit, while
silent John listened and enjoyed the talk, but said not a word, and Mr.
Laurence found it impossible to go to sleep.

If Jo had not been otherwise engaged, Laurie\textquotesingle s behavior
would have amused her; for a faint twinge, not of jealousy, but
something like suspicion, caused that gentleman to stand aloof at first,
and observe the new-comer with brotherly circumspection. But it did not
last long. He got interested in spite of himself, and, before he knew
it, was drawn into the circle; for Mr. Bhaer talked well in this genial
atmosphere, and did himself justice. He seldom spoke to Laurie, but he
looked at him often, and a shadow would pass across his face, as if
regretting his own lost youth, as he watched the young man in his prime.
Then his eye would turn to Jo so wistfully that she would have surely
answered the mute inquiry if she had seen it; but Jo had her own eyes to
take care of, and, feeling that they could not be trusted, she prudently
kept them on the little sock she was knitting, like a model maiden aunt.

A stealthy glance now and then refreshed her like sips of fresh water
after a dusty walk, for the sidelong peeps showed her several propitious
omens. Mr. Bhaer\textquotesingle s face had lost the absent-minded
expression, and looked all alive with interest in the present moment,
actually young and handsome, she thought, forgetting to compare him with
Laurie, as she usually did strange men, to their great detriment. Then
he seemed quite inspired, though the burial customs of the ancients, to
which the conversation had strayed, might not be considered an
exhilarating topic. Jo quite glowed with triumph when Teddy got quenched
in an argument, and thought to herself, as she watched her
father\textquotesingle s absorbed face, "How he would enjoy having such
a man as my Professor to talk with every day!" Lastly, Mr. Bhaer was
dressed in a new suit of black, which made him look more like a
gentleman than ever. His bushy hair had been cut and smoothly brushed,
but didn\textquotesingle t stay in order long, for, in exciting moments,
he rumpled it up in the droll way he used to do; and Jo liked it
rampantly erect better than flat, because she thought it gave his fine
forehead a Jove-like aspect. Poor Jo, how she did glorify that plain
man, as she sat knitting away so quietly, yet letting nothing escape
her, not even the fact that Mr. Bhaer actually had gold sleeve-buttons
in his immaculate wristbands!

"Dear old fellow! He couldn\textquotesingle t have got himself up with
more care if he\textquotesingle d been going a-wooing," said Jo to
herself; and then a sudden thought, born of the words, made her blush so
dreadfully that she had to drop her ball, and go down after it to hide
her face.

The manœuvre did not succeed as well as she expected, however; for,
though just in the act of setting fire to a funeral-pile, the Professor
dropped his torch, metaphorically speaking, and made a dive after the
little blue ball. Of course they bumped their heads smartly together,
saw stars, and both came up flushed and laughing, without the ball, to
resume their seats, wishing they had not left them.

Nobody knew where the evening went to; for Hannah skilfully abstracted
the babies at an early hour, nodding like two rosy poppies, and Mr.
Laurence went home to rest. The others sat round the fire, talking away,
utterly regardless of the lapse of time, till Meg, whose maternal mind
was impressed with a firm conviction that Daisy had tumbled out of bed,
and Demi set his night-gown afire studying the structure of matches,
made a move to go.

"We must have our sing, in the good old way, for we are all together
again once more," said Jo, feeling that a good shout would be a safe and
pleasant vent for the jubilant emotions of her soul.

They were not \emph{all} there. But no one found the words thoughtless
or untrue; for Beth still seemed among them, a peaceful presence,
invisible, but dearer than ever, since death could not break the
household league that love made indissoluble. The little chair stood in
its old place; the tidy basket, with the bit of work she left unfinished
when the needle grew "so heavy," was still on its accustomed shelf; the
beloved instrument, seldom touched now, had not been moved; and above it
Beth\textquotesingle s face, serene and smiling, as in the early days,
looked down upon them, seeming to say, "Be happy. I am here."

"Play something, Amy. Let them hear how much you have improved," said
Laurie, with pardonable pride in his promising pupil.

But Amy whispered, with full eyes, as she twirled the faded stool,---

"Not to-night, dear. I can\textquotesingle t show off to-night."

But she did show something better than brilliancy or skill; for she sung
Beth\textquotesingle s songs with a tender music in her voice which the
best master could not have taught, and touched the
listeners\textquotesingle{} hearts with a sweeter power than any other
inspiration could have given her. The room was very still, when the
clear voice failed suddenly at the last line of Beth\textquotesingle s
favorite hymn. It was hard to say,---

"Earth hath no sorrow that heaven cannot heal;"

and Amy leaned against her husband, who stood behind her, feeling that
her welcome home was not quite perfect without Beth\textquotesingle s
kiss.

"Now, we must finish with Mignon\textquotesingle s song; for Mr. Bhaer
sings that," said Jo, before the pause grew painful. And Mr. Bhaer
cleared his throat with a gratified "Hem!" as he stepped into the corner
where Jo stood, saying,---

"You will sing with me? We go excellently well together."

\protect\phantomsection\label{6672479776654687619_37106-h-7.htm.xhtml_b185.png}{}
\pandocbounded{\includegraphics[keepaspectratio]{303483661336987339_b185.png}}

A pleasing fiction, by the way; for Jo had no more idea of music than a
grasshopper. But she would have consented if he had proposed to sing a
whole opera, and warbled away, blissfully regardless of time and tune.
It didn\textquotesingle t much matter; for Mr. Bhaer sang like a true
German, heartily and well; and Jo soon subsided into a subdued hum, that
she might listen to the mellow voice that seemed to sing for her alone.

"Know\textquotesingle st thou the land where the citron blooms,"

used to be the Professor\textquotesingle s favorite line, for "das land"
meant Germany to him; but now he seemed to dwell, with peculiar warmth
and melody, upon the words,---

"There, oh there, might I with thee,

O my beloved, go!"

and one listener was so thrilled by the tender invitation that she
longed to say she did know the land, and would joyfully depart thither
whenever he liked.

The song was considered a great success, and the singer retired covered
with laurels. But a few minutes afterward, he forgot his manners
entirely, and stared at Amy putting on her bonnet; for she had been
introduced simply as "my sister," and no one had called her by her new
name since he came. He forgot himself still further when Laurie said, in
his most gracious manner, at parting,---

"My wife and I are very glad to meet you, sir. Please remember that
there is always a welcome waiting for you over the way."

Then the Professor thanked him so heartily, and looked so suddenly
illuminated with satisfaction, that Laurie thought him the most
delightfully demonstrative old fellow he ever met.

"I too shall go; but I shall gladly come again, if you will gif me
leave, dear madame, for a little business in the city will keep me here
some days."

He spoke to Mrs. March, but he looked at Jo; and the
mother\textquotesingle s voice gave as cordial an assent as did the
daughter\textquotesingle s eyes; for Mrs. March was not so blind to her
children\textquotesingle s interest as Mrs. Moffat supposed.

"I suspect that is a wise man," remarked Mr. March, with placid
satisfaction, from the hearth-rug, after the last guest had gone.

"I know he is a good one," added Mrs. March, with decided approval, as
she wound up the clock.

"I thought you\textquotesingle d like him," was all Jo said, as she
slipped away to her bed.

She wondered what the business was that brought Mr. Bhaer to the city,
and finally decided that he had been appointed to some great honor,
somewhere, but had been too modest to mention the fact. If she had seen
his face when, safe in his own room, he looked at the picture of a
severe and rigid young lady, with a good deal of hair, who appeared to
be gazing darkly into futurity, it might have thrown some light upon the
subject, especially when he turned off the gas, and kissed the picture
in the dark.

\begin{center}\rule{0.5\linewidth}{0.5pt}\end{center}

\subsection{XLIV. My Lord and
Lady.}\label{6672479776654687619_37106-h-7.htm.xhtml_pgepubid00046}

\protect\phantomsection\label{6672479776654687619_37106-h-7.htm.xhtml_b186.png}{}
\pandocbounded{\includegraphics[keepaspectratio]{303483661336987339_b186.png}}

\protect\phantomsection\label{6672479776654687619_37106-h-7.htm.xhtml_XLIV}{}\hyperref[6672479776654687619_37106-h-0.htm.xhtml_contents2b]{XLIV.}

MY LORD AND LADY.

"{Please}, Madam Mother, could you lend me my wife for half an hour? The
luggage has come, and I\textquotesingle ve been making hay of
Amy\textquotesingle s Paris finery, trying to find some things I want,"
said Laurie, coming in the next day to find Mrs. Laurence sitting in her
mother\textquotesingle s lap, as if being made "the baby" again.

"Certainly. Go, dear; I forget that you have any home but this," and
Mrs. March pressed the white hand that wore the wedding-ring, as if
asking pardon for her maternal covetousness.

"I shouldn\textquotesingle t have come over if I could have helped it;
but I can\textquotesingle t get on without my little woman any more than
a---"

"Weathercock can without wind," suggested Jo, as he paused for a simile;
Jo had grown quite her own saucy self again since Teddy came home.

"Exactly; for Amy keeps me pointing due west most of the time, with only
an occasional whiffle round to the south, and I haven\textquotesingle t
had an easterly spell since I was married; don\textquotesingle t know
anything about the north, but am altogether salubrious and balmy, hey,
my lady?"

"Lovely weather so far; I don\textquotesingle t know how long it will
last, but I\textquotesingle m not afraid of storms, for
I\textquotesingle m learning how to sail my ship. Come home, dear, and
I\textquotesingle ll find your bootjack; I suppose
that\textquotesingle s what you are rummaging after among my things. Men
are \emph{so} helpless, mother," said Amy, with a matronly air, which
delighted her husband.

"What are you going to do with yourselves after you get settled?" asked
Jo, buttoning Amy\textquotesingle s cloak as she used to button her
pinafores.

"We have our plans; we don\textquotesingle t mean to say much about them
yet, because we are such very new brooms, but we don\textquotesingle t
intend to be idle. I\textquotesingle m going into business with a
devotion that shall delight grandfather, and prove to him that
I\textquotesingle m not spoilt. I need something of the sort to keep me
steady. I\textquotesingle m tired of dawdling, and mean to work like a
man."

"And Amy, what is she going to do?" asked Mrs. March, well pleased at
Laurie\textquotesingle s decision, and the energy with which he spoke.

"After doing the civil all round, and airing our best bonnet, we shall
astonish you by the elegant hospitalities of our mansion, the brilliant
society we shall draw about us, and the beneficial influence we shall
exert over the world at large. That\textquotesingle s about it,
isn\textquotesingle t it, Madame Récamier?" asked Laurie, with a
quizzical look at Amy.

"Time will show. Come away, Impertinence, and don\textquotesingle t
shock my family by calling me names before their faces," answered Amy,
resolving that there should be a home with a good wife in it before she
set up a \emph{salon} as a queen of society.

"How happy those children seem together!" observed Mr. March, finding it
difficult to become absorbed in his Aristotle after the young couple had
gone.

"Yes, and I think it will last," added Mrs. March, with the restful
expression of a pilot who has brought a ship safely into port.

"I know it will. Happy Amy!" and Jo sighed, then smiled brightly as
Professor Bhaer opened the gate with an impatient push.

Later in the evening, when his mind had been set at rest about the
bootjack, Laurie said suddenly to his wife, who was flitting about,
arranging her new art treasures,---

"Mrs. Laurence."

"My lord!"

"That man intends to marry our Jo!"

"I hope so; don\textquotesingle t you, dear?"

"Well, my love, I consider him a trump, in the fullest sense of that
expressive word, but I do wish he was a little younger and a good deal
richer."

"Now, Laurie, don\textquotesingle t be too fastidious and
worldly-minded. If they love one another it doesn\textquotesingle t
matter a particle how old they are nor how poor. Women \emph{never}
should marry for money---" Amy caught herself up short as the words
escaped her, and looked at her husband, who replied, with malicious
gravity,---

"Certainly not, though you do hear charming girls say that they intend
to do it sometimes. If my memory serves me, you once thought it your
duty to make a rich match; that accounts, perhaps, for your marrying a
good-for-nothing like me."

"O my dearest boy, don\textquotesingle t, don\textquotesingle t say
that! I forgot you were rich when I said
\textquotesingle Yes.\textquotesingle{} I\textquotesingle d have married
you if you hadn\textquotesingle t a penny, and I sometimes wish you
\emph{were} poor that I might show how much I love you;" and Amy, who
was very dignified in public and very fond in private, gave convincing
proofs of the truth of her words.

"You don\textquotesingle t really think I am such a mercenary creature
as I tried to be once, do you? It would break my heart if you
didn\textquotesingle t believe that I\textquotesingle d gladly pull in
the same boat with you, even if you had to get your living by rowing on
the lake."

"Am I an idiot and a brute? How could I think so, when you refused a
richer man for me, and won\textquotesingle t let me give you half I want
to now, when I have the right? Girls do it every day, poor things, and
are taught to think it is their only salvation; but you had better
lessons, and, though I trembled for you at one time, I was not
disappointed, for the daughter was true to the mother\textquotesingle s
teaching. I told mamma so yesterday, and she looked as glad and grateful
as if I\textquotesingle d given her a check for a million, to be spent
in charity. You are not listening to my moral remarks, Mrs. Laurence;"
and Laurie paused, for Amy\textquotesingle s eyes had an absent look,
though fixed upon his face.

"Yes, I am, and admiring the dimple in your chin at the same time. I
don\textquotesingle t wish to make you vain, but I must confess that
I\textquotesingle m prouder of my handsome husband than of all his
money. Don\textquotesingle t laugh, but your nose is \emph{such} a
comfort to me;" and Amy softly caressed the well-cut feature with
artistic satisfaction.

Laurie had received many compliments in his life, but never one that
suited him better, as he plainly showed, though he did laugh at his
wife\textquotesingle s peculiar taste, while she said slowly,---

"May I ask you a question, dear?"

"Of course you may."

"Shall you care if Jo does marry Mr. Bhaer?"

"Oh, that\textquotesingle s the trouble, is it? I thought there was
something in the dimple that didn\textquotesingle t suit you. Not being
a dog in the manger, but the happiest fellow alive, I assure you I can
dance at Jo\textquotesingle s wedding with a heart as light as my heels.
Do you doubt it, my darling?"

Amy looked up at him, and was satisfied; her last little jealous fear
vanished forever, and she thanked him, with a face full of love and
confidence.

"I wish we could do something for that capital old Professor.
Couldn\textquotesingle t we invent a rich relation, who shall obligingly
die out there in Germany, and leave him a tidy little fortune?" said
Laurie, when they began to pace up and down the long drawing-room,
arm-in-arm, as they were fond of doing, in memory of the chateau garden.

\protect\phantomsection\label{6672479776654687619_37106-h-7.htm.xhtml_b187.png}{}
\pandocbounded{\includegraphics[keepaspectratio]{303483661336987339_b187.png}}

"Jo would find us out, and spoil it all; she is very proud of him, just
as he is, and said yesterday that she thought poverty was a beautiful
thing."

"Bless her dear heart! she won\textquotesingle t think so when she has a
literary husband, and a dozen little professors and professorins to
support. We won\textquotesingle t interfere now, but watch our chance,
and do them a good turn in spite of themselves. I owe Jo for a part of
my education, and she believes in people\textquotesingle s paying their
honest debts, so I\textquotesingle ll get round her in that way."

"How delightful it is to be able to help others, isn\textquotesingle t
it? That was always one of my dreams, to have the power of giving
freely; and, thanks to you, the dream has come true."

"Ah! we\textquotesingle ll do quantities of good, won\textquotesingle t
we? There\textquotesingle s one sort of poverty that I particularly like
to help. Out-and-out beggars get taken care of, but poor gentlefolks
fare badly, because they won\textquotesingle t ask, and people
don\textquotesingle t dare to offer charity; yet there are a thousand
ways of helping them, if one only knows how to do it so delicately that
it does not offend. I must say, I like to serve a decayed gentleman
better than a blarneying beggar; I suppose it\textquotesingle s wrong,
but I do, though it is harder."

"Because it takes a gentleman to do it," added the other member of the
domestic admiration society.

"Thank you, I\textquotesingle m afraid I don\textquotesingle t deserve
that pretty compliment. But I was going to say that while I was dawdling
about abroad, I saw a good many talented young fellows making all sorts
of sacrifices, and enduring real hardships, that they might realize
their dreams. Splendid fellows, some of them, working like heroes, poor
and friendless, but so full of courage, patience, and ambition, that I
was ashamed of myself, and longed to give them a right good lift. Those
are people whom it\textquotesingle s a satisfaction to help, for if
they\textquotesingle ve got genius, it\textquotesingle s an honor to be
allowed to serve them, and not let it be lost or delayed for want of
fuel to keep the pot boiling; if they haven\textquotesingle t,
it\textquotesingle s a pleasure to comfort the poor souls, and keep them
from despair when they find it out."

"Yes, indeed; and there\textquotesingle s another class who
can\textquotesingle t ask, and who suffer in silence. I know something
of it, for I belonged to it before you made a princess of me, as the
king does the beggar-maid in the old story. Ambitious girls have a hard
time, Laurie, and often have to see youth, health, and precious
opportunities go by, just for want of a little help at the right minute.
People have been very kind to me; and whenever I see girls struggling
along, as we used to do, I want to put out my hand and help them, as I
was helped."

"And so you shall, like an angel as you are!" cried Laurie, resolving,
with a glow of philanthropic zeal, to found and endow an institution for
the express benefit of young women with artistic tendencies. "Rich
people have no right to sit down and enjoy themselves, or let their
money accumulate for others to waste. It\textquotesingle s not half so
sensible to leave legacies when one dies as it is to use the money
wisely while alive, and enjoy making one\textquotesingle s
fellow-creatures happy with it. We\textquotesingle ll have a good time
ourselves, and add an extra relish to our own pleasure by giving other
people a generous taste. Will you be a little Dorcas, going about
emptying a big basket of comforts, and filling it up with good deeds?"

"With all my heart, if you will be a brave St. Martin, stopping, as you
ride gallantly through the world, to share your cloak with the beggar."

"It\textquotesingle s a bargain, and we shall get the best of it!"

So the young pair shook hands upon it, and then paced happily on again,
feeling that their pleasant home was more home-like because they hoped
to brighten other homes, believing that their own feet would walk more
uprightly along the flowery path before them, if they smoothed rough
ways for other feet, and feeling that their hearts were more closely
knit together by a love which could tenderly remember those less blest
than they.

\protect\phantomsection\label{6672479776654687619_37106-h-7.htm.xhtml_b188.png}{}
\pandocbounded{\includegraphics[keepaspectratio]{303483661336987339_b188.png}}

\begin{center}\rule{0.5\linewidth}{0.5pt}\end{center}

\subsection{XLV. Daisy and
Demi.}\label{6672479776654687619_37106-h-7.htm.xhtml_pgepubid00047}

\protect\phantomsection\label{6672479776654687619_37106-h-7.htm.xhtml_XLV}{}\hyperref[6672479776654687619_37106-h-0.htm.xhtml_contents2b]{XLV.}

DAISY AND DEMI.

{I cannot} feel that I have done my duty as humble historian of the
March family, without devoting at least one chapter to the two most
precious and important members of it. Daisy and Demi had now arrived at
years of discretion; for in this fast age babies of three or four assert
their rights, and get them, too, which is more than many of their elders
do. If there ever were a pair of twins in danger of being utterly spoilt
by adoration, it was these prattling Brookes. Of course they were the
most remarkable children ever born, as will be shown when I mention that
they walked at eight months, talked fluently at twelve months, and at
two years they took their places at table, and behaved with a propriety
which charmed all beholders. At three, Daisy demanded a "needler," and
actually made a bag with four stitches in it; she likewise set up
housekeeping in the sideboard, and managed a microscopic cooking-stove
with a skill that brought tears of pride to Hannah\textquotesingle s
eyes, while Demi learned his letters with his grandfather, who invented
a new mode of teaching the alphabet by forming the letters with his arms
and legs, thus uniting gymnastics for head and heels. The boy early
developed a mechanical genius which delighted his father and distracted
his mother, for he tried to imitate every machine he saw, and kept the
nursery in a chaotic condition, with his "sewin-sheen,"---a mysterious
structure of string, chairs, clothes-pins, and spools, for wheels to go
"wound and wound;" also a basket hung over the back of a big chair, in
which he vainly tried to hoist his too confiding sister, who, with
feminine devotion, allowed her little head to be bumped till rescued,
when the young inventor indignantly remarked, "Why, marmar,
dat\textquotesingle s my lellywaiter, and me\textquotesingle s trying to
pull her up."

Though utterly unlike in character, the twins got on remarkably well
together, and seldom quarrelled more than thrice a day. Of course, Demi
tyrannized over Daisy, and gallantly defended her from every other
aggressor; while Daisy made a galley-slave of herself, and adored her
brother as the one perfect being in the world. A rosy, chubby, sunshiny
little soul was Daisy, who found her way to everybody\textquotesingle s
heart, and nestled there. One of the captivating children, who seem made
to be kissed and cuddled, adorned and adored like little goddesses, and
produced for general approval on all festive occasions. Her small
virtues were so sweet that she would have been quite angelic if a few
small naughtinesses had not kept her delightfully human. It was all fair
weather in her world, and every morning she scrambled up to the window
in her little night-gown to look out, and say, no matter whether it
rained or shone, "Oh, pitty day, oh, pitty day!" Every one was a friend,
and she offered kisses to a stranger so confidingly that the most
inveterate bachelor relented, and baby-lovers became faithful
worshippers.

\protect\phantomsection\label{6672479776654687619_37106-h-7.htm.xhtml_b189.png}{}
\pandocbounded{\includegraphics[keepaspectratio]{303483661336987339_b189.png}}

"Me loves evvybody," she once said, opening her arms, with her spoon in
one hand, and her mug in the other, as if eager to embrace and nourish
the whole world.

As she grew, her mother began to feel that the Dove-cote would be blest
by the presence of an inmate as serene and loving as that which had
helped to make the old house home, and to pray that she might be spared
a loss like that which had lately taught them how long they had
entertained an angel unawares. Her grandfather often called her "Beth,"
and her grandmother watched over her with untiring devotion, as if
trying to atone for some past mistake, which no eye but her own could
see.

Demi, like a true Yankee, was of an inquiring turn, wanting to know
everything, and often getting much disturbed because he could not get
satisfactory answers to his perpetual "What for?"

He also possessed a philosophic bent, to the great delight of his
grandfather, who used to hold Socratic conversations with him, in which
the precocious pupil occasionally posed his teacher, to the undisguised
satisfaction of the womenfolk.

\protect\phantomsection\label{6672479776654687619_37106-h-7.htm.xhtml_b190.png}{}
\pandocbounded{\includegraphics[keepaspectratio]{303483661336987339_b190.png}}

"What makes my legs go, dranpa?" asked the young philosopher, surveying
those active portions of his frame with a meditative air, while resting
after a go-to-bed frolic one night.

"It\textquotesingle s your little mind, Demi," replied the sage,
stroking the yellow head respectfully.

"What is a little mine?"

"It is something which makes your body move, as the spring made the
wheels go in my watch when I showed it to you."

"Open me; I want to see it go wound."

"I can\textquotesingle t do that any more than you could open the watch.
God winds you up, and you go till He stops you."

"Does I?" and Demi\textquotesingle s brown eyes grew big and bright as
he took in the new thought. "Is I wounded up like the watch?"

"Yes; but I can\textquotesingle t show you how; for it is done when we
don\textquotesingle t see."

Demi felt of his back, as if expecting to find it like that of the
watch, and then gravely remarked,---

"I dess Dod does it when I\textquotesingle s asleep."

A careful explanation followed, to which he listened so attentively that
his anxious grandmother said,---

"My dear, do you think it wise to talk about such things to that baby?
He\textquotesingle s getting great bumps over his eyes, and learning to
ask the most unanswerable questions."

"If he is old enough to ask the questions he is old enough to receive
true answers. I am not putting the thoughts into his head, but helping
him unfold those already there. These children are wiser than we are,
and I have no doubt the boy understands every word I have said to him.
Now, Demi, tell me where you keep your mind?"

If the boy had replied like Alcibiades, "By the gods, Socrates, I cannot
tell," his grandfather would not have been surprised; but when, after
standing a moment on one leg, like a meditative young stork, he
answered, in a tone of calm conviction, "In my little belly," the old
gentleman could only join in grandma\textquotesingle s laugh, and
dismiss the class in metaphysics.

There might have been cause for maternal anxiety, if Demi had not given
convincing proofs that he was a true boy, as well as a budding
philosopher; for, often, after a discussion which caused Hannah to
prophesy, with ominous nods, "That child ain\textquotesingle t long for
this world," he would turn about and set her fears at rest by some of
the pranks with which dear, dirty, naughty little rascals distract and
delight their parents\textquotesingle{} souls.

Meg made many moral rules, and tried to keep them; but what mother was
ever proof against the winning wiles, the ingenious evasions, or the
tranquil audacity of the miniature men and women who so early show
themselves accomplished Artful Dodgers?

"No more raisins, Demi, they\textquotesingle ll make you sick," says
mamma to the young person who offers his services in the kitchen with
unfailing regularity on plum-pudding day.

"Me likes to be sick."

"I don\textquotesingle t want to have you, so run away and help Daisy
make patty-cakes."

He reluctantly departs, but his wrongs weigh upon his spirit; and, by
and by, when an opportunity comes to redress them, he outwits mamma by a
shrewd bargain.

"Now you have been good children, and I\textquotesingle ll play anything
you like," says Meg, as she leads her assistant cooks upstairs, when the
pudding is safely bouncing in the pot.

"Truly, marmar?" asks Demi, with a brilliant idea in his well-powdered
head.

"Yes, truly; anything you say," replies the short-sighted parent,
preparing herself to sing "The Three Little Kittens" half a dozen times
over, or to take her family to "Buy a penny bun," regardless of wind or
limb. But Demi corners her by the cool reply,---

"Then we\textquotesingle ll go and eat up all the raisins."

Aunt Dodo was chief playmate and \emph{confidante} of both children, and
the trio turned the little house topsy-turvy. Aunt Amy was as yet only a
name to them, Aunt Beth soon faded into a pleasantly vague memory, but
Aunt Dodo was a living reality, and they made the most of her, for which
compliment she was deeply grateful. But when Mr. Bhaer came, Jo
neglected her playfellows, and dismay and desolation fell upon their
little souls. Daisy, who was fond of going about peddling kisses, lost
her best customer and became bankrupt; Demi, with infantile penetration,
soon discovered that Dodo liked to play with "the bear-man" better than
she did with him; but, though hurt, he concealed his anguish, for he
hadn\textquotesingle t the heart to insult a rival who kept a mine of
chocolate-drops in his waistcoat-pocket, and a watch that could be taken
out of its case and freely shaken by ardent admirers.

Some persons might have considered these pleasing liberties as bribes;
but Demi didn\textquotesingle t see it in that light, and continued to
patronize the "bear-man" with pensive affability, while Daisy bestowed
her small affections upon him at the third call, and considered his
shoulder her throne, his arm her refuge, his gifts treasures of
surpassing worth.

Gentlemen are sometimes seized with sudden fits of admiration for the
young relatives of ladies whom they honor with their regard; but this
counterfeit philoprogenitiveness sits uneasily upon them, and does not
deceive anybody a particle. Mr. Bhaer\textquotesingle s devotion was
sincere, however likewise effective,---for honesty is the best policy in
love as in law; he was one of the men who are at home with children, and
looked particularly well when little faces made a pleasant contrast with
his manly one. His business, whatever it was, detained him from day to
day, but evening seldom failed to bring him out to see---well, he always
asked for Mr. March, so I suppose \emph{he} was the attraction. The
excellent papa labored under the delusion that he was, and revelled in
long discussions with the kindred spirit, till a chance remark of his
more observing grandson suddenly enlightened him.

Mr. Bhaer came in one evening to pause on the threshold of the study,
astonished by the spectacle that met his eye. Prone upon the floor lay
Mr. March, with his respectable legs in the air, and beside him,
likewise prone, was Demi, trying to imitate the attitude with his own
short, scarlet-stockinged legs, both grovellers so seriously absorbed
that they were unconscious of spectators, till Mr. Bhaer laughed his
sonorous laugh, and Jo cried out, with a scandalized face,---

"Father, father, here\textquotesingle s the Professor!"

Down went the black legs and up came the gray head, as the preceptor
said, with undisturbed dignity,---

"Good evening, Mr. Bhaer. Excuse me for a moment; we are just finishing
our lesson. Now, Demi, make the letter and tell its name."

"I knows him!" and, after a few convulsive efforts, the red legs took
the shape of a pair of compasses, and the intelligent pupil triumphantly
shouted, "It\textquotesingle s a We, dranpa, it\textquotesingle s a We!"

\protect\phantomsection\label{6672479776654687619_37106-h-8.htm.xhtml}{}

\protect\phantomsection\label{6672479776654687619_37106-h-8.htm.xhtml_b191.png}{}
\pandocbounded{\includegraphics[keepaspectratio]{303483661336987339_b191.png}}

"He\textquotesingle s a born Weller," laughed Jo, as her parent gathered
himself up, and her nephew tried to stand on his head, as the only mode
of expressing his satisfaction that school was over.

"What have you been at to-day, bübchen?" asked Mr. Bhaer, picking up the
gymnast.

"Me went to see little Mary."

"And what did you there?"

"I kissed her," began Demi, with artless frankness.

"Prut! thou beginnest early. What did the little Mary say to that?"
asked Mr. Bhaer, continuing to confess the young sinner, who stood upon
his knee, exploring the waistcoat-pocket.

"Oh, she liked it, and she kissed me, and I liked it.
\emph{Don\textquotesingle t} little boys like little girls?" added Demi,
with his mouth full, and an air of bland satisfaction.

"You precocious chick! Who put that into your head?" said Jo, enjoying
the innocent revelations as much as the Professor.

"\textquotesingle Tisn\textquotesingle t in mine head;
it\textquotesingle s in mine mouf," answered literal Demi, putting out
his tongue, with a chocolate-drop on it, thinking she alluded to
confectionery, not ideas.

"Thou shouldst save some for the little friend: sweets to the sweet,
mannling;" and Mr. Bhaer offered Jo some, with a look that made her
wonder if chocolate was not the nectar drunk by the gods. Demi also saw
the smile, was impressed by it, and artlessly inquired,---

"Do great boys like great girls, too, \textquotesingle Fessor?"

Like young Washington, Mr. Bhaer "couldn\textquotesingle t tell a lie;"
so he gave the somewhat vague reply that he believed they did sometimes,
in a tone that made Mr. March put down his clothes-brush, glance at
Jo\textquotesingle s retiring face, and then sink into his chair,
looking as if the "precocious chick" had put an idea into \emph{his}
head that was both sweet and sour.

Why Dodo, when she caught him in the china-closet half an hour
afterward, nearly squeezed the breath out of his little body with a
tender embrace, instead of shaking him for being there, and why she
followed up this novel performance by the unexpected gift of a big slice
of bread and jelly, remained one of the problems over which Demi puzzled
his small wits, and was forced to leave unsolved forever.

\protect\phantomsection\label{6672479776654687619_37106-h-8.htm.xhtml_b192.png}{}
\pandocbounded{\includegraphics[keepaspectratio]{303483661336987339_b192.png}}

\begin{center}\rule{0.5\linewidth}{0.5pt}\end{center}

\subsection{XLVI. Under the
Umbrella.}\label{6672479776654687619_37106-h-8.htm.xhtml_pgepubid00048}

\protect\phantomsection\label{6672479776654687619_37106-h-8.htm.xhtml_b193.png}{}
\pandocbounded{\includegraphics[keepaspectratio]{303483661336987339_b193.png}}

\protect\phantomsection\label{6672479776654687619_37106-h-8.htm.xhtml_XLVI}{}\hyperref[6672479776654687619_37106-h-0.htm.xhtml_contents2b]{XLVI.}

UNDER THE UMBRELLA.

{While} Laurie and Amy were taking conjugal strolls over velvet carpets,
as they set their house in order, and planned a blissful future, Mr.
Bhaer and Jo were enjoying promenades of a different sort, along muddy
roads and sodden fields.

"I always do take a walk toward evening, and I don\textquotesingle t
know why I should give it up, just because I often happen to meet the
Professor on his way out," said Jo to herself, after two or three
encounters; for, though there were two paths to Meg\textquotesingle s,
whichever one she took she was sure to meet him, either going or
returning. He was always walking rapidly, and never seemed to see her
till quite close, when he would look as if his short-sighted eyes had
failed to recognize the approaching lady till that moment. Then, if she
was going to Meg\textquotesingle s, he always had something for the
babies; if her face was turned homeward, he had merely strolled down to
see the river, and was just about returning, unless they were tired of
his frequent calls.

Under the circumstances, what could Jo do but greet him civilly, and
invite him in? If she \emph{was} tired of his visits, she concealed her
weariness with perfect skill, and took care that there should be coffee
for supper, "as Friedrich---I mean Mr. Bhaer---doesn\textquotesingle t
like tea."

By the second week, every one knew perfectly well what was going on, yet
every one tried to look as if they were stone-blind to the changes in
Jo\textquotesingle s face. They never asked why she sang about her work,
did up her hair three times a day, and got so blooming with her evening
exercise; and no one seemed to have the slightest suspicion that
Professor Bhaer, while talking philosophy with the father, was giving
the daughter lessons in love.

Jo couldn\textquotesingle t even lose her heart in a decorous manner,
but sternly tried to quench her feelings; and, failing to do so, led a
somewhat agitated life. She was mortally afraid of being laughed at for
surrendering, after her many and vehement declarations of independence.
Laurie was her especial dread; but, thanks to the new manager, he
behaved with praiseworthy propriety, never called Mr. Bhaer "a capital
old fellow" in public, never alluded, in the remotest manner, to
Jo\textquotesingle s improved appearance, or expressed the least
surprise at seeing the Professor\textquotesingle s hat on the
Marches\textquotesingle{} hall-table nearly every evening. But he
exulted in private and longed for the time to come when he could give Jo
a piece of plate, with a bear and a ragged staff on it as an appropriate
coat-of-arms.

For a fortnight, the Professor came and went with lover-like regularity;
then he stayed away for three whole days, and made no sign,---a
proceeding which caused everybody to look sober, and Jo to become
pensive, at first, and then---alas for romance!---very cross.

"Disgusted, I dare say, and gone home as suddenly as he came.
It\textquotesingle s nothing to me, of course; but I \emph{should} think
he would have come and bid us good-by, like a gentleman," she said to
herself, with a despairing look at the gate, as she put on her things
for the customary walk, one dull afternoon.

"You\textquotesingle d better take the little umbrella, dear; it looks
like rain," said her mother, observing that she had on her new bonnet,
but not alluding to the fact.

"Yes, Marmee; do you want anything in town? I\textquotesingle ve got to
run in and get some paper," returned Jo, pulling out the bow under her
chin before the glass as an excuse for not looking at her mother.

"Yes; I want some twilled silesia, a paper of number nine needles, and
two yards of narrow lavender ribbon. Have you got your thick boots on,
and something warm under your cloak?"

"I believe so," answered Jo absently.

"If you happen to meet Mr. Bhaer, bring him home to tea. I quite long to
see the dear man," added Mrs. March.

Jo heard \emph{that}, but made no answer, except to kiss her mother, and
walk rapidly away, thinking with a glow of gratitude, in spite of her
heartache,---

"How good she is to me! What \emph{do} girls do who
haven\textquotesingle t any mothers to help them through their
troubles?"

The dry-goods stores were not down among the counting-houses, banks, and
wholesale warerooms, where gentlemen most do congregate; but Jo found
herself in that part of the city before she did a single errand,
loitering along as if waiting for some one, examining engineering
instruments in one window and samples of wool in another with most
unfeminine interest; tumbling over barrels, being half-smothered by
descending bales, and hustled unceremoniously by busy men who looked as
if they wondered "how the deuce she got there." A drop of rain on her
cheek recalled her thoughts from baffled hopes to ruined ribbons; for
the drops continued to fall, and, being a woman as well as a lover, she
felt that, though it was too late to save her heart, she might her
bonnet. Now she remembered the little umbrella, which she had forgotten
to take in her hurry to be off; but regret was unavailing, and nothing
could be done but borrow one or submit to a drenching. She looked up at
the lowering sky, down at the crimson bow already flecked with black,
forward along the muddy street, then one long, lingering look behind, at
a certain grimy warehouse, with "Hoffmann, Swartz, \& Co." over the
door, and said to herself, with a sternly reproachful air,---

"It serves me right! What business had I to put on all my best things
and come philandering down here, hoping to see the Professor? Jo,
I\textquotesingle m ashamed of you! No, you shall \emph{not} go there to
borrow an umbrella, or find out where he is, from his friends. You shall
trudge away, and do your errands in the rain; and if you catch your
death and ruin your bonnet, it\textquotesingle s no more than you
deserve. Now then!"

With that she rushed across the street so impetuously that she narrowly
escaped annihilation from a passing truck, and precipitated herself into
the arms of a stately old gentleman, who said, "I beg pardon,
ma\textquotesingle am," and looked mortally offended. Somewhat daunted,
Jo righted herself, spread her handkerchief over the devoted ribbons,
and, putting temptation behind her, hurried on, with increasing dampness
about the ankles, and much clashing of umbrellas overhead. The fact that
a somewhat dilapidated blue one remained stationary above the
unprotected bonnet, attracted her attention; and, looking up, she saw
Mr. Bhaer looking down.

\protect\phantomsection\label{6672479776654687619_37106-h-8.htm.xhtml_b194.png}{}
\pandocbounded{\includegraphics[keepaspectratio]{303483661336987339_b194.png}}

"I feel to know the strong-minded lady who goes so bravely under many
horse-noses, and so fast through much mud. What do you down here, my
friend?"

"I\textquotesingle m shopping."

Mr. Bhaer smiled, as he glanced from the pickle-factory on one side, to
the wholesale hide and leather concern on the other; but he only said
politely,---

"You haf no umbrella. May I go also, and take for you the bundles?"

"Yes, thank you."

Jo\textquotesingle s cheeks were as red as her ribbon, and she wondered
what he thought of her; but she didn\textquotesingle t care, for in a
minute she found herself walking away arm-in-arm with her Professor,
feeling as if the sun had suddenly burst out with uncommon brilliancy,
that the world was all right again, and that one thoroughly happy woman
was paddling through the wet that day.

"We thought you had gone," said Jo hastily, for she knew he was looking
at her. Her bonnet wasn\textquotesingle t big enough to hide her face,
and she feared he might think the joy it betrayed unmaidenly.

"Did you believe that I should go with no farewell to those who haf been
so heavenly kind to me?" he asked so reproachfully that she felt as if
she had insulted him by the suggestion, and answered heartily,---

"No, \emph{I} didn\textquotesingle t; I knew you were busy about your
own affairs, but we rather missed you,---father and mother especially."

"And you?"

"I\textquotesingle m always glad to see you, sir."

In her anxiety to keep her voice quite calm, Jo made it rather cool, and
the frosty little monosyllable at the end seemed to chill the Professor,
for his smile vanished, as he said gravely,---

"I thank you, and come one time more before I go."

"You \emph{are} going, then?"

"I haf no longer any business here; it is done."

"Successfully, I hope?" said Jo, for the bitterness of disappointment
was in that short reply of his.

"I ought to think so, for I haf a way opened to me by which I can make
my bread and gif my Jünglings much help."

"Tell me, please! I like to know all about the---the boys," said Jo
eagerly.

"That is so kind, I gladly tell you. My friends find for me a place in a
college, where I teach as at home, and earn enough to make the way
smooth for Franz and Emil. For this I should be grateful, should I not?"

"Indeed you should. How splendid it will be to have you doing what you
like, and be able to see you often, and the boys!" cried Jo, clinging to
the lads as an excuse for the satisfaction she could not help betraying.

"Ah! but we shall not meet often, I fear; this place is at the West."

"So far away!" and Jo left her skirts to their fate, as if it
didn\textquotesingle t matter now what became of her clothes or herself.

Mr. Bhaer could read several languages, but he had not learned to read
women yet. He flattered himself that he knew Jo pretty well, and was,
therefore, much amazed by the contradictions of voice, face, and manner,
which she showed him in rapid succession that day, for she was in half a
dozen different moods in the course of half an hour. When she met him
she looked surprised, though it was impossible to help suspecting that
she had come for that express purpose. When he offered her his arm, she
took it with a look that filled him with delight; but when he asked if
she missed him, she gave such a chilly, formal reply that despair fell
upon him. On learning his good fortune she almost clapped her hands: was
the joy all for the boys? Then, on hearing his destination, she said,
"So far away!" in a tone of despair that lifted him on to a pinnacle of
hope; but the next minute she tumbled him down again by observing, like
one entirely absorbed in the matter,---

"Here\textquotesingle s the place for my errands; will you come in? It
won\textquotesingle t take long."

Jo rather prided herself upon her shopping capabilities, and
particularly wished to impress her escort with the neatness and despatch
with which she would accomplish the business. But, owing to the flutter
she was in, everything went amiss; she upset the tray of needles, forgot
the silesia was to be "twilled" till it was cut off, gave the wrong
change, and covered herself with confusion by asking for lavender ribbon
at the calico counter. Mr. Bhaer stood by, watching her blush and
blunder; and, as he watched, his own bewilderment seemed to subside, for
he was beginning to see that on some occasions women, like dreams, go by
contraries.

When they came out, he put the parcel under his arm with a more cheerful
aspect, and splashed through the puddles as if he rather enjoyed it, on
the whole.

"Should we not do a little what you call shopping for the babies, and
haf a farewell feast to-night if I go for my last call at your so
pleasant home?" he asked, stopping before a window full of fruit and
flowers.

"What will we buy?" said Jo, ignoring the latter part of his speech, and
sniffing the mingled odors with an affectation of delight as they went
in.

"May they haf oranges and figs?" asked Mr. Bhaer, with a paternal air.

"They eat them when they can get them."

"Do you care for nuts?"

"Like a squirrel."

"Hamburg grapes; yes, we shall surely drink to the Fatherland in those?"

Jo frowned upon that piece of extravagance, and asked why he
didn\textquotesingle t buy a frail of dates, a cask of raisins, and a
bag of almonds, and done with it? Whereat Mr. Bhaer confiscated her
purse, produced his own, and finished the marketing by buying several
pounds of grapes, a pot of rosy daisies, and a pretty jar of honey, to
be regarded in the light of a demijohn. Then, distorting his pockets
with the knobby bundles, and giving her the flowers to hold, he put up
the old umbrella, and they travelled on again.

"Miss Marsch, I haf a great favor to ask of you," began the Professor,
after a moist promenade of half a block.

"Yes, sir;" and Jo\textquotesingle s heart began to beat so hard she was
afraid he would hear it.

"I am bold to say it in spite of the rain, because so short a time
remains to me."

"Yes, sir;" and Jo nearly crushed the small flower-pot with the sudden
squeeze she gave it.

"I wish to get a little dress for my Tina, and I am too stupid to go
alone. Will you kindly gif me a word of taste and help?"

"Yes, sir;" and Jo felt as calm and cool, all of a sudden, as if she had
stepped into a refrigerator.

"Perhaps also a shawl for Tina\textquotesingle s mother, she is so poor
and sick, and the husband is such a care. Yes, yes, a thick, warm shawl
would be a friendly thing to take the little mother."

"I\textquotesingle ll do it with pleasure, Mr. Bhaer.
I\textquotesingle m going very fast and he\textquotesingle s getting
dearer every minute," added Jo to herself; then, with a mental shake,
she entered into the business with an energy which was pleasant to
behold.

Mr. Bhaer left it all to her, so she chose a pretty gown for Tina, and
then ordered out the shawls. The clerk, being a married man,
condescended to take an interest in the couple, who appeared to be
shopping for their family.

"Your lady may prefer this; it\textquotesingle s a superior article, a
most desirable color, quite chaste and genteel," he said, shaking out a
comfortable gray shawl, and throwing it over Jo\textquotesingle s
shoulders.

\protect\phantomsection\label{6672479776654687619_37106-h-8.htm.xhtml_b195.png}{}
\pandocbounded{\includegraphics[keepaspectratio]{303483661336987339_b195.png}}

"Does this suit you, Mr. Bhaer?" she asked, turning her back to him, and
feeling deeply grateful for the chance of hiding her face.

"Excellently well; we will haf it," answered the Professor, smiling to
himself as he paid for it, while Jo continued to rummage the counters
like a confirmed bargain-hunter.

"Now shall we go home?" he asked, as if the words were very pleasant to
him.

"Yes; it\textquotesingle s late, and I\textquotesingle m \emph{so}
tired." Jo\textquotesingle s voice was more pathetic than she knew; for
now the sun seemed to have gone in as suddenly as it came out, the world
grew muddy and miserable again, and for the first time she discovered
that her feet were cold, her head ached, and that her heart was colder
than the former, fuller of pain than the latter. Mr. Bhaer was going
away; he only cared for her as a friend; it was all a mistake, and the
sooner it was over the better. With this idea in her head, she hailed an
approaching omnibus with such a hasty gesture that the daisies flew out
of the pot and were badly damaged.

"This is not our omniboos," said the Professor, waving the loaded
vehicle away, and stopping to pick up the poor little flowers.

"I beg your pardon, I didn\textquotesingle t see the name distinctly.
Never mind, I can walk. I\textquotesingle m used to plodding in the
mud," returned Jo, winking hard, because she would have died rather than
openly wipe her eyes.

Mr. Bhaer saw the drops on her cheeks, though she turned her head away;
the sight seemed to touch him very much, for, suddenly stooping down, he
asked in a tone that meant a great deal,---

"Heart\textquotesingle s dearest, why do you cry?"

Now, if Jo had not been new to this sort of thing she would have said
she wasn\textquotesingle t crying, had a cold in her head, or told any
other feminine fib proper to the occasion; instead of which that
undignified creature answered, with an irrepressible sob,---

"Because you are going away."

"Ach, mein Gott, that is \emph{so} good!" cried Mr. Bhaer, managing to
clasp his hands in spite of the umbrella and the bundles. "Jo, I haf
nothing but much love to gif you; I came to see if you could care for
it, and I waited to be sure that I was something more than a friend. Am
I? Can you make a little place in your heart for old Fritz?" he added,
all in one breath.

"Oh, yes!" said Jo; and he was quite satisfied, for she folded both
hands over his arm, and looked up at him with an expression that plainly
showed how happy she would be to walk through life beside him, even
though she had no better shelter than the old umbrella, if he carried
it.

It was certainly proposing under difficulties, for, even if he had
desired to do so, Mr. Bhaer could not go down upon his knees, on account
of the mud; neither could he offer Jo his hand, except figuratively, for
both were full; much less could he indulge in tender demonstrations in
the open street, though he was near it: so the only way in which he
could express his rapture was to look at her, with an expression which
glorified his face to such a degree that there actually seemed to be
little rainbows in the drops that sparkled on his beard. If he had not
loved Jo very much, I don\textquotesingle t think he could have done it
\emph{then}, for she looked far from lovely, with her skirts in a
deplorable state, her rubber boots splashed to the ankle, and her bonnet
a ruin. Fortunately, Mr. Bhaer considered her the most beautiful woman
living, and she found him more "Jove-like" than ever, though his
hat-brim was quite limp with the little rills trickling thence upon his
shoulders (for he held the umbrella all over Jo), and every finger of
his gloves needed mending.

Passers-by probably thought them a pair of harmless lunatics, for they
entirely forgot to hail a \textquotesingle bus, and strolled leisurely
along, oblivious of deepening dusk and fog. Little they cared what
anybody thought, for they were enjoying the happy hour that seldom comes
but once in any life, the magical moment which bestows youth on the old,
beauty on the plain, wealth on the poor, and gives human hearts a
foretaste of heaven. The Professor looked as if he had conquered a
kingdom, and the world had nothing more to offer him in the way of
bliss; while Jo trudged beside him, feeling as if her place had always
been there, and wondering how she ever could have chosen any other lot.
Of course, she was the first to speak---intelligibly, I mean, for the
emotional remarks which followed her impetuous "Oh, yes!" were not of a
coherent or reportable character.

"Friedrich, why didn\textquotesingle t you---"

"Ah, heaven, she gifs me the name that no one speaks since Minna died!"
cried the Professor, pausing in a puddle to regard her with grateful
delight.

"I always call you so to myself---I forgot; but I won\textquotesingle t,
unless you like it."

"Like it? it is more sweet to me than I can tell. Say
\textquotesingle thou,\textquotesingle{} also, and I shall say your
language is almost as beautiful as mine."

"Isn\textquotesingle t \textquotesingle thou\textquotesingle{} a little
sentimental?" asked Jo, privately thinking it a lovely monosyllable.

"Sentimental? Yes. Thank Gott, we Germans believe in sentiment, and keep
ourselves young mit it. Your English
\textquotesingle you\textquotesingle{} is so cold, say
\textquotesingle thou,\textquotesingle{} heart\textquotesingle s
dearest, it means so much to me," pleaded Mr. Bhaer, more like a
romantic student than a grave professor.

"Well, then, why didn\textquotesingle t thou tell me all this sooner?"
asked Jo bashfully.

"Now I shall haf to show thee all my heart, and I so gladly will,
because thou must take care of it hereafter. See, then, my Jo,---ah, the
dear, funny little name!---I had a wish to tell something the day I said
good-by, in New York; but I thought the handsome friend was betrothed to
thee, and so I spoke not. Wouldst thou have said
\textquotesingle Yes,\textquotesingle{} then, if I \emph{had} spoken?"

"I don\textquotesingle t know; I\textquotesingle m afraid not, for I
didn\textquotesingle t have any heart just then."

"Prut! that I do not believe. It was asleep till the fairy prince came
through the wood, and waked it up. Ah, well, \textquotesingle Die erste
Liebe ist die beste;\textquotesingle{} but that I should not expect."

"Yes, the first love \emph{is} the best; so be contented, for I never
had another. Teddy was only a boy, and soon got over his little fancy,"
said Jo, anxious to correct the Professor\textquotesingle s mistake.

"Good! then I shall rest happy, and be sure that thou givest me all. I
haf waited so long, I am grown selfish, as thou wilt find, Professorin."

"I like that," cried Jo, delighted with her new name. "Now tell me what
brought you, at last, just when I most wanted you?"

"This;" and Mr. Bhaer took a little worn paper out of his
waistcoat-pocket.

Jo unfolded it, and looked much abashed, for it was one of her own
contributions to a paper that paid for poetry, which accounted for her
sending it an occasional attempt.

"How could that bring you?" she asked, wondering what he meant.

"I found it by chance; I knew it by the names and the initials, and in
it there was one little verse that seemed to call me. Read and find him;
I will see that you go not in the wet."

Jo obeyed, and hastily skimmed through the lines which she had
christened---

\hfill\break

"IN THE GARRET.

"Four little chests all in a row,

Dim with dust, and worn by time,

All fashioned and filled, long ago,

By children now in their prime.

Four little keys hung side by side,

With faded ribbons, brave and gay

When fastened there, with childish pride,

Long ago, on a rainy day.

Four little names, one on each lid,

Carved out by a boyish hand,

And underneath there lieth hid

Histories of the happy band

Once playing here, and pausing oft

To hear the sweet refrain,

That came and went on the roof aloft,

In the falling summer rain.

\hfill\break

"\textquotesingle Meg\textquotesingle{} on the first lid, smooth and
fair.

I look in with loving eyes,

For folded here, with well-known care,

A goodly gathering lies,

The record of a peaceful life,---

Gifts to gentle child and girl,

A bridal gown, lines to a wife,

A tiny shoe, a baby curl.

No toys in this first chest remain,

For all are carried away,

In their old age, to join again

In another small Meg\textquotesingle s play.

Ah, happy mother! well I know

You hear, like a sweet refrain,

Lullabies ever soft and low

In the falling summer rain.

\hfill\break

"\textquotesingle Jo\textquotesingle{} on the next lid, scratched and
worn,

And within a motley store

Of headless dolls, of school-books torn,

Birds and beasts that speak no more;

Spoils brought home from the fairy ground

Only trod by youthful feet,

Dreams of a future never found,

Memories of a past still sweet;

Half-writ poems, stories wild,

April letters, warm and cold,

Diaries of a wilful child,

Hints of a woman early old;

A woman in a lonely home,

Hearing, like a sad refrain,---

\textquotesingle Be worthy love, and love will come,\textquotesingle{}

In the falling summer rain.

\hfill\break

"My Beth! the dust is always swept

From the lid that bears your name,

As if by loving eyes that wept,

By careful hands that often came.

Death canonized for us one saint,

Ever less human than divine,

And still we lay, with tender plaint,

Relics in this household shrine.---

The silver bell, so seldom rung,

The little cap which last she wore,

The fair, dead Catherine that hung

By angels borne above her door;

The songs she sang, without lament,

In her prison-house of pain,

Forever are they sweetly blent

With the falling summer rain.

\hfill\break

"Upon the last lid\textquotesingle s polished field---

Legend now both fair and true---

A gallant knight bears on his shield,

\textquotesingle Amy,\textquotesingle{} in letters gold and blue.

Within lie snoods that bound her hair,

Slippers that have danced their last,

Faded flowers laid by with care,

Fans whose airy toils are past;

Gay valentines, all ardent flames,

Trifles that have borne their part

In girlish hopes and fears and shames,---

The record of a maiden heart

Now learning fairer, truer spells,

Hearing, like a blithe refrain,

The silver sound of bridal bells

In the falling summer rain.

\hfill\break

"Four little chests all in a row,

Dim with dust, and worn by time,

Four women, taught by weal and woe

To love and labor in their prime.

Four sisters, parted for an hour,

None lost, one only gone before,

Made by love\textquotesingle s immortal power,

Nearest and dearest evermore.

Oh, when these hidden stores of ours

Lie open to the Father\textquotesingle s sight,

May they be rich in golden hours,

Deeds that show fairer for the light,

Lives whose brave music long shall ring,

Like a spirit-stirring strain,

Souls that shall gladly soar and sing

In the long sunshine after rain.

"J. M."

"It\textquotesingle s very bad poetry, but I felt it when I wrote it,
one day when I was very lonely, and had a good cry on a rag-bag. I never
thought it would go where it could tell tales," said Jo, tearing up the
verses the Professor had treasured so long.

"Let it go, it has done its duty, and I will haf a fresh one when I read
all the brown book in which she keeps her little secrets," said Mr.
Bhaer, with a smile, as he watched the fragments fly away on the wind.
"Yes," he added earnestly, "I read that, and I think to myself,
\textquotesingle She has a sorrow, she is lonely, she would find comfort
in true love. I haf a heart full, full for her; shall I not go and say,
\textquotesingle If this is not too poor a thing to gif for what I shall
hope to receive, take it in Gott\textquotesingle s
name?\textquotesingle"

"And so you came to find that it was not too poor, but the one precious
thing I needed," whispered Jo.

"I had no courage to think that at first, heavenly kind as was your
welcome to me. But soon I began to hope, and then I said,
\textquotesingle I will haf her if I die for it,\textquotesingle{} and
so I will!" cried Mr. Bhaer, with a defiant nod, as if the walls of mist
closing round them were barriers which he was to surmount or valiantly
knock down.

Jo thought that was splendid, and resolved to be worthy of her knight,
though he did not come prancing on a charger in gorgeous array.

"What made you stay away so long?" she asked presently, finding it so
pleasant to ask confidential questions and get delightful answers that
she could not keep silent.

"It was not easy, but I could not find the heart to take you from that
so happy home until I could haf a prospect of one to give you, after
much time, perhaps, and hard work. How could I ask you to gif up so much
for a poor old fellow, who has no fortune but a little learning?"

"I\textquotesingle m glad you \emph{are} poor; I
couldn\textquotesingle t bear a rich husband," said Jo decidedly,
adding, in a softer tone, "Don\textquotesingle t fear poverty;
I\textquotesingle ve known it long enough to lose my dread, and be happy
working for those I love; and don\textquotesingle t call yourself
old,---forty is the prime of life. I couldn\textquotesingle t help
loving you if you were seventy!"

The Professor found that so touching that he would have been glad of his
handkerchief, if he could have got at it; as he
couldn\textquotesingle t, Jo wiped his eyes for him, and said, laughing,
as she took away a bundle or two,---

"I may be strong-minded, but no one can say I\textquotesingle m out of
my sphere now, for woman\textquotesingle s special mission is supposed
to be drying tears and bearing burdens. I\textquotesingle m to carry my
share, Friedrich, and help to earn the home. Make up your mind to that,
or I\textquotesingle ll never go," she added resolutely, as he tried to
reclaim his load.

"We shall see. Haf you patience to wait a long time, Jo? I must go away
and do my work alone. I must help my boys first, because, even for you,
I may not break my word to Minna. Can you forgif that, and be happy
while we hope and wait?"

"Yes, I know I can; for we love one another, and that makes all the rest
easy to bear. I have my duty, also, and my work. I
couldn\textquotesingle t enjoy myself if I neglected them even for you,
so there\textquotesingle s no need of hurry or impatience. You can do
your part out West, I can do mine here, and both be happy hoping for the
best, and leaving the future to be as God wills."

"Ah! thou gifest me such hope and courage, and I haf nothing to gif back
but a full heart and these empty hands," cried the Professor, quite
overcome.

\protect\phantomsection\label{6672479776654687619_37106-h-8.htm.xhtml_b196.png}{}
\pandocbounded{\includegraphics[keepaspectratio]{303483661336987339_b196.png}}

Jo never, never would learn to be proper; for when he said that as they
stood upon the steps, she just put both hands into his, whispering
tenderly, "Not empty now;" and, stooping down, kissed her Friedrich
under the umbrella. It was dreadful, but she would have done it if the
flock of draggle-tailed sparrows on the hedge had been human beings, for
she was very far gone indeed, and quite regardless of everything but her
own happiness. Though it came in such a very simple guise, that was the
crowning moment of both their lives, when, turning from the night and
storm and loneliness to the household light and warmth and peace waiting
to receive them, with a glad "Welcome home!" Jo led her lover in, and
shut the door.

\protect\phantomsection\label{6672479776654687619_37106-h-8.htm.xhtml_b197.png}{}
\pandocbounded{\includegraphics[keepaspectratio]{303483661336987339_b197.png}}

\begin{center}\rule{0.5\linewidth}{0.5pt}\end{center}

\subsection{XLVII. Harvest
Time.}\label{6672479776654687619_37106-h-8.htm.xhtml_pgepubid00049}

\protect\phantomsection\label{6672479776654687619_37106-h-8.htm.xhtml_b198.png}{}
\pandocbounded{\includegraphics[keepaspectratio]{303483661336987339_b198.png}}

\protect\phantomsection\label{6672479776654687619_37106-h-8.htm.xhtml_XLVII}{}\hyperref[6672479776654687619_37106-h-0.htm.xhtml_contents2b]{XLVII.}

HARVEST TIME.

{For} a year Jo and her Professor worked and waited, hoped and loved,
met occasionally, and wrote such voluminous letters that the rise in the
price of paper was accounted for, Laurie said. The second year began
rather soberly, for their prospects did not brighten, and Aunt March
died suddenly. But when their first sorrow was over,---for they loved
the old lady in spite of her sharp tongue,---they found they had cause
for rejoicing, for she had left Plumfield to Jo, which made all sorts of
joyful things possible.

"It\textquotesingle s a fine old place, and will bring a handsome sum;
for of course you intend to sell it," said Laurie, as they were all
talking the matter over, some weeks later.

"No, I don\textquotesingle t," was Jo\textquotesingle s decided answer,
as she petted the fat poodle, whom she had adopted, out of respect to
his former mistress.

"You don\textquotesingle t mean to live there?"

"Yes, I do."

"But, my dear girl, it\textquotesingle s an immense house, and will take
a power of money to keep it in order. The garden and orchard alone need
two or three men, and farming isn\textquotesingle t in
Bhaer\textquotesingle s line, I take it."

"He\textquotesingle ll try his hand at it there, if I propose it."

"And you expect to live on the produce of the place? Well, that sounds
paradisiacal, but you\textquotesingle ll find it desperate hard work."

"The crop we are going to raise is a profitable one;" and Jo laughed.

"Of what is this fine crop to consist, ma\textquotesingle am?"

"Boys. I want to open a school for little lads,---a good, happy,
homelike school, with me to take care of them, and Fritz to teach them."

"There\textquotesingle s a truly Joian plan for you!
Isn\textquotesingle t that just like her?" cried Laurie, appealing to
the family, who looked as much surprised as he.

"I like it," said Mrs. March decidedly.

"So do I," added her husband, who welcomed the thought of a chance for
trying the Socratic method of education on modern youth.

"It will be an immense care for Jo," said Meg, stroking the head of her
one all-absorbing son.

"Jo can do it, and be happy in it. It\textquotesingle s a splendid idea.
Tell us all about it," cried Mr. Laurence, who had been longing to lend
the lovers a hand, but knew that they would refuse his help.

"I knew you\textquotesingle d stand by me, sir. Amy does too---I see it
in her eyes, though she prudently waits to turn it over in her mind
before she speaks. Now, my dear people," continued Jo earnestly, "just
understand that this isn\textquotesingle t a new idea of mine, but a
long-cherished plan. Before my Fritz came, I used to think how, when
I\textquotesingle d made my fortune, and no one needed me at home,
I\textquotesingle d hire a big house, and pick up some poor, forlorn
little lads, who hadn\textquotesingle t any mothers, and take care of
them, and make life jolly for them before it was too late. I see so many
going to ruin, for want of help at the right minute; I love so to do
anything for them; I seem to feel their wants, and sympathize with their
troubles, and, oh, I should \emph{so} like to be a mother to them!"

Mrs. March held out her hand to Jo, who took it, smiling, with tears in
her eyes, and went on in the old enthusiastic way, which they had not
seen for a long while.

"I told my plan to Fritz once, and he said it was just what he would
like, and agreed to try it when we got rich. Bless his dear heart,
he\textquotesingle s been doing it all his life,---helping poor boys, I
mean, not getting rich; that he\textquotesingle ll never be; money
doesn\textquotesingle t stay in his pocket long enough to lay up any.
But now, thanks to my good old aunt, who loved me better than I ever
deserved, \emph{I\textquotesingle m} rich, at least I feel so, and we
can live at Plumfield perfectly well, if we have a flourishing school.
It\textquotesingle s just the place for boys, the house is big, and the
furniture strong and plain. There\textquotesingle s plenty of room for
dozens inside, and splendid grounds outside. They could help in the
garden and orchard: such work is healthy, isn\textquotesingle t it, sir?
Then Fritz can train and teach in his own way, and father will help him.
I can feed and nurse and pet and scold them; and mother will be my
stand-by. I\textquotesingle ve always longed for lots of boys, and never
had enough; now I can fill the house full, and revel in the little dears
to my heart\textquotesingle s content. Think what luxury,---Plumfield my
own, and a wilderness of boys to enjoy it with me!"

As Jo waved her hands, and gave a sigh of rapture, the family went off
into a gale of merriment, and Mr. Laurence laughed till they thought
he\textquotesingle d have an apoplectic fit.

"I don\textquotesingle t see anything funny," she said gravely, when she
could be heard. "Nothing could be more natural or proper than for my
Professor to open a school, and for me to prefer to reside on my own
estate."

"She is putting on airs already," said Laurie, who regarded the idea in
the light of a capital joke. "But may I inquire how you intend to
support the establishment? If all the pupils are little ragamuffins,
I\textquotesingle m afraid your crop won\textquotesingle t be profitable
in a worldly sense, Mrs. Bhaer."

"Now don\textquotesingle t be a wet-blanket, Teddy. Of course I shall
have rich pupils, also,---perhaps begin with such altogether; then, when
I\textquotesingle ve got a start, I can take a ragamuffin or two, just
for a relish. Rich people\textquotesingle s children often need care and
comfort, as well as poor. I\textquotesingle ve seen unfortunate little
creatures left to servants, or backward ones pushed forward, when
it\textquotesingle s real cruelty. Some are naughty through
mismanagement or neglect, and some lose their mothers. Besides, the best
have to get through the hobbledehoy age, and that\textquotesingle s the
very time they need most patience and kindness. People laugh at them,
and hustle them about, try to keep them out of sight, and expect them to
turn, all at once, from pretty children into fine young men. They
don\textquotesingle t complain much,---plucky little souls,---but they
feel it. I\textquotesingle ve been through something of it, and I know
all about it. I\textquotesingle ve a special interest in such young
bears, and like to show them that I see the warm, honest, well-meaning
boys\textquotesingle{} hearts, in spite of the clumsy arms and legs and
the topsy-turvy heads. I\textquotesingle ve had experience, too, for
haven\textquotesingle t I brought up one boy to be a pride and honor to
his family?"

"I\textquotesingle ll testify that you tried to do it," said Laurie,
with a grateful look.

"And I\textquotesingle ve succeeded beyond my hopes; for here you are, a
steady, sensible business man, doing heaps of good with your money, and
laying up the blessings of the poor, instead of dollars. But you are not
merely a business man: you love good and beautiful things, enjoy them
yourself, and let others go halves, as you always did in the old times.
I \emph{am} proud of you, Teddy, for you get better every year, and
every one feels it, though you won\textquotesingle t let them say so.
Yes, and when I have my flock, I\textquotesingle ll just point to you,
and say, \textquotesingle There\textquotesingle s your model, my
lads.\textquotesingle"

Poor Laurie didn\textquotesingle t know where to look; for, man though
he was, something of the old bashfulness came over him as this burst of
praise made all faces turn approvingly upon him.

"I say, Jo, that\textquotesingle s rather too much," he began, just in
his old boyish way. "You have all done more for me than I can ever thank
you for, except by doing my best not to disappoint you. You have rather
cast me off lately, Jo, but I\textquotesingle ve had the best of help,
nevertheless; so, if I\textquotesingle ve got on at all, you may thank
these two for it;" and he laid one hand gently on his
grandfather\textquotesingle s white head, the other on
Amy\textquotesingle s golden one, for the three were never far apart.

"I do think that families are the most beautiful things in all the
world!" burst out Jo, who was in an unusually uplifted frame of mind
just then. "When I have one of my own, I hope it will be as happy as the
three I know and love the best. If John and my Fritz were only here, it
would be quite a little heaven on earth," she added more quietly. And
that night, when she went to her room, after a blissful evening of
family counsels, hopes, and plans, her heart was so full of happiness
that she could only calm it by kneeling beside the empty bed always near
her own, and thinking tender thoughts of Beth.

It was a very astonishing year altogether, for things seemed to happen
in an unusually rapid and delightful manner. Almost before she knew
where she was, Jo found herself married and settled at Plumfield. Then a
family of six or seven boys sprung up like mushrooms, and flourished
surprisingly, poor boys as well as rich; for Mr. Laurence was
continually finding some touching case of destitution, and begging the
Bhaers to take pity on the child, and he would gladly pay a trifle for
its support. In this way the sly old gentleman got round proud Jo, and
furnished her with the style of boy in which she most delighted.

Of course it was up-hill work at first, and Jo made queer mistakes; but
the wise Professor steered her safely into calmer waters, and the most
rampant ragamuffin was conquered in the end. How Jo did enjoy her
"wilderness of boys," and how poor, dear Aunt March would have lamented
had she been there to see the sacred precincts of prim, well-ordered
Plumfield overrun with Toms, Dicks, and Harrys! There was a sort of
poetic justice about it, after all, for the old lady had been the terror
of the boys for miles round; and now the exiles feasted freely on
forbidden plums, kicked up the gravel with profane boots unreproved, and
played cricket in the big field where the irritable "cow with a crumpled
horn" used to invite rash youths to come and be tossed. It became a sort
of boys\textquotesingle{} paradise, and Laurie suggested that it should
be called the "Bhaer-garten," as a compliment to its master and
appropriate to its inhabitants.

It never was a fashionable school, and the Professor did not lay up a
fortune; but it \emph{was} just what Jo intended it to be,---"a happy,
homelike place for boys, who needed teaching, care, and kindness." Every
room in the big house was soon full; every little plot in the garden
soon had its owner; a regular menagerie appeared in barn and shed, for
pet animals were allowed; and, three times a day, Jo smiled at her Fritz
from the head of a long table lined on either side with rows of happy
young faces, which all turned to her with affectionate eyes, confiding
words, and grateful hearts, full of love for "Mother Bhaer." She had
boys enough now, and did not tire of them, though they were not angels,
by any means, and some of them caused both Professor and Professorin
much trouble and anxiety. But her faith in the good spot which exists in
the heart of the naughtiest, sauciest, most tantalizing little
ragamuffin gave her patience, skill, and, in time, success; for no
mortal boy could hold out long with Father Bhaer shining on him as
benevolently as the sun, and Mother Bhaer forgiving him seventy times
seven. Very precious to Jo was the friendship of the lads; their
penitent sniffs and whispers after wrong-doing; their droll or touching
little confidences; their pleasant enthusiasms, hopes, and plans; even
their misfortunes, for they only endeared them to her all the more.
There were slow boys and bashful boys; feeble boys and riotous boys;
boys that lisped and boys that stuttered; one or two lame ones; and a
merry little quadroon, who could not be taken in elsewhere, but who was
welcome to the "Bhaer-garten," though some people predicted that his
admission would ruin the school.

Yes; Jo was a very happy woman there, in spite of hard work, much
anxiety, and a perpetual racket. She enjoyed it heartily, and found the
applause of her boys more satisfying than any praise of the world; for
now she told no stories except to her flock of enthusiastic believers
and admirers. As the years went on, two little lads of her own came to
increase her happiness,---Rob, named for grandpa, and Teddy, a
happy-go-lucky baby, who seemed to have inherited his
papa\textquotesingle s sunshiny temper as well as his
mother\textquotesingle s lively spirit. How they ever grew up alive in
that whirlpool of boys was a mystery to their grandma and aunts; but
they flourished like dandelions in spring, and their rough nurses loved
and served them well.

There were a great many holidays at Plumfield, and one of the most
delightful was the yearly apple-picking; for then the Marches,
Laurences, Brookes, and Bhaers turned out in full force, and made a day
of it. Five years after Jo\textquotesingle s wedding, one of these
fruitful festivals occurred,---a mellow October day, when the air was
full of an exhilarating freshness which made the spirits rise, and the
blood dance healthily in the veins. The old orchard wore its holiday
attire; golden-rod and asters fringed the mossy walls; grasshoppers
skipped briskly in the sere grass, and crickets chirped like fairy
pipers at a feast; squirrels were busy with their small harvesting;
birds twittered their adieux from the alders in the lane; and every tree
stood ready to send down its shower of red or yellow apples at the first
shake. Everybody was there; everybody laughed and sang, climbed up and
tumbled down; everybody declared that there never had been such a
perfect day or such a jolly set to enjoy it; and every one gave
themselves up to the simple pleasures of the hour as freely as if there
were no such things as care or sorrow in the world.

Mr. March strolled placidly about, quoting Tusser, Cowley, and Columella
to Mr. Laurence, while enjoying---

"The gentle apple\textquotesingle s winey juice."

The Professor charged up and down the green aisles like a stout Teutonic
knight, with a pole for a lance, leading on the boys, who made a hook
and ladder company of themselves, and performed wonders in the way of
ground and lofty tumbling. Laurie devoted himself to the little ones,
rode his small daughter in a bushel-basket, took Daisy up among the
birds\textquotesingle{} nests, and kept adventurous Rob from breaking
his neck. Mrs. March and Meg sat among the apple piles like a pair of
Pomonas, sorting the contributions that kept pouring in; while Amy, with
a beautiful motherly expression in her face, sketched the various
groups, and watched over one pale lad, who sat adoring her with his
little crutch beside him.

\protect\phantomsection\label{6672479776654687619_37106-h-8.htm.xhtml_b199.png}{}
\pandocbounded{\includegraphics[keepaspectratio]{303483661336987339_b199.png}}

Jo was in her element that day, and rushed about, with her gown pinned
up, her hat anywhere but on her head, and her baby tucked under her arm,
ready for any lively adventure which might turn up. Little Teddy bore a
charmed life, for nothing ever happened to him, and Jo never felt any
anxiety when he was whisked up into a tree by one lad, galloped off on
the back of another, or supplied with sour russets by his indulgent
papa, who labored under the Germanic delusion that babies could digest
anything, from pickled cabbage to buttons, nails, and their own small
shoes. She knew that little Ted would turn up again in time, safe and
rosy, dirty and serene, and she always received him back with a hearty
welcome, for Jo loved her babies tenderly.

At four o\textquotesingle clock a lull took place, and baskets remained
empty, while the apple-pickers rested, and compared rents and bruises.
Then Jo and Meg, with a detachment of the bigger boys, set forth the
supper on the grass, for an out-of-door tea was always the crowning joy
of the day. The land literally flowed with milk and honey on such
occasions, for the lads were not required to sit at table, but allowed
to partake of refreshment as they liked,---freedom being the sauce best
beloved by the boyish soul. They availed themselves of the rare
privilege to the fullest extent, for some tried the pleasing experiment
of drinking milk while standing on their heads, others lent a charm to
leap-frog by eating pie in the pauses of the game, cookies were sown
broadcast over the field, and apple-turnovers roosted in the trees like
a new style of bird. The little girls had a private tea-party, and Ted
roved among the edibles at his own sweet will.

When no one could eat any more, the Professor proposed the first regular
toast, which was always drunk at such times,---"Aunt March, God bless
her!" A toast heartily given by the good man, who never forgot how much
he owed her, and quietly drunk by the boys, who had been taught to keep
her memory green.

"Now, grandma\textquotesingle s sixtieth birthday! Long life to her,
with three times three!"

That was given with a will, as you may well believe; and the cheering
once begun, it was hard to stop it. Everybody\textquotesingle s health
was proposed, from Mr. Laurence, who was considered their special
patron, to the astonished guinea-pig, who had strayed from its proper
sphere in search of its young master. Demi, as the oldest grandchild,
then presented the queen of the day with various gifts, so numerous that
they were transported to the festive scene in a wheelbarrow. Funny
presents, some of them, but what would have been defects to other eyes
were ornaments to grandma\textquotesingle s,---for the
children\textquotesingle s gifts were all their own. Every stitch
Daisy\textquotesingle s patient little fingers had put into the
handkerchiefs she hemmed was better than embroidery to Mrs. March;
Demi\textquotesingle s shoe-box was a miracle of mechanical skill,
though the cover wouldn\textquotesingle t shut; Rob\textquotesingle s
footstool had a wiggle in its uneven legs, that she declared was very
soothing; and no page of the costly book Amy\textquotesingle s child
gave her was so fair as that on which appeared, in tipsy capitals, the
words,---"To dear Grandma, from her little Beth."

During this ceremony the boys had mysteriously disappeared; and, when
Mrs. March had tried to thank her children, and broken down, while Teddy
wiped her eyes on his pinafore, the Professor suddenly began to sing.
Then, from above him, voice after voice took up the words, and from tree
to tree echoed the music of the unseen choir, as the boys sung, with all
their hearts, the little song Jo had written, Laurie set to music, and
the Professor trained his lads to give with the best effect. This was
something altogether new, and it proved a grand success; for Mrs. March
couldn\textquotesingle t get over her surprise, and insisted on shaking
hands with every one of the featherless birds, from tall Franz and Emil
to the little quadroon, who had the sweetest voice of all.

After this, the boys dispersed for a final lark, leaving Mrs. March and
her daughters under the festival tree.

\protect\phantomsection\label{6672479776654687619_37106-h-8.htm.xhtml_b200.png}{}
\pandocbounded{\includegraphics[keepaspectratio]{303483661336987339_b200.png}}\\
\protect\phantomsection\label{6672479776654687619_37106-h-8.htm.xhtml_ebm_caption6}{
"Leaving Mrs. March and her daughters under the festival tree."---Page
583}

"I don\textquotesingle t think I ever ought to call myself
\textquotesingle Unlucky Jo\textquotesingle{} again, when my greatest
wish has been so beautifully gratified," said Mrs. Bhaer, taking
Teddy\textquotesingle s little fist out of the milk-pitcher, in which he
was rapturously churning.

"And yet your life is very different from the one you pictured so long
ago. Do you remember our castles in the air?" asked Amy, smiling as she
watched Laurie and John playing cricket with the boys.

"Dear fellows! It does my heart good to see them forget business, and
frolic for a day," answered Jo, who now spoke in a maternal way of all
mankind. "Yes, I remember; but the life I wanted then seems selfish,
lonely, and cold to me now. I haven\textquotesingle t given up the hope
that I may write a good book yet, but I can wait, and
I\textquotesingle m sure it will be all the better for such experiences
and illustrations as these;" and Jo pointed from the lively lads in the
distance to her father, leaning on the Professor\textquotesingle s arm,
as they walked to and fro in the sunshine, deep in one of the
conversations which both enjoyed so much, and then to her mother,
sitting enthroned among her daughters, with their children in her lap
and at her feet, as if all found help and happiness in the face which
never could grow old to them.

"My castle was the most nearly realized of all. I asked for splendid
things, to be sure, but in my heart I knew I should be satisfied, if I
had a little home, and John, and some dear children like these.
I\textquotesingle ve got them all, thank God, and am the happiest woman
in the world;" and Meg laid her hand on her tall boy\textquotesingle s
head, with a face full of tender and devout content.

"My castle is very different from what I planned, but I would not alter
it, though, like Jo, I don\textquotesingle t relinquish all my artistic
hopes, or confine myself to helping others fulfil their dreams of
beauty. I\textquotesingle ve begun to model a figure of baby, and Laurie
says it is the best thing I\textquotesingle ve ever done. I think so
myself, and mean to do it in marble, so that, whatever happens, I may at
least keep the image of my little angel."

As Amy spoke, a great tear dropped on the golden hair of the sleeping
child in her arms; for her one well-beloved daughter was a frail little
creature and the dread of losing her was the shadow over
Amy\textquotesingle s sunshine. This cross was doing much for both
father and mother, for one love and sorrow bound them closely together.
Amy\textquotesingle s nature was growing sweeter, deeper, and more
tender; Laurie was growing more serious, strong, and firm; and both were
learning that beauty, youth, good fortune, even love itself, cannot keep
care and pain, loss and sorrow, from the most blest; for---

"Into each life some rain must fall,

Some days must be dark and sad and dreary."

"She is growing better, I am sure of it, my dear. Don\textquotesingle t
despond, but hope and keep happy," said Mrs. March, as tender-hearted
Daisy stooped from her knee, to lay her rosy cheek against her little
cousin\textquotesingle s pale one.

"I never ought to, while I have you to cheer me up, Marmee, and Laurie
to take more than half of every burden," replied Amy warmly. "He never
lets me see his anxiety, but is so sweet and patient with me, so devoted
to Beth, and such a stay and comfort to me always, that I
can\textquotesingle t love him enough. So, in spite of my one cross, I
can say with Meg, \textquotesingle Thank God, I\textquotesingle m a
happy woman.\textquotesingle"

"There\textquotesingle s no need for me to say it, for every one can see
that I\textquotesingle m far happier than I deserve," added Jo, glancing
from her good husband to her chubby children, tumbling on the grass
beside her. "Fritz is getting gray and stout; I\textquotesingle m
growing as thin as a shadow, and am thirty; we never shall be rich, and
Plumfield may burn up any night, for that incorrigible Tommy Bangs
\emph{will} smoke sweet-fern cigars under the bed-clothes, though
he\textquotesingle s set himself afire three times already. But in spite
of these unromantic facts, I have nothing to complain of, and never was
so jolly in my life. Excuse the remark, but living among boys, I
can\textquotesingle t help using their expressions now and then."

"Yes, Jo, I think your harvest will be a good one," began Mrs. March,
frightening away a big black cricket that was staring Teddy out of
countenance.

"Not half so good as yours, mother. Here it is, and we never can thank
you enough for the patient sowing and reaping you have done," cried Jo,
with the loving impetuosity which she never could outgrow.

"I hope there will be more wheat and fewer tares every year," said Amy
softly.

"A large sheaf, but I know there\textquotesingle s room in your heart
for it, Marmee dear," added Meg\textquotesingle s tender voice.

Touched to the heart, Mrs. March could only stretch out her arms, as if
to gather children and grandchildren to herself, and say, with face and
voice full of motherly love, gratitude, and humility,---

"O, my girls, however long you may live, I never can wish you a greater
happiness than this!"

\protect\phantomsection\label{6672479776654687619_37106-h-8.htm.xhtml_b201.png}{}
\pandocbounded{\includegraphics[keepaspectratio]{303483661336987339_b201.png}}

\hfill\break
\hfill\break
\hfill\break
\hfill\break
\hfill\break
\hfill\break
\hfill\break
\hfill\break

\pandocbounded{\includegraphics[keepaspectratio]{303483661336987339_b202.png}}

\begin{center}\rule{0.5\linewidth}{0.5pt}\end{center}

\subsection{The Works of Louisa May
Alcott}\label{6672479776654687619_37106-h-8.htm.xhtml_pgepubid00050}

Louisa M. Alcott\textquotesingle s Writings

\begin{center}\rule{0.5\linewidth}{0.5pt}\end{center}

\protect\phantomsection\label{6672479776654687619_37106-h-8.htm.xhtml_alcottContents}
\begin{itemize}
\item
  {THE LITTLE WOMEN SERIES.}

  \begin{description}
  \tightlist
  \item[\protect\phantomsection\label{6672479776654687619_37106-h-8.htm.xhtml_id-748401321037488359}\href{http://www.gutenberg.org/ebooks/514}{Little
  Women;}]
  or Meg, Jo, Beth, and Amy.

  Illustrated. 16mo. \$1.50.
  \item[\protect\phantomsection\label{6672479776654687619_37106-h-8.htm.xhtml_id-7782933466195000528}\href{http://www.gutenberg.org/ebooks/2788}{Little
  Men.}]
  Life at Plumfield with Jo\textquotesingle s Boys.

  Illustrated. 16mo. \$1.50.
  \item[\protect\phantomsection\label{6672479776654687619_37106-h-8.htm.xhtml_id-979967502029679593}\href{http://www.gutenberg.org/ebooks/3499}{Jo\textquotesingle s
  Boys and How They Turned Out.}]
  A Sequel to "Little Men."

  Portrait of Author. 16mo. \$1.50.
  \item[\protect\phantomsection\label{6672479776654687619_37106-h-8.htm.xhtml_id-2908273491269169562}\href{http://www.gutenberg.org/ebooks/2787}{An
  Old-Fashioned Girl.}]
  Illustrated. 16mo. \$1.50.
  \item[\protect\phantomsection\label{6672479776654687619_37106-h-8.htm.xhtml_id-4913089920197718664}\href{http://www.gutenberg.org/ebooks/2726}{Eight
  Cousins;}]
  or, The Aunt-Hill.

  Illustrated. 16mo. \$1.50.
  \item[\protect\phantomsection\label{6672479776654687619_37106-h-8.htm.xhtml_id-7411460194094654903}\href{http://www.gutenberg.org/ebooks/2804}{Rose
  in Bloom.}]
  A Sequel to "Eight Cousins."

  Illustrated. 16mo. \$1.50.
  \item[\protect\phantomsection\label{6672479776654687619_37106-h-8.htm.xhtml_id-2647610106804844572}\href{http://www.gutenberg.org/ebooks/3795}{Under
  the Lilacs.}]
  Illustrated. 16mo. \$1.50.
  \item[\protect\phantomsection\label{6672479776654687619_37106-h-8.htm.xhtml_id-5168079675692591739}\href{http://www.gutenberg.org/ebooks/2786}{Jack
  and Jill.}]
  A Village Story.

  Illustrated. 16mo. \$1.50.
  \end{description}

  The above eight volumes, uniformly bound in cloth, gilt, in box,
  \$12.00.
\item
  {THE LITTLE WOMEN SERIES.} \emph{New Illustrated Edition.}

  Printed from new plates with new cover designs, and illustrated with
  84 full-page plates from drawings especially made for this edition by
  Reginald B. Birch, Alice Barber Stephens, Jessie Willcox Smith, and
  Harriet Roosevelt Richards. 8 vols. Crown 8vo. Decorated cloth, gilt,
  in box. \$16.00. Separately, \$2.00.
\item
  {THE SPINNING-WHEEL SERIES}

  \begin{itemize}
  \tightlist
  \item
    \protect\phantomsection\label{6672479776654687619_37106-h-8.htm.xhtml_id-5458839721589999178}\href{http://www.gutenberg.org/ebooks/36221}{Spinning-Wheel
    Stories.}
  \item
    \protect\phantomsection\label{6672479776654687619_37106-h-8.htm.xhtml_id-4824022405360122348}\href{http://www.gutenberg.org/ebooks/34920}{Silver
    Pitchers.}
  \item
    \protect\phantomsection\label{6672479776654687619_37106-h-8.htm.xhtml_id-201597026279407396}\href{http://www.gutenberg.org/ebooks/10360}{Proverb
    Stories.}
  \item
    \protect\phantomsection\label{6672479776654687619_37106-h-8.htm.xhtml_id-1834362957080139491}\href{http://www.gutenberg.org/ebooks/5830}{A
    Garland for Girls.}
  \end{itemize}

  4 vols. 16mo. Each, \$1.25. In box, \$5.00.
\item
  {THE SPINNING-WHEEL SERIES.} \emph{New Illustrated Edition.}

  Uniform in size with the Illustrated Edition of the Little Women
  Series. With 36 full-page plates by well-known artists. 4 vols. Crown
  8vo. Decorated cloth. In box, \$6.00. Separately, \$1.50.
\item
  {AUNT JO\textquotesingle S SCRAP-BAG}

  \begin{itemize}
  \tightlist
  \item
    \protect\phantomsection\label{6672479776654687619_37106-h-8.htm.xhtml_id-5465700865761979236}\href{http://www.gutenberg.org/ebooks/26041}{My
    Boys.}
  \item
    \protect\phantomsection\label{6672479776654687619_37106-h-8.htm.xhtml_id-3666124022233252501}\href{http://www.gutenberg.org/ebooks/22022}{Shawl-Straps.}
  \item
    Cupid and Chow-Chow.
  \item
    My Girls.
  \item
    \protect\phantomsection\label{6672479776654687619_37106-h-8.htm.xhtml_id-4584704215947509649}\href{http://www.gutenberg.org/ebooks/22234}{Jimmy\textquotesingle s
    Cruise in the Pinafore.}
  \item
    \protect\phantomsection\label{6672479776654687619_37106-h-8.htm.xhtml_id-5427435354055137834}\href{http://www.gutenberg.org/ebooks/27567}{An
    Old-Fashioned Thanksgiving.}
  \end{itemize}

  6 vols. 16mo. Illustrated. Each, \$1.00. In box, \$6.00.
\item
  {LULU\textquotesingle S LIBRARY}

  \begin{itemize}
  \tightlist
  \item
    \protect\phantomsection\label{6672479776654687619_37106-h-8.htm.xhtml_id-5765634543277580674}\href{http://www.gutenberg.org/ebooks/40682}{Volume
    1}
  \item
    \protect\phantomsection\label{6672479776654687619_37106-h-8.htm.xhtml_id-4539843575477618054}\href{http://www.gutenberg.org/ebooks/32357}{Volume
    2}
  \item
    \protect\phantomsection\label{6672479776654687619_37106-h-8.htm.xhtml_id-530706071221874052}\href{http://www.gutenberg.org/ebooks/40683}{Volume
    3}
  \end{itemize}

  3 vols. Each, \$1.00. The set uniformly bound in cloth, gilt, in box,
  \$3.00.
\item
  {NOVELS, ETC.}

  \begin{itemize}
  \tightlist
  \item
    \protect\phantomsection\label{6672479776654687619_37106-h-8.htm.xhtml_id-4580573623381370066}\href{http://www.gutenberg.org/ebooks/3837}{Hospital
    Sketches.}
  \item
    \protect\phantomsection\label{6672479776654687619_37106-h-8.htm.xhtml_id-8753513551664593068}\href{http://www.gutenberg.org/ebooks/4770}{Work.}
  \item
    \protect\phantomsection\label{6672479776654687619_37106-h-8.htm.xhtml_id-3626127986129971520}\href{http://www.gutenberg.org/ebooks/33986}{Comic
    Tragedies.}
  \item
    \protect\phantomsection\label{6672479776654687619_37106-h-8.htm.xhtml_id-1138832304076097056}\href{http://www.gutenberg.org/ebooks/28203}{Moods.}
  \item
    \protect\phantomsection\label{6672479776654687619_37106-h-8.htm.xhtml_id-7236899497256576680}\href{http://www.gutenberg.org/ebooks/54212}{A
    Modern Mephistopheles.}
  \item
    \protect\phantomsection\label{6672479776654687619_37106-h-8.htm.xhtml_id-9083468320340013230}\href{http://www.gutenberg.org/ebooks/38049}{Life
    of Louisa May Alcott.}
  \end{itemize}

  6 vols. 16mo. Each, \$1.50.
\end{itemize}

\begin{center}\rule{0.5\linewidth}{0.5pt}\end{center}

LITTLE, BROWN, \& COMPANY

\emph{Publishers}, 254 WASHINGTON STREET, BOSTON, MASS.

\begin{center}\rule{0.5\linewidth}{0.5pt}\end{center}

\subsubsection{The Little Women
Series}\label{6672479776654687619_37106-h-8.htm.xhtml_pgepubid00051}

The Little Women Series\\
By LOUISA M. ALCOTT

\begin{center}\rule{0.5\linewidth}{0.5pt}\end{center}

\protect\phantomsection\label{6672479776654687619_37106-h-8.htm.xhtml_alcottLittleWomen}
\begin{enumerate}
\item
  {LITTLE WOMEN;} or Meg, Jo, Beth, and Amy

  Illustrated. 16mo. \$1.50.

  A simple story of the home life of four girls. A portrayal of child
  life, natural, wholesome, and inspiring. One of the best and most
  popular children\textquotesingle s books ever written.
\item
  {LITTLE MEN:} Life at Plumfield with Jo\textquotesingle s Boys

  Illustrated. 16mo. \$1.50.

  Gives delightful pictures of boy life at old Plumfield, and is brimful
  of activity, merriment, health, and happiness.
\item
  {JO\textquotesingle S BOYS,} and How They Turned Out

  Illustrated. 16mo. \$1.50.

  This sequel to "Little Men" takes up the story and carries
  Jo\textquotesingle s boys through the home struggles and adventures in
  the outside world until they are fairly launched on the sea of
  manhood.
\item
  {AN OLD-FASHIONED GIRL}

  Illustrated. 16mo. \$1.50.

  The heroine of this book is shown as a possible improvement upon the
  girl of the period, who seems sadly ignorant or ashamed of the good
  old fashions which made women truly beautiful and honored.
\item
  {EIGHT COUSINS;} or, the Aunt-Hill

  Illustrated. 16mo. \$1.50.

  The story of a pretty-faced and sunny-tempered little girl, obliged by
  the death of her parents to live with her uncle and her aunts, thereby
  coming in contact with seven cousins---all boys.
\item
  {ROSE IN BLOOM}

  Illustrated. 16mo. \$1.50.

  This sequel to "Eight Cousins" carries on the story of Rose and the
  cousins, and is full of vivacity, fresh and stirring incident, and
  brilliant character painting.
\item
  {UNDER THE LILACS}

  Illustrated. 16mo. \$1.50.

  Ben and his dog Sancho run away from a circus and find a home with Bob
  and Betty in the old house under the lilacs. Told in Miss
  Alcott\textquotesingle s best style.
\item
  {JACK AND JILL}

  Illustrated. 16mo. \$1.50.

  A vivid yet natural portrayal of home and school life in a New England
  village, full of the sympathetic quality which lends such a charm to
  Miss Alcott\textquotesingle s writings. It is a lively and jolly
  narrative.
\end{enumerate}

{the above eight volumes, uniformly bound, in box, \$12.00}

\hfill\break

\emph{Uniform with "The Little Women Series."}

\begin{description}
\tightlist
\item[COMIC TRAGEDIES]
Written by "Jo" and "Meg," and acted by the "Little Women," with a
Foreword by "Meg." Portraits, etc. 16mo. \$1.50.
\item[LOUISA MAY ALCOTT]
Her Life, Letters, and Journals. Edited by Ednah D. Cheney. With
photogravure portraits, etc. 16mo. \$1.50.
\end{description}

\begin{center}\rule{0.5\linewidth}{0.5pt}\end{center}

\subsubsection{Other Alcott
Collections.}\label{6672479776654687619_37106-h-8.htm.xhtml_pgepubid00052}

Other Stories by LOUISA M. ALCOTT

\begin{center}\rule{0.5\linewidth}{0.5pt}\end{center}

\protect\phantomsection\label{6672479776654687619_37106-h-8.htm.xhtml_alcottCollections}
\begin{itemize}
\item
  {SPINNING-WHEEL STORIES}

  Four volumes of healthy and hearty short stories so told as to
  fascinate the young people, while inculcating sturdy courage and
  kindness to the weak in the boys, and in the girls those virtues which
  fit them for filling a woman\textquotesingle s place in the home.

  \begin{enumerate}
  \item
    SPINNING-WHEEL STORIES

    With twelve initial illustrations. 16mo. \$1.25.
  \item
    SILVER PITCHERS: and Independence

    16mo. \$1.25.
  \item
    PROVERB STORIES

    16mo. \$1.25.
  \item
    A GARLAND FOR GIRLS

    With illustrations. 16mo. \$1.25.
  \end{enumerate}

  The above four volumes, uniformly bound in cloth, gilt, in box,
  \$5.00.

  \hfill\break
\item
  {AUNT JO\textquotesingle S SCRAP BAG}

  Six books of jolly, readable stories told in Miss
  Alcott\textquotesingle s best style and sure to please young people.

  \begin{enumerate}
  \item
    MY BOYS

    Illustrated. 16mo. \$1.00.
  \item
    SHAWL-STRAPS

    Illustrated. Story of a voyage abroad. 16mo. \$1.00
  \item
    CUPID AND CHOW-CHOW

    Illustrated. 16mo. \$1.00.
  \item
    MY GIRLS

    Illustrated. 16mo. \$1.00.
  \item
    JIMMY\textquotesingle S CRUISE IN THE PINAFORE, ETC.

    Illustrated. 16mo. \$1.00.
  \item
    AN OLD-FASHIONED THANKSGIVING

    Illustrated. 16mo. \$1.00.

    The above six volumes, uniformly bound in cloth, gilt, in box,
    \$6.00.
  \end{enumerate}
\item
  {LULU\textquotesingle S LIBRARY}

  Delightful short stories, many of them founded on incidents from Miss
  Alcott\textquotesingle s life. Told so as to attract children, and all
  showing the spirit of cheerful accomplishment in the face of
  discouragements.

  Three volumes. Each, \$1.00. The set, uniformly bound in cloth, gilt,
  in box, \$3.00.

  \hfill\break
\item
  {MISS ALCOTT\textquotesingle S NOVELS}

  \begin{itemize}
  \item
    HOSPITAL SKETCHES

    and Camp and Fireside Stories. With illustrations. 16mo. \$1.50.
  \item
    WORK

    A Story of Experience. Illustrated by Sol Eytinge. 16mo. \$1.50.
  \item
    MOODS

    A Novel. 16mo. \$1.50.
  \item
    A MODERN MEPHISTOPHELES

    and a Whisper in the Dark. 16mo. \$1.50.
  \end{itemize}
\end{itemize}

\begin{center}\rule{0.5\linewidth}{0.5pt}\end{center}

\subsubsection{The Children\textquotesingle s Friend
Series.}\label{6672479776654687619_37106-h-8.htm.xhtml_pgepubid00053}

Other Stories by LOUISA M. ALCOTT

\begin{center}\rule{0.5\linewidth}{0.5pt}\end{center}

\protect\phantomsection\label{6672479776654687619_37106-h-8.htm.xhtml_alcottStories}
\begin{description}
\tightlist
\item[A HOLE IN THE WALL.]
Illustrated. 12mo. 50 cents.

An account of a poor boy\textquotesingle s admiration for a beautiful
garden to which he is invited by a little girl friend. ("How They Camped
Out" in same volume.)
\item[\protect\phantomsection\label{6672479776654687619_37106-h-8.htm.xhtml_id-2264673739379660960}\href{http://www.gutenberg.org/ebooks/5352}{MARJORIE\textquotesingle S
THREE GIFTS.}]
Illustrated. 12mo. 50 cents.

A fairy tale told Marjorie comes true, and there enter into her life
three good fairies: Industry, Cheerfulness, and Love. ("Roses and
Forget-me-nots" in same volume.)
\item[\protect\phantomsection\label{6672479776654687619_37106-h-8.htm.xhtml_id-163893235460928324}\href{http://www.gutenberg.org/ebooks/37981}{MAY
FLOWERS.}]
Illustrated. 12mo. 50 cents.

The experiences of six earnest young girls who try to make the sad lives
about them happier. Full of sensible hints as to wisest methods of
charity.
\item[A CHRISTMAS DREAM.]
Illustrated. 12mo. 50 cents.

A rather spoiled child gets her first real enjoyment of Christmas by
making others happy. ("Baa! Baa!" in same volume.)
\item[\protect\phantomsection\label{6672479776654687619_37106-h-8.htm.xhtml_id-4852434796179260346}\href{http://www.gutenberg.org/ebooks/25165}{THE
CANDY COUNTRY}]
Illustrated. 12mo. 50 cents.

A quaint little fable in which the young heroine visits Candy-land and
is finally contented to return to Bread-land. ("How They Ran Away" in
same volume.)
\item[LITTLE BUTTON ROSE.]
Illustrated. 12mo. 50 cents.

A bright, vivacious child visits her maiden aunts. Her influence on the
somewhat narrow characters about her is delightfully described.
\item[POPPIES AND WHEAT.]
Illustrated. 12mo. 50 cents.

Two girls travel through Europe. The frivolous Ethel learns the
advantages of culture and simple dignity from her companion.
\item[\protect\phantomsection\label{6672479776654687619_37106-h-8.htm.xhtml_id-4531771237995600594}\href{http://www.gutenberg.org/ebooks/37807}{MOUNTAIN-LAUREL
AND MAIDENHAIR.}]
Illustrated. 12mo. 50 cents.

The story of a New Hampshire farmer\textquotesingle s daughter who is
fond of writing verses.
\item[PANSIES AND WATER-LILIES.]
Illustrated. 12mo. 50 cents.

"Pansies" is a story of a girls\textquotesingle{} discussion of books;
"Water-Lilies" a romance by the sea-shore.
\item[THE DOLLS\textquotesingle{} JOURNEY.]
Illustrated. 12mo. 50 cents.

A droll account of the travels of two dolls. ("Shadow-Children" and "The
Moss People" in same volume.)
\item[MORNING GLORIES AND QUEEN ASTOR.]
Illustrated. 12mo. 50 cents.

Aunt Wee changes Daisy from a petulant to a cheerful girl by interesting
her in the wonderful world of nature which Daisy has never before
learned to appreciate justly.
\item[THE LITTLE MEN PLAY.]
\item[THE LITTLE WOMEN PLAY.]
Adapted from Miss Alcott\textquotesingle s famous stories, "Little Men"
and "Little Women," by Elizabeth L. Gould.

Illustrated. 12mo. Price 50 cents each.

Two forty-five minute plays of two acts each, for eight or ten little
people. They will prove a source of limitless delight.
\end{description}

\begin{center}\rule{0.5\linewidth}{0.5pt}\end{center}

LITTLE, BROWN, \& COMPANY, Publishers

254 Washington St., Boston, Mass.

\begin{center}\rule{0.5\linewidth}{0.5pt}\end{center}

\subsection{Transcriber\textquotesingle s
Notes}\label{6672479776654687619_37106-h-8.htm.xhtml_pgepubid00054}

Transcriber\textquotesingle s Notes:

\begin{center}\rule{0.5\linewidth}{0.5pt}\end{center}

{Project Gutenberg} has two versions of Little Women by Louisa May
Alcott.

\begin{itemize}
\tightlist
\item
  \protect\phantomsection\label{6672479776654687619_37106-h-8.htm.xhtml_id-4219720723409952157}\href{http://www.gutenberg.org/ebooks/514}{Original
  Edition}
\item
  \protect\phantomsection\label{6672479776654687619_37106-h-8.htm.xhtml_id-8265715706667296828}\href{http://www.gutenberg.org/ebooks/37106}{Illustrated
  Edition}
\end{itemize}

Welcome to the {Project Gutenberg} Illustrated Edition of Little Women
by Louisa May Alcott, published by Little, Brown, and Company. Some
versions of the book, such as this one, use the full title of the book
from the title page, \emph{Little Women; Or Meg, Jo, Beth, and Amy.}

We used the version of the book from Little, Brown, and Company:
copyright 1896, for this transcription. A scanned copy of this book is
available through the internet archive, courtesy of the New York Public
Library.

A copy of the first version of the novel, published in 1869, was
consulted for emendations, the proper rendering of words hyphenated and
split between two lines for spacing, and other issues in transcribing
the novel. We are not trying to change this version of the novel back to
the 1869 novel, but correct the errors that were made in re-transcribing
and updating the text.

Throughout the dialogues, words were used to mimic accents of the
speakers. Those words were retained as-is.

Errors in punctuations and hyphenation were not corrected unless
otherwise noted below.

On page vii, in the Table of Contents, change page 7 to 1 for the
beginning of Chapter One.

In the List of Illustrations, for the illustration on page 147,
"postmistress" was replaced with "post-mistress".

In the List of Illustrations, for the illustration on page 235,
"tea-pot" was replaced with "teapot".

On page 30, the punctuation after \textquotesingle stained they
are\textquotesingle{} may be a colon, but on page 41 of the 1869 book,
it is a semicolon. We used the semi-colon.

On page 34, transcribe red-headed with the hyphen. See page 44 of the
1869 novel.

On page 40, a period was added after "room where old Mr". See page 50 of
the 1869 novel.

On page 41, the semicolon after "Laurie went on the box" was replaced
with a comma. See page 52 of the 1869 novel.

On page 62, mantel-piece was transcribed with the hyphen. See Page 75 of
the 1869 novel.

On page 63, checked the clause "and I\textquotesingle ve been trying to
do it this ever so long." It is written the same way on page 77 of the
1869 book. No change was made.

On page 64, add period after "red and shining with merriment." See page
79 of the 1869 book.

On page 68, changed weet to sweet in \textquotesingle the damp weet
air.\textquotesingle{} See page 84 of the 1869 novel.

On page 79, add comma after I remain in the letter. See page 95 of the
1869 novel.

On page 84, ferrule was an adjustment from the 1869 book, which only
used one r in spelling the word (see page 101).
Webster\textquotesingle s dictionary spells ferrule with two rs.

On page 109, a period was added after "and the old man quite dotes on
them". See page 130 of the 1869 novel.

On page 109, a period was added after "asked another voice". See page
131 of the 1869 novel.

On page 112, change colon to semicolon after "if you
don\textquotesingle t;"---see page 134 of the 1869 novel.

On page 113, transcribe ear-rings with the hyphen. See page 135 of the
1869 novel.

On Page 123, "One thing remember, my girls:" is written as it appears in
the 1896 novel. A comma instead of a colon was used after my girls in
the 1869 novel (see page 146). "One thing \textbf{to} remember," may
work better today, or even "Remember one thing," but we left this as Ms.
Alcott wrote it.

On Page 124, the P. C. is the Pickwick Club from a novel by Charles
Dickens. Samuel Pickwick, Tracy Tupman, Augustus Snodgrass, and
Nathaniel Winkle were introdued by Charles Dickens in the first chapter
of \emph{The Pickwick Papers}. Samuel Weller makes his first appearance
in Chapter Ten of that novel.

On page 128, in the Pickwick Portfolio, there is no period after "it is
nearly school time" in N.~Winkle\textquotesingle s letter. This period
was also missing on page 151 of the 1869 novel. The missing period was
intentional.

On page 135, the ambiguous punctuation after "Oh, dear, no!" is an
exclamation point. See page 160 of the 1869 novel.

On page 146, change buttonholes to button-holes. On page 173 of the 1869
novel, this word was hyphenated and split between two lines for spacing.
Transcribing the word with the hyphen matches seven other occurrences of
button-hole or botton-holes in the novel. We therefore used the hyphen.

On page 150, "Betty" was replaced with "Bethy". This error was also made
in the 1869 novel---see page 177. The character addressed is Beth.

On page 158, a period was added after "still kneeling". See page 187 of
the 1869 novel.

On page 160, "hard a lee" is spelled the same way in the 1869 novel (see
page 189) and this novel. We left this as is.

On page 166, a comma was added after "Meg" in "Meg obediently
following". See page 195 of the 1869 book.

On page 185, "receptable" was replaced with "receptacle". See page 217
of the 1869 novel.

On page 185, transcribe door-way with the hyphen. See page 217 of the
1869 novel. Also, change doorway to door-way a few lines down the same
page. See page 218 of the 1869 novel.

On page 189, the comma after "published every day" was replaced with a
period. See page 225 of 1869 book.

On page 198, the clause: "Beth, go and ask Mr. Laurence for a couple of
bottles of old wine:" was written as it appeared in the 1896 novel. The
clause ended in with a semi-colon in the 1869 book (see page 234).

On page 200, change needlework to needle-work. See page 236 of the 1869
novel.

On page 209, "turnovers" was replaced with "turn-overs". See page 246 of
the 1869 novel.

On page 214, the single quotation mark before "Head Nurse of Ward" was
replaced with a double quotation mark. See page 252 of the 1869 novel.

On page 218, "Year\textquotesingle s ago" was replaced with "Years ago".
See page 257 of the 1869 novel.

On page 219, "ask him so help" was replaced with "ask him to help". See
page 257 of the 1869 novel.

On page 219, add period after "give it to her." See page 258 of the 1869
novel.

On page 230, "two, A.M." is spelled the same way, with the comma, in
this book and in the 1869 novel (on page 272). The comma was retained.

On page 244, "postscrips" was replaced with "postscripts". See page 287
of the 1869 novel.

On page 279, place exclamation point after won\textquotesingle t in
\textquotesingle No, I won\textquotesingle t!\textquotesingle{} See page
329 in the 1869 novel.

On page 286, "actingly" was replaced with "acting". See page 337 of the
1869 novel.

On page 288, add comma after mankind in the clause "who felt at peace
with all mankind even his mischievous pupil." See page 339 of the 1869
novel.

On page 294, transcribe gray-headed with the hyphen. See page 5 of the
1869 novel.

On page 295, add a comma after salary in the phrase "with an
honestly-earned salary." See page 7 of the 1869 book.

Checked the clause "But once get used to these slight blemishes" on page
297. The sentence appears the same way on page 10 of the 1869 novel.

Checked the clause "People who hire all these things done for them never
know what they lose" on page 298. The sentence has a comma after them,
but is otherwise written the same way on page 11 of the 1869 novel.

On page 299, transcribe door-handles with the hyphen. See page 13 of the
1869 novel.

On page 339, "shortcomings" was replaced with "short-comings". See page
62 of the 1869 novel.

On page 345, "furbelows and notions" was written "furbelows and
quinny-dingles" in the 1869 novel. See page 59 of the 1869 novel. We
made no change, and only point this out because quinny-dingles is such a
memorable word that those intimate with the novel may notice the change.

On page 353, change snowbank to snow-bank. See page 79 of the 1869
novel.

On page 363, a double quotation mark was added before "Cross-patch, draw
the latch". See page 91 of the 1869 book.

On page 379, change period after Jo to a comma in the clause "as for Jo.
she would have gone up". See page 109 of the 1869 book.

On page 380, a comma was added after "all lying down". See page 111 of
the 1869 book.

On page 393, the punctuation after \textquotesingle but so was
everybody\textquotesingle s\textquotesingle{} is difficult to read. It
could be a colon or semicolon. In the 1869 novel, the mark is a
semi-colon (see page 126). We used the semi-colon.

On page 396, the second line of the verse beginning with
"\textquotesingle Out upon you," is indented. In the 1869 version, the
capital B of "Bold-faced jig!\textquotesingle" is lined-up under Out. We
aligned the verse as the 1869 version of the novel-\/-see page 131.

On page 404, add period after heaviness. See page 140 of the 1869 novel.

On page 405, transcribe needle-work with the hyphen. See page 141 of the
1869 book.

On Page 411, a letter is curiously addressed to Betsey, both here and on
page 148 of the 1869 book.

On Page 413, removed double quotes around Yes in "Yes," they say to one
another, these so kind ladies. Instead, place a single quote in front of
Yes, because Bhaer is resuming his quote. The resumed quote concludes
with a single quote after me and mine. See page 151 of the 1869 novel.
The double quote before \textquotesingle he is a stupid old
fellow\textquotesingle{} is actually a triple-nested quote, ending in
make themselves.

On page 417, transcribe Teddy-ism as Teddyism. See page 155 of the 1869
novel.

On page 451, a period was added after "I can\textquotesingle t let you
go". See page 196 of the 1869 novel.

On page 463, "Tarantula" was used as the name of a dance, but the author
might have meant "Tarantella," which is the name of an Italian dance
about tarantulas.

On page 468, transcribe chess-board with the hyphen. See page 218 of the
1869 novel.

On page 512, a period was added after "she said softly". If you see page
272 of the 1869 novel, you will also notice a comma in
\textquotesingle she said, softly.\textquotesingle{} We added the period
but not the comma.

On page 514, the 1869 novel did not have a comma after oar in the
sentence: "I\textquotesingle m not tired; but you may take an oar, if
you like. See page 525 of the 1869 novel. We did not remove the comma.

On page 527, \textquotesingle the "the best nevvy in the
world."\textquotesingle{} was replaced with "the best nevvy in the
world." See page 290 of the 1869 novel.

On page 527, change he to the in \textquotesingle like Jenny and
\textbf{he} ballad\textquotesingle. See page 291 of the 1869 novel.

On page 531, David and Peggotty refer to two characters from the novel
"David Copperfield" by Charles Dickens.

On page 534, change of to off in the clause: Daisy found it impossible
to keep her eyes \textbf{of} her "pitty aunty," ... see page 300 of the
1869 novel.

On page 541, "know\textquotesingle st thou the land where the citron
blooms," was broken into two stanzas in the book for spacing. We
transcribed this as one line. See page 308 in 1869 novel.

On page 551, transcribe Dove-cote with the hyphen. See page 319 of the
1869 novel.

On the first page of ads, a period was added after "THE LITTLE WOMEN
SERIES. \emph{New Illustrated Edition}".

On the second page of ads, in the blurb for the book Comic Tragedies, a
period as placed after "Portraits, etc".

After the novel is a list of The Works of Louisa May Alcott. The list is
not complete: for example, there are no listings for her work as Flora
Fairfield or A.~M. Barnard. Nevertheless, the pages are a fine
structured outline of Ms. Alcott\textquotesingle s best work.

Most of the novels and stories in these four pages are published by
{Project Gutenberg.} We included links to these titles for the
reader\textquotesingle s convenience. A change had to be made for one
item. We had to list each of the three volumes of Lulu\textquotesingle s
Library to provide the links to that book. All links will only work in
the HTML document.

The final page is a listing of eleven stories originally published in
other volumes, such as \emph{Jo\textquotesingle s Scrap-Bag},
\emph{Lulu\textquotesingle s Library}, and \emph{A Garland for Girls.}
These works were subsequently published separately in small volumes,
generally less than 100 pages, in \emph{The Children\textquotesingle s
Friend Series}.

\protect\phantomsection\label{6672479776654687619_37106-h-9.htm.xhtml}{}

\protect\phantomsection\label{6672479776654687619_37106-h-9.htm.xhtml_pg-footer}
\begin{otherlanguage}{english}

\protect\phantomsection\label{6672479776654687619_37106-h-9.htm.xhtml_pg-end-separator}
{*** END OF THE PROJECT GUTENBERG EBOOK LITTLE WOMEN; OR, MEG, JO, BETH,
AND AMY ***}

Updated editions will replace the previous one---the old editions will
be renamed.

Creating the works from print editions not protected by U.S. copyright
law means that no one owns a United States copyright in these works, so
the Foundation (and you!) can copy and distribute it in the United
States without permission and without paying copyright royalties.
Special rules, set forth in the General Terms of Use part of this
license, apply to copying and distributing Project Gutenberg™ electronic
works to protect the PROJECT GUTENBERG™ concept and trademark. Project
Gutenberg is a registered trademark, and may not be used if you charge
for an eBook, except by following the terms of the trademark license,
including paying royalties for use of the Project Gutenberg trademark.
If you do not charge anything for copies of this eBook, complying with
the trademark license is very easy. You may use this eBook for nearly
any purpose such as creation of derivative works, reports, performances
and research. Project Gutenberg eBooks may be modified and printed and
given away---you may do practically ANYTHING in the United States with
eBooks not protected by U.S. copyright law. Redistribution is subject to
the trademark license, especially commercial redistribution.

\protect\phantomsection\label{6672479776654687619_37106-h-9.htm.xhtml_project-gutenberg-license}
START: FULL LICENSE

\subsection{THE FULL PROJECT GUTENBERG
LICENSE}\label{6672479776654687619_37106-h-9.htm.xhtml_pg-footer-heading}

PLEASE READ THIS BEFORE YOU DISTRIBUTE OR USE THIS WORK

To protect the Project Gutenberg™ mission of promoting the free
distribution of electronic works, by using or distributing this work (or
any other work associated in any way with the phrase ``Project
Gutenberg''), you agree to comply with all the terms of the Full Project
Gutenberg™ License available with this file or online at
www.gutenberg.org/license.

Section 1. General Terms of Use and Redistributing Project Gutenberg™
electronic works

1.A. By reading or using any part of this Project Gutenberg™ electronic
work, you indicate that you have read, understand, agree to and accept
all the terms of this license and intellectual property
(trademark/copyright) agreement. If you do not agree to abide by all the
terms of this agreement, you must cease using and return or destroy all
copies of Project Gutenberg™ electronic works in your possession. If you
paid a fee for obtaining a copy of or access to a Project Gutenberg™
electronic work and you do not agree to be bound by the terms of this
agreement, you may obtain a refund from the person or entity to whom you
paid the fee as set forth in paragraph 1.E.8.

1.B. ``Project Gutenberg'' is a registered trademark. It may only be
used on or associated in any way with an electronic work by people who
agree to be bound by the terms of this agreement. There are a few things
that you can do with most Project Gutenberg™ electronic works even
without complying with the full terms of this agreement. See paragraph
1.C below. There are a lot of things you can do with Project Gutenberg™
electronic works if you follow the terms of this agreement and help
preserve free future access to Project Gutenberg™ electronic works. See
paragraph 1.E below.

1.C. The Project Gutenberg Literary Archive Foundation (``the
Foundation'' or PGLAF), owns a compilation copyright in the collection
of Project Gutenberg™ electronic works. Nearly all the individual works
in the collection are in the public domain in the United States. If an
individual work is unprotected by copyright law in the United States and
you are located in the United States, we do not claim a right to prevent
you from copying, distributing, performing, displaying or creating
derivative works based on the work as long as all references to Project
Gutenberg are removed. Of course, we hope that you will support the
Project Gutenberg™ mission of promoting free access to electronic works
by freely sharing Project Gutenberg™ works in compliance with the terms
of this agreement for keeping the Project Gutenberg™ name associated
with the work. You can easily comply with the terms of this agreement by
keeping this work in the same format with its attached full Project
Gutenberg™ License when you share it without charge with others.

1.D. The copyright laws of the place where you are located also govern
what you can do with this work. Copyright laws in most countries are in
a constant state of change. If you are outside the United States, check
the laws of your country in addition to the terms of this agreement
before downloading, copying, displaying, performing, distributing or
creating derivative works based on this work or any other Project
Gutenberg™ work. The Foundation makes no representations concerning the
copyright status of any work in any country other than the United
States.

1.E. Unless you have removed all references to Project Gutenberg:

1.E.1. The following sentence, with active links to, or other immediate
access to, the full Project Gutenberg™ License must appear prominently
whenever any copy of a Project Gutenberg™ work (any work on which the
phrase ``Project Gutenberg'' appears, or with which the phrase ``Project
Gutenberg'' is associated) is accessed, displayed, performed, viewed,
copied or distributed:

\begin{quote}
This eBook is for the use of anyone anywhere in the United States and
most other parts of the world at no cost and with almost no restrictions
whatsoever. You may copy it, give it away or re-use it under the terms
of the Project Gutenberg License included with this eBook or online at
\href{https://www.gutenberg.org}{www.gutenberg.org}. If you are not
located in the United States, you will have to check the laws of the
country where you are located before using this eBook.
\end{quote}

1.E.2. If an individual Project Gutenberg™ electronic work is derived
from texts not protected by U.S. copyright law (does not contain a
notice indicating that it is posted with permission of the copyright
holder), the work can be copied and distributed to anyone in the United
States without paying any fees or charges. If you are redistributing or
providing access to a work with the phrase ``Project Gutenberg''
associated with or appearing on the work, you must comply either with
the requirements of paragraphs 1.E.1 through 1.E.7 or obtain permission
for the use of the work and the Project Gutenberg™ trademark as set
forth in paragraphs 1.E.8 or 1.E.9.

1.E.3. If an individual Project Gutenberg™ electronic work is posted
with the permission of the copyright holder, your use and distribution
must comply with both paragraphs 1.E.1 through 1.E.7 and any additional
terms imposed by the copyright holder. Additional terms will be linked
to the Project Gutenberg™ License for all works posted with the
permission of the copyright holder found at the beginning of this work.

1.E.4. Do not unlink or detach or remove the full Project Gutenberg™
License terms from this work, or any files containing a part of this
work or any other work associated with Project Gutenberg™.

1.E.5. Do not copy, display, perform, distribute or redistribute this
electronic work, or any part of this electronic work, without
prominently displaying the sentence set forth in paragraph 1.E.1 with
active links or immediate access to the full terms of the Project
Gutenberg™ License.

1.E.6. You may convert to and distribute this work in any binary,
compressed, marked up, nonproprietary or proprietary form, including any
word processing or hypertext form. However, if you provide access to or
distribute copies of a Project Gutenberg™ work in a format other than
``Plain Vanilla ASCII'' or other format used in the official version
posted on the official Project Gutenberg™ website (www.gutenberg.org),
you must, at no additional cost, fee or expense to the user, provide a
copy, a means of exporting a copy, or a means of obtaining a copy upon
request, of the work in its original ``Plain Vanilla ASCII'' or other
form. Any alternate format must include the full Project Gutenberg™
License as specified in paragraph 1.E.1.

1.E.7. Do not charge a fee for access to, viewing, displaying,
performing, copying or distributing any Project Gutenberg™ works unless
you comply with paragraph 1.E.8 or 1.E.9.

1.E.8. You may charge a reasonable fee for copies of or providing access
to or distributing Project Gutenberg™ electronic works provided that:

\begin{itemize}
\tightlist
\item
  • You pay a royalty fee of 20\% of the gross profits you derive from
  the use of Project Gutenberg™ works calculated using the method you
  already use to calculate your applicable taxes. The fee is owed to the
  owner of the Project Gutenberg™ trademark, but he has agreed to donate
  royalties under this paragraph to the Project Gutenberg Literary
  Archive Foundation. Royalty payments must be paid within 60 days
  following each date on which you prepare (or are legally required to
  prepare) your periodic tax returns. Royalty payments should be clearly
  marked as such and sent to the Project Gutenberg Literary Archive
  Foundation at the address specified in Section 4, ``Information about
  donations to the Project Gutenberg Literary Archive Foundation.''
\item
  • You provide a full refund of any money paid by a user who notifies
  you in writing (or by e-mail) within 30 days of receipt that s/he does
  not agree to the terms of the full Project Gutenberg™ License. You
  must require such a user to return or destroy all copies of the works
  possessed in a physical medium and discontinue all use of and all
  access to other copies of Project Gutenberg™ works.
\item
  • You provide, in accordance with paragraph 1.F.3, a full refund of
  any money paid for a work or a replacement copy, if a defect in the
  electronic work is discovered and reported to you within 90 days of
  receipt of the work.
\item
  • You comply with all other terms of this agreement for free
  distribution of Project Gutenberg™ works.
\end{itemize}

1.E.9. If you wish to charge a fee or distribute a Project Gutenberg™
electronic work or group of works on different terms than are set forth
in this agreement, you must obtain permission in writing from the
Project Gutenberg Literary Archive Foundation, the manager of the
Project Gutenberg™ trademark. Contact the Foundation as set forth in
Section 3 below.

1.F.

1.F.1. Project Gutenberg volunteers and employees expend considerable
effort to identify, do copyright research on, transcribe and proofread
works not protected by U.S. copyright law in creating the Project
Gutenberg™ collection. Despite these efforts, Project Gutenberg™
electronic works, and the medium on which they may be stored, may
contain ``Defects,'' such as, but not limited to, incomplete, inaccurate
or corrupt data, transcription errors, a copyright or other intellectual
property infringement, a defective or damaged disk or other medium, a
computer virus, or computer codes that damage or cannot be read by your
equipment.

1.F.2. LIMITED WARRANTY, DISCLAIMER OF DAMAGES - Except for the ``Right
of Replacement or Refund'' described in paragraph 1.F.3, the Project
Gutenberg Literary Archive Foundation, the owner of the Project
Gutenberg™ trademark, and any other party distributing a Project
Gutenberg™ electronic work under this agreement, disclaim all liability
to you for damages, costs and expenses, including legal fees. YOU AGREE
THAT YOU HAVE NO REMEDIES FOR NEGLIGENCE, STRICT LIABILITY, BREACH OF
WARRANTY OR BREACH OF CONTRACT EXCEPT THOSE PROVIDED IN PARAGRAPH 1.F.3.
YOU AGREE THAT THE FOUNDATION, THE TRADEMARK OWNER, AND ANY DISTRIBUTOR
UNDER THIS AGREEMENT WILL NOT BE LIABLE TO YOU FOR ACTUAL, DIRECT,
INDIRECT, CONSEQUENTIAL, PUNITIVE OR INCIDENTAL DAMAGES EVEN IF YOU GIVE
NOTICE OF THE POSSIBILITY OF SUCH DAMAGE.

1.F.3. LIMITED RIGHT OF REPLACEMENT OR REFUND - If you discover a defect
in this electronic work within 90 days of receiving it, you can receive
a refund of the money (if any) you paid for it by sending a written
explanation to the person you received the work from. If you received
the work on a physical medium, you must return the medium with your
written explanation. The person or entity that provided you with the
defective work may elect to provide a replacement copy in lieu of a
refund. If you received the work electronically, the person or entity
providing it to you may choose to give you a second opportunity to
receive the work electronically in lieu of a refund. If the second copy
is also defective, you may demand a refund in writing without further
opportunities to fix the problem.

1.F.4. Except for the limited right of replacement or refund set forth
in paragraph 1.F.3, this work is provided to you `AS-IS', WITH NO OTHER
WARRANTIES OF ANY KIND, EXPRESS OR IMPLIED, INCLUDING BUT NOT LIMITED TO
WARRANTIES OF MERCHANTABILITY OR FITNESS FOR ANY PURPOSE.

1.F.5. Some states do not allow disclaimers of certain implied
warranties or the exclusion or limitation of certain types of damages.
If any disclaimer or limitation set forth in this agreement violates the
law of the state applicable to this agreement, the agreement shall be
interpreted to make the maximum disclaimer or limitation permitted by
the applicable state law. The invalidity or unenforceability of any
provision of this agreement shall not void the remaining provisions.

1.F.6. INDEMNITY - You agree to indemnify and hold the Foundation, the
trademark owner, any agent or employee of the Foundation, anyone
providing copies of Project Gutenberg™ electronic works in accordance
with this agreement, and any volunteers associated with the production,
promotion and distribution of Project Gutenberg™ electronic works,
harmless from all liability, costs and expenses, including legal fees,
that arise directly or indirectly from any of the following which you do
or cause to occur: (a) distribution of this or any Project Gutenberg™
work, (b) alteration, modification, or additions or deletions to any
Project Gutenberg™ work, and (c) any Defect you cause.

Section 2. Information about the Mission of Project Gutenberg™

Project Gutenberg™ is synonymous with the free distribution of
electronic works in formats readable by the widest variety of computers
including obsolete, old, middle-aged and new computers. It exists
because of the efforts of hundreds of volunteers and donations from
people in all walks of life.

Volunteers and financial support to provide volunteers with the
assistance they need are critical to reaching Project Gutenberg™'s goals
and ensuring that the Project Gutenberg™ collection will remain freely
available for generations to come. In 2001, the Project Gutenberg
Literary Archive Foundation was created to provide a secure and
permanent future for Project Gutenberg™ and future generations. To learn
more about the Project Gutenberg Literary Archive Foundation and how
your efforts and donations can help, see Sections 3 and 4 and the
Foundation information page at www.gutenberg.org.

Section 3. Information about the Project Gutenberg Literary Archive
Foundation

The Project Gutenberg Literary Archive Foundation is a non-profit
501(c)(3) educational corporation organized under the laws of the state
of Mississippi and granted tax exempt status by the Internal Revenue
Service. The Foundation's EIN or federal tax identification number is
64-6221541. Contributions to the Project Gutenberg Literary Archive
Foundation are tax deductible to the full extent permitted by U.S.
federal laws and your state's laws.

The Foundation's business office is located at 809 North 1500 West, Salt
Lake City, UT 84116, (801) 596-1887. Email contact links and up to date
contact information can be found at the Foundation's website and
official page at www.gutenberg.org/contact

Section 4. Information about Donations to the Project Gutenberg Literary
Archive Foundation

Project Gutenberg™ depends upon and cannot survive without widespread
public support and donations to carry out its mission of increasing the
number of public domain and licensed works that can be freely
distributed in machine-readable form accessible by the widest array of
equipment including outdated equipment. Many small donations (\$1 to
\$5,000) are particularly important to maintaining tax exempt status
with the IRS.

The Foundation is committed to complying with the laws regulating
charities and charitable donations in all 50 states of the United
States. Compliance requirements are not uniform and it takes a
considerable effort, much paperwork and many fees to meet and keep up
with these requirements. We do not solicit donations in locations where
we have not received written confirmation of compliance. To SEND
DONATIONS or determine the status of compliance for any particular state
visit
\href{https://www.gutenberg.org/donate/}{www.gutenberg.org/donate}.

While we cannot and do not solicit contributions from states where we
have not met the solicitation requirements, we know of no prohibition
against accepting unsolicited donations from donors in such states who
approach us with offers to donate.

International donations are gratefully accepted, but we cannot make any
statements concerning tax treatment of donations received from outside
the United States. U.S. laws alone swamp our small staff.

Please check the Project Gutenberg web pages for current donation
methods and addresses. Donations are accepted in a number of other ways
including checks, online payments and credit card donations. To donate,
please visit: www.gutenberg.org/donate.

Section 5. General Information About Project Gutenberg™ electronic works

Professor Michael S. Hart was the originator of the Project Gutenberg™
concept of a library of electronic works that could be freely shared
with anyone. For forty years, he produced and distributed Project
Gutenberg™ eBooks with only a loose network of volunteer support.

Project Gutenberg™ eBooks are often created from several printed
editions, all of which are confirmed as not protected by copyright in
the U.S. unless a copyright notice is included. Thus, we do not
necessarily keep eBooks in compliance with any particular paper edition.

Most people start at our website which has the main PG search facility:
\href{https://www.gutenberg.org}{www.gutenberg.org}.

This website includes information about Project Gutenberg™, including
how to make donations to the Project Gutenberg Literary Archive
Foundation, how to help produce our new eBooks, and how to subscribe to
our email newsletter to hear about new eBooks.

\end{otherlanguage}

\end{document}
